% !Mode:: "TeX:UTF-8"
\part{深度网络:现代实践}
\label{part:deep_networks_modern_practices}

\newpage
本书这一部分总结现代\gls{DL}用于解决实际应用的现状。

\gls{DL}有着悠久的历史和许多愿景。
数种提出的方法尚未完全结出果实。
数个雄心勃勃的目标尚未实现。
这些较不发达的\gls{DL}分支将出现在本书的最后部分。

这一部分仅关注那些基本上已在工业中大量使用的技术方法。

现代\gls{DL}为\gls{supervised_learning}提供了一个强大的框架。
通过添加更多层以及向层内添加更多单元,\gls{deep_network}可以表示复杂性不断增加的函数。
给定足够大的模型和足够大的标注训练数据集,我们可以通过\gls{DL}将输入向量映射到输出向量,完成大多数对人来说能迅速处理的任务。
其他任务,比如不能被描述为将一个向量与另一个相关联的任务,或者对于一个人来说足够困难并需要时间思考和反复琢磨才能完成的任务,现在仍然超出了\gls{DL}的能力范围。

% ??
本书这一部分描述参数化函数近似技术的核心,几乎所有现代实际应用的\gls{DL}背后都用到了这一技术。
首先,我们描述用于表示这些函数的前馈\gls{deep_network}模型。
接着,我们提出正则化和优化这种模型的高级技术。
将这些模型扩展到大输入(如高分辨率图像或长时间序列)需要专门化。
我们将会介绍扩展到大图像的\gls{convolutional_network}和用于处理时间序列的\gls{RNN}。
最后,我们提出实用方法的一般准则,有助于设计、构建和配置一些涉及\gls{DL}的应用,并回顾其中一些应用。

这些章节对于从业者来说是最重要的,也就是现在想开始实现和使用\gls{DL}算法解决现实问题的人需要阅读这些章节。


% !Mode:: "TeX:UTF-8"
% Translator: Kai Li
\chapter{\glsentrytext{deep_feedforward_network}}
\label{chap:deep_feedforward_networks}

\firstgls{deep_feedforward_network},也叫作\firstgls{feedforward_neural_network}或者\firstall{MLP},是典型的深度学习模型。\gls{feedforward_network}的目标是近似某个函数$f^*$。
例如,对于分类器,$y = f^*(\Vx)$将输入$\Vx$映射到一个类别$y$。
\gls{feedforward_network}定义了一个映射$\Vy = f(\Vx; \Vtheta)$,并且学习参数$\Vtheta$的值,使它能够得到最佳的函数近似。

这种模型被称为\firstgls{feedforward}的,是因为信息流过$\Vx$的函数,流经用于定义$f$的中间计算过程,最终到达输出$\Vy$。
%这里的feedforward是翻译成前向的,还是前馈的?
在模型的输出和模型本身之间没有\firstgls{feedback}连接。
当前馈神经网络被扩展成包含\gls{feedback}连接时,它们被称为\firstgls{RNN},在\chapref{chap:sequence_modeling_recurrent_and_recursive_nets}介绍。

\gls{feedforward_network}对于机器学习的从业者是极其重要的。
它们是许多重要商业应用的基础。
例如,用于对照片中的对象进行识别的\gls{CNN}就是一种专门的\gls{feedforward_network}。
\gls{feedforward_network}是通往\gls{recurrent_network}之路的概念基石,后者在自然语言的许多应用中发挥着巨大作用。

\gls{feedforward_neural_network}被称作\firstgls{network}是因为它们通常用许多不同函数复合在一起来表示。
该模型与一个有向无环图相关联,而图描述了函数是如何复合在一起的。
例如,我们有三个函数$f^{(1)}, f^{(2)}$和$f^{(3)}$连接在一个链上以形成$f(\Vx) = f^{(3)}(f^{(2)}(f^{(1)}(\Vx)) )$。
这些链式结构是神经网络中最常用的结构。
在这种情况下,$f^{(1)}$被称为网络的\firstgls{first_layer},$f^{(2)}$被称为\firstgls{second_layer},以此类推。
链的全长称为模型的\firstgls{depth}。
正是因为这个术语才出现了``深度学习''这个名字。
\gls{feedforward_network}的最后一层被称为\firstgls{output_layer}。
在神经网络训练的过程中,我们让$f(\Vx)$去匹配$f^*(\Vx)$的值。
训练数据为我们提供了在不同训练点上取值的、含有噪声的$f^*(\Vx)$的近似实例。
每个样本$\Vx$都伴随着一个标签$y\approx f^*(\Vx)$。
训练样本直接指明了输出层在每一点$\Vx$上必须做什么;它必须产生一个接近$y$的值。
但是训练数据并没有直接指明其他层应该怎么做。
学习算法必须决定如何使用这些层来产生想要的输出,但是训练数据并没有说每个单独的层应该做什么。
相反,学习算法必须决定如何使用这些层来最好地实现$f^*$的近似。
因为训练数据并没有给出这些层中的每一层所需的输出,所以这些层被称为\firstgls{hidden_layer}。

% -- 163 --

最后,这些网络被称为\emph{神经网络}是因为它们或多或少地受到神经科学的启发。
网络中的每个\gls{hidden_layer}通常都是向量值的。这些\gls{hidden_layer}的维数决定了模型的\firstgls{width}。
向量的每个元素都可以被视为起到类似一个神经元的作用。
除了将层想象成向量到向量的单个函数,我们也可以把层想象成由许多并行操作的\firstgls{unit}组成,每个单元表示一个向量到标量的函数。
每个单元在某种意义上类似一个神经元,它接收的输入来源于许多其他的单元,并计算它自己的激活值。
使用多层向量值表示的想法来源于神经科学。
用于计算这些表示的函数$f^{(i)}(\Vx)$的选择,也或多或少地受到神经科学观测的指引,这些观测是关于生物神经元计算功能的。
然而,现代的神经网络研究受到更多的是来自许多数学和工程学科的指引,并且神经网络的目标并不是完美地给大脑建模。
我们最好将\gls{feedforward_neural_network}想成是为了实现统计\gls{generalization}而设计出的函数近似机,它偶尔从我们了解的大脑中提取灵感,但并不是大脑功能的模型。

一种理解\gls{feedforward_network}的方式是从线性模型开始,并考虑如何克服它的局限性。
线性模型,例如\gls{logistic_regression}和线性回归,是非常吸引人的,因为无论是通过闭解形式还是使用凸优化,它们都能高效且可靠地拟合。
线性模型也有明显的缺陷,那就是该模型的能力被局限在线性函数里,所以它无法理解任何两个输入变量间的相互作用。

为了扩展线性模型来表示$\Vx$的非线性函数,我们可以不把线性模型用于$\Vx$本身,而是用在一个变换后的输入$\phi(\Vx)$上,这里$\phi$是一个非线性变换。
同样,我们可以使用\secref{sec:support_vector_machines}中描述的\gls{kernel_trick},来得到一个基于隐含地使用$\phi$映射的非线性学习算法。
我们可以认为$\phi$提供了一组描述$\Vx$的特征,或者认为它提供了$\Vx$的一个新的表示。

% -- 164 --

剩下的问题就是如何选择映射$\phi$。
\begin{enumerate}
\item 其中一种选择是使用一个通用的$\phi$,例如无限维的$\phi$,它隐含地用在基于~\glssymbol{RBF}~核的\gls{kernel_machines}上。
如果$\phi(\Vx)$具有足够高的维数,我们总是有足够的能力来拟合训练集,但是对于测试集的\gls{generalization}往往不佳。
非常通用的特征映射通常只基于局部光滑的原则,并且没有将足够的先验信息进行编码来解决高级问题。

\item 另一种选择是手动地设计$\phi$。
在深度学习出现以前,这一直是主流的方法。
这种方法对于每个单独的任务都需要人们数十年的努力,从业者各自擅长特定的领域(如语音识别或计算机视觉),并且不同领域之间很难迁移(transfer)。

\item 深度学习的策略是去学习$\phi$。
在这种方法中,我们有一个模型$y = f(\Vx;\theta, \Vw) = \phi(\Vx; \theta)^\top \Vw$。
我们现在有两种参数:用于从一大类函数中学习$\phi$的参数$\Vtheta$,以及用于将$\phi(\Vx)$映射到所需的输出的参数$\Vw$。
这是\gls{deep_feedforward_network}的一个例子,其中$\phi$定义了一个\gls{hidden_layer}。
这是三种方法中唯一一种放弃了训练问题的凸性的,但是利大于弊。
在这种方法中,我们将表示参数化为$\phi(\Vx; \Vtheta)$,并且使用优化算法来寻找$\Vtheta$,使它能够得到一个好的表示。
如果我们想要的话,这种方法也可以通过使它变得高度通用以获得第一种方法的优点——我们只需使用一个非常广泛的函数族$\phi(\Vx; \Vtheta)$。
这种方法也可以获得第二种方法的优点。
人类专家可以将他们的知识编码进网络来帮助\gls{generalization},他们只需要设计那些他们期望能够表现优异的函数族$\phi(\Vx; \Vtheta)$即可。
这种方法的优点是人类设计者只需要寻找正确的函数族即可,而不需要去寻找精确的函数。
\end{enumerate}

这种通过学习特征来改善模型的一般化原则不仅仅适用于本章描述的\gls{feedforward_neural_network}。
它是深度学习中反复出现的主题,适用于全书描述的所有种类的模型。
\gls{feedforward_neural_network}是这个原则的应用,它学习从$\Vx$到$\Vy$的确定性映射并且没有反馈连接。
后面出现的其他模型会把这些原则应用到学习随机映射、学习带有反馈的函数以及学习单个向量的概率分布。

% -- 165 --

本章我们先从\gls{feedforward_network}的一个简单例子说起。
接着,我们讨论部署一个\gls{feedforward_network}所需的每个设计决策。
首先,训练一个\gls{feedforward_network}至少需要做和线性模型同样多的设计决策:选择一个优化模型、代价函数以及输出单元的形式。
我们先回顾这些基于梯度学习的基本知识,然后去面对那些只出现在\gls{feedforward_network}中的设计决策。
\gls{feedforward_network}已经引入了\gls{hidden_layer}的概念,这需要我们去选择用于计算\gls{hidden_layer}值的\firstgls{activation_function}。
我们还必须设计网络的结构,包括网络应该包含多少层、这些层应该如何连接,以及每一层包含多少单元。
在深度神经网络的学习中需要计算复杂函数的梯度。
我们给出\firstgls{back_propagation}算法和它的现代推广,它们可以用来高效地计算这些梯度。
最后,我们以某些历史观点来结束这一章。

\section{实例:学习XOR}
\label{sec:example_learning_xor}

为了使\gls{feedforward_network}的想法更加具体,我们首先从一个可以完整工作的\gls{feedforward_network}说起。
这个例子解决一个非常简单的任务:学习XOR函数。

XOR函数(``异或''逻辑)是两个二进制值$x_1$和$x_2$的运算。
当这些二进制值中恰好有一个为1时,XOR函数返回值为1。
其余情况下返回值为0。
XOR函数提供了我们想要学习的目标函数$y = f^*(\Vx)$。
我们的模型给出了一个函数$y=f(\Vx; \Vtheta)$并且我们的学习算法会不断调整参数$\Vtheta$来使得$f$尽可能接近$f^*$。

在这个简单的例子中,我们不会关心统计\gls{generalization}。
我们希望网络在这四个点$\SetX =\{[0, 0]^\top, [0, 1]^\top, [1, 0]^\top, [1, 1]^\top\}$上表现正确。
我们会用全部这四个点来训练我们的网络,唯一的挑战是拟合训练集。

我们可以把这个问题当作是回归问题,并使用\gls{mean_squared_error}损失函数。
我们选择这个损失函数是为了尽可能简化本例中用到的数学。
在应用领域,对于二进制数据建模时,\glssymbol{mean_squared_error}通常并不是一个合适的损失函数。
更加合适的方法将在\secref{sec:sigmoid_units_for_bernoulli_output_distributions}中讨论。

% -- 166 --

评估整个训练集上表现的~\glssymbol{mean_squared_error}~损失函数为
\begin{equation}
J(\Vtheta) = \frac{1}{4} \sum_{\Vx\in \SetX} (f^*(\Vx) - f(\Vx; \Vtheta))^2.
\end{equation}

我们现在必须要选择我们模型$f(\Vx; \Vtheta)$的形式。
假设我们选择一个线性模型,$\Vtheta$包含$\Vw$和$b$,那么我们的模型被定义成
\begin{equation}
f(\Vx; \Vw, b) = \Vx^\top \Vw + b.
\end{equation}
我们可以使用\gls{normal_equations}关于$\Vw$和$b$最小化$J(\Vtheta)$,来得到一个闭式解。

解\gls{normal_equations}以后,我们得到$\Vw = 0$以及$b = \frac{1}{2}$。
线性模型仅仅是在任意一点都输出0.5。
为什么会发生这种事?
\figref{fig:chap6_xor_space_gray}演示了线性模型为什么不能用来表示XOR函数。
解决这个问题的其中一种方法是使用一个模型来学习一个不同的特征空间,在这个空间上线性模型能够表示这个解。
% fig 6.1
\begin{figure}[!htb]
\ifOpenSource
\centerline{\includegraphics{figure.pdf}}
\else
\centerline{\includegraphics{Chapter6/figures/xor_space_gray}}
\fi
\captionsetup{singlelinecheck=off}
\caption{通过学习一个表示来解决XOR问题。
图上的粗体数字标明了学得的函数必须在每个点输出的值。
\emph{(左)}直接应用于原始输入的线性模型不能实现XOR函数。
当$x_1 = 0$时,模型的输出必须随着$x_2$的增大而增大。
当$x_1 = 1$时,模型的输出必须随着$x_2$的增大而减小。
线性模型必须对$x_2$使用固定的系数$w_2$。
因此,线性模型不能使用$x_1$的值来改变$x_2$的系数,从而不能解决这个问题。 
\emph{(右)}在由神经网络提取的特征表示的变换空间中,线性模型现在可以解决这个问题了。
在我们的示例解决方案中,输出必须为1的两个点折叠到了特征空间中的单个点。
换句话说,非线性特征将$\Vx = [1, 0]^\top$和$\Vx = [0, 1]^\top$都映射到了特征空间中的单个点$\Vh = [1, 0]^\top$。
线性模型现在可以将函数描述为$h_1$增大和$h_2$减小。
在该示例中,学习特征空间的动机仅仅是使得模型的能力更大,使得它可以拟合训练集。
在更现实的应用中,学习的表示也可以帮助模型\gls{generalization}。}
\label{fig:chap6_xor_space_gray}
\end{figure}


具体来说,我们这里引入一个非常简单的\gls{feedforward_neural_network},它有一层\gls{hidden_layer}并且\gls{hidden_layer}中包含两个单元。
见\figref{fig:chap6_example_mlp}中对该模型的解释。
这个\gls{feedforward_network}有一个通过函数$f^{(1)}(\Vx;\MW, \Vc)$计算得到的\gls{hidden_unit}的向量$\Vh$。
这些\gls{hidden_unit}的值随后被用作第二层的输入。
第二层就是这个网络的输出层。
输出层仍然只是一个线性回归模型,只不过现在它作用于$\Vh$而不是$\Vx$。
网络现在包含链接在一起的两个函数:$\Vh=f^{(1)}(\Vx; \MW, \Vc)$和$y = f^{(2)}(\Vh; \Vw, b)$,完整的模型是$f(\Vx; \MW, \Vc, \Vw, b) = f^{(2)}(f^{(1)}(\Vx))$。
% fig 6.2
\begin{figure}[!htb]
\ifOpenSource
\centerline{\includegraphics{figure.pdf}}
\else
\centerline{\includegraphics{Chapter6/figures/example_mlp}}
\fi
\captionsetup{singlelinecheck=off}
\caption{使用两种不同样式绘制的\gls{feedforward_network}的示例。
具体来说,这是我们用来解决XOR问题的\gls{feedforward_network}。
它有单个\gls{hidden_layer},包含两个单元。
\emph{(左)}在这种样式中,我们将每个单元绘制为图中的一个节点。
这种风格是清楚而明确的,但对于比这个例子更大的网络,它可能会消耗太多的空间。
\emph{(右)}在这种样式中,我们将表示每一层激活的整个向量绘制为图中的一个节点。
这种样式更加紧凑。
有时,我们对图中的边使用参数名进行注释,这些参数是用来描述两层之间的关系的。
这里,我们用矩阵$\MW$描述从$\Vx$到$\Vh$的映射,用向量$\Vw$描述从$\Vh$到$y$的映射。
当标记这种图时,我们通常省略与每个层相关联的截距参数。}
\label{fig:chap6_example_mlp}
\end{figure}


$f^{(1)}$应该是哪种函数?线性模型到目前为止都表现不错,让$f^{(1)}$也是线性的似乎很有诱惑力。
可惜的是,如果$f^{(1)}$是线性的,那么\gls{feedforward_network}作为一个整体对于输入仍然是线性的。
暂时忽略截距项,假设$f^{(1)}(\Vx)= \MW^\top \Vx$并且$f^{(2)}(\Vh)=\Vh^\top \Vw$,那么$f(\Vx) = \Vw^\top\MW^\top \Vx$。
我们可以将这个函数重新表示成$f(\Vx) = \Vx^\top\Vw'$其中$\Vw' = \MW\Vw$。

% -- 167 --

显然,我们必须用非线性函数来描述这些特征。
大多数神经网络通过仿射变换之后紧跟着一个被称为激活函数的固定非线性函数来实现这个目标,其中仿射变换由学得的参数控制。
我们这里使用这种策略,定义$\Vh=g(\MW^\top \Vx+\Vc)$,其中$\MW$是线性变换的权重矩阵,$\Vc$是偏置。
此前,为了描述线性回归模型,我们使用权重向量和一个标量的偏置参数来描述从输入向量到输出标量的仿射变换。
现在,因为我们描述的是向量$\Vx$到向量$\Vh$的仿射变换,所以我们需要一整个向量的偏置参数。
激活函数$g$通常选择对每个元素分别起作用的函数,有$h_i =g(\Vx^\top \MW_{:, i} + c_i)$。
在现代神经网络中,默认的推荐是使用由激活函数$g(z)=\max\{0, z\}$定义的\firstgls{ReLU}或者称为~\glssymbol{ReLU}~\citep{Jarrett-ICCV2009-small,Nair-2010-small,Glorot+al-AI-2011-small},如\figref{fig:chap6_relu_color}所示。
% fig 6.3
\begin{figure}[!htb]
\ifOpenSource
\centerline{\includegraphics{figure.pdf}}
\else
\centerline{\includegraphics{Chapter6/figures/relu_color}}
\fi
\caption{\gls{rectified_linear}激活函数。
该激活函数是被推荐用于大多数\gls{feedforward_neural_network}的默认激活函数。
将此函数用于线性变换的输出将产生非线性变换。 然而,函数仍然非常接近线性,在这种意义上它是具有两个线性部分的分段线性函数。
由于\gls{ReLU}几乎是线性的,因此它们保留了许多使得线性模型易于使用基于梯度的方法进行优化的属性。 它们还保留了许多使得线性模型能够\gls{generalization}良好的属性。 计算机科学的一个通用原则是,我们可以从最小的组件构建复杂的系统。
就像图灵机的内存只需要能够存储0或1的状态,我们可以从\gls{rectified_linear}函数构建一个\gls{universal_function_approximator}。}
\label{fig:chap6_relu_color}
\end{figure}


我们现在可以指明我们的整个网络是
\begin{equation}
f(\Vx; \MW, \Vc, \Vw, b) = \Vw^\top \max\{ 0, \MW^\top \Vx + \Vc \} +b.
\end{equation}

% -- 168 --

% -- 169 --

我们现在可以给出XOR问题的一个解。
令
\begin{align}
\MW &= \begin{bmatrix}
1 & 1\\
1 & 1
\end{bmatrix},\\
\Vc &= \begin{bmatrix}
0\\
-1
\end{bmatrix},\\
\Vw &= \begin{bmatrix}
1\\
-2
\end{bmatrix},
\end{align}
以及$b=0$。

我们现在可以了解这个模型如何处理一批输入。
令$\MX$表示设计矩阵,它包含二进制输入空间中全部的四个点,每个样本占一行,那么矩阵表示为:
\begin{equation} 
\MX = \begin{bmatrix}
0 & 0\\
0 & 1\\ 
1 & 0\\ 
1 & 1
\end{bmatrix}. 
\end{equation} 
神经网络的第一步是将输入矩阵乘以第一层的权重矩阵: 
\begin{equation}
\MX\MW = \begin{bmatrix} 
0 & 0\\ 
1 & 1\\ 
1 & 1\\ 
2 & 2 
\end{bmatrix}.
\end{equation} 
然后,我们加上偏置向量$\Vc$,得到 
\begin{equation} 
\begin{bmatrix} 
0 & -1\\
1 & 0\\ 
1 & 0\\ 
2 & 1 
\end{bmatrix}. 
\end{equation}
在这个空间中,所有的样本都处在一条斜率为1的直线上。
当我们沿着这条直线移动时,输出需要从0升到1,然后再降回0。
线性模型不能实现这样一种函数。
为了用$\Vh$对每个样本求值,我们使用\gls{rectified_linear_transformation}: 
\begin{equation}
\begin{bmatrix} 
0 & 0\\ 
1 & 0\\ 
1 & 0\\ 
2 & 1 
\end{bmatrix}. 
\end{equation}

% -- 170 --

这个变换改变了样本间的关系。它们不再处于同一条直线上了。
如\figref{fig:chap6_xor_space_gray}所示,它们现在处在一个可以用线性模型解决的空间上。

我们最后乘以一个权重向量$\Vw$:
\begin{equation}
\begin{bmatrix}
0\\
1\\
1\\
0
\end{bmatrix}.
\end{equation}
神经网络对这一批次中的每个样本都给出了正确的结果。

在这个例子中,我们简单地指定了解决方案,然后说明它得到的误差为零。
在实际情况中,可能会有数十亿的模型参数以及数十亿的训练样本,所以不能像我们这里做的那样进行简单地猜解。
与之相对的,基于梯度的优化算法可以找到一些参数使得产生的误差非常小。
我们这里给出的XOR问题的解处在损失函数的全局最小点,所以梯度下降算法可以收敛到这一点。
梯度下降算法还可以找到XOR问题一些其他的等价解。
梯度下降算法的收敛点取决于参数的初始值。
在实践中,梯度下降通常不会找到像我们这里给出的那种干净的、容易理解的、整数值的解。

\section{基于梯度的学习}
\label{sec:gradient_based_learning}

设计和训练神经网络与使用梯度下降训练其他任何机器学习模型并没有太大不同。
在\secref{sec:building_a_machine_learning_algorithm}中,我们描述了如何通过指定一个优化过程、代价函数和一个模型族来构建一个机器学习算法。

我们到目前为止看到的线性模型和神经网络的最大区别,在于神经网络的非线性导致大多数我们感兴趣的\gls{cost_function}都变得非凸。
这意味着神经网络的训练通常使用迭代的、基于梯度的优化,仅仅使得代价函数达到一个非常小的值;而不是像用于训练线性回归模型的线性方程求解器,或者用于训练\gls{logistic_regression}或~\glssymbol{SVM}~的凸优化算法那样保证全局收敛。
凸优化从任何一种初始参数出发都会收敛(理论上如此——在实践中也很鲁棒但可能会遇到数值问题)。
用于非凸损失函数的随机梯度下降没有这种收敛性保证,并且对参数的初始值很敏感。
对于\gls{feedforward_neural_network},将所有的权重值初始化为小随机数是很重要的。
偏置可以初始化为零或者小的正值。
这种用于训练\gls{feedforward_neural_network}以及几乎所有深度模型的迭代的基于梯度的优化算法会在\chapref{chap:optimization_for_training_deep_models}详细介绍,参数初始化会在\secref{sec:parameter_initialization_strategies}中具体说明。
就目前而言,只需要懂得,训练算法几乎总是基于使用梯度来使得代价函数下降的各种方法即可。
一些特别的算法是对梯度下降思想的改进和提纯(在\secref{sec:gradient_based_optimization}中介绍)还有一些更特别的,大多数是对随机梯度下降算法的改进(在\secref{sec:stochastic_gradient_descent_chap5}中介绍)。

% -- 171 --

我们当然也可以用梯度下降来训练诸如线性回归和\gls{SVM}之类的模型,并且事实上当训练集相当大时这是很常用的。
从这点来看,训练神经网络和训练其他任何模型并没有太大区别。
计算梯度对于神经网络会略微复杂一些,但仍然可以很高效而精确地实现。
\secref{sec:back_propagation_and_other_differentiation_algorithms}将会介绍如何用\gls{BP}算法以及它的现代扩展算法来求得梯度。

和其他的机器学习模型一样,为了使用基于梯度的学习方法我们必须选择一个代价函数,并且我们必须选择如何表示模型的输出。
现在,我们重温这些设计上的考虑,并且特别强调神经网络的情景。

\subsection{代价函数}
\label{sec:cost_functions}

深度神经网络设计中的一个重要方面是代价函数的选择。
幸运的是,神经网络的代价函数或多或少是和其他的参数模型例如线性模型的代价函数相同的。

在大多数情况下, 我们的参数模型定义了一个分布$p(\Vy\mid\Vx;\Vtheta)$并且我们简单地使用最大似然原理。
这意味着我们使用训练数据和模型预测间的\gls{cross_entropy}作为代价函数。

有时,我们使用一个更简单的方法,不是预测$\Vy$的完整概率分布,而是仅仅预测在给定$\Vx$的条件下$\Vy$的某种统计量。
某些专门的损失函数允许我们来训练这些估计量的预测器。

用于训练神经网络的完整的代价函数,通常在我们这里描述的基本代价函数的基础上结合一个正则项。
我们已经在\secref{sec:regularization}中看到正则化应用到线性模型中的一些简单的例子。
用于线性模型的权重衰减方法也直接适用于深度神经网络,而且是最流行的正则化策略之一。
用于神经网络的更高级的正则化策略将在\chapref{chap:regularization_for_deep_learning}中讨论。

% -- 172 --

\subsubsection{使用最大似然学习条件分布}
\label{sec:learning_conditional_distributions_with_maximum_likelihood}

大多数现代的神经网络使用最大似然来训练。
这意味着代价函数就是负的对数似然,它与训练数据和模型分布间的\gls{cross_entropy}等价。
这个代价函数表示为
\begin{equation}
J(\Vtheta) = -\SetE_{\RVx, \RVy \sim \hat{p}_\text{data}} \log p_\text{model} (\Vy \mid \Vx).
\end{equation}

代价函数的具体形式随着模型而改变,取决于$\log p_\text{model}$的具体形式。
上述方程的展开形式通常会有一些项不依赖于模型的参数,我们可以舍去。
例如,正如我们在\secref{sec:the_task_t}中看到的,如果$p_\text{model}(\Vy\mid\Vx) = \CalN(\Vy;f(\Vx;\Vtheta), \MI)$,那么我们就重新得到了\gls{mean_squared_error}代价,
\begin{equation}
J(\theta) = \frac{1}{2} \SetE_{\RVx, \RVy \sim  \hat{p}_\text{data}} || \Vy - f(\Vx; \Vtheta) ||^2 + \text{const},
\end{equation}
至少系数$\frac{1}{2}$和常数项不依赖于$\Vtheta$。
舍弃的常数是基于\gls{gaussian_distribution}的方差,在这种情况下我们选择不把它参数化。
之前,我们看到了对输出分布的最大似然估计和对线性模型\gls{mean_squared_error}的最小化之间的等价性,但事实上,这种等价性并不要求$f(\Vx; \Vtheta)$用于预测\gls{gaussian_distribution}的均值。

使用最大似然来导出代价函数的方法的一个优势是,它减轻了为每个模型设计代价函数的负担。
明确一个模型$p(\Vy\mid\Vx)$则自动地确定了一个代价函数$\log p(\Vy\mid\Vx)$。

贯穿神经网络设计的一个反复出现的主题是代价函数的梯度必须足够的大和具有足够的预测性,来为学习算法提供一个好的指引。
饱和(变得非常平)的函数破坏了这一目标,因为它们把梯度变得非常小。
这在很多情况下都会发生,因为用于产生\gls{hidden_unit}或者输出单元的输出的激活函数会饱和。
负的对数似然帮助我们在很多模型中避免这个问题。
很多输出单元都会包含一个指数函数,这在它的变量取绝对值非常大的负值时会造成饱和。
负对数似然代价函数中的对数函数消除了某些输出单元中的指数效果。
我们将会在\secref{sec:output_units}中讨论代价函数和输出单元的选择间的相互作用。

% -- 173 --

用于实现最大似然估计的\gls{cross_entropy}代价函数有一个不同寻常的特性,那就是当它被应用于实践中经常遇到的模型时,它通常没有最小值。
对于离散型输出变量,大多数模型以一种特殊的形式来参数化,即它们不能表示概率零和一,但是可以无限接近。
\gls{logistic_regression}是其中一个例子。
对于实值的输出变量,如果模型可以控制输出分布的密度(例如,通过学习\gls{gaussian_output_distribution}的方差参数),那么它可能对正确的训练集输出赋予极其高的密度,这将导致\gls{cross_entropy}趋向负无穷。
\chapref{chap:regularization_for_deep_learning}中描述的正则化技术提供了一些不同的方法来修正学习问题,使得模型不会通过这种方式来获得无限制的收益。

\subsubsection{学习条件统计量}
\label{sec:learning_conditional_statistics}

有时我们并不是想学习一个完整的概率分布$p(\Vy\mid\Vx; \Vtheta)$,而仅仅是想学习在给定$\Vx$时$\Vy$的某个条件统计量。

例如,我们可能有一个预测器$f(\Vx; \Vtheta)$,我们想用它来预测$\Vy$的均值。
如果我们使用一个足够强大的神经网络,我们可以认为这个神经网络能够表示一大类函数中的任何一个函数$f$,这个类仅仅被一些特征所限制,例如连续性和有界,而不是具有特殊的参数形式。
从这个角度来看,我们可以把代价函数看作是一个\firstgls{functional}而不仅仅是一个函数。
泛函是函数到实数的映射。我们因此可以将学习看作是选择一个函数而不仅仅是选择一组参数。
我们可以设计代价泛函在我们想要的某些特殊函数处取得最小值。
例如,我们可以设计一个代价泛函,使它的最小值处于一个特殊的函数上,这个函数将$\Vx$映射到给定$\Vx$时$\Vy$的期望值。
对函数求解优化问题需要用到\firstgls{calculus_of_variations}这个数学工具,我们将在\secref{sec:calculus_of_variations}中讨论。
理解\gls{calculus_of_variations}对于理解本章的内容不是必要的。
目前,只需要知道\gls{calculus_of_variations}可以被用来导出下面的两个结果。

% -- 174 --

我们使用\gls{calculus_of_variations}导出的第一个结果是解优化问题
\begin{equation}
f^* = \underset{f}{\argmin}  \ \SetE_{\RVx, \RVy \sim  p_\text{data}} ||\Vy-f(\Vx)||^2
\end{equation}
得到
\begin{equation}
f^*(\Vx) = \SetE_{\RVy\sim p_\text{data}(\Vy|\Vx)} [\Vy],
\end{equation}
要求这个函数处在我们要优化的类里。
换句话说,如果我们能够用无穷多的、来源于真实的数据生成分布的样本进行训练,最小化\gls{mean_squared_error}代价函数将得到一个函数,它可以用来对每个$\Vx$的值预测出$\Vy$的均值。

不同的代价函数给出不同的统计量。
第二个使用\gls{calculus_of_variations}得到的结果是
\begin{equation}
f^* = \underset{f}{\argmin} \ \SetE_{\RVx, \RVy \sim  p_\text{data}} ||\Vy - f(\Vx)||_1
\end{equation}
将得到一个函数可以对每个$\Vx$预测$\Vy$取值的\emph{中位数},只要这个函数在我们要优化的函数族里。
这个代价函数通常被称为\firstgls{mean_absolute_error}。

可惜的是,\gls{mean_squared_error}和\gls{mean_absolute_error}在使用基于梯度的优化方法时往往成效不佳。
一些饱和的输出单元当结合这些代价函数时会产生非常小的梯度。
这就是为什么\gls{cross_entropy}代价函数比\gls{mean_squared_error}或者\gls{mean_absolute_error}更受欢迎的原因之一了,即使是在没必要估计整个$p(\Vy\mid\Vx)$分布时。


\subsection{输出单元}
\label{sec:output_units}

代价函数的选择与输出单元的选择紧密相关。
大多数时候,我们简单地使用数据分布和模型分布间的\gls{cross_entropy}。
选择如何表示输出决定了\gls{cross_entropy}函数的形式。

任何可用作输出的神经网络单元,也可以被用作\gls{hidden_unit}。
这里,我们着重讨论将这些单元用作模型输出时的情况,不过原则上它们也可以在内部使用。
我们将在\secref{sec:hidden_units}中重温这些单元,并且给出当它们被用作\gls{hidden_unit}时一些额外的细节。

在本节中,我们假设\gls{feedforward_network}提供了一组定义为$\Vh=f(\Vx;\Vtheta)$的隐藏特征。
输出层的作用是随后对这些特征进行一些额外的变换来完成整个网络必须完成的任务。

% -- 175 --

\subsubsection{用于\glsentrytext{gaussian_output_distribution}的线性单元}
\label{sec:linear_units_for_gaussian_output_distributions}

一种简单的输出单元是基于仿射变换的输出单元,仿射变换不具有非线性。
这些单元往往被直接称为线性单元。

给定特征$\Vh$,线性输出单元层产生一个向量$\hat{\Vy} = \MW^\top \Vh+\Vb$。

线性输出层经常被用来产生条件\gls{gaussian_distribution}的均值:
\begin{equation}
p(\Vy\mid\Vx) = \CalN(\Vy; \hat{\Vy}, \MI ).
\end{equation}
最大化其对数似然此时等价于最小化\gls{mean_squared_error}。

最大似然框架也使得学习\gls{gaussian_distribution}的协方差矩阵更加容易,或更容易地使\gls{gaussian_distribution}的协方差矩阵作为输入的函数。
然而,对于所有输入,协方差矩阵都必须被限定成一个正定矩阵。
线性输出层很难满足这种限定,所以通常使用其他的输出单元来对协方差参数化。
对协方差建模的方法将在\secref{sec:other_output_types}中简要介绍。

因为线性模型不会饱和,所以它们易于采用基于梯度的优化算法,甚至可以使用其他多种优化算法。

\subsubsection{用于\glsentrytext{bernoulli_output_distribution}的\glsentrytext{sigmoid}单元}
\label{sec:sigmoid_units_for_bernoulli_output_distributions}

许多任务需要预测二值型变量$y$的值。
具有两个类的分类问题可以归结为这种形式。

此时最大似然的方法是定义$y$在$\Vx$条件下的~\gls{bernoulli_distribution}。

\gls{bernoulli_distribution}仅需单个参数来定义。
神经网络只需要预测$P(y =1\mid\Vx)$即可。
为了使这个数是有效的概率,它必须处在区间$[0, 1]$中。

为满足该约束条件需要一些细致的设计工作。
假设我们打算使用线性单元,并且通过阈值来限制它成为一个有效的概率:
\begin{equation}
P(y=1 \mid \Vx) = \max \left \{ 0, \min \{1, \Vw^\top \Vh+b \} \right \}.
\end{equation}
这的确定义了一个有效的条件概率分布,但我们无法使用梯度下降来高效地训练它。
当$\Vw^\top \Vh+b$处于单位区间外时,模型的输出对其参数的梯度都将为$\bm{0}$。
梯度为$\bm{0}$通常是有问题的,因为学习算法对于如何改善相应的参数不再具有指导意义。

% -- 176 --

相反,最好是使用一种新的方法来保证无论何时模型给出了错误的答案时,总能有一个较大的梯度。
这种方法是基于使用~\gls{sigmoid}~输出单元结合最大似然来实现的。

\gls{sigmoid}~输出单元定义为
\begin{equation}
\hat{y} = \sigma \left (\Vw^\top \Vh + b \right ),
\end{equation}
这里$\sigma$是\secref{sec:useful_properties_of_common_functions}中介绍的~\gls{logistic_sigmoid}~函数。

我们可以认为~\gls{sigmoid}~输出单元具有两个部分。
首先,它使用一个线性层来计算$z=\Vw^\top \Vh+b$。
接着,它使用~\gls{sigmoid}~激活函数将$z$转化成概率。

我们暂时忽略对于$\Vx$的依赖性,只讨论如何用$z$的值来定义$y$的概率分布。
\gls{sigmoid}~可以通过构造一个非归一化(和不为1)的概率分布$\tilde{P}(y)$来得到。
我们可以随后除以一个合适的常数来得到有效的概率分布。
如果我们假定非归一化的对数概率对$y$ 和$z$是线性的,可以对它取指数来得到非归一化的概率。
我们然后对它归一化,可以发现这服从~\gls{bernoulli_distribution},该分布受$z$的~\gls{sigmoid}~变换控制:
\begin{align}
\log \tilde{P}(y) &= yz,\\
\tilde{P}(y) &= \exp(yz),\\
P(y) &= \frac{\exp(yz)}{\sum_{y' = 0}^1 \exp(y' z)},\\
P(y) &= \sigma((2y-1)z).
\end{align}
基于指数和归一化的概率分布在统计建模的文献中很常见。
用于定义这种二值型变量分布的变量$z$被称为\firstgls{logit}。

% -- 177 --

这种在对数空间里预测概率的方法可以很自然地使用最大似然学习。
因为用于最大似然的代价函数是$-\log P(y\mid\Vx)$,代价函数中的log抵消了~\gls{sigmoid}~中的exp。
如果没有这个效果,\gls{sigmoid}~的饱和性会阻止基于梯度的学习做出好的改进。
我们使用最大似然来学习一个由~\gls{sigmoid}~参数化的~\gls{bernoulli_distribution},它的损失函数为
\begin{align}
J(\Vtheta) &= -\log P(y\mid\Vx)\\
&= -\log \sigma ((2y-1)z)\\
&= \zeta((1-2y)z).
\end{align}

这个推导使用了\secref{sec:useful_properties_of_common_functions}中的一些性质。
通过将损失函数写成~\gls{softplus_function}的形式,我们可以看到它仅仅在$(1-2y)z$取绝对值非常大的负值时才会饱和。
因此饱和只会出现在模型已经得到正确答案时——当$y=1$且$z$取非常大的正值时,或者$y=0$且$z$取非常小的负值时。
当$z$的符号错误时,\gls{softplus_function}的变量$(1-2y)z$可以简化为$|z|$。
当$|z|$变得很大并且$z$的符号错误时,\gls{softplus_function}渐近地趋向于它的变量$|z|$。
对$z$求导则渐近地趋向于$\text{sign}(z)$,所以,对于极限情况下极度不正确的$z$,\gls{softplus_function}完全不会收缩梯度。
这个性质很有用,因为它意味着基于梯度的学习可以很快地改正错误的$z$。

当我们使用其他的损失函数,例如\gls{mean_squared_error}之类的,损失函数会在$\sigma(z)$饱和时饱和。
\gls{sigmoid}~激活函数在$z$取非常小的负值时会饱和到0,当$z$取非常大的正值时会饱和到1。
这种情况一旦发生,梯度会变得非常小以至于不能用来学习,无论此时模型给出的是正确还是错误的答案。
因此,最大似然几乎总是训练~\gls{sigmoid}~输出单元的优选方法。

理论上,\gls{sigmoid}~的对数总是确定和有限的,因为~\gls{sigmoid}~的返回值总是被限制在开区间$(0, 1)$上,而不是使用整个闭区间$[0, 1]$的有效概率。
在软件实现时,为了避免数值问题,最好将负的对数似然写作$z$的函数,而不是$\hat{y}=\sigma(z)$的函数。
如果~\gls{sigmoid}~函数下溢到零,那么之后对$\hat{y}$取对数会得到负无穷。

\subsubsection{用于
\glsentrytext{multinoulli_output_distribution}的softmax单元}
\label{sec:softmax_units_for_multinoulli_output_distributions}

任何时候当我们想要表示一个具有$n$个可能取值的离散型随机变量的分布时,我们都可以使用softmax函数。
它可以看作是~\gls{sigmoid}~函数的扩展,其中~\gls{sigmoid}~函数用来表示二值型变量的分布。

% -- 178 --

softmax函数最常用作分类器的输出,来表示$n$个不同类上的概率分布。
比较少见的是,softmax函数可以在模型内部使用,例如如果我们想要在某个内部变量的$n$个不同选项中进行选择。

在二值型变量的情况下,我们希望计算一个单独的数
\begin{equation}
\hat{y} = P(y=1\mid\Vx).
\end{equation}
因为这个数需要处在0和1之间,并且我们想要让这个数的对数可以很好地用于对数似然的基于梯度的优化,我们选择去预测另外一个数$z=\log \hat{P}(y=1\mid\Vx)$。
对其指数化和归一化,我们就得到了一个由~\gls{sigmoid}~函数控制的~\gls{bernoulli_distribution}。

为了推广到具有$n$个值的离散型变量的情况,我们现在需要创造一个向量$\hat{\Vy}$,它的每个元素是$\hat{y}_i = P(y=i\mid\Vx)$。
我们不仅要求每个$\hat{y}_i$元素介于0和1之间,还要使得整个向量的和为1,使得它表示一个有效的概率分布。
用于~\gls{bernoulli_distribution}的方法同样可以推广到~\gls{multinoulli_distribution}。
首先,线性层预测了未归一化的对数概率:
\begin{equation}
\Vz = \MW^\top \Vh+\Vb,
\end{equation}
其中$z_i=\log \hat{P}(y=i\mid\Vx)$。
softmax函数然后可以对$z$指数化和归一化来获得需要的$\hat{\Vy}$。
最终,softmax函数的形式为
\begin{equation}
\text{softmax}(\Vz)_i = \frac{\exp(z_i)}{\sum_j \exp(z_j)}.
\end{equation}

和~\gls{logistic_sigmoid}一样,当使用最大化对数似然训练softmax来输出目标值$\RSy$时,使用指数函数工作地非常好。
这种情况下,我们想要最大化$\log P(\RSy =i; \Vz)=\log \text{softmax}(\Vz)_i$。
将softmax定义成指数的形式是很自然的因为对数似然中的log可以抵消softmax中的exp:
\begin{equation}
\log \text{softmax}(\Vz)_i = z_i - \log \sum_j \exp(z_j).
\label{eq:6.30}
\end{equation}

% -- 179 --

\eqnref{eq:6.30}中的第一项表示输入$z_i$总是对代价函数有直接的贡献。
因为这一项不会饱和,所以即使$z_i$对\eqnref{eq:6.30}的第二项的贡献很小,学习依然可以进行。
当最大化对数似然时,第一项鼓励$z_i$被推高,而第二项则鼓励所有的$\Vz$被压低。
为了对第二项$\log \sum_j \exp(z_j)$有一个直观的理解,注意到这一项可以大致近似为$\max_j z_j$。
这种近似是基于对任何明显小于$\max_j z_j$的$z_k$,$\exp(z_k)$都是不重要的。
我们能从这种近似中得到的直觉是,负对数似然代价函数总是强烈地惩罚最活跃的不正确预测。
如果正确答案已经具有了softmax的最大输入,那么$-z_i$项和$\log\sum_j \exp(z_j) \approx \max_j z_j = z_i$项将大致抵消。
这个样本对于整体训练代价贡献很小,这个代价主要由其他未被正确分类的样本产生。

到目前为止我们只讨论了一个例子。
总体来说,未正则化的最大似然会驱动模型去学习一些参数,而这些参数会驱动softmax函数来预测在训练集中观察到的每个结果的比率:
\begin{equation}
\text{softmax}(\Vz(\Vx; \Vtheta))_i \approx \frac{\sum_{j=1}^m \bm{1}_{y^{(j)}=i, \Vx^{(j)} = \Vx}  }{ \sum_{j=1}^{m} \bm{1}_{\Vx^{(j)} = \Vx} }.
\end{equation}
因为最大似然是一致的估计量,所以只要模型族能够表示训练的分布,这就能保证发生。
在实践中,有限的模型能力和不完美的优化将意味着模型只能近似这些比率。

除了对数似然之外的许多目标函数对softmax函数不起作用。
具体来说,那些不使用对数来抵消softmax中的指数的目标函数,当指数函数的变量取非常小的负值时会造成梯度消失,从而无法学习。
特别是,平方误差对于softmax单元来说是一个很差的损失函数,即使模型做出高度可信的不正确预测,也不能训练模型改变其输出\citep{Bridle90}。
要理解为什么这些损失函数可能失败,我们需要检查softmax函数本身。

像~\gls{sigmoid}~一样,softmax激活函数可能会饱和。
\gls{sigmoid}~函数具有单个输出,当它的输入极端负或者极端正时会饱和。
对于softmax的情况,它有多个输出值。
当输入值之间的差异变得极端时,这些输出值可能饱和。
当softmax饱和时,基于softmax的许多代价函数也饱和,除非它们能够转化饱和的激活函数。

% -- 180 --

为了说明softmax函数对于输入之间差异的响应,观察到当对所有的输入都加上一个相同常数时softmax的输出不变:
\begin{equation}
\text{softmax}(\Vz) = \text{softmax}(\Vz+c).
\end{equation}
使用这个性质,我们可以导出一个数值方法稳定的softmax函数的变体:
\begin{equation}
\text{softmax}(\Vz) = \text{softmax}(\Vz- \max_i z_i).
\end{equation}
变换后的形式允许我们在对softmax函数求值时只有很小的数值误差,即使是当$z$包含极正或者极负的数时。
观察softmax数值稳定的变体,可以看到softmax函数由它的变量偏离$\max_i z_i$的量来驱动。

当其中一个输入是最大($z_i = \max_i z_i$)并且$z_i$远大于其他的输入时,相应的输出$\text{softmax}(\Vz)_i$会饱和到1。
当$z_i$不是最大值并且最大值非常大时,相应的输出$\text{softmax}(\Vz)_i$也会饱和到0。
这是~\gls{sigmoid}~单元饱和方式的一般化,并且如果损失函数不被设计成对其进行补偿,那么也会造成类似的学习困难。

softmax函数的变量$\Vz$可以通过两种方式产生。
最常见的是简单地使神经网络中softmax之前的层输出$\Vz$的每个元素,就像先前描述的使用线性层$\Vz={W}^\top\Vh+\Vb$。
虽然很直观,但这种方法是对分布的过度参数化。
$n$个输出总和必须为1的约束意味着只有$n-1$个参数是必要的;第$n$个概率值可以通过1减去前面$n-1$个概率来获得。
因此,我们可以强制要求$\Vz$的一个元素是固定的。
例如,我们可以要求$z_n=0$。
事实上,这正是~\gls{sigmoid}~单元所做的。
定义$P(y=1\mid\Vx)=\sigma(z)$等价于用二维的$\Vz$以及$z_1=0$来定义$P(y=1\mid\Vx)=\text{softmax}(\Vz)_1$。
无论是$n-1$个变量还是$n$个变量的方法,都描述了相同的概率分布,但会产生不同的学习机制。
在实践中,无论是过度参数化的版本还是限制的版本都很少有差别,并且实现过度参数化的版本更为简单。

从神经科学的角度看,有趣的是认为softmax是一种在参与其中的单元之间形成竞争的方式:softmax输出总是和为1,所以一个单元的值增加必然对应着其他单元值的减少。
这与被认为存在于皮质中相邻神经元间的侧抑制类似。
在极端情况下(当最大的$a_i$和其他的在幅度上差异很大时),它变成了\firstgls{winner_take_all}的形式(其中一个输出接近1,其他的接近0)。

``softmax''的名称可能会让人产生困惑。
这个函数更接近于argmax函数而不是max函数。
``soft''这个术语来源于softmax函数是连续可微的。
``argmax''函数的结果表示为一个\gls{one_hot}向量(只有一个元素为1,其余元素都为0的向量),不是连续和可微的。
softmax函数因此提供了argmax的``软化''版本。max函数相应的软化版本是$\text{softmax}(\Vz)^\top \Vz$。
可能最好是把softmax函数称为``softargmax'',但当前名称已经是一个根深蒂固的习惯了。

% -- 181 --

\subsubsection{其他的输出类型}
\label{sec:other_output_types}

之前描述的线性、\gls{sigmoid}~和softmax输出单元是最常见的。
神经网络可以推广到我们希望的几乎任何种类的输出层。
最大似然原则给如何为几乎任何种类的输出层设计一个好的代价函数提供了指导。

一般的,如果我们定义了一个条件分布$p(\Vy\mid\Vx; \Vtheta)$,最大似然原则建议我们使用$-\log p(\Vy\mid \Vx;\Vtheta)$作为代价函数。

一般来说,我们可以认为神经网络表示函数$f(\Vx;\Vtheta)$。
这个函数的输出不是对$\Vy$值的直接预测。
相反,$f(\Vx;\Vtheta)=\Vomega$提供了$y$分布的参数。
我们的损失函数就可以表示成$-\log p(\RVy; \Vomega(\Vx))$。

例如,我们想要学习在给定$\RVx$时,$\RVy$的条件\gls{gaussian_distribution}的方差。
简单情况下,方差$\sigma^2$是一个常数,此时有一个解析表达式,这是因为方差的最大似然估计量仅仅是观测值$\RVy$与它们的期望值的差值的平方平均。
一种计算上代价更加高但是不需要写特殊情况代码的方法是简单地将方差作为分布$p(\RVy\mid\Vx)$的其中一个属性,这个分布由$\Vomega=f(\Vx;\Vtheta)$控制。
负对数似然$-\log p(\Vy;\Vomega(\Vx))$将为代价函数提供一个必要的合适项来使我们的优化过程可以逐渐地学到方差。
在标准差不依赖于输入的简单情况下,我们可以在网络中创建一个直接复制到$\Vomega$中的新参数。
这个新参数可以是$\sigma$本身,或者可以是表示$\sigma^2$的参数$v$,或者可以是表示$\frac{1}{\sigma^2}$的参数$\beta$,取决于我们怎样对分布参数化。
我们可能希望模型对不同的$\RVx$值预测出$\RVy$不同的方差。
这被称为\firstgls{heteroscedastic}模型。
在异方差情况下,我们简单地把方差指定为$f(\RVx;\Vtheta)$其中一个输出值。
实现它的典型方法是使用精度而不是方差来表示\gls{gaussian_distribution},就像\eqnref{eq:3.22}所描述的。
在多维变量的情况下,最常见的是使用一个对角精度矩阵
\begin{equation}
\text{diag}(\Vbeta).
\end{equation}
这个公式适用于梯度下降,因为由$\Vbeta$参数化的\gls{gaussian_distribution}的对数似然的公式仅涉及$\beta_i$的乘法和$\log \beta_i$的加法。
乘法、加法和对数运算的梯度表现良好。
相比之下,如果我们用方差来参数化输出,我们需要用到除法。
除法函数在零附近会变得任意陡峭。
虽然大梯度可以帮助学习,但任意大的梯度通常导致不稳定。
如果我们用标准差来参数化输出,对数似然仍然会涉及除法,并且还将涉及平方。
通过平方运算的梯度可能在零附近消失,这使得学习被平方的参数变得困难。
无论我们使用的是标准差,方差还是精度,我们必须确保\gls{gaussian_distribution}的协方差矩阵是正定的。
因为精度矩阵的特征值是协方差矩阵特征值的倒数,所以这等价于确保精度矩阵是正定的。
如果我们使用对角矩阵,或者是一个常数乘以单位矩阵\footnote{译者注:这里原文是``If we use a diagonal matrix, or a scalar times the diagonal matrix...''即``如果我们使用对角矩阵,或者是一个标量乘以对角矩阵...'',但一个标量乘以对角矩阵和对角矩阵没区别,结合上下文可以看出,这里原作者误把``identity''写成了``diagonal matrix'',因此这里采用``常数乘以单位矩阵''的译法。},那么我们需要对模型输出强加的唯一条件是它的元素都为正。
如果我们假设$\Va$是用于确定对角精度的模型的原始激活,那么可以用softplus 函数来获得正的精度向量:$\Vbeta=\Vzeta(\Va)$。
这种相同的策略对于方差或标准差同样适用,也适用于常数乘以单位阵的情况。

% -- 182 --

学习一个比对角矩阵具有更丰富结构的协方差或者精度矩阵是很少见的。
如果协方差矩阵是满的和有条件的,那么参数化的选择就必须要保证预测的协方差矩阵是正定的。
这可以通过写成$\VSigma(\Vx)=\MB(\Vx)\MB^\top (\Vx)$来实现,这里$\MB$是一个无约束的方阵。
如果矩阵是满秩的,那么一个实际问题是计算似然的代价是很高的,计算一个$d\times d$的矩阵的行列式或者$\VSigma(\Vx)$的逆(或者等价地并且更常用地,对它特征值分解或者$\MB(\Vx)$的特征值分解)需要$O(d^3)$的计算量。

% -- 183 --

我们经常想要执行多峰回归(multimodal regression),即预测条件分布$p(\Vy\mid\Vx)$的实值,该条件分布对于相同的$\Vx$值在$\Vy$空间中有多个不同的峰值。
在这种情况下,高斯混合是输出的自然表示\citep{Jacobs-nc91,bishop1994mixture}。
将高斯混合作为其输出的神经网络通常被称为\firstgls{mixture_density_network}。
具有$n$个分量的高斯混合输出由下面的条件分布定义:
\begin{equation}
p(\Vy\mid\Vx) = \sum_{i=1}^n p(\RSc = i \mid \Vx) \CalN(\Vy; \Vmu^{(i)}(\Vx), \VSigma^{(i)}(\Vx)).
\end{equation}
神经网络必须有三个输出:定义$p(\RSc=i\mid\Vx)$的向量,对所有的$i$给出$\Vmu^{(i)}(\Vx)$的矩阵,以及对所有的$i$给出$\VSigma^{(i)}(\Vx)$的张量。
这些输出必须满足不同的约束:
\begin{enumerate}
\item 混合组件$p(\RSc=i\mid\Vx)$:它们由\gls{latent_variable}\footnote{我们之所以认为$\RSc$是\gls{latent}的,是因为我们不能直接在数据中观测到它:给定输入$\RVx$和目标$\RVy$,不可能确切地知道是哪个高斯组件产生$\RVy$,但我们可以想象$\RVy$是通过选择其中一个来产生的,并且将那个未被观测到的选择作为随机变量。}~$\RSc$关联着,在$n$个不同组件上形成~\gls{multinoulli_distribution}。
这个分布通常可以由$n$维向量的softmax来获得,以确保这些输出是正的并且和为1。

\item 均值$\Vmu^{(i)}(\Vx)$:它们指明了与第$i$个高斯组件相关联的中心或者均值,并且是无约束的(通常对于这些输出单元完全没有非线性)。
如果$\RVy$是个$d$维向量,那么网络必须输出一个由$n$个这种$d$维向量组成的$n\times d$ 的矩阵。
用最大似然来学习这些均值要比学习只有一个输出模式的分布的均值稍稍复杂一些。我们只想更新那个真正产生观测数据的组件的均值。
在实践中,我们并不知道是哪个组件产生了观测数据。
负对数似然表达式将每个样本对每个组件的贡献进行赋权,权重的大小由相应的组件产生这个样本的概率来决定。

\item 协方差$\VSigma^{(i)}(\Vx)$:它们指明了每个组件$i$的协方差矩阵。
和学习单个高斯组件时一样,我们通常使用对角矩阵来避免计算行列式。
和学习混合均值时一样,最大似然是很复杂的,它需要将每个点的部分责任分配给每个混合组件。
如果给定了混合模型的正确的负对数似然,梯度下降将自动地遵循正确的过程。
\end{enumerate}
有报告说基于梯度的优化方法对于混合条件高斯(作为神经网络的输出)可能是不可靠的,部分是因为涉及到除法(除以方差)可能是数值不稳定的(当某个方差对于特定的实例变得非常小时,会导致非常大的梯度)。
一种解决方法是\firstgls{clip_gradients}(见\secref{sec:clipping_gradients}),另外一种是启发式缩放梯度\citep{Uria+al-ICML2014}。

% -- 184 --

高斯混合输出在语音生成模型\citep{schuster1999supervised}和物理运动\citep{Graves-arxiv2013}中特别有效。
混合密度策略为网络提供了一种方法来表示多种输出模式,并且控制输出的方差,这对于在这些实数域中获得高质量的结果是至关重要的。
混合密度网络的一个实例如\figref{fig:chap6_mdn_color}所示。
% fig 6.4
\begin{figure}[!htb]
\ifOpenSource
\centerline{\includegraphics{figure.pdf}}
\else
\centerline{\includegraphics{Chapter6/figures/mdn_color}}
\fi
\captionsetup{singlelinecheck=off}
\caption{从具有混合密度输出层的神经网络中抽取的样本。
输入$x$从\gls{uniform_distribution}中采样,输出$y$从$p_{\text{model}}(y\mid x)$中采样。 神经网络能够学习从输入到输出分布的参数的非线性映射。 这些参数包括控制三个组件中的哪一个将产生输出的概率,以及每个组件各自的参数。
每个混合组件都是\gls{gaussian_distribution},具有预测的均值和方差。 输出分布的这些方面都能够相对输入$x$变化,并且以非线性的方式改变。}
\label{fig:chap6_mdn_color}
\end{figure}


一般的,我们可能希望继续对包含更多变量的、更大的向量$\Vy$来建模,并在这些输出变量上施加更多更丰富的结构。
例如,我们可能希望神经网络输出字符序列形成一个句子。
在这些情况下,我们可以继续使用最大似然原理应用到我们的模型$p(\Vy;\Vomega(\Vx))$上,但我们用来描述$\Vy$的模型会变得非常复杂,超出了本章的范畴。
\chapref{chap:sequence_modeling_recurrent_and_recursive_nets}描述了如何使用循环神经网络来定义这种序列上的模型,第\ref{part:deep_learning_research}部分描述了对任意概率分布进行建模的高级技术。

\section{\glsentrytext{hidden_unit}}
\label{sec:hidden_units}

到目前为止,我们集中讨论了神经网络的设计选择,这对于使用基于梯度的优化方法来训练的大多数参数化机器学习模型都是通用的。
现在我们转向一个\gls{feedforward_neural_network}独有的问题:该如何选择\gls{hidden_unit}的类型,这些\gls{hidden_unit}用在模型的\gls{hidden_layer}中。

% -- 185 --

\gls{hidden_unit}的设计是一个非常活跃的研究领域,并且还没有许多明确的指导性理论原则。

\gls{ReLU}是\gls{hidden_unit}极好的默认选择。
许多其他类型的\gls{hidden_unit}也是可用的。
决定何时使用哪种类型的\gls{hidden_unit}是困难的事(尽管\gls{ReLU}通常是一个可接受的选择)。
我们这里描述对于每种\gls{hidden_unit}的一些基本直觉。
这些直觉可以用来建议我们何时来尝试一些单元。
通常不可能预先预测出哪种\gls{hidden_unit}工作得最好。
设计过程充满了试验和错误,先直觉认为某种\gls{hidden_unit}可能表现良好,然后用它组成神经网络进行训练,最后用验证集来评估它的性能。

这里列出的一些\gls{hidden_unit}可能并不是在所有的输入点上都是可微的。
例如,\gls{ReLU} $g(z)=\max\{0, z\}$在$z=0$处不可微。
这似乎使得$g$对于基于梯度的学习算法无效。
在实践中,梯度下降对这些机器学习模型仍然表现得足够好。
部分原因是神经网络训练算法通常不会达到代价函数的局部最小值,而是仅仅显著地减小它的值,如\figref{fig:chap4_approx_opt_color}所示。
这些想法会在\chapref{chap:optimization_for_training_deep_models}中进一步描述。
因为我们不再期望训练能够实际到达梯度为$\bm{0}$的点,所以代价函数的最小值对应于梯度未定义的点是可以接受的。
不可微的\gls{hidden_unit}通常只在少数点上不可微。
一般来说,函数$g(z)$具有左导数和右导数,左导数定义为紧邻在$z$左边的函数的斜率,右导数定义为紧邻在$z$右边的函数的斜率。
只有当函数在$z$处的左导数和右导数都有定义并且相等时,函数在$z$点处才是可微的。
神经网络中用到的函数通常对左导数和右导数都有定义。
在$g(z)=\max\{0,z\}$的情况下,在$z=0$处的左导数是0,右导数是1。
神经网络训练的软件实现通常返回左导数或右导数的其中一个,而不是报告导数未定义或产生一个错误。
这可以通过观察到在数字计算机上基于梯度的优化总是会受到数值误差的影响来启发式地给出理由。
当一个函数被要求计算$g(0)$时,底层值真正为0 是不太可能的。
相对的,它可能是被舍入为0的一个小量$\epsilon$。
在某些情况下,理论上有更好的理由,但这些通常对神经网络训练并不适用。
重要的是,在实践中,我们可以放心地忽略下面描述的\gls{hidden_unit}激活函数的不可微性。

% -- 186 --

除非另有说明,大多数的\gls{hidden_unit}都可以描述为接受输入向量$\Vx$,计算仿射变换$\Vz=\MW^\top \Vx+\Vb$,然后使用一个逐元素的非线性函数$g(\Vz)$。
大多数\gls{hidden_unit}的区别仅仅在于激活函数$g(\Vz)$的形式。

\subsection{\glsentrytext{ReLU}及其扩展}
\label{sec:rectified_linear_units_and_their_generalizations}

\gls{ReLU}使用激活函数$g(z)=\max\{0, z\}$。

\gls{ReLU}易于优化,因为它们和线性单元非常类似。
线性单元和\gls{ReLU}的唯一区别在于\gls{ReLU}在其一半的定义域上输出为零。
这使得只要\gls{ReLU}处于激活状态,它的导数都能保持较大。
它的梯度不仅大而且一致。
整流操作的二阶导数几乎处处为0,并且在\gls{ReLU}处于激活状态时,它的一阶导数处处为1。
这意味着相比于引入二阶效应的激活函数来说,它的梯度方向对于学习来说更加有用。

\gls{ReLU}通常作用于仿射变换之上:
\begin{equation}
\Vh = g(\MW^\top \Vx + \Vb).
\end{equation}
当初始化仿射变换的参数时,可以将$\Vb$的所有元素设置成一个小的正值,例如0.1。
这使得\gls{ReLU}很可能初始时就对训练集中的大多数输入呈现激活状态,并且允许导数通过。

有很多\gls{ReLU}的扩展存在。
大多数这些扩展的表现比得上\gls{ReLU},并且偶尔表现得更好。

\gls{ReLU}的一个缺陷是它们不能通过基于梯度的方法学习那些使它们激活为零的样本。
\gls{ReLU}的各种扩展保证了它们能在各个位置都接收到梯度。

\gls{ReLU}的三个扩展基于当$z_i<0$时使用一个非零的斜率$\alpha_i$:$h_i =g(\Vz, \Valpha)_i = \max(0, z_i) + \alpha_i \min(0, z_i)$。
\firstgls{absolute_value_rectification}固定$\alpha_i=-1$来得到$g(z)=|z|$。
它用于图像中的对象识别\citep{Jarrett-ICCV2009},其中寻找在输入照明极性反转下不变的特征是有意义的。
\gls{ReLU}的其他扩展比这应用地更广泛。
\firstgls{leaky_ReLU}\citep{Maas-et-al-ICML2013}将$\alpha_i$固定成一个类似0.01的小值,\firstgls{PReLU}或者~\textbf{\glssymbol{PReLU}}~将$\alpha_i$作为学习的参数\citep{He-et-al-arxiv2015}。

% -- 187 --

\firstgls{maxout_unit}\citep{Goodfellow_maxout_2013}进一步扩展了\gls{ReLU}。
\gls{maxout_unit}将$\Vz$划分为每组具有$k$个值的组,而不是使用作用于每个元素的函数$g(z)$。
每个\gls{maxout_unit}则输出每组中的最大元素:
\begin{equation}
g(\Vz)_i = \underset{j\in \SetG^{(i)}}{\max} z_j
\end{equation}
这里$\SetG^{(i)}$是组$i$的输入索引集$\{(i-1)k+1, \ldots, ik\}$。
这提供了一种方法来学习对输入$\Vx$空间中多个方向响应的分段线性函数。

\gls{maxout_unit}可以学习具有多达$k$段的分段线性的凸函数。
\gls{maxout_unit}因此可以视为\emph{学习激活函数}本身而不仅仅是单元之间的关系。
使用足够大的$k$,\gls{maxout_unit}可以以任意的精确度来近似任何凸函数。
特别地,具有两块的maxout层可以学习实现和传统层相同的输入$\Vx$的函数,这些传统层可以使用\gls{rectified_linear}激活函数、\gls{absolute_value_rectification}、\gls{leaky_ReLU} 或\gls{PReLU},或者可以学习实现与这些都不同的函数。
maxout层的参数化当然也将与这些层不同,所以即使是maxout学习去实现和其他种类的层相同的$\Vx$的函数这种情况下,学习的机理也是不一样的。

每个~\gls{maxout_unit}现在由$k$个权重向量来参数化,而不仅仅是一个,所以~\gls{maxout_unit}通常比\gls{ReLU}需要更多的正则化。
如果训练集很大并且每个单元的块数保持很低的话,它们可以在没有正则化的情况下工作得不错\citep{cai2013deep}。

\gls{maxout_unit}还有一些其他的优点。
在某些情况下,要求更少的参数可以获得一些统计和计算上的优点。
具体来说,如果由$n$个不同的线性过滤器描述的特征可以在不损失信息的情况下,用每一组$k$个特征的最大值来概括的话,那么下一层可以获得$k$倍更少的权重数 。

因为每个单元由多个过滤器驱动,\gls{maxout_unit}具有一些冗余来帮助它们抵抗一种被称为\firstgls{catastrophic_forgetting}的现象,这个现象是说神经网络忘记了如何执行它们过去训练的任务\citep{Goodfellow+al-ICLR2014-small}。

\gls{ReLU}和它们的这些扩展都是基于一个原则,那就是如果它们的行为更接近线性,那么模型更容易优化。
使用线性行为更容易优化的一般性原则同样也适用于除深度线性网络以外的情景。
循环网络可以从序列中学习并产生状态和输出的序列。
当训练它们时,需要通过一些\gls{time_step}来传播信息,当其中包含一些线性计算(具有大小接近1的某些方向导数)时,这会更容易。
作为性能最好的循环网络结构之一,LSTM通过求和在时间上传播信息,这是一种特别直观的线性激活。
它将在\secref{sec:the_long_short_term_memory_and_other_gated_rnns}中进一步讨论。

% -- 188 --

\subsection{\gls{logistic_sigmoid}与双曲正切函数}
\label{sec:logistic_sigmoid_and_hyperbolic_tangent}

在引入\gls{ReLU}之前,大多数神经网络使用~\gls{logistic_sigmoid}~激活函数
\begin{equation}
g(z) = \sigma(z)
\end{equation}
或者是双曲正切激活函数
\begin{equation}
g(z) = \text{tanh}(z).
\end{equation}
这些激活函数紧密相关,因为$\text{tanh}(z)=2\sigma(2z)-1$。

我们已经看过~\gls{sigmoid}~单元作为输出单元用来预测二值型变量取值为1的概率。
与分段线性单元不同,\gls{sigmoid}~单元在其大部分定义域内都饱和——当$z$取绝对值很大的正值时,它们饱和到一个高值,当$z$取绝对值很大的负值时,它们饱和到一个低值,并且仅仅当$z$接近0时它们才对输入强烈敏感。
\gls{sigmoid}~单元的广泛饱和性会使得基于梯度的学习变得非常困难。
因为这个原因,现在不鼓励将它们用作前馈网络中的\gls{hidden_unit}。
当使用一个合适的代价函数来抵消~\gls{sigmoid}~的饱和性时,它们作为输出单元可以与基于梯度的学习相兼容。

当必须要使用~\gls{sigmoid}~激活函数时,双曲正切激活函数通常要比~\gls{logistic_sigmoid}~函数表现更好。
在$\text{tanh}(0)=0$而$\sigma(0)=\frac{1}{2}$的意义上,它更像是单位函数。
因为$\text{tanh}$在0附近与单位函数类似,训练深层神经网络$\hat{y}=\Vw^\top \text{tanh}(\MU^\top \text{tanh}(\MV^\top \Vx))$类似于训练一个线性模型$\hat{y}= \Vw^\top \MU^\top \MV^\top \Vx$,只要网络的激活能够被保持地很小。
这使得训练$\text{tanh}$网络更加容易。

% -- 189 --

\gls{sigmoid}~激活函数在除了\gls{feedforward_network}以外的情景中更为常见。
循环网络、许多概率模型以及一些\gls{AE}有一些额外的要求使得它们不能使用分段线性激活函数,并且使得~\gls{sigmoid}~单元更具有吸引力,尽管它存在饱和性的问题。

\subsection{其他\gls{hidden_unit}}
\label{sec:other_hidden_units}

也存在许多其他种类的\gls{hidden_unit},但它们并不常用。

一般来说,很多种类的可微函数都表现得很好。
许多未发布的激活函数与流行的激活函数表现得一样好。
为了提供一个具体的例子,作者在MNIST数据集上使用$\Vh=\cos(\MW\Vx+\Vb)$测试了一个\gls{feedforward_network},并获得了小于1\%的误差率,这可以与更为传统的激活函数获得的结果相媲美。
在新技术的研究和开发期间,通常会测试许多不同的激活函数,并且会发现许多标准方法的变体表现非常好。
这意味着,通常新的\gls{hidden_unit}类型只有在被明确证明能够提供显著改进时才会被发布。
新的\gls{hidden_unit}类型如果与已有的\gls{hidden_unit}表现大致相当的话,那么它们是非常常见的,不会引起别人的兴趣。

列出文献中出现的所有\gls{hidden_unit}类型是不切实际的。
我们只对一些特别有用和独特的类型进行强调。

其中一种是完全没有激活函数$g(z)$。
也可以认为这是使用单位函数作为激活函数的情况。
我们已经看过线性单元可以用作神经网络的输出。
它也可以用作\gls{hidden_unit}。
如果神经网络的每一层都仅由线性变换组成,那么网络作为一个整体也将是线性的。
然而,神经网络的一些层是纯线性也是可以接受的。
考虑具有$n$个输入和$p$个输出的神经网络层$\Vh=g(\MW^\top \Vx+\Vb)$。
我们可以用两层来代替它,一层使用权重矩阵$\MU$,另一层使用权重矩阵$\MV$。
如果第一层没有激活函数,那么我们(用$\MU$和$\MV$替代$\MW$)实际上是对基于$\MW$的原始层的权重矩阵进行了因式分解。
分解方法是计算$\Vh=g(\MV^\top \MU^\top \Vx+\Vb)$。
如果$U$产生了$q$个输出,那么$\MU$和$V$一起仅包含$(n+p)q$个参数,而$\MW$包含$np$个参数。
如果$q$很小,这可以在很大程度上节省参数。
这是以将线性变换约束为低秩的代价来实现的,但这些低秩关系往往是足够的。
线性\gls{hidden_unit}因此提供了一种减少网络中参数数量的有效方法。

% -- 190 --


softmax单元是另外一种经常用作输出的单元(如\secref{sec:softmax_units_for_multinoulli_output_distributions}中所描述的),但有时也可以用作\gls{hidden_unit}。
softmax单元很自然地表示具有$k$个可能值的离散型随机变量的概率分布,所以它们可以用作一种开关。
这些类型的\gls{hidden_unit}通常仅用于明确地学习操作内存的高级结构中,将在\secref{sec:explicit_memory}中描述。

其他一些常见的\gls{hidden_unit}类型包括:
\begin{itemize}
\item \firstall{RBF}:$h_i = \exp \left (-\frac{1}{\sigma_i^2}|| \MW_{:,i}-\Vx||^2 \right )$。
这个函数在$\Vx$接近模板$\MW_{:,i}$时更加活跃。
因为它对大部分$\Vx$都饱和到0,因此很难优化。

\item \textbf{\gls{softplus}}函数:$g(a)=\zeta(a)=\log(1+e^a)$。
这是\gls{ReLU}的平滑版本,由~\cite{Dugas01}引入用于函数近似,由~\cite{Nair-2010-small}引入用于无向概率模型的条件分布。
\cite{Glorot+al-AI-2011-small}比较了softplus和\gls{ReLU},发现后者的结果更好。
通常不鼓励使用~\gls{softplus_function}。
softplus表明\gls{hidden_unit}类型的性能可能是非常反直觉的——因为它处处可导或者因为它不完全饱和,人们可能希望它具有优于\gls{ReLU}的点,但根据经验来看,它并没有。

\item \firstgls{hard_tanh}:它的形状和$\text{tanh}$以及\gls{ReLU}类似,但是不同于后者,它是有界的,$g(a)=\max(-1, \min(1,a))$。
它由~\cite{Collobert04}引入。
\end{itemize}

\gls{hidden_unit}的设计仍然是一个活跃的研究领域,许多有用的\gls{hidden_unit}类型仍有待发现。

\section{架构设计}
\label{sec:architecture_design}

神经网络设计的另一个关键点是确定它的架构。
\firstgls{architecture}一词是指网络的整体结构:它应该具有多少单元,以及这些单元应该如何连接。

% -- 191 --

大多数神经网络被组织成称为层的单元组。
大多数神经网络架构将这些层布置成链式结构,其中每一层都是前一层的函数。
在这种结构中,第一层由下式给出:
\begin{equation}
\Vh^{(1)}= g^{(1)}\left ( \MW^{(1)\top} \Vx + \Vb^{(1)}\right );
\end{equation}
第二层由
\begin{equation}
\Vh^{(2)} = g^{(2)}\left ( \MW^{(2)\top}\Vh^{(1)}+\Vb^{(2)} \right );
\end{equation}
给出,以此类推。

在这些链式架构中,主要的架构考虑是选择网络的深度和每一层的宽度。
我们将会看到,即使只有一个\gls{hidden_layer}的网络也足够适应训练集。
更深层的网络通常能够对每一层使用更少的单元数和更少的参数,并且经常容易泛化到测试集,但是通常也更难以优化。
对于一个具体的任务,理想的网络架构必须通过实验,观测在验证集上的误差来找到。

\subsection{万能近似性质和深度}
\label{sec:universal_approximation_properties_and_depth}

线性模型,通过矩阵乘法将特征映射到输出,顾名思义,仅能表示线性函数。
它具有易于训练的优点,因为当使用线性模型时,许多损失函数会导出凸优化问题。
可惜的是,我们经常希望我们的系统学习非线性函数。

乍一看,我们可能认为学习非线性函数需要为我们想要学习的那种非线性专门设计一类模型族。
幸运的是,具有\gls{hidden_layer}的\gls{feedforward_network}提供了一种万能近似框架。
具体来说,\firstgls{universal_approximation_theorem}\citep{Hornik89,Cybenko89}表明,一个\gls{feedforward_neural_network}如果具有线性输出层和至少一层具有任何一种``挤压''性质的激活函数(例如\gls{logistic_sigmoid}激活函数)的\gls{hidden_layer},只要给予网络足够数量的\gls{hidden_unit},它可以以任意的精度来近似任何从一个有限维空间到另一个有限维空间的Borel可测函数。
\gls{feedforward_network}的导数也可以任意好地来近似函数的导数\citep{hornik1990universal}。
Borel可测的概念超出了本书的范畴;对于我们想要实现的目标,只需要知道定义在$\SetR^n$的有界闭集上的任意连续函数是Borel可测的,因此可以用神经网络来近似。
神经网络也可以近似从任何有限维离散空间映射到另一个的任意函数。
虽然原始定理最初以具有特殊激活函数的单元的形式来描述,这个激活函数当变量取绝对值非常大的正值和负值时都会饱和,\gls{universal_approximation_theorem}也已经被证明对于更广泛类别的激活函数也是适用的,其中就包括现在常用的\gls{ReLU}~\citep{Leshno-et-al-1993}。

% -- 192 --

\gls{universal_approximation_theorem}意味着无论我们试图学习什么函数,我们知道一个大的MLP一定能够\emph{表示}这个函数。
然而,我们不能保证训练算法能够\emph{学得}这个函数。
即使MLP能够表示该函数,学习也可能因两个不同的原因而失败。
首先,用于训练的优化算法可能找不到用于期望函数的参数值。
其次,训练算法可能由于过拟合而选择了错误的函数。
回忆\secref{sec:the_no_free_lunch_theorem}中的``没有免费的午餐''定理,说明了没有普遍优越的机器学习算法。
\gls{feedforward_network}提供了表示函数的万能系统,在这种意义上,给定一个函数,存在一个\gls{feedforward_network}能够近似该函数。
不存在万能的过程能够通过检查训练集上的样本,来找到一个能对训练集上没有的点也适用的函数。

\gls{universal_approximation_theorem}说明了,存在一个足够大的网络能够达到我们所希望的任意精度,但是定理并没有说这个网络有多大。
\cite{Barron93}提供了单层网络近似一大类函数所需大小的一些界。
不幸的是,在最坏情况下,可能需要指数数量的\gls{hidden_unit}(可能一个\gls{hidden_unit}对应着一个需要区分的输入配置)。
这在二进制值的情况下很容易看到:向量$\Vv \in \{0,1\}^n$上的可能的二值型函数的数量是$2^{2^n}$,并且选择一个这样的函数需要$2^n$位,这通常需要$O(2^n)$的自由度。

总之,具有单层的\gls{feedforward_network}足以表示任何函数,但是网络层可能大得不可实现,并且可能无法正确地学习和\gls{generalization}。
在很多情况下,使用更深的模型能够减少表示期望函数所需的单元的数量,并且可以减少\gls{generalization}误差。

% -- 193 --

存在一些函数族能够被深度大于某个值$d$的网络高效地近似,而当深度被限制到小于或等于$d$时需要一个远远大于之前的模型。
在很多情况下,浅层模型所需的\gls{hidden_unit}的数量是$n$的指数级。
这个结果最初是在那些不与连续可微的神经网络类似的机器学习模型中被证明,但现在已经扩展到了这些模型。
第一个结果是关于逻辑门电路的\citep{Hastad86}。
后来的工作将这些结果扩展到了具有非负权重的线性阈值单元\citep{Hastad91,Hajnal-et-al-1993},然后扩展到了具有连续值激活的网络\citep{Maass-1992,Maass-et-al-1994}。
许多现代神经网络使用\gls{ReLU}。
\cite{Leshno-et-al-1993}证明带有一大类非多项式激活函数族的浅层网络,包括\gls{ReLU},具有万能的近似性质,但是这些结果并没有强调深度或效率的问题——它们仅指出足够宽的\gls{rectifier_network}能够表示任意函数。
\cite{Montufar-et-al-NIPS2014}指出一些用深度\gls{rectifier_network}表示的函数可能需要浅层网络(一个\gls{hidden_layer})指数级的\gls{hidden_unit}才能表示。
更确切的说,他们说明分段线性网络(可以通过整流非线性或~\gls{maxout_unit}获得)可以表示区域的数量是网络深度的指数级的函数。%这里为什么是非线性?
\figref{fig:chap6_space_folding}解释了带有\gls{absolute_value_rectification}的网络是如何创建函数的镜像图像的,这些函数在某些\gls{hidden_unit}的顶部计算,作用于\gls{hidden_unit}的输入。
每个\gls{hidden_unit}指定在哪里折叠输入空间,来创造镜像响应(在绝对值非线性的两侧)。
通过组合这些折叠操作,我们获得指数级的分段线性区域,他们可以概括所有种类的规则模式(例如,重复)。
% fig 6.5
\begin{figure}[!htb]
\ifOpenSource
\centerline{\includegraphics{figure.pdf}}
\else
\centerline{\includegraphics[width=\textwidth]{Chapter6/figures/space_folding.jpg}}
\fi
\caption{ % This is a screenshot.
关于更深的整流网络具有指数优势的一个直观的几何解释,来自~\cite{Montufar-et-al-NIPS2014}。 
\emph{(左)}\gls{absolute_value_rectification}单元对其输入中的每对镜像点有相同的输出。
镜像的对称轴由单元的权重和偏置定义的超平面给出。 在该单元顶部计算的函数(绿色决策面)将是横跨该对称轴的更简单模式的一个镜像。 
\emph{(中)}该函数可以通过折叠对称轴周围的空间来得到。
\emph{(右)}另一个重复模式可以在第一个的顶部折叠(由另一个下游单元)以获得另外的对称性(现在重复四次,使用了两个\gls{hidden_layer})。 
经~\cite{Montufar-et-al-NIPS2014}许可改编此图。}
\label{fig:chap6_space_folding}
\end{figure}


\cite{Montufar-et-al-NIPS2014}的主要定理指出,具有$d$个输入、深度为$l$、每个\gls{hidden_layer}具有$n$个单元的深度\gls{rectifier_network}可以描述的线性区域的数量是
\begin{equation}
O \left ( {n \choose d}^{d(l-1)} n^d \right ),
\end{equation}
意味着,这是深度$l$的指数级。
在每个单元具有$k$个过滤器的maxout网络中,线性区域的数量是
\begin{equation}
O \left ( k^{(l-1)+d} \right ).
\end{equation}

% -- 194 --

当然,我们不能保证在机器学习(特别是AI)的应用中我们想要学得的函数类型享有这样的属性。

我们还可能出于统计原因来选择深度模型。
任何时候,当我们选择一个特定的机器学习算法时,我们隐含地陈述了一些先验,这些先验是关于算法应该学得什么样的函数的。
选择深度模型默许了一个非常普遍的信念,那就是我们想要学得的函数应该涉及几个更加简单的函数的组合。
这可以从表示学习的观点来解释, 我们相信学习的问题包含发现一组潜在的\gls{factors_of_variation},它们可以根据其他更简单的潜在的\gls{factors_of_variation}来描述。
或者,我们可以将深度结构的使用解释为另一种信念,那就是我们想要学得的函数是包含多个步骤的计算机程序,其中每个步骤使用前一步骤的输出。
这些中间输出不一定是\gls{factors_of_variation},而是可以类似于网络用来组织其内部处理的计数器或指针。
根据经验,更深的模型似乎确实在广泛的任务中泛化得更好\citep{Bengio-NIPS2007,Erhan2009,Bengio-2009-book,UTLC+LISA-2011-small,Ciresan-et-al-2012,Krizhevsky-2012-small,sermanet-cvpr-13,Farabet-et-al-2013,couprie-iclr-13,LISA-EmotiW2013,Goodfellow+et+al-ICLR2014a,Szegedy-et-al-arxiv2014}。
\figref{fig:chap6_depth_color}和\figref{fig:chap6_model_size_color}展示了一些实验结果的例子。
这表明使用深层架构确实在模型学习的函数空间上表示了一个有用的先验。
% fig 6.6
\begin{figure}[!htb]
\ifOpenSource
\centerline{\includegraphics{figure.pdf}}
\else
\centerline{\includegraphics{Chapter6/figures/depth_color}}
\fi
\caption{深度的影响。
实验结果表明,当从地址照片转录多位数字时,更深层的网络能够更好地\gls{generalization}。
数据来自~\cite{Goodfellow+et+al-ICLR2014a}。
测试集上的准确率随着深度的增加而不断增加。
\figref{fig:chap6_model_size_color}给出了一个对照实验,它说明了对模型尺寸其他方面的增加并不能产生相同的效果。}
\label{fig:chap6_depth_color}
\end{figure}

% fig 6.7
\begin{figure}[!htb]
\ifOpenSource
\centerline{\includegraphics{figure.pdf}}
\else
\centerline{\includegraphics{Chapter6/figures/model_size_color}}
\fi
\caption{参数数量的影响。
更深的模型往往表现更好。
这不仅仅是因为模型更大。
\cite{Goodfellow+et+al-ICLR2014a}的这项实验表明,增加\gls{convolutional_network}层中参数的数量,但是不增加它们的深度,在提升测试集性能方面没有同样显著的效果,如此图所示。
图例标明了每条曲线对应的网络深度,以及曲线表示的是卷积层还是全连接层的大小变化。
我们可以观察到,在这种情况下,浅层模型在参数数量达到2000万时就过拟合,而深层模型在参数数量超过6000万时仍然表现良好。
这表明,使用深层模型表达出了对模型可以学习的函数空间的有用偏好。
具体来说,它表达了一种信念,即该函数应该由许多更简单的函数复合在一起而得到。
这可能导致学习由更简单的表示所组成的表示(例如,由边所定义的角)或者学习具有顺序依赖步骤的程序(例如,首先定位一组对象,然后分割它们,之后识别它们)。}
\label{fig:chap6_model_size_color}
\end{figure}

\subsection{其他架构上的考虑}
\label{sec:other_architectural_considerations}

目前为止,我们都将神经网络描述成层的简单链式结构,主要的考虑因素是网络的深度和每层的宽度。
在实践中,神经网络显示出相当的多样性。

许多神经网络架构已经被开发用于特定的任务。
用于计算机视觉的卷积神经网络的特殊架构将在\chapref{chap:convolutional_networks}中介绍。
\gls{feedforward_network}也可以推广到用于序列处理的循环神经网络,但有它们自己的架构考虑,将在\chapref{chap:sequence_modeling_recurrent_and_recursive_nets}中介绍。

% -- 195 --

一般的,层并不一定都需要连接在一条链中,尽管这是最常见的做法。
许多架构构建了一个主链,但随后又添加了额外的架构特性,例如从层$i$到层$i+2$或者更高层的跳跃连接。
这些跳跃连接使得梯度更容易从输出层流向更接近输入的层。

架构设计考虑的另外一个关键点是如何将层与层之间连接起来。
默认的神经网络层采用矩阵$\MW$描述的线性变换,每个输入单元连接到每个输出单元。
在之后章节中的许多专用网络具有较少的连接,使得输入层中的每个单元仅连接到输出层单元的一个小子集。
这些用于减少连接数量的策略减少了参数的数量以及用于评估网络的计算量,但通常高度依赖于问题。
例如,\chapref{chap:convolutional_networks}描述的卷积神经网络使用对于计算机视觉问题非常有效的稀疏连接的专用模式。
在这一章中,很难对通用神经网络的架构给出更多具体的建议。
我们在随后的章节中介绍一些特殊的架构策略,可以在不同的领域工作良好。

% -- 196 --

\section{\glsentrytext{BP}和其他的微分算法}
\label{sec:back_propagation_and_other_differentiation_algorithms}

当我们使用\gls{feedforward_neural_network}接收输入$\Vx$并产生输出$\hat{\Vy}$时,信息通过网络向前流动。
输入$\Vx$提供初始信息,然后传播到每一层的\gls{hidden_unit},最终产生输出$\hat{\Vy}$。
这称之为~\firstgls{forward_propagation}。
在训练过程中,前向传播可以持续向前直到它产生一个标量代价函数$J(\Vtheta)$。
\firstgls{back_propagation}算法\citep{RHW},经常简称为\textbf{\glsdesc{BP}},允许来自代价函数的信息通过网络向后流动,以便计算梯度。

% -- 197 --

计算梯度的解析表达式是很直观的,但是数值化地求解这样的表达式在计算上的代价可能很大。
反向传播算法使用简单和廉价的程序来实现这个目标。

反向传播这个术语经常被误解为用于多层神经网络的整个学习算法。
实际上,反向传播仅指用于计算梯度的方法,而另一种算法,例如随机梯度下降,使用该梯度来进行学习。
此外,反向传播经常被误解为仅适用于多层神经网络,但是原则上它可以计算任何函数的导数(对于一些函数,正确的响应是报告函数的导数是未定义的)。
特别地,我们会描述如何计算一个任意函数$f$的梯度$\nabla_{\Vx} f(\Vx, \Vy)$,其中$\Vx$是一组变量,我们需要它们的导数,而$\Vy$是函数的另外一组输入变量,但我们并不需要它们的导数。
在学习算法中,我们最常需要的梯度是代价函数关于参数的梯度,即$\nabla_{\Vtheta} J(\Vtheta)$。
许多机器学习任务需要计算其他导数,来作为学习过程的一部分,或者用来分析学得的模型。
反向传播算法也适用于这些任务,不局限于计算代价函数关于参数的梯度。
通过在网络中传播信息来计算导数的想法非常普遍,它还可以用于计算诸如多输出函数$f$的~\gls{jacobian}~的值。
我们这里描述的是最常用的情况,其中$f$只有单个输出。

\subsection{计算图}
\label{sec:computational_graphs}

目前为止,我们已经用相对非正式的图形语言讨论了神经网络。
为了更精确地描述反向传播算法,使用更精确的\firstgls{computational_graph}语言是很有帮助的。

将计算形式化为图形的方法有很多。

这里,我们使用图中的每一个节点来表示一个变量。
变量可以是标量、向量、矩阵、张量、或者甚至是另一类型的变量。

为了形式化我们的图形,我们还需引入\firstgls{operation}这一概念。
操作是指一个或多个变量的简单函数。
我们的图形语言伴随着一组被允许的操作。
我们可以通过将多个操作复合在一起来描述更为复杂的函数。

% -- 198 --

不失一般性,我们定义一个操作仅返回单个输出变量。
这并没有失去一般性,是因为输出变量可以有多个条目,例如向量。
反向传播的软件实现通常支持具有多个输出的操作,但是我们在描述中避免这种情况,因为它引入了对概念理解不重要的许多额外细节。

如果变量$y$是变量$x$通过一个操作计算得到的,那么我们画一条从$x$到$y$的有向边。
我们有时用操作的名称来注释输出的节点,当上下文很明确时,有时也会省略这个标注。

计算图的实例可以参考\figref{fig:chap6_computational_graph}。
% fig 6.8
\begin{figure}[!htb]
\ifOpenSource
\centerline{\includegraphics{figure.pdf}}
\else
\centerline{\includegraphics{Chapter6/figures/computational_graph}}
\fi
\captionsetup{singlelinecheck=off}
\caption{一些计算图的示例。
\emph{(a)}使用$\times$操作计算$z = xy$的图。 
\emph{(b)}用于\gls{logistic_regression}预测$\hat{y} = \sigma(\Vx^\top \Vw + b)$的图。 一些中间表达式在代数表达式中没有名称,但在图形中却需要。
我们简单地将第$i$个这样的变量命名为$\Vu^{(i)}$。
\emph{(c)}表达式$\MH = \max \{ 0, \MX\MW+ \Vb \}$的计算图,在给定包含小批量输入数据的设计矩阵$\MX$时,它计算\gls{ReLU}激活的设计矩阵$\MH$。
\emph{(d)}示例a-c对每个变量最多只实施一个操作,但是对变量实施多个操作也是可能的。 这里我们展示一个计算图,它对线性回归模型的权重$\Vw$实施多个操作。
这个权重不仅用于预测$\hat{y}$,也用于权重衰减罚项$\lambda \sum_i w_i^2$。}
\label{fig:chap6_computational_graph}
\end{figure}

\subsection{微积分中的链式法则}
\label{sec:chain_rule_of_calculus}

微积分中的链式法则(为了不与概率中的链式法则相混淆)用于计算复合函数的导数。
反向传播是一种计算链式法则的算法,使用高效的特定运算顺序。

设$x$是实数,$f$和$g$是从实数映射到实数的函数。
假设$y=g(x)$并且$z=f(g(x))=f(y)$。
那么链式法则是说
\begin{equation}
\frac{dz}{dx}=\frac{dz}{dy} \frac{dy}{dx}.
\label{eq:6.44}
\end{equation}

我们可以将这种标量情况进行扩展。
假设$\Vx\in \SetR^m, \Vy\in \SetR^n$,$g$是从$\SetR^m$到$\SetR^n$的映射,$f$是从$\SetR^n$到$\SetR$的映射。
如果$\Vy=g(\Vx)$并且$z=f(\Vy)$,那么
\begin{equation}
\frac{\partial z}{\partial x_i} = \sum_j \frac{\partial z}{\partial y_j} \frac{\partial y_j}{\partial x_i}.
\end{equation}
使用向量记法,可以等价地写成
\begin{equation}
\nabla_{\Vx}z = \left ( \frac{\partial \Vy}{\partial \Vx} \right )^\top \nabla_{\Vy} z,
\end{equation}
这里$\frac{\partial \Vy}{\partial \Vx}$是$g$的$n\times m$的~\gls{jacobian_matrix}。

从这里我们看到,变量$\Vx$的梯度可以通过~\gls{jacobian_matrix}$\frac{\partial \Vy}{\partial \Vx}$和梯度$\nabla_{\Vy} z$相乘来得到。
反向传播算法由图中每一个这样的~\gls{jacobian}~梯度的乘积操作所组成。

% -- 199 --

通常我们将反向传播算法应用于任意维度的张量,而不仅仅用于向量。
从概念上讲,这与使用向量的反向传播完全相同。 
唯一的区别是如何将数字排列成网格以形成张量。 
我们可以想象,在我们运行反向传播之前,将每个张量变平为一个向量,计算一个向量值梯度,然后将该梯度重新构造成一个张量。
从这种重新排列的观点上看,反向传播仍然只是将~\gls{jacobian}~乘以梯度。

% -- 200 --

为了表示值$z$关于张量$\TSX$的梯度,我们记为$\nabla_\TSX z$,就像$\TSX$是向量一样。
$\TSX$的索引现在有多个坐标——例如,一个3维的张量由三个坐标索引。
我们可以通过使用单个变量$i$来表示完整的索引元组,从而完全抽象出来。
对所有可能的元组$i$,$(\nabla_\TSX z)_i$给出$\frac{\partial z}{\partial \TEX_i}$。
这与向量中索引的方式完全一致,$(\nabla_{\Vx} z)_i$给出$\frac{\partial z}{\partial x_i}$。
使用这种记法,我们可以写出适用于张量的链式法则。
如果$\TSY=g(\TSX)$并且$z=f(\TSY)$,那么
\begin{equation}
  \nabla_\TSX z = \sum_j (\nabla_\TSX \TEY_j)\frac{\partial z}{\partial \TEY_j}.
  \label{eq:6.47}
\end{equation}

\subsection{递归地使用链式法则来实现\glsentrytext{BP}}
\label{sec:recursively_applying_the_chain_rule_to_obtain_backprop}

使用链式规则,我们可以直接写出某个标量关于计算图中任何产生该标量的节点的梯度的代数表达式。
然而,实际在计算机中计算该表达式时会引入一些额外的考虑。

具体来说,许多子表达式可能在梯度的整个表达式中重复若干次。
任何计算梯度的程序都需要选择是存储这些子表达式还是重新计算它们几次。
\figref{fig:chap6_repeated_subexpression}给出了一个例子来说明这些重复的子表达式是如何出现的。
在某些情况下,计算两次相同的子表达式纯粹是浪费。
在复杂图中,可能存在指数多的这种计算上的浪费,使得简单的链式法则不可实现。
在其他情况下,计算两次相同的子表达式可能是以较高的运行时间为代价来减少内存开销的有效手段。

我们首先给出一个版本的反向传播算法,它指明了梯度的直接计算方式(\algref{alg:bprop}以及相关的正向计算的\algref{alg:fprop}),按照它实际完成的顺序并且递归地使用链式法则。
我们可以直接执行这些计算或者将算法的描述视为用于计算反向传播的计算图的符号表示。
然而,这些公式并没有明确地操作和构造用于计算梯度的符号图。
这些公式将在后面的\secref{sec:general_back_propagation}和\algref{alg:backprop}中给出,其中我们还推广到了包含任意张量的节点。

首先考虑描述如何计算单个标量$u^{(n)}$(例如训练样本上的损失函数)的计算图。
我们想要计算这个标量对$n_i$个输入节点$u^{(1)}$到$u^{(n_i)}$的梯度。
换句话说,我们希望对所有的$i\in\{1,2,\ldots,n_i\}$计算$\frac{\partial u^{(n)}}{\partial u^{(i)}}$。
在使用反向传播计算梯度来实现参数的梯度下降时,$u^{(n)}$将对应单个或者小批量实例的代价函数,而$u^{(1)}$到$u^{(n_i)}$则对应于模型的参数。

% -- 201 --

我们假设图的节点已经以一种特殊的方式被排序,使得我们可以一个接一个地计算他们的输出,从$u^{(n_i+1)}$开始,一直上升到$u^{(n)}$。
如\algref{alg:fprop}中所定义的,每个节点$u^{(i)}$与操作$f^{(i)}$相关联,并且通过对以下函数求值来得到
\begin{equation}
  u^{(i)} = f(\SetA^{(i)}),
\end{equation}
其中$\SetA^{(i)}$是$u^{(i)}$所有父节点的集合。
% alg 6.1
\begin{algorithm}[htbp]
\caption{计算将$n_i$个输入$u^{(1)}$到$u^{(n_i)}$映射到一个输出$u^{(n)}$的程序。
这定义了一个计算图,其中每个节点通过将函数$f^{(i)}$应用到变量集合$\SetA^{(i)}$上来计算$u^{(i)}$的值,$\SetA^{(i)}$包含先前节点$u^{(j)}$的值满足$j<i$且$j \in Pa(u^{(i)})$。
计算图的输入是向量$\Vx$,并且被分配给前$n_i$个节点$u^{(1)}$到$u^{(n_i)}$。计算图的输出可以从最后一个(输出)节点$u^{(n)}$读出。}
\label{alg:fprop}
\begin{algorithmic}
\FOR {$i=1, \ldots, n_i$}
 \STATE $u^{(i)} \leftarrow x_i$
\ENDFOR
\FOR {$i=n_i+1, \ldots, n$}
 \STATE $\SetA^{(i)} \leftarrow \{ u^{(j)} \mid j \in Pa(u^{(i)}) \}$
 \STATE $u^{(i)} \leftarrow f^{(i)}(\SetA^{(i)})$
\ENDFOR
\STATE {\bf return} $u^{(n)}$
\end{algorithmic}
\end{algorithm}

% -- 202 --

该算法详细说明了前向传播的计算,我们可以将其放入图$\CalG$中。
为了执行反向传播,我们可以构造一个依赖于$\CalG$并添加额外一组节点的计算图。
这形成了一个子图$\CalB$,它的每个节点都是$\CalG$的节点。
$\CalB$中的计算和$\CalG$中的计算顺序完全相反,而且$\CalB$中的每个节点计算导数$\frac{\partial u^{(n)}}{\partial u^{(i)}}$与前向图中的节点$u^{(i)}$相关联。
这通过对标量输出$u^{(n)}$使用链式法则来完成:
\begin{equation}
  \frac{\partial u^{(n)}}{\partial u^{(j)}} = \sum_{i:j \in Pa(u^{(i)})} \frac{\partial u^{(n)} }{ \partial u^{(i)} } \frac{ \partial u^{(i)} }{ \partial u^{(j)} }
  \label{eq:6.49}
\end{equation}
这在\algref{alg:bprop}中详细说明。
子图$\CalB$恰好包含每一条对应着$\CalG$中从节点$u^{(j)}$到节点$u^{(i)}$的边。
从$u^{(j)}$到$u^{(i)}$的边对应着计算$\frac{\partial u^{(i)}}{\partial u^{(j)}}$。
另外,对于每个节点都要执行一个内积,内积的一个因子是对于$u^{j}$子节点$u^{(i)}$的已经计算的梯度,另一个因子是对于相同子节点$u^{(i)}$ 的偏导数$\frac{\partial u^{(i)}}{\partial u^{(j)}}$组成的向量。
总而言之,执行反向传播所需的计算量与$\CalG$中的边的数量成比例,其中每条边的计算包括计算偏导数(节点关于它的一个父节点的偏导数)以及执行一次乘法和一次加法。
下面,我们将此分析推广到张量值节点,这只是在同一节点中对多个标量值进行分组并能够更高效地实现。
% fig 6.9
\begin{figure}[!htb]
\ifOpenSource
\centerline{\includegraphics{figure.pdf}}
\else
\centerline{\includegraphics{Chapter6/figures/repeated_subexpression}}
\fi
\captionsetup{singlelinecheck=off}
\caption[.]{
计算梯度时导致重复子表达式的计算图。
令$w \in \SetR$为图的输入。
我们对链中的每一步使用相同的操作函数$f: \SetR \to \SetR$,这样$x=f(w), y=f(x), z=f(y)$。
为了计算$\frac{\partial z}{\partial w}$,我们应用\eqnref{eq:6.44}得到:
\begin{align}
& \frac{\partial z}{\partial w}\\
=& \frac{\partial z}{\partial y} \frac{\partial y}{\partial x} \frac{\partial x}{\partial w}\\
\label{eq:6.52}
=& f'(y)f'(x)f'(w)\\ 
\label{eq:6.53}
=& f'(f(f(w))) f'(f(w)) f'(w). 
\end{align}
\eqnref{eq:6.52}建议我们采用的实现方式是,仅计算$f(w)$的值一次并将它存储在变量$x$中。
这是\gls{BP}算法所采用的方法。
\eqnref{eq:6.53}提出了一种替代方法,其中子表达式$f(w)$出现了不止一次。
在替代方法中,每次只在需要时重新计算$f(w)$。
当存储这些表达式的值所需的存储较少时,\eqnref{eq:6.52}的\gls{BP}方法显然是较优的,因为它减少了运行时间。
然而,\eqnref{eq:6.53}也是链式法则的有效实现,并且当存储受限时它是有用的。}
\label{fig:chap6_repeated_subexpression}
\end{figure}

% -- 203 --

反向传播算法被设计为减少公共子表达式的数量而不考虑存储的开销。
具体来说,它大约对图中的每个节点执行一个~\gls{jacobian}~乘积。
这可以从\algref{alg:bprop}中看出,反向传播算法访问了图中的节点$u^{(j)}$到节点$u^{(i)}$的每条边一次,以获得相关的偏导数$\frac{\partial u^{(i)}}{\partial u^{(j)}}$。
反向传播因此避免了重复子表达式的指数爆炸。
然而,其他算法可能通过对计算图进行简化来避免更多的子表达式,或者也可能通过重新计算而不是存储这些子表达式来节省内存。
我们将在描述完反向传播算法本身后再重新审视这些想法。
% alg 6.2
\begin{algorithm}[htb!]
\caption{\gls{BP}算法的简化版本,用于计算$u^{(n)}$关于图中变量的导数。
这个示例旨在通过演示所有变量都是标量的简化情况来进一步理解\gls{BP}算法,这里我们希望计算关于$u^{(1)},\ldots,u^{(n_i)}$的导数。
这个简化版本计算了关于图中所有节点的导数。
假定与每条边相关联的偏导数计算需要恒定的时间的话,该算法的计算成本与图中边的数量成比例。
这与\gls{forward_propagation}的计算次数具有相同的阶。
每个$\frac{\partial u^{(i)}}{\partial u^{(j)}}$是$u^{(i)}$的父节点$u^{(j)}$的函数,从而将前向图的节点链接到\gls{BP}图中添加的节点。}
\label{alg:bprop}
\begin{algorithmic}
\STATE 运行前向传播 (对于此例是\algref{alg:fprop}) 获得网络的激活。
\STATE 初始化 {\tt grad\_table},用于存储计算好的导数的数据结构。 ${\tt grad\_table}[u^{(i)}]$将存储$\frac{\partial u^{(n)}}{\partial u^{(i)}}$计算好的值。
\STATE ${\tt grad\_table}[u^{(n)}] \leftarrow 1$
\FOR {$j=n-1$ down to 1}
\STATE 下一行使用存储的值计算 $\frac{\partial u^{(n)}}{\partial u^{(j)}} =
  \sum_{i: j \in Pa(u^{(i)})} \frac{\partial u^{(n)}}{\partial u^{(i)}} \frac{\partial u^{(i)}}{\partial u^{(j)}}$:
\STATE ${\tt grad\_table}[ u^{(j)}] \leftarrow 
\sum_{i: j \in Pa(u^{(i)})} {\tt grad\_table}[u^{(i)}]
\frac{\partial u^{(i)}}{\partial u^{(j)}}$
\ENDFOR
\STATE {\bf return} $\{ {\tt grad\_table}[ u^{(i)}] \mid i=1, \dots, n_i \} $
\end{algorithmic}
\end{algorithm}


% -- 204 --

\subsection{全连接MLP中的\glsentrytext{BP}计算}
\label{sec:back_propagation_computation_in_fully_connected_mlp}

为了阐明反向传播的上述定义,让我们考虑一个与全连接的多层MLP相关联的特定图。

\algref{alg:mlp-fprop}首先给出了前向传播,它将参数映射到与单个训练样本(输入,目标)$(\Vx,\Vy)$相关联的监督损失函数$L(\hat{\Vy}, \Vy)$,其中$\hat{\Vy}$是当$\Vx$提供输入时神经网络的输出。
% alg 6.3
\begin{algorithm}[ht]
\caption{典型深度神经网络中的\gls{forward_propagation}和代价函数的计算。
损失函数$L(\hat{\Vy}, \Vy)$取决于输出$\hat{\Vy}$和目标$\Vy$(参考\secref{sec:learning_conditional_distributions_with_maximum_likelihood}中损失函数的示例)。
为了获得总代价$J$,损失函数可以加上正则项$\Omega(\theta)$,其中$\theta$包含所有参数(权重和偏置)。
\algref{alg:mlp-bprop}说明了如何计算$J$关于参数$\MW$和$\Vb$的梯度。 为简单起见,该演示仅使用单个输入样本$\Vx$。
实际应用应该使用\gls{minibatch}。
请参考\secref{sec:example_back_propagation_for_mlp_training}以获得更加真实的演示。}
\label{alg:mlp-fprop}
\begin{algorithmic}
\REQUIRE 网络深度, $l$
\REQUIRE $\MW^{(i)}, i \in \{ 1, \dots, l\}$, 模型的权重矩阵
\REQUIRE $\Vb^{(i)}, i \in \{ 1, \dots, l\}$, 模型的偏置参数
\REQUIRE $\Vx$,程序的输入
\REQUIRE $\Vy$,目标输出
\STATE $\Vh^{(0)}=\Vx$
\FOR {$k=1, \ldots, l$}
 \STATE $\Va^{(k)} = \Vb^{(k)} + \MW^{(k)} \Vh^{(k-1)}$
 \STATE $\Vh^{(k)} = f(\Va^{(k)})$
\ENDFOR
\STATE $\hat{\Vy} = \Vh^{(l)}$
\STATE $J = L(\hat{\Vy},\Vy) + \lambda \Omega(\theta)$
\end{algorithmic}
\end{algorithm}

\algref{alg:mlp-bprop}随后说明了将反向传播应用于该图所需的相关计算。
% alg 6.4
\begin{algorithm}[htbp]
\caption{深度神经网络中\algref{alg:mlp-fprop}的反向计算,它不止使用了输入$\Vx $和目标$\Vy$。
该计算对于每一层$k$都产生了对激活$\Va^{(k)}$的梯度,从输出层开始向后计算一直到第一个\gls{hidden_layer}。
这些梯度可以看作是对每层的输出应如何调整以减小误差的指导,根据这些梯度可以获得对每层参数的梯度。
权重和偏置上的梯度可以立即用作随机梯度更新的一部分(梯度算出后即可执行更新),或者与其他基于梯度的优化方法一起使用。
}
\label{alg:mlp-bprop}
\begin{algorithmic}
\STATE 在前向计算完成后,计算顶层的梯度:
\STATE $\Vg \leftarrow \nabla_{\hat{\Vy}} J = \nabla_{\hat{\Vy}} L(\hat{\Vy},\Vy)$
\FOR {$k=l, l-1, \dots, 1$}
\STATE 将关于层输出的梯度转换为非线性激活输入前的梯度(如果 $f$ 是逐元素的,则逐元素地相乘):
  \STATE $\Vg \leftarrow \nabla_{\Va^{(k)}} J = \Vg \odot f'(\Va^{(k)})$
  \STATE 计算关于权重和偏置的梯度(如果需要的话,还要包括正则项):
  \STATE $\nabla_{\Vb^{(k)}} J = \Vg + \lambda \nabla_{\Vb^{(k)}} \Omega(\theta)$
  \STATE $\nabla_{\MW^{(k)}} J = \Vg \; \Vh^{(k-1)\top} + \lambda \nabla_{\MW^{(k)}} \Omega(\theta)$
  \STATE 关于下一更低层的\gls{hidden_layer}传播梯度:
  \STATE $\Vg \leftarrow \nabla_{\Vh^{(k-1)}} J = \MW^{(k)\top} \; \Vg$
\ENDFOR
\end{algorithmic}
\end{algorithm}

\algref{alg:mlp-fprop}和\algref{alg:mlp-bprop}是简单而直观的演示。
然而,它们专门针对特定的问题。

现在的软件实现基于之后\secref{sec:general_back_propagation}中描述的一般形式的反向传播,它可以通过显式地操作表示符号计算的数据结构,来适应任何计算图。


% -- 205 --

\subsection{符号到符号的导数}
\label{sec:symbol_to_symbol_derivatives}

代数表达式和计算图都对\firstgls{symbol}或不具有特定值的变量进行操作。
这些代数或者基于图的表达式被称为\firstgls{symbolic_representation}。
当我们实际使用或者训练神经网络时,我们必须给这些符号赋特定的值。
我们用一个特定的\firstgls{numeric_value}来替代网络的符号输入$\Vx$,例如$[1.2, 3,765, -1.8]^\top$。

一些反向传播的方法采用计算图和一组用于图的输入的数值,然后返回在这些输入值处梯度的一组数值。
我们将这种方法称为\,\textbf{符号到数值}的微分。
这种方法用在诸如Torch~\citep{Torch-2011}和Caffe~\citep{Jia13caffe}之类的库中。

% -- 206 --
 
另一种方法是采用计算图以及添加一些额外的节点到计算图中,这些额外的节点提供了我们所需导数的符号描述。
这是Theano~\citep{bergstra+al:2010-scipy-small,Bastien-Theano-2012}和TensorFlow~\citep{tensorflow}所采用的方法。
\figref{fig:chap6_symbol_to_symbol}给出了该方法如何工作的一个例子。
这种方法的主要优点是导数可以使用与原始表达式相同的语言来描述。
因为导数只是另外一张计算图,我们可以再次运行反向传播,对导数再进行求导就能得到更高阶的导数。
高阶导数的计算在\secref{sec:higher_order_derivatives}中描述。
% fig 6.10
\begin{figure}[!htb]
\ifOpenSource
\centerline{\includegraphics{figure.pdf}}
\else
\centerline{\includegraphics{Chapter6/figures/symbol_to_symbol}}
\fi
\captionsetup{singlelinecheck=off}
\caption{使用符号到符号的方法计算导数的示例。
在这种方法中,\gls{BP}算法不需要访问任何实际的特定数值。
相反,它将节点添加到计算图中来描述如何计算这些导数。
通用图形求值引擎可以在随后计算任何特定数值的导数。
\emph{(左)}在这个例子中,我们从表示$z=f(f(f(w)))$的图开始。
\emph{(右)}我们运行\gls{BP}算法,指导它构造表达式$\frac{dz}{dw}$对应的图。 在这个例子中,我们不解释\gls{BP}算法如何工作。
我们的目的只是说明想要的结果是什么:符号描述的导数的计算图。}
\label{fig:chap6_symbol_to_symbol}
\end{figure}

我们将使用后一种方法,并且使用构造导数的计算图的方法来描述反向传播算法。
图的任意子集之后都可以使用特定的数值来求值。
这允许我们避免精确地指明每个操作应该在何时计算。
相反,通用的图计算引擎只要当一个节点的父节点的值都可用时就可以进行求值。

% -- 207 --
  
基于符号到符号的方法的描述包含了符号到数值的方法。
符号到数值的方法可以理解为执行了与符号到符号的方法中构建图的过程中完全相同的计算。
关键的区别是符号到数值的方法不向外暴露出计算图。

\subsection{一般化的\glsentrytext{BP}}
\label{sec:general_back_propagation}

反向传播算法非常简单。
为了计算某个标量$z$关于图中它的一个祖先$\Vx$的梯度,我们首先观察到它关于$z$的梯度由$\frac{dz}{dz}=1$给出。
然后,我们可以计算对图中$z$的每个父节点的梯度,通过现有的梯度乘以产生$z$的操作的~\gls{jacobian}。
我们继续乘以~\gls{jacobian},以这种方式向后穿过图,直到我们到达$\Vx$。
对于从$z$出发可以经过两个或更多路径向后行进而到达的任意节点,我们简单地对该节点来自不同路径上的梯度进行求和。

更正式地,图$\CalG$中的每个节点对应着一个变量。
为了实现最大的一般化,我们将这个变量描述为一个张量$\TSV$。
张量通常可以具有任意维度,并且包含标量、向量和矩阵。

我们假设每个变量$\TSV$与下列子程序相关联:
\begin{itemize}
    \item \verb|get_operation|($\TSV$):它返回用于计算$\TSV$的操作,代表了在计算图中流入$\TSV$的边。
    例如,可能有一个Python或者C++的类表示矩阵乘法操作,以及~\verb|get_operation|函数。
    假设我们的一个变量是由矩阵乘法产生的,$\MC=\MA\MB$。
    那么,\verb|get_operation|($\TSV$)返回一个指向相应C++类的实例的指针。

    \item \verb|get_consumers|($\TSV, \CalG$):它返回一组变量,是计算图$\CalG$中$\TSV$的子节点。

    \item \verb|get_inputs|($\TSV, \CalG$):它返回一组变量,是计算图$\CalG$中$\TSV$的父节点。
\end{itemize}

% -- 208 --
  
每个操作~\verb|op|也与~\verb|bprop|操作相关联。
该~\verb|bprop|操作可以计算如\eqnref{eq:6.47}所描述的~\gls{jacobian}~向量积。
这是反向传播算法能够实现很大通用性的原因。
每个操作负责了解如何通过它参与的图中的边来反向传播。
例如,我们可以使用矩阵乘法操作来产生变量$\MC=\MA\MB$。
假设标量$z$关于$\MC$的梯度是$\MG$。
矩阵乘法操作负责定义两个反向传播规则,每个规则对应于一个输入变量。
如果我们调用~\verb|bprop|方法来请求关于$\MA$的梯度,那么在给定输出的梯度为$\MG$的情况下,矩阵乘法操作的~\verb|bprop|方法必须说明关于$\MA$的梯度是$\MG\MB^\top$。
类似的,如果我们调用~\verb|bprop|方法来请求关于$\MB$的梯度,那么矩阵操作负责实现~\verb|bprop|方法并指定希望的梯度是$\MA^\top\MG$。
反向传播算法本身并不需要知道任何微分法则。
它只需要使用正确的参数调用每个操作的~\verb|bprop|方法即可。
正式地,~\verb|op.bprop(inputs|$, \TSX, \TSG)$必须返回
\begin{equation}
  \sum_i (\nabla_{\TSX} \verb|op.f(inputs|)_i) \textsf{G}_i,
\end{equation}
这只是如\eqnref{eq:6.47}所表达的链式法则的实现。
这里,\verb|inputs|是提供给操作的一组输入,\verb|op.f|是操作实现的数学函数,$\TSX$是输入,我们想要计算关于它的梯度,$\TSG$是操作对于输出的梯度。

\verb|op.bprop|方法应该总是假装它的所有输入彼此不同,即使它们不是。
例如,如果~\verb|mul|操作传递两个$x$来计算$x^2$,\verb|op.bprop|方法应该仍然返回$x$作为对于两个输入的导数。
反向传播算法后面会将这些变量加起来获得$2x$,这是$x$上总的正确的导数。

反向传播算法的软件实现通常提供操作和其~\verb|bprop|方法,所以深度学习软件库的用户能够对使用诸如矩阵乘法、指数运算、对数运算等等常用操作构建的图进行反向传播。
构建反向传播新实现的软件工程师或者需要向现有库添加自己的操作的高级用户通常必须手动为新操作推导~\verb|op.bprop|方法。

反向传播算法的正式描述参考\algref{alg:backprop}。

% -- 209 --
% alg 6.5
\begin{algorithm}[ht]
\caption{\gls{BP}算法最外围的骨架。
这部分做简单的设置和清理工作。
大多数重要的工作发生在\algref{alg:build_grad}的子程序{\tt build\_grad}中。
}
\label{alg:backprop}
\begin{algorithmic}
\REQUIRE $\SetT$,需要计算梯度的目标变量集
\REQUIRE $\CalG$,计算图
\REQUIRE $z$, 要微分的变量
\STATE 令 $\CalG'$ 为$\CalG$剪枝后的计算图,其中仅包括$z$的祖先以及$\SetT$中节点的后代。
\STATE 初始化 {\tt grad\_table},它是关联张量和对应导数的数据结构。
\STATE ${\tt grad\_table}[z] \leftarrow 1$
\FOR{$\TSV$ in $\SetT$}
\STATE ${\tt build\_grad}(\TSV, \CalG, \CalG', {\tt grad\_table})$
\ENDFOR
\STATE Return {\tt grad\_table} restricted to $\SetT$
\end{algorithmic}
\end{algorithm}

\begin{algorithm}[ht]
\caption{\gls{BP}算法的内循环子程序${\tt build\_grad}(\TSV, \CalG, \CalG', {\tt grad\_table})$,
由\algref{alg:backprop}中定义的\gls{BP}算法调用。
}
\label{alg:build_grad}
\begin{algorithmic}
\REQUIRE $\TSV$,应该被加到$\CalG$和{\tt grad\_table}的变量。
\REQUIRE $\CalG$,要修改的图。
\REQUIRE $\CalG'$,根据参与梯度的节点$\CalG$的受限图。
\REQUIRE {\tt grad\_table},将节点映射到对应梯度的数据结构。
\IF{$\SetV$ is in {\tt grad\_table} }
 \STATE Return ${\tt grad\_table}[\TSV]$
\ENDIF
\STATE $i \leftarrow 1$
\FOR{$\TSC$ in ${\tt get\_consumers}(\TSV, \CalG')$ }
\STATE ${\tt op} \leftarrow {\tt get\_operation}(\TSC)$
\STATE $\TSD \leftarrow {\tt build\_grad}(\TSC, \CalG, \CalG', {\tt grad\_table})$
\STATE $\TSG^{(i)} \leftarrow {\tt op.bprop}({\tt get\_inputs}(\TSC, \CalG'), \TSV, \TSD)$ 
\STATE $i \leftarrow i + 1$
\ENDFOR
\STATE $\TSG \leftarrow \sum_i \TSG^{(i)}$
\STATE ${\tt grad\_table}[\TSV] = \TSG$
\STATE 插入 $\TSG$ 和将其生成到$\CalG$中的操作
\STATE Return $\TSG$
\end{algorithmic}
\end{algorithm}


% -- 210 --
  
在\secref{sec:chain_rule_of_calculus}中,我们使用反向传播作为一种策略来避免多次计算链式法则中的相同子表达式。
由于这些重复子表达式的存在,简单的算法可能具有指数运行时间。
现在我们已经详细说明了反向传播算法,我们可以去理解它的计算成本。
如果我们假设每个操作的执行都有大致相同的开销,那么我们可以依据执行操作的数量来分析计算成本。
注意这里我们将一个操作记为计算图的基本单位,它实际可能包含许多算术运算(例如,我们可能将矩阵乘法视为单个操作)。
在具有$n$个节点的图中计算梯度,将永远不会执行超过$O(n^2)$个操作,或者存储超过$O(n^2)$个操作的输出。
这里我们是对计算图中的操作进行计数,而不是由底层硬件执行的单独操作,所以重要的是要记住每个操作的运行时间可能是高度可变的。
例如,两个矩阵相乘可能对应着图中的一个单独的操作,但这两个矩阵可能每个都包含数百万个元素。
我们可以看到,计算梯度至多需要$O(n^2)$的操作,因为在最坏的情况下,前向传播的步骤将在原始图的全部$n$个节点上运行(取决于我们想要计算的值,我们可能不需要执行整个图)。
反向传播算法在原始图的每条边添加一个~\gls{jacobian}~向量积,可以用$O(1)$个节点来表达。
因为计算图是有向无环图,它至多有$O(n^2)$条边。对于实践中常用图的类型,情况会更好。
大多数神经网络的代价函数大致是链式结构的,使得反向传播只有$O(n)$的成本。
这远远胜过简单的方法,简单方法可能需要在指数级的节点上运算。
这种潜在的指数级代价可以通过非递归地扩展和重写递归链式法则(\eqnref{eq:6.49})来看出:
\begin{equation}
  \frac{\partial u^{(n)}}{\partial u^{(j)}} =
  \sum_{\substack{\text{path}(u^{(\pi_1)}, u^{(\pi_2)}, \ldots, u^{(\pi_t)}  ),\\ \text{from } \pi_1=j \text{ to }\pi_t = n}}
  \prod_{k=2}^t \frac{\partial u^{(\pi_k)}}{\partial u^{(\pi_{k-1})}}.
\end{equation}
由于节点$j$到节点$n$的路径数目可以关于这些路径的长度上指数地增长,所以上述求和符号中的项数(这些路径的数目),可能以前向传播图的深度的指数级增长。
会产生如此大的成本是因为对于$\frac{\partial u^{(i)}}{\partial u^{(j)}}$,相同的计算会重复进行很多次。
为了避免这种重新计算,我们可以将反向传播看作一种表填充算法,利用存储的中间结果$\frac{\partial u^{(n)}}{\partial u^{(i)}}$来对表进行填充。
图中的每个节点对应着表中的一个位置,这个位置存储对该节点的梯度。
通过顺序填充这些表的条目,反向传播算法避免了重复计算许多公共子表达式。
这种表填充策略有时被称为\firstgls{dynamic_programming}。
  
% -- 211 --
  
\subsection{实例:用于MLP训练的\glsentrytext{BP}}
\label{sec:example_back_propagation_for_mlp_training}

作为一个例子,我们利用反向传播算法来训练\gls{MLP}。

这里,我们考虑一个具有单个\gls{hidden_layer}的非常简单的\gls{MLP}。
为了训练这个模型,我们将使用\gls{minibatch}随机梯度下降算法。
反向传播算法用于计算单个\gls{minibatch}上的代价的梯度。
具体来说,我们使用训练集上的一\gls{minibatch}实例,将其规范化为一个设计矩阵$\MX$以及相关联的类标签向量$\Vy$。
网络计算隐藏特征层$\MH=\max\{0, \MX\MW^{(1)}\}$。
为了简化表示,我们在这个模型中不使用偏置。
假设我们的图语言包含~\verb|relu|操作,该操作可以对$\max\{0,\MZ\}$表达式的每个元素分别进行计算。
类的非归一化对数概率的预测将随后由$\MH\MW^{(2)}$ 给出。
假设我们的图语言包含~\verb|cross_entropy|操作,用以计算目标$\Vy$和由这些未归一化对数概率定义的概率分布间的\gls{cross_entropy}。
所得到的\gls{cross_entropy}定义了代价函数$J_\text{MLE}$。最小化这个\gls{cross_entropy}将执行对分类器的最大似然估计。
然而,为了使得这个例子更加真实,我们也包含一个正则项。
总的代价函数为
\begin{equation}
  J = J_{\text{MLE}} + \lambda \left ( \sum_{i, j} \left (W_{i, j}^{(1)} \right )^2 + \sum_{i, j} \left (W_{i, j}^{(2)} \right)^2 \right )
\end{equation}
包含了\gls{cross_entropy}和系数为$\lambda$的权重衰减项。
它的计算图在\figref{fig:chap6_mlp_example}中给出。
% fig 6.11
\begin{figure}[!htb]
\ifOpenSource
\centerline{\includegraphics{figure.pdf}}
\else
\centerline{\includegraphics{Chapter6/figures/mlp_example}}
\fi
\caption{用于计算代价函数的计算图,这个代价函数是使用\gls{cross_entropy}损失以及权重衰减训练我们的单层~\glssymbol{MLP}~示例所产生的。}
\label{fig:chap6_mlp_example}
\end{figure}

这个示例的梯度计算图实在太大,以致绘制或者阅读都将是乏味的。
这显示出了反向传播算法的优点之一,即它可以自动生成梯度,而这种计算对于软件工程师来说需要进行直观但冗长的手动推导。

我们可以通过观察\figref{fig:chap6_mlp_example}中的正向传播图来粗略地描述反向传播算法的行为。
为了训练,我们希望计算$\nabla_{\MW^{(1)}} J$和$\nabla_{\MW^{(2)}} J$。
有两种不同的路径从$J$后退到权重:一条通过\gls{cross_entropy}代价,另一条通过权重衰减代价。
权重衰减代价相对简单,它总是对$\MW^{(i)}$上的梯度贡献$2\lambda \MW^{(i)}$。
  
% -- 212 --
  
另一条通过\gls{cross_entropy}代价的路径稍微复杂一些。
令$\MG$是由~\verb|cross_entropy|操作提供的对未归一化对数概率$\MU^{(2)}$的梯度。
反向传播算法现在需要探索两个不同的分支。
在较短的分支上,它使用对矩阵乘法的第二个变量的反向传播规则,将$\MH^\top \MG$加到$\MW^{(2)}$的梯度上。
另一条更长些的路径沿着网络逐步下降。
首先,反向传播算法使用对矩阵乘法的第一个变量的反向传播规则,计算$\nabla_{\MH} J = \MG\MW^{(2)\top}$。
接下来,\verb|relu|操作使用其反向传播规则对先前梯度的部分位置清零,这些位置对应着$\MU^{(1)}$中所有小于0的元素。记上述结果为$\MG'$。 
反向传播算法的最后一步是使用对~\verb|matmul|操作的第二个变量的反向传播规则,将$\MX^\top \MG'$加到$\MW^{(1)}$的梯度上。

在计算了这些梯度以后,梯度下降算法或者其他优化算法所要做的就是使用这些梯度来更新参数。

对于MLP,计算成本主要来源于矩阵乘法。
在前向传播阶段,我们乘以每个权重矩阵,得到了$O(w)$数量的乘-加,其中$w$是权重的数量。
在反向传播阶段,我们乘以每个权重矩阵的转置,这具有相同的计算成本。
算法主要的存储成本是我们需要将输入存储到\gls{hidden_layer}的非线性中去。
这些值从被计算时开始存储,直到反向过程回到了同一点。
因此存储成本是$O(mn_h)$,其中$m$是\gls{minibatch}中样本的数目,$n_h$是\gls{hidden_unit}的数量。
  
% -- 213 --
  
\subsection{复杂化}
\label{sec:complications}

我们这里描述的反向传播算法要比实践中实际使用的实现要简单。

正如前面提到的,我们将操作的定义限制为返回单个张量的函数。
大多数软件实现需要支持可以返回多个张量的操作。 
例如,如果我们希望计算张量中的最大值和该值的索引,则最好在单次运算中计算两者,因此将该过程实现为具有两个输出的操作效率更高。

我们还没有描述如何控制反向传播的内存消耗。
反向传播经常涉及将许多张量加在一起。
在朴素方法中,将分别计算这些张量中的每一个,然后在第二步中对所有这些张量求和。 
朴素方法具有过高的存储瓶颈,可以通过保持一个缓冲器,并且在计算时将每个值加到该缓冲器中来避免该瓶颈。

反向传播的现实实现还需要处理各种数据类型,例如32位浮点数、64位浮点数和整型。
处理这些类型的策略需要特别的设计考虑。

一些操作具有未定义的梯度,重要的是跟踪这种情况并且确定用户请求的梯度是否是未定义的。

各种其他技术的特性使现实世界的微分更加复杂。 这些技术性并不是不可逾越的,本章已经描述了计算微分所需的关键知识工具,但重要的是要知道还有许多的精妙之处存在。
  
% -- 214 --
  
\subsection{深度学习界以外的微分}
\label{sec:differentiation_outside_the_deep_learning_community}

深度学习界在某种程度上已经与更广泛的计算机科学界隔离开来,并且在很大程度上发展了自己关于如何进行微分的文化态度。
更一般地,\firstgls{automatic_differentiation}领域关心如何以算法方式计算导数。 
这里描述的反向传播算法只是自动微分的一种方法。 
它是一种称为\firstgls{reverse_mode_accumulation}的更广泛类型的技术的特殊情况。 
其他方法以不同的顺序来计算链式法则的子表达式。 
一般来说,确定一种计算的顺序使得计算开销最小,是困难的问题。 
找到计算梯度的最优操作序列是NP完全问题\citep{naumann2008optimal},在这种意义上,它可能需要将代数表达式简化为它们最廉价的形式。

例如,假设我们有变量$p_1,p_2\ldots,p_n$表示概率,以及变量$z_1,z_2,\ldots,z_n$表示未归一化的对数概率。
假设我们定义
\begin{equation}
  q_i = \frac{\exp(z_i)}{\sum_i \exp(z_i)},
\end{equation}
其中我们通过指数化、求和与除法运算构建softmax函数,并构造\gls{cross_entropy}损失函数$J=-\sum_i p_i\log q_i$。
人类数学家可以观察到$J$对$z_i$的导数有一个非常简单的形式:$q_i-p_i$
%\footnote{译者注:$\frac{\partial J}{\partial z_i} = -\frac{p_i}{q_i}\frac{\partial q_i}{\partial z_i} -\sum_{j \ne i} \frac{p_j}{q_j} \frac{\partial q_j}{\partial z_i}  = -\frac{p_i}{q_i}(q_i - q_i^2) - \sum_{j \ne i} \frac{p_j}{q_j}(-q_jq_i) =p_i(q_i - 1) + (1-p_i)q_i = q_i - p_i.$}。
反向传播算法不能够以这种方式来简化梯度,而是会通过原始图中的所有对数和指数操作显式地传播梯度。
一些软件库如Theano~\citep{bergstra+al:2010-scipy-small,Bastien-Theano-2012}能够执行某些种类的代数替换来改进由纯反向传播算法提出的图。
  
% -- 215 --
  
当前向图$\CalG$具有单个输出节点,并且每个偏导数$\frac{\partial u^{(i)}}{\partial u^{(j)}}$都可以用恒定的计算量来计算时,反向传播保证梯度计算的计算数目和前向计算的计算数目是同一个量级:这可以在\algref{alg:bprop}中看出,因为每个局部偏导数$\frac{\partial u^{(i)}}{\partial u^{(j)}}$以及递归链式公式(\eqnref{eq:6.49})中相关的乘和加都只需计算一次。
因此,总的计算量是$O(\#\text{edges})$。
然而,可能通过对反向传播算法构建的计算图进行简化来减少这些计算量,并且这是NP完全问题。
诸如Theano和TensorFlow的实现使用基于匹配已知简化模式的试探法,以便重复地尝试去简化图。
我们定义反向传播仅用于计算标量输出的梯度,但是反向传播可以扩展到计算~\gls{jacobian_matrix}(该~\gls{jacobian_matrix}或者来源于图中的$k$个不同标量节点,或者来源于包含$k$个值的张量值节点)。
朴素的实现可能需要$k$倍的计算:对于原始前向图中的每个内部标量节点,朴素的实现计算$k$个梯度而不是单个梯度。
当图的输出数目大于输入的数目时,有时更偏向于使用另外一种形式的自动微分,称为\firstgls{forward_mode_accumulation}。
前向模式计算已经被提出用于循环神经网络梯度的实时计算,例如\citep{Williams89b}。
这也避免了存储整个图的值和梯度的需要,是计算效率和内存使用的折中。
前向模式和后向模式的关系类似于左乘和右乘一系列矩阵之间的关系,例如
\begin{equation}
  \MA \MB \MC \MD,
\end{equation}
其中的矩阵可以认为是~\gls{jacobian_matrix}。
例如,如果$\MD$是列向量,而$\MA$有很多行,那么这对应于一幅具有单个输出和多个输入的图,并且从最后开始乘,反向进行,只需要矩阵-向量的乘积。
这对应着反向模式。
相反,从左边开始乘将涉及一系列的矩阵-矩阵乘积,这使得总的计算变得更加昂贵。
然而,如果$\MA$的行数小于$D$的列数,则从左到右乘更为便宜,这对应着前向模式。

在机器学习以外的许多社区中,更常见的是使用传统的编程语言来直接实现微分软件,例如用Python或者C来编程,并且自动生成使用这些语言编写的不同函数的程序。
在深度学习界中,计算图通常使用由专用库创建的明确的数据结构表示。
专用方法的缺点是需要库开发人员为每个操作定义~\verb|bprop|方法,并且限制了库的用户仅使用定义好的那些操作。
然而,专用方法也允许定制每个操作的反向传播规则,允许开发者以非显而易见的方式提高速度或稳定性,对于这种方式自动的过程可能不能复制。

因此,反向传播不是计算梯度的唯一方式或最佳方式,但它是一个非常实用的方法,继续为深度学习社区服务。 
在未来,深度网络的微分技术可能会提高,因为深度学习的从业者更加懂得了更广泛的自动微分领域的进步。
  
% -- 216 --
  
\subsection{高阶微分}
\label{sec:higher_order_derivatives}

一些软件框架支持使用高阶导数。 
在深度学习软件框架中,这至少包括Theano和TensorFlow。
这些库使用一种数据结构来描述要被微分的原始函数,它们使用相同类型的数据结构来描述这个函数的导数表达式。
这意味着符号微分机制可以应用于导数(从而产生高阶导数)。

在深度学习的相关领域,很少会计算标量函数的单个二阶导数。
相反,我们通常对Hessian矩阵的性质比较感兴趣。
如果我们有函数$f:\SetR^n \to \SetR$,那么Hessian矩阵的大小是$n\times n$。
在典型的深度学习应用中,$n$将是模型的参数数量,可能很容易达到数十亿。
因此,完整的Hessian矩阵甚至不能表示。

典型的深度学习方法是使用\firstgls{Krylov_methods},而不是显式地计算Hessian矩阵。
Krylov方法是用于执行各种操作的一组迭代技术,这些操作包括像近似求解矩阵的逆、或者近似矩阵的特征值或特征向量等,而不使用矩阵-向量乘法以外的任何操作。

为了在Hesssian矩阵上使用Krylov方法,我们只需要能够计算Hessian矩阵$\MH$和一个任意向量$\Vv$间的乘积即可。
实现这一目标的一种直观方法\citep{christianson1992automatic}是
\begin{equation}
  \MH \Vv=\nabla_{\Vx} \left [ (\nabla_{\Vx} f(x))^\top \Vv\right ].
\end{equation}
该表达式中两个梯度的计算都可以由适当的软件库自动完成。
注意,外部梯度表达式是内部梯度表达式的函数的梯度。

如果$\Vv$本身是由计算图产生的一个向量,那么重要的是指定自动微分软件不要对产生$\Vv$的图进行微分。

虽然计算Hessian通常是不可取的,但是可以使用Hessian向量积。
可以对所有的$i=1,\ldots,n$简单地计算$\MH \Ve^{(i)}$,其中$\Ve^{(i)}$是$e_i^{(i)}=1$并且其他元素都为0的\gls{one_hot}向量。

\section{历史小记}
\label{sec:historical_notes}

\gls{feedforward_network}可以被视为一种高效的非线性函数近似器,它以使用梯度下降来最小化函数近似误差为基础。
从这个角度来看,现代\gls{feedforward_network}是一般函数近似任务的几个世纪进步的结晶。
  
% -- 217 --
  
处于反向传播算法底层的链式法则是17世纪发明的\citep{Leibniz-1676,LHopital-1696}。
微积分和代数长期以来被用于求解优化问题的封闭形式,但梯度下降直到19世纪才作为优化问题的一种迭代近似的求解方法被引入\citep{cauchy1847}。

从20世纪40年代开始,这些函数近似技术被用于导出诸如感知机的机器学习模型。 
然而,最早的模型都是基于线性模型。 
来自包括Marvin Minsky的批评指出了线性模型族的几个缺陷,例如它无法学习XOR函数,这导致了对整个神经网络方法的抵制。

学习非线性函数需要\gls{MLP}的发展和计算该模型梯度的方法。
基于动态规划的链式法则的高效应用开始出现在20世纪60年代和70年代,主要用于控制领域\citep{Kelley-1960,Bryson-et-al-1961,Dreyfus-1962,Bryson-1969,Dreyfus-1973},也用于灵敏度分析\citep{Linnainmaa-1976}。 
\cite{Werbos-1981}提出应用这些技术来训练人工神经网络。
这个想法以不同的方式被独立地重新发现后\citep{LeCun85,Parker-1985,Rumelhart86b-small},最终在实践中得以发展。
\firstgls{parallel_distributed_processing}一书在其中一章提供了第一次成功使用反向传播的一些实验的结果\citep{Rumelhart86c},这对反向传播的普及做出了巨大的贡献,并且开启了一个研究多层神经网络非常活跃的时期。
然而,该书作者提出的想法,特别是Rumelhart 和Hinton提出的想法远远超过了反向传播。
它们包括一些关键思想,关于可能通过计算实现认知和学习的几个核心方面,后来被冠以`` 联结主义''的名称,因为它强调了神经元之间的连接作为学习和记忆的轨迹的重要性。
特别地,这些想法包括分布式表示的概念\citep{Hinton-et-al-PDP1986}。

在反向传播的成功之后,神经网络研究获得了普及,并在20世纪90年代初达到高峰。 
随后,其他机器学习技术变得更受欢迎,直到2006年开始的现代深度学习复兴。

现代\gls{feedforward_network}的核心思想自20世纪80年代以来没有发生重大变化。
仍然使用相同的反向传播算法和相同的梯度下降方法。
1986年至2015年神经网络性能的大部分改进可归因于两个因素。
首先,较大的数据集减少了统计\gls{generalization}对神经网络的挑战的程度。
第二,神经网络由于更强大的计算机和更好的软件基础设施已经变得更大。
然而,少量算法上的变化也显著改善了神经网络的性能。
  
% -- 218 --
  
其中一个算法上的变化是用\gls{cross_entropy}族损失函数替代\gls{mean_squared_error}损失函数。
\gls{mean_squared_error}在20世纪80年代和90年代流行,但逐渐被\gls{cross_entropy}损失替代,并且最大似然原理的想法在统计学界和机器学习界之间广泛传播。
使用\gls{cross_entropy}损失大大提高了具有~\gls{sigmoid}~和softmax输出的模型的性能,而当使用\gls{mean_squared_error}损失时会存在饱和和学习缓慢的问题。

另一个显著改善\gls{feedforward_network}性能的算法上的主要变化是使用分段线性\gls{hidden_unit}来替代~\gls{sigmoid}~\gls{hidden_unit},例如用\gls{ReLU}。
使用$\max\{0, z\}$函数的整流在早期神经网络中已经被引入,并且至少可以追溯到认知机(Cognitron)和神经认知机(Neocognitron)\citep{Fukushima75,Fukushima80}。
这些早期的模型没有使用\gls{ReLU},而是将整流用于非线性函数。
尽管整流在早期很普及,在20世纪80年代,整流很大程度上被~\gls{sigmoid}~所取代,也许是因为当神经网络非常小时,\gls{sigmoid}~表现更好。
到21世纪初,由于有些迷信的观念,认为必须避免具有不可导点的激活函数,所以避免了\gls{ReLU}。
这在2009年开始发生改变。
\cite{Jarrett-ICCV2009-small}观察到,在神经网络结构设计的几个不同因素中``使用整流非线性是提高识别系统性能的最重要的唯一因素''。

对于小的数据集,\cite{Jarrett-ICCV2009-small}观察到,使用整流非线性甚至比学习\gls{hidden_layer}的权重值更加重要。
随机的权重足以通过\gls{rectifier_network}传播有用的信息,允许在顶部的分类器层学习如何将不同的特征向量映射到类标识。

当有更多数据可用时,学习开始提取足够的有用知识来超越随机选择参数的性能。
\cite{Glorot+al-AI-2011-small}说明,在深度\gls{rectifier_network}中的学习比在激活函数具有曲率或两侧饱和的深度网络中的学习更容易。

\gls{ReLU}还具有历史意义,因为它们表明神经科学继续对深度学习算法的发展产生影响。
\cite{Glorot+al-AI-2011-small}从生物学考虑\gls{ReLU}的导出。
半整流非线性旨在描述生物神经元的这些性质:(1) 对于某些输入,生物神经元是完全不活跃的。
(2) 对于某些输入,生物神经元的输出和它的输入成比例。
(3) 大多数时间,生物神经元是在它们不活跃的状态下进行操作(即它们应该具有\firstgls{sparse_activation})。
  
% -- 219 --
  
当2006年深度学习开始现代复兴时,\gls{feedforward_network}仍然有不良的声誉。
从2006年至2012年,人们普遍认为,\gls{feedforward_network}不会表现良好,除非它们得到其他模型的辅助,例如概率模型。
现在已经知道,只要具备适当的资源和工程实践,\gls{feedforward_network}表现得非常好。
今天,\gls{feedforward_network}中基于梯度的学习被用作发展概率模型的工具,例如\chapref{chap:deep_generative_models}中描述的\gls{VAE}和生成式对抗网络。
\gls{feedforward_network}中基于梯度的学习自2012年以来一直被视为一种强大的技术,并应用于许多其他机器学习任务,而不是被视为必须由其他技术支持的不可靠技术。
在2006 年,业内使用无监督学习来支持监督学习,现在更讽刺的是,更常见的是使用监督学习来支持无监督学习。

\gls{feedforward_network}还有许多未实现的潜力。
未来,我们期望它们用于更多的任务,优化算法和模型设计的进步将进一步提高它们的性能。
本章主要描述了神经网络模型族。
在接下来的章节中,我们将讨论如何使用这些模型——如何对它们进行正则化和训练。

  
% -- 220 --
  

% !Mode:: "TeX:UTF-8"
% Translator: Shenjian Zhao
\chapter{\glsentrytext{DL}中的正则化}
\label{chap:regularization_for_deep_learning}
\gls{ML}中的一个核心问题是设计不仅在训练数据上表现好,并且能在新输入上泛化好的算法。
在\gls{ML}中,许多策略显式地被设计来减少测试误差(可能会以增大训练误差为代价)。
这些策略被统称为\gls{regularization}。
我们将在后文看到,\gls{DL}工作者可以使用许多不同形式的\gls{regularization}策略。
事实上,开发更有效的\gls{regularization}策略已成为本领域的主要研究工作之一。

\chapref{chap:machine_learning_basics}介绍了\gls{generalization}、\gls{underfitting}、\gls{overfitting}、\gls{bias_sta}、\gls{variance}和\gls{regularization}的基本概念。
如果你不熟悉这些概念,请参考该章节再继续阅读本章。

在本章中,我们会更详细地介绍\gls{regularization},重点介绍深度模型(或组成深度模型的模块)的\gls{regularization}策略。

本章中的某些章节涉及\gls{ML}中的标准概念。
如果你已经熟悉了这些概念,可以随意跳过相关章节。
然而,本章的大多数内容是关于这些基本概念在特定\gls{NN}中的扩展概念。

在\secref{sec:regularization}中,我们将\gls{regularization}定义为``对学习算法的修改——旨在减少\gls{generalization}误差而不是训练误差''。
目前有许多\gls{regularization}策略。
有些策略向\gls{ML}模型添加限制参数值的额外约束。
有些策略向\gls{objective_function}增加额外项来对参数值进行软约束。
如果我们细心选择,这些额外的约束和惩罚可以改善模型在测试集上的表现。
有时侯,这些约束和惩罚被设计为编码特定类型的先验知识;
其他时候,这些约束和惩罚被设计为偏好简单模型,以便提高\gls{generalization}能力。
有时,惩罚和约束对于确定欠定的问题是必要的。
其他形式的\gls{regularization},如被称为\gls{ensemble}的方法,则结合多个假说来解释训练数据。

% -- 221 --

在\gls{DL}的背景下,大多数\gls{regularization}策略都会对\gls{estimator}进行\gls{regularization}。
\gls{estimator}的\gls{regularization}以\gls{bias_sta}的增加换取\gls{variance}的减少。
一个有效的\gls{regularization}是有利的``交易'',也就是能显著减少\gls{variance}而不过度增加\gls{bias_sta}。
我们在\chapref{chap:machine_learning_basics}中讨论\gls{generalization}和\gls{overfitting}时,主要侧重模型族训练的3个情形:(1)不包括真实的数据生成过程——对应\gls{underfitting}和含有\gls{bias_sta}的情况,(2)匹配真实数据生成过程,(3)包括真实的数据生成过程,同时还包括许多其他可能的生成过程——\gls{variance}(而不是\gls{bias_sta})主导的\gls{overfitting}。
\gls{regularization}的目标是使模型从第三种情况转化为第二种情况。

在实践中,过于复杂的模型族不一定包括目标函数或真实数据生成过程,甚至也不包括近似过程。
我们几乎从未知晓真实数据的生成过程,所以我们永远不知道被估计的模型族是否包括生成过程。
然而,\gls{DL}算法的大多数应用都是针对这样的情况,其中真实数据的生成过程几乎肯定在模型族之外。
\gls{DL}算法通常应用于极为复杂的领域,如图像、音频序列和文本,本质上这些领域的真实生成过程涉及模拟整个宇宙。
从某种程度上说,我们总是持方枘(数据生成过程)而欲内圆凿(我们的模型族)。

这意味着控制模型的复杂度不是找到合适规模的模型(带有正确的参数个数)这样一个简单的事情。
相反,我们可能会发现,或者说在实际的深度学习场景中我们几乎总是会发现,最好的拟合模型(从最小化\gls{generalization}误差的意义上)是一个适当\gls{regularization}的大型模型。

现在我们回顾几种策略,以创建这些\gls{regularization}的大型深度模型。

% -- 222 --

\section{参数范数惩罚}
\label{sec:parameter_norm_penalties}
\gls{regularization}在\gls{DL}的出现前就已经被使用了数十年。
\gls{linear_model},如\gls{linear_regression}和\gls{logistic_regression}可以使用简单、直接、有效的\gls{regularization}策略。

许多\gls{regularization}方法通过对\gls{objective_function} $J$添加一个参数范数惩罚$\Omega(\Vtheta)$,限制模型(如\gls{NN}、\gls{linear_regression}或\gls{logistic_regression})的学习能力。
我们将正则化后的\gls{objective_function}记为$\tilde{J}$:
\begin{align}
 \tilde{J}(\Vtheta;\MX, \Vy) = J(\Vtheta;\MX, \Vy) + \alpha \Omega(\Vtheta),
\end{align}
其中$\alpha \in [0, \infty)$是权衡范数惩罚项$\Omega$和标准\gls{objective_function} $J(\MX;\Vtheta)$相对贡献的超参数。
将$\alpha$设为0表示没有\gls{regularization}。
$\alpha$越大,对应\gls{regularization}惩罚越大。

当我们的训练算法最小化\gls{regularization}后的\gls{objective_function} $\tilde{J}$时,它会降低原始目标$J$关于训练数据的误差并同时减小在某些衡量标准下参数$\Vtheta$(或参数子集)的规模。
选择不同的参数范数$\Omega$会偏好不同的解。
在本节中,我们会讨论各种范数惩罚对模型的影响。

在探究不同范数的\gls{regularization}表现之前,我们需要说明一下,在神经网络中,参数包括每一层仿射变换的权重和\gls{bias_aff},我们通常\emph{只对权重}做惩罚而不对\gls{bias_aff}做正则惩罚。
精确拟合\gls{bias_aff}所需的数据通常比拟合权重少得多。
每个权重会指定两个变量如何相互作用。
我们需要在各种条件下观察这两个变量才能良好地拟合权重。
而每个\gls{bias_aff}仅控制一个单变量。
这意味着,我们不对其进行\gls{regularization}也不会导致太大的\gls{variance}。
另外,\gls{regularization}\gls{bias_aff}参数可能会导致明显的\gls{underfitting}。
因此,我们使用向量$\Vw$表示所有应受范数惩罚影响的权重,而向量$\Vtheta$表示所有参数(包括$\Vw$和无需\gls{regularization}的参数)。

在\gls{NN}的情况下,有时希望对网络的每一层使用单独的惩罚,并分配不同的$\alpha$系数。
寻找合适的多个超参数的代价很大,因此为了减少搜索空间,我们会在所有层使用相同的\gls{weight_decay}。

% -- 223 --

\subsection{$L^2$参数\glsentrytext{regularization}}
\label{sec:l2_parameter_regularization}
在\secref{sec:capacity_overfitting_and_underfitting}中我们已经看到过最简单而又最常见的参数范数惩罚,即通常被称为\firstgls{weight_decay}的$L^2$参数范数惩罚。
这个\gls{regularization}策略通过向\gls{objective_function}添加一个正则项$\Omega(\Vtheta) = \frac{1}{2} \norm{\Vw}_2^2$,使权重更加接近原点\footnote{更一般地,我们可以将参数\gls{regularization}为接近空间中的任意特定点,令人惊讶的是这样也仍有\gls{regularization}效果,但是特定点越接近真实值结果越好。
当我们不知道正确的值应该是正还是负时,零是有意义的默认值。
由于模型参数\gls{regularization}为零的情况更为常见,我们将只探讨这种特殊情况。}。
在其他学术圈,$L^2$也被称为\gls{ridge_regression}或~\gls{tikhonov_regularization}。

我们可以通过研究\gls{regularization}后\gls{objective_function}的\gls{gradient},洞察一些\gls{weight_decay}的\gls{regularization}表现。
为了简单起见,我们假定其中没有\gls{bias_aff}参数,因此$\Vtheta$就是$\Vw$。
这样一个模型具有以下总的\gls{objective_function}:
\begin{align}
  \tilde{J}(\Vw;\MX, \Vy) =\frac{\alpha}{2} \Vw^\top \Vw +  J(\Vw;\MX, \Vy),
\end{align}
与之对应的\gls{gradient}为
\begin{align}
 \nabla_{\Vw} \tilde{J}(\Vw;\MX,\Vy) =\alpha \Vw +  \nabla_{\Vw} J(\Vw;\MX, \Vy).
\end{align}
使用单步\gls{gradient}下降更新权重,即执行以下更新:
\begin{align}
 \Vw \leftarrow \Vw - \epsilon(\alpha \Vw + \nabla_{\Vw} J(\Vw;\MX, \Vy)).
\end{align}
换种写法就是:
\begin{align}
 \Vw \leftarrow (1-\epsilon \alpha)\Vw - \epsilon \nabla_{\Vw} J(\Vw;\MX, \Vy).
\end{align}
我们可以看到,加入\gls{weight_decay}后会引起学习规则的修改,即在每步执行通常的\gls{gradient}更新之前先收缩权重向量(将权重向量乘以一个常数因子)。
这是单个步骤发生的变化。
但是,在训练的整个过程会发生什么呢?

% -- 224 --

我们进一步简化分析,令$\Vw^*$为未\gls{regularization}的\gls{objective_function}取得最小训练误差时的权重向量,即$\Vw^* = \argmin_{\Vw} J(\Vw)$, 并在$\Vw^*$的邻域对\gls{objective_function}做二次近似。  
如果\gls{objective_function}确实是二次的(如以均方误差拟合\gls{linear_regression}模型的情况),则该近似是完美的。
近似的$\hat J(\Vtheta)$如下
\begin{align}
 \hat J(\Vtheta) = J(\Vw^*) + \frac{1}{2}(\Vw - \Vw^*)^\top \MH (\Vw - \Vw^*),
\end{align}
其中$\MH$是$J$在$\Vw^*$处计算的~\gls{hessian}~矩阵(关于$\Vw$)。
因为$\Vw^*$被定义为最优,即梯度消失为$0$,所以该二次近似中没有一阶项。
同样地,因为$\Vw^*$是$J$的一个最优点,我们可以得出$\MH$是半正定的结论。

当$\hat J$取得最小时,其梯度
\begin{align}
\label{eq:gradient}
  \nabla_{\Vw} \hat{J}(\Vw) = \MH (\Vw - \Vw^*)
\end{align}
为$0$。

为了研究\gls{weight_decay}带来的影响,我们在\eqnref{eq:gradient}中添加\gls{weight_decay}的梯度。 
现在我们探讨最小化\gls{regularization}后的$\hat J$。
我们使用变量$\tilde{\Vw}$表示此时的最优点:
\begin{align}
 \alpha \tilde{\Vw} + \MH (\tilde{\Vw} - \Vw^*) = 0 \\
 (\MH + \alpha \MI) \tilde{\Vw} = \MH \Vw^* \\
 \tilde{\Vw} = (\MH + \alpha \MI)^{-1} \MH \Vw^* \label{eq:final_w}
 \end{align}

当$\alpha$趋向于$0$时,\gls{regularization}的解$\tilde{\Vw}$会趋向$\Vw^*$。 
那么当$\alpha$增加时会发生什么呢?
因为$\MH$是实对称的,所以我们可以将其分解为一个对角矩阵$\VLambda$和一组特征向量的标准正交基$\MQ$,并且有$\MH = \MQ \VLambda \MQ^\top$。
将其应用于\eqnref{eq:final_w},可得:
\begin{align}
 \tilde \Vw &= ( \MQ \VLambda \MQ^\top + \alpha \MI)^{-1} \MQ \VLambda \MQ^\top \Vw^* \\
                 &=  [ \MQ( \VLambda+ \alpha \MI)  \MQ^\top ]^{-1} \MQ \VLambda \MQ^\top \Vw^* \\
                 &= \MQ( \VLambda+ \alpha \MI)^{-1} \VLambda \MQ^\top \Vw^*. \label{eq:713L2}
\end{align}
我们可以看到\gls{weight_decay}的效果是沿着由$\MH$的特征向量所定义的轴缩放$\Vw^*$。
具体来说,我们会根据$\frac{\lambda_i}{\lambda_i + \alpha}$因子缩放与$\MH$第$i$个特征向量对齐的$\Vw^*$的分量。
(不妨查看\figref{fig:chap2_eigen_ellipse}回顾这种缩放的原理)。

沿着$\MH$特征值较大的方向(如$\lambda_i \gg \alpha$)\gls{regularization}的影响较小。
而$\lambda_i \ll \alpha$的分量将会收缩到几乎为零。
这种效应如\figref{fig:chap7_reg_l2}所示。
\begin{figure}[!htb]
\ifOpenSource
\centerline{\includegraphics{figure.pdf}}
\else
\centerline{\includegraphics{Chapter7/figures/reg_l2}}
\fi
\caption{$L^2$(或\gls{weight_decay})\gls{regularization}对最佳$\Vw$值的影响。
实线椭圆表示没有\gls{regularization}目标的等值线。
虚线圆圈表示$L^2$正则化项的等值线。
在$\tilde{\Vw}$点,这两个竞争目标达到平衡。
目标函数$J$的~\gls{hessian}~的第一维特征值很小。
当从$\Vw^*$水平移动时,\gls{objective_function}不会增加得太多。
因为\gls{objective_function}对这个方向没有强烈的偏好,所以正则化项对该轴具有强烈的影响。
正则化项将$w_1$拉向零。
而\gls{objective_function}对沿着第二维远离$\Vw^*$的移动非常敏感。
对应的特征值较大,表示高\gls{curvature}。
因此,\gls{weight_decay}对$w_2$的位置影响相对较小。
}
\label{fig:chap7_reg_l2}
\end{figure}

% -- 225 --

只有在显著减小\gls{objective_function}方向上的参数会保留得相对完好。
在无助于\gls{objective_function}减小的方向(对应~\gls{hessian}~矩阵较小的特征值)上改变参数不会显著增加\gls{gradient}。
这种不重要方向对应的分量会在训练过程中因\gls{regularization}而衰减掉。


目前为止,我们讨论了\gls{weight_decay}对优化一个抽象通用的二次\gls{cost_function}的影响。
这些影响具体是怎么和机器学习关联的呢?
我们可以研究\gls{linear_regression},它的真实\gls{cost_function}是二次的,因此我们可以使用相同的方法分析。
再次应用分析,我们会在这种情况下得到相同的结果,但这次我们使用训练数据的术语表述。
\gls{linear_regression}的\gls{cost_function}是平方误差之和:
\begin{align}
 (\MX \Vw - \Vy)^\top (\MX \Vw - \Vy).
\end{align}
我们添加$L^2$正则项后,\gls{objective_function}变为
\begin{align}
  (\MX \Vw - \Vy)^\top (\MX \Vw - \Vy) + \frac{1}{2}\alpha \Vw^\top \Vw.
\end{align}
这将普通方程的解从
\begin{align}
\label{eq:716w}
  \Vw = (\MX^\top \MX)^{-1} \MX^\top \Vy
\end{align}
变为
\begin{align}
\label{eq:717mw}
   \Vw = (\MX^\top \MX + \alpha \MI)^{-1} \MX^\top \Vy .
\end{align}
\eqnref{eq:716w}中的矩阵$\MX^\top\MX$与\gls{covariance}矩阵$\frac{1}{m}\MX^\top\MX$成正比。
$L^2$正则项将这个矩阵替换为\eqnref{eq:717mw}中的$ (\MX^\top \MX + \alpha \MI)^{-1}$
这个新矩阵与原来的是一样的,不同的仅仅是在对角加了$\alpha$。
这个矩阵的对角项对应每个输入特征的\gls{variance}。
我们可以看到,$L^2$\gls{regularization}能让学习算法``感知''到输入$\Vx$具有较高的方差,这会使得学习算法压缩那些与输出目标的\gls{covariance}较小(相对增加方差)的特征的权重。

\subsection{$L^1$参数\glsentrytext{regularization}}
\label{sec:l1_regularization}
$L^2$\gls{weight_decay}是\gls{weight_decay}最常见的形式,我们还可以使用其他的方法限制模型参数的规模。
一个选择是使用$L^1$\gls{regularization}。

形式地,对模型参数$\Vw$的$L^1$\gls{regularization}被定义为:
\begin{align}
 \Omega(\Vtheta) = \norm{ \Vw }_1 = \sum_i | w_i |,
 \end{align}
即各个参数的绝对值之和\footnote{如同$L^2$\gls{regularization},我们也可以将参数\gls{regularization}到其他非零值$\Vw^{(o)}$。在这种情况下,$L^1$\gls{regularization}将会引入不同的项$\Omega(\Vtheta)=
\|\Vw - \Vw^{(o)} \|_1 = \sum_i | w_i - w_i^{(o)} |$。}。
接着我们将讨论$L^1$\gls{regularization}对简单\gls{linear_regression}模型的影响,与分析$L^2$\gls{regularization}时一样不考虑\gls{bias_aff}参数。  
我们尤其感兴趣的是找出$L^1$和$L^2$\gls{regularization}之间的差异。
与$L^2$\gls{weight_decay}类似,我们也可以通过缩放惩罚项$\Omega$的正超参数$\alpha$来控制$L^1$\gls{weight_decay}的强度。 
因此,\gls{regularization}的\gls{objective_function} $\tilde{J}(\Vw;\MX, \Vy)$如下所示
\begin{align}
\tilde{J}(\Vw;\MX, \Vy) = \alpha \| \Vw \|_1 +  J(\Vw;\MX, \Vy) ,
\end{align}
对应的梯度(实际上是次梯度):
\begin{align}
\label{eq:subgradient}
  \nabla_{\Vw} \tilde{J}(\Vw; \MX, \Vy) = \alpha \text{sign}(\Vw) + \nabla_{\Vw} J(\Vw; \MX, \Vy), % ?? may be wrong
\end{align}
其中$\text{sign}(\Vw)$只是简单地取$\Vw$各个元素的正负号。

% -- 227 --

观察\eqnref{eq:subgradient},我们立刻发现$L^1$的\gls{regularization}效果与$L^2$大不一样。
具体来说,我们可以看到\gls{regularization}对\gls{gradient}的影响不再是线性地缩放每个$w_i$;而是添加了一项与$\text{sign}(w_i)$同号的常数。
使用这种形式的\gls{gradient}之后,我们不一定能得到$J(\MX, \Vy;\Vw)$二次近似的直接算术解($L^2$\gls{regularization}时可以)。 
 
简单\gls{linear_model}具有二次\gls{cost_function},我们可以通过\gls{taylor}级数表示。
或者我们可以设想,这是逼近更复杂模型的\gls{cost_function}的截断\gls{taylor}级数。
在这个设定下,\gls{gradient}由下式给出
\begin{align}
  \nabla_{\Vw} \hat{J}(\Vw) = \MH (\Vw - \Vw^*),
\end{align}
同样,$\MH$是$J$在$\Vw^*$处的\gls{hessian}矩阵(关于$\Vw$)。

由于$L^1$惩罚项在完全一般化的~\gls{hessian}~的情况下,无法得到直接清晰的代数表达式,因此我们将进一步简化假设~\gls{hessian}~是对角的,即$\MH = \text{diag}([H_{1,1},\dots, H_{n,n}])$,其中每个$H_{i,i}>0$。
如果\gls{linear_regression}问题中的数据已被预处理(如可以使用PCA),去除了输入特征之间的相关性,那么这一假设成立。

我们可以将$L^1$\gls{regularization}目标函数的二次近似分解成关于参数的求和:
\begin{align}
 \hat J(\Vw; \MX, \Vy) = J(\Vw^*; \MX, \Vy) + \sum_i \Bigg [\frac{1}{2} H_{i,i} (w_i - w_i^*)^2 
 + \alpha |w_i| \Bigg].  % I think this is correct
\end{align}
如下列形式的解析解(对每一维$i$)可以最小化这个近似\gls{cost_function}:
\begin{align}
w_i = \text{sign}(w_i^*) \max\Big\{ |w_i^*| - \frac{\alpha}{H_{i,i}} , 0\Big\} .
\end{align}
对每个$i$,考虑$w_i^* > 0$的情形,会有两种可能结果:
\begin{enumerate}
\item $w_i^* \leq \frac{\alpha}{H_{i,i}}$的情况。
\gls{regularization}后目标中的$w_i$最优值是$w_i = 0$。
这是因为在方向$i$上$J(\Vw; \MX, \Vy) $对$ \hat J(\Vw; \MX, \Vy)$的贡献被抵消,$L^1$\gls{regularization}项将$w_i$推至$0$。
\item  $w_i^* > \frac{\alpha}{H_{i,i}}$的情况。在这种情况下,\gls{regularization}不会将$w_i$的最优值推至0,而仅仅在那个方向上移动$\frac{\alpha}{H_{i,i}}$的距离。
\end{enumerate}
$w_i^* < 0$的情况与之类似,但是$L^1$惩罚项使$w_i$更接近0(增加$ \frac{\alpha}{H_{i,i}}$)或者为0。

相比$L^2$\gls{regularization},$L^1$\gls{regularization}会产生更\firstgls{sparse}的解。
此处\gls{sparse}性指的是最优值中的一些参数为$0$。
和$L^2$\gls{regularization}相比,$L^1$\gls{regularization}的\gls{sparse}性具有本质的不同。
\eqnref{eq:713L2}给出了$L^2$\gls{regularization}的解$\tilde \Vw$。 
如果我们使用~\gls{hessian}~矩阵$\MH$为对角正定矩阵的假设(与$L^1$\gls{regularization}分析时一样),重新考虑这个等式,我们发现
$\tilde{w_i} = \frac{H_{i,i}}{H_{i,i} + \alpha} w_i^*$。
如果$w_i^*$不是零,那么$\tilde{w_i}$也会保持非零。 
这表明$L^2$\gls{regularization}不会使参数变得\gls{sparse},而$L^1$\gls{regularization}有可能通过足够大的$\alpha$实现\gls{sparse}。
 
由$L^1$\gls{regularization}导出的\gls{sparse}性质已经被广泛地用于\firstgls{feature_selection}机制。
\gls{feature_selection}从可用的特征子集选择出有意义的特征,化简\gls{ML}问题。
著名的LASSO~\citep{Tibshirani95}(Least Absolute Shrinkage and
Selection Operator)模型将$L^1$惩罚和\gls{linear_model}结合,并使用最小二乘\gls{cost_function}。 
$L^1$惩罚使部分子集的权重为零,表明相应的特征可以被安全地忽略。
 
在\secref{sec:maximum_a_posteriori_map_estimation},我们看到许多\gls{regularization}策略可以被解释为~\glssymbol{MAP}~贝叶斯推断,
特别是$L^2$\gls{regularization}相当于权重是高斯先验的~\glssymbol{MAP}~贝叶斯推断。
对于$L^1$\gls{regularization},用于\gls{regularization}\gls{cost_function}的惩罚项$\alpha \Omega(\Vw) =  \alpha \sum_i |w_i |$与通过~\glssymbol{MAP}~贝叶斯推断最大化的对数先验项是等价的($\Vw \in \SetR^n$并且权重先验是各向同性的拉普拉斯分布(\eqnref{eq:chap3_laplace})):
\begin{align}
\log p(\Vw) = \sum_i \log \text{Laplace}(w_i;0,\frac{1}{\alpha}) = 
  -\alpha \norm{\Vw}_1 + n \log \alpha - n \log 2.
\end{align}
因为是关于$\Vw$最大化进行学习,我们可以忽略$\log \alpha - \log 2$项,因为它们与$\Vw$无关。
 
 % -- 229 --
 
 \section{作为约束的范数惩罚}
 \label{sec:7.2}
考虑经过参数范数\gls{regularization}的\gls{cost_function}:
\begin{align}
 \tilde{J}(\Vtheta;\MX, \Vy) = J(\Vtheta;\MX, \Vy) + \alpha \Omega(\Vtheta) .
\end{align}

回顾\secref{sec:constrained_optimization}我们可以构造一个\gls{generalized_lagrange_function}来最小化带约束的函数,即在原始\gls{objective_function}上添加一系列惩罚项。
每个惩罚是一个被称为\firstgls{KKT}乘子的系数以及一个表示约束是否满足的函数之间的乘积。
如果我们想约束$\Omega(\Vtheta)$小于某个常数$k$,我们可以构建\gls{generalized_lagrange_function}
\begin{align}
 \CalL(\Vtheta, \alpha; \MX, \Vy) = J(\Vtheta; \MX, \Vy) + \alpha (\Omega(\Vtheta) - k).
\end{align}

这个约束问题的解由下式给出
\begin{align}
 \Vtheta^* = \underset{\Vtheta}{ \argmin} \underset{\alpha, \alpha \geq 0}{\max} \CalL(\Vtheta, \alpha).
\end{align}

如\secref{sec:constrained_optimization}中描述的,解决这个问题我们需要对$\Vtheta$和$\alpha$都做出调整。
\secref{sec:example_linear_least_squares}给出了一个带$L^2$约束的\gls{linear_regression}实例。
还有许多不同的优化方法,有些可能会使用\gls{GD}而其他可能会使用\gls{gradient}为0的解析解,但在所有过程中$\alpha$在$\Omega(\Vtheta) > k$时必须增加,在$\Omega(\Vtheta) < k$时必须减小。
所有正值的$\alpha$都鼓励$\Omega(\Vtheta)$收缩。
最优值$\alpha^*$也将鼓励$\Omega(\Vtheta)$收缩,但不会强到使得$\Omega(\Vtheta)$小于$k$。

为了洞察约束的影响,我们可以固定$\alpha^*$,把这个问题看成只跟$\Vtheta$有关的函数:
\begin{align}
 \Vtheta^* =  \underset{\Vtheta}{ \argmin} ~\CalL(\Vtheta, \alpha^*) = 
 \underset{\Vtheta}{ \argmin}~
 J(\Vtheta; \MX, \Vy) + \alpha^* \Omega(\Vtheta).
\end{align}
这和最小化$\tilde J$的\gls{regularization}训练问题是完全一样的。
因此,我们可以把参数范数惩罚看作对权重强加的约束。
如果$\Omega$是$L^2$范数,那么权重就是被约束在一个$L^2$球中。
如果$\Omega$是$L^1$范数,那么权重就是被约束在一个$L^1$范数限制的区域中。
通常我们不知道\gls{weight_decay}系数$\alpha^*$约束的区域大小,因为$\alpha^*$的值不直接告诉我们$k$的值。
原则上我们可以解得$k$,但$k$和$\alpha^*$之间的关系取决于$J$的形式。
虽然我们不知道约束区域的确切大小,但我们可以通过增加或者减小$\alpha$来大致扩大或收缩约束区域。
较大的$\alpha$,将得到一个较小的约束区域。
较小的$\alpha$,将得到一个较大的约束区域。

% -- 230 --

有时候,我们希望使用显式的限制,而不是惩罚。
如\secref{sec:constrained_optimization}所述,我们可以修改下降算法(如\gls{SGD}算法),使其先计算$J(\Vtheta)$的下降步,然后将$\Vtheta$投影到满足$\Omega(\Vtheta) < k$的最近点。
如果我们知道什么样的$k$是合适的,而不想花时间寻找对应于此$k$处的$\alpha$值,这会非常有用。

另一个使用显式约束和重投影而不是使用惩罚强加约束的原因是惩罚可能会导致\gls{objective_function}非凸而使算法陷入局部极小(对应于小的$\Vtheta$)。
当训练\gls{NN}时,这通常表现为训练带有几个``死亡单元''的\gls{NN}。
这些单元不会对网络学到的函数有太大影响,因为进入或离开它们的权重都非常小。
当使用权重范数的惩罚训练时,即使可以通过增加权重以显著减少$J$,这些配置也可能是局部最优的。
因为重投影实现的显式约束不鼓励权重接近原点,所以在这些情况下效果更好。
通过重投影实现的显式约束只在权重变大并试图离开限制区域时产生作用。

最后,因为重投影的显式约束还对优化过程增加了一定的稳定性,所以这是另一个好处。
当使用较高的学习率时,很可能进入正反馈,即大的权重诱导大\gls{gradient},然后使得权重获得较大更新。
如果这些更新持续增加权重的大小,$\Vtheta$就会迅速增大,直到离原点很远而发生溢出。
重投影的显式约束可以防止这种反馈环引起权重无限制地持续增加。
\cite{Hinton-et-al-arxiv2012}建议结合使用约束和高学习速率,这样能更快地探索参数空间,并保持一定的稳定性。

% -- 231 --

\cite{Hinton-et-al-arxiv2012}尤其推荐由\cite{Srebro05}引入的策略:约束\gls{NN}层的权重矩阵每列的范数,而不是限制整个权重矩阵的~\ENNAME{Frobenius}~范数。
分别限制每一列的范数可以防止某一\gls{hidden_unit}有非常大的权重。
如果我们将此约束转换成~\ENNAME{Lagrange}~函数中的一个惩罚,这将与$L^2$ \gls{weight_decay}类似但每个\gls{hidden_unit}的权重都具有单独的~\glssymbol{KKT}~乘子。
每个~\glssymbol{KKT}~乘子分别会被动态更新,以使每个\gls{hidden_unit}服从约束。
在实践中,列范数的限制总是通过重投影的显式约束来实现。

\section{\glsentrytext{regularization}和欠约束问题}
\label{sec:regularization_and_under_constrained_problems}
在某些情况下,为了正确定义\gls{ML}问题,\gls{regularization}是必要的。
\gls{ML}中许多\gls{linear_model},包括\gls{linear_regression}和PCA,都依赖于对矩阵$\MX^\top\MX$求逆。
只要$\MX^\top\MX$是奇异的,这些方法就会失效。
当数据生成分布在一些方向上确实没有差异时,或因为例子较少(即相对输入特征的维数来说)而在一些方向上没有观察到\gls{variance}时,这个矩阵就是奇异的。
在这种情况下,(人们往往使得)\gls{regularization}的许多形式对应求逆$\MX^\top\MX + \alpha \MI$。
(因为)这个\gls{regularization}矩阵可以保证是可逆的。

相关矩阵可逆时,这些线性问题有\gls{closed_form_solution}。
没有\gls{closed_form_solution}的问题也可能是欠定的。
一个例子是应用于线性可分问题的\gls{logistic_regression}。
如果权重向量$\Vw$能够实现完美分类,那么$2 \Vw$也会以更高似然实现完美分类。
类似\gls{SGD}的迭代优化算法将持续增加$\Vw$的大小,理论上永远不会停止。
在实践中,数值实现的\gls{GD}最终会达到导致数值溢出的超大权重,此时的行为将取决于程序员如何处理这些不是真正数字的值。

大多数形式的\gls{regularization}能够保证应用于欠定问题的迭代方法收敛。
例如,当似然的斜率等于\gls{weight_decay}的系数时, \gls{weight_decay}将阻止\gls{GD}继续增加权重的大小。

使用\gls{regularization}解决欠定问题的想法不局限于\gls{ML}。
同样的想法在几个基本线性代数问题中也非常有用。

% -- 232 --

正如我们在\secref{sec:the_moore_penrose_pseudoinverse}看到的,我们可以使用~\ENNAME{Moore-Penrose}~求解欠定线性方程。 
回想$\MX$伪逆$\MX^+$的一个定义:
\begin{align} 
\label{eq:729pseudo}
 \MX^+ = \lim_{\alpha \searrow 0} (\MX^\top \MX + \alpha \MI)^{-1}\MX^\top.
\end{align}
现在我们可以将\secref{eq:729pseudo}看作进行具有\gls{weight_decay}的\gls{linear_regression}。
具体来说,当\gls{regularization}系数趋向0时,\eqnref{eq:729pseudo}是\eqnref{eq:717mw}的极限。
因此,我们可以将伪逆解释为使用\gls{regularization}来稳定欠定问题。


\section{数据集增强}
\label{sec:dataset_augmentation_chap7}
让\gls{ML}模型泛化得更好的最好办法是使用更多的数据进行训练。
当然,在实践中,我们拥有的数据量是很有限的。
解决这个问题的一种方法是创建假数据并添加到训练集中。
对于一些\gls{ML}任务,创建新的假数据相当简单。

对分类来说这种方法是最简单的。
分类器需要一个复杂的高维输入$\Vx$,并用单个类别标识$y$概括$\Vx$。
这意味着分类面临的一个主要任务是要对各种各样的变换保持不变。
我们可以轻易通过转换训练集中的$\Vx$来生成新的$(\Vx, y)$对。

这种方法对于其他许多任务来说并不那么容易。
例如,除非我们已经解决了密度估计问题,否则在密度估计任务中生成新的假数据是很困难的。

数据集增强对一个具体的分类问题来说是特别有效的方法:对象识别。
图像是高维的并包括各种巨大的变化因素,其中有许多可以轻易地模拟。
即使模型已使用卷积和\gls{pooling}技术(\chapref{chap:convolutional_networks})对部分平移保持不变,沿训练图像每个方向平移几个像素的操作通常可以大大改善泛化。
许多其他操作如旋转图像或缩放图像也已被证明非常有效。

我们必须要小心,不能使用会改变类别的转换。
例如,光学字符识别任务需要认识到``b''和``d''以及``6''和``9''的区别,所以对这些任务来说,水平翻转和旋转$180^{\circ}$并不是合适的数据集增强方式。

% -- 233 --

能保持我们希望的分类不变,但不容易执行的转换也是存在的。
例如,平面外绕轴转动难以通过简单的几何运算在输入像素上实现。

数据集增强对语音识别任务也是有效的\citep{Jaitly_VTLP_2013}。

在\gls{NN}的输入层注入噪声\citep{SietsmaDow91}也可以被看作是数据增强的一种方式。
对于许多分类甚至一些回归任务而言,即使小的随机噪声被加到输入,任务仍应该是能够被解决的。
然而 ,\gls{NN}被证明对噪声不是非常健壮\citep{TangElias10}。
改善\gls{NN}健壮性的方法之一是简单地将随机噪声添加到输入再进行训练。
输入噪声注入是一些\gls{unsupervised_learning}算法的一部分,如\gls{DAE}\citep{VincentPLarochelleH2008}。
向\gls{hidden_unit}施加噪声也是可行的,这可以被看作在多个抽象层上进行的数据集增强。
\cite{Poole14}最近表明,噪声的幅度被细心调整后,该方法是非常高效的。
我们将在\secref{sec:dropout}介绍一个强大的\gls{regularization}策略~\gls{dropout},该策略可以被看作是通过与噪声\emph{相乘}构建新输入的过程。

在比较\gls{ML}基准测试的结果时,考虑其采取的数据集增强是很重要的。
通常情况下,人工设计的数据集增强方案可以大大减少\gls{ML}技术的泛化误差。
将一个\gls{ML}算法的性能与另一个进行对比时,对照实验是必要的。
在比较\gls{ML}算法A和\gls{ML}算法B时,应该确保这两个算法使用同一人工设计的数据集增强方案。
假设算法$A$在没有数据集增强时表现不佳,而$B$结合大量人工转换的数据后表现良好。
在这样的情况下,很可能是合成转化引起了性能改进,而不是\gls{ML}算法$B$比算法$A$更好。 
有时候,确定实验是否已经适当控制需要主观判断。
例如,向输入注入噪声的\gls{ML}算法是执行数据集增强的一种形式。
通常,普适操作(例如,向输入添加高斯噪声)被认为是\gls{ML}算法的一部分,
而特定于一个应用领域(如随机地裁剪图像)的操作被认为是独立的预处理步骤。

% -- 234 --

\section{噪声鲁棒性}
\label{sec:noise_robustness}

\secref{sec:dataset_augmentation_chap7}已经提出将噪声作用于输入,作为数据集增强策略。
对于某些模型而言,向输入添加方差极小的噪声等价于对权重施加范数惩罚\citep{Bishop1995,bishop95training}。
在一般情况下,注入噪声远比简单地收缩参数强大,特别是噪声被添加到\gls{hidden_unit}时会更加强大。
向\gls{hidden_unit}添加噪声是值得单独讨论重要的话题;在\secref{sec:dropout}所述~\gls{dropout}~算法是这种做法的主要发展方向。

另一种\gls{regularization}模型的噪声使用方式是将其加到权重。
这项技术主要用于\gls{RNN}~\citep{JimGilesHorne1996,Graves-2011}。
这可以被解释为关于权重的贝叶斯\gls{inference}的随机实现。
贝叶斯学习过程将权重视为不确定的,并且可以通过概率分布表示这种不确定性。
向权重添加噪声是反映这种不确定性的一种实用的随机方法。

在某些假设下,施加于权重的噪声可以被解释为与更传统的\gls{regularization}形式等同,鼓励要学习的函数保持稳定。
我们研究回归的情形,也就是训练将一组特征$\Vx$映射成一个标量的函数$\hat y(\Vx)$,并使用最小二乘\gls{cost_function}衡量模型预测值$\hat y(\Vx)$与真实值$y$的误差:
\begin{align}
 J = \SetE_{p(x,y)}[(\hat y(\Vx) - y)^2].
\end{align}
训练集包含$m$对标注样例$\{(\Vx^{(1)}, y^{(1)}),\dots,(\Vx^{(m)}, y^{(m)})\}$。

现在我们假设对每个输入表示,网络权重添加随机扰动$\epsilon_{\Vw} \sim \CalN(\Vepsilon;0, \eta\MI \, )$。
想象我们有一个标准的$l$层MLP。
我们将扰动模型记为$\hat y_{\epsilon_{\MW}} (\Vx)$。
尽管有噪声注入,我们仍然希望减少网络输出误差的平方。
因此目标函数变为:
\begin{align}
 \tilde J_{\MW} &= \SetE_{p(\Vx,y,\epsilon_{\MW})}[(\hat y_{\epsilon_{\MW}}(\Vx) - y)^2] \\
   &=  \SetE_{p(\Vx,y,\epsilon_{\MW})}[\hat y_{\epsilon_{\MW}}^2(\Vx) -  2y\hat y_{\epsilon_{\MW}}
   (\Vx)+ y^2] .
\end{align}

对于小的$\eta$,最小化带权重噪声(方差为$\eta \MI$\,)的$J$等同于最小化附加\gls{regularization}项:
$ \eta \SetE_{p(\Vx,y)}[\norm{\nabla_{\MW}~\hat y(\Vx)}^2]$的$J$。
这种形式的\gls{regularization}鼓励参数进入权重小扰动对输出相对影响较小的参数空间区域。
换句话说,它推动模型进入对权重小的变化相对不敏感的区域,找到的点不只是极小点,还是由平坦区域所包围的极小点\citep{Hochreiter95}。
在简化的线性回归中(例如,$\hat y(\Vx) = \Vw^\top \Vx + b$),正则项退化为$ \eta \SetE_{p(\Vx)}[\norm{\Vx}^2]$,这与函数的参数无关,因此不会对$\tilde J_{\Vw}$关于模型参数的梯度有影响。

% -- 235 --

\subsection{向输出目标注入噪声}
\label{sec:injecting_noise_at_the_output_targets}
大多数数据集的$y$标签都有一定错误。
错误的$y$不利于最大化$\log p(y \mid \Vx)$。
避免这种情况的一种方法是显式地对标签上的噪声进行建模。
例如,我们可以假设,对于一些小常数$\epsilon$,训练集标记$y$是正确的概率是$1-\epsilon$,(以$\epsilon$的概率)任何其他可能的标签也可能是正确的。
这个假设很容易就能解析地与\gls{cost_function}结合,而不用显式地抽取噪声样本。
例如,\textbf{标签平滑}(label smoothing)通过把确切分类目标从0和1替换成$\frac{\epsilon}{k-1}$和$1-\epsilon$,\gls{regularization}具有$k$个输出的~\gls{softmax}~的模型。
标准交叉熵损失可以用在这些非确切目标的输出上。
使用~\gls{softmax}~和明确目标的最大似然学习可能永远不会收敛——
\gls{softmax}~永远无法真正预测0概率或1概率,因此它会继续学习越来越大的权重,使预测更极端。
使用如\gls{weight_decay}等其他\gls{regularization}策略能够防止这种情况。
标签平滑的优势是能够防止模型追求确切概率而不影响模型学习正确分类。
这种策略自20世纪80年代就已经被使用,并在现代神经网络继续保持显著特色\citep{Szegedy-et-al-2015}。

% -- 236 --

\section{\glsentrytext{semi_supervised_learning}}
\label{sec:semi_supervised_learning}
在\gls{semi_supervised_learning}的框架下,$P(\RVx)$产生的未标记样本和$P(\RVx, \RVy)$中的标记样本都用于估计$P(\RVy \mid \RVx)$或者根据$\RVx$预测$\RVy$。

在\gls{DL}的背景下,\gls{semi_supervised_learning}通常指的是学习一个\gls{representation} $\Vh = f(\Vx)$。 
学习\gls{representation}的目的是使相同类中的\gls{example}有类似的表示。
\gls{unsupervised_learning}可以为如何在\gls{representation}空间聚集\gls{example}提供有用线索。
在输入空间紧密聚集的\gls{example}应该被映射到类似的表示。
在许多情况下,新空间上的线性分类器可以达到较好的泛化\citep{Belkin+Niyogi-2002,Chapelle+al-2003}。
这种方法的一个经典变种是使用\gls{PCA}作为分类前(在投影后的数据上分类)的预处理步骤。

我们可以构建这样一个模型,其中生成模型$P(\RVx)$或$P(\RVx, \RVy)$与判别模型$P(\RVy \mid \RVx)$共享参数,而不用分离\gls{unsupervised}和\gls{supervised}部分。
我们权衡\gls{supervised}模型\gls{criterion} $-\log P(\RVy \mid \RVx)$和\gls{unsupervised}或生成模型\gls{criterion}(如$-\log P(\RVx)$或$-\log P(\RVx, \RVy)$)。
生成模型\gls{criterion}表达了对\gls{supervised_learning}问题解的特殊形式的先验知识\citep{LasserreJ2006},即$P(\RVx)$的结构通过某种共享参数的方式连接到$P(\RVy \mid \RVx)$。
通过控制在总\gls{criterion}中的生成\gls{criterion},我们可以获得比纯生成或纯判别训练\gls{criterion}更好的权衡\citep{LasserreJ2006,Larochelle+Bengio-2008-small}。

\cite{Russ+Geoff-nips-2007}描述了一种学习回归\gls{kernel_machines}中核函数的方法,其中建模$P(\RVx)$时使用的未标记样本大大提高了$P(\RVy \mid \RVx)$的效果。

更多\gls{semi_supervised_learning}的信息,请参阅~\cite{Chapelle-2006}。

\section{\glsentrytext{multitask_learning}}
\label{sec:multitask_learning}
\gls{multitask_learning}~\citep{caruana93a}是通过合并几个任务中的样例(可以视为对参数施加的软约束)来提高泛化的一种方式。
正如额外的训练样本能够将模型参数推向具有更好泛化能力的值一样,当模型的一部分被多个额外的任务共享时,这部分将被约束为良好的值(如果共享合理),通常会带来更好的泛化能力。

\figref{fig:chap7_multi_factor_output}展示了\gls{multitask_learning}中非常普遍的一种形式,其中不同的\gls{supervised}任务(给定$\RVx$预测$\RVy^{(i)}$)共享相同的输入$\RVx$以及一些中间层表示$\Vh^{(\text{share})}$,能学习共同的因素池。
该模型通常可以分为两类相关的参数:
\begin{enumerate}
 \item 具体任务的参数 (只能从各自任务的样本中实现良好的泛化)。如\figref{fig:chap7_multi_factor_output}中的上层。
 \item 所有任务共享的通用参数(从所有任务的汇集数据中获益)。如\figref{fig:chap7_multi_factor_output}中的下层。
\end{enumerate}
\begin{figure}[!htb]
\ifOpenSource
\centerline{\includegraphics{figure.pdf}}
\else
\centerline{\includegraphics{Chapter7/figures/multi_factor_output}}
\fi
\caption{\gls{multitask_learning}在深度学习框架中可以以多种方式进行,该图说明了任务共享相同输入但涉及不同目标随机变量的常见情况。
\gls{deep_network}的较低层(无论是\gls{supervised}前馈的,还是包括向下箭头的生成组件)可以跨这样的任务共享,而任务特定的参数(分别关联于从$\Vh^{(1)}$和$\Vh^{(2)}$进入和发出的权重)可以在共享表示$\Vh^{(\text{shared})}$之上学习。
这里的基本假设是存在解释输入$\RVx$变化的共同因素池,而每个任务与这些因素的子集相关联。
在该示例中,额外假设顶层\gls{hidden_unit} $\Vh^{(1)}$和$\Vh^{(2)}$专用于每个任务(分别预测$\RVy^{(1)}$和$\RVy^{(2)}$),而一些中间层表示$\Vh^{(\text{shared})}$在所有任务之间共享。
在\gls{unsupervised_learning}情况下,一些顶层因素不与输出任务$(\Vh^{(3)})$的任意一个关联是有意义的:这些因素可以解释一些输入变化但与预测$\RVy^{(1)}$或$\RVy^{(2)}$不相关。
}
\label{fig:chap7_multi_factor_output}
\end{figure}

因为共享参数,其统计强度可大大提高(共享参数的样本数量相对于单任务模式增加的比例),并能改善泛化和泛化误差的范围\citep{baxter95a}。
当然,仅当不同的任务之间存在某些统计关系的假设是合理(意味着某些参数能通过不同任务共享)时才会发生这种情况。

从\gls{DL}的观点看,底层的先验知识如下:\emph{能解释数据变化(在与之相关联的不同任务中观察到)的因素中,某些因素是跨两个或更多任务共享的。}

% -- 238 --

\section{\glsentrytext{early_stopping}}
\label{sec:early_stopping}
当训练有足够的表示能力甚至会过拟合的大模型时,我们经常观察到,训练误差会随着时间的推移逐渐降低但验证集的误差会再次上升。
\figref{fig:chap7_learning_curve}是这些现象的一个例子,这种现象总是非常可靠地反复出现。


这意味着我们只要返回使验证集误差最低的参数设置,就可以获得验证集误差更低的模型(并且因此有希望获得更好的测试误差)。
在每次验证集误差有所改善后,我们存储模型参数的副本。
当训练算法终止时,我们返回这些参数而不是最新的参数。
当验证集上的误差在事先指定的循环次数内没有进一步改善时,算法就会终止。
此过程在\algref{alg:early_stopping}中有更正式的说明。


% -- 239 --

这种策略被称为\firstgls{early_stopping}。
这可能是\gls{DL}中最常用的\gls{regularization}形式。
它的流行主要是因为有效性和简单性。

\begin{algorithm}[ht]
\caption{用于确定最佳训练时间量的\gls{early_stopping}元算法。
这种元算法是一种通用策略,可以很好地在各种训练算法和各种量化验证集误差的方法上工作。
}
\label{alg:early_stopping}
\begin{algorithmic}
\STATE 令 $n$ 为评估间隔的步数。
\STATE 令 $p$ 为 ``耐心(patience)'',即观察到较坏的验证集表现$p$次后终止。 
\STATE 令 $\Vtheta_{o}$ 为初始参数。
\STATE $\Vtheta \leftarrow \Vtheta_{o}$
\STATE $i \leftarrow 0$
\STATE $j \leftarrow 0$
\STATE $v \leftarrow \infty$
\STATE $\Vtheta^* \leftarrow \Vtheta$
\STATE $i^* \leftarrow i$
\WHILE{$j < p$}
    \STATE 运行训练算法$n$步,更新 $\Vtheta$ 。
    \STATE $i \leftarrow i + n$
    \STATE $v' \leftarrow \text{ValidationSetError}(\Vtheta)$
    \IF{$v' < v$}
        \STATE $j \leftarrow 0$
        \STATE $\Vtheta^* \leftarrow \Vtheta$
        \STATE $i^* \leftarrow i$
        \STATE $v \leftarrow v'$
    \ELSE
        \STATE $j \leftarrow j + 1$
    \ENDIF
\ENDWHILE
% pdflatex told me \RETURN was not recognized, wtf
\STATE 最佳参数为 $\Vtheta^*$,最佳训练步数为$i^*$
\end{algorithmic}
\end{algorithm}


我们可以认为\gls{early_stopping}是非常高效的超参数选择算法。
按照这种观点,训练步数仅是另一个超参数。
我们从\figref{fig:chap7_learning_curve}可以看到,这个超参数在验证集上具有U型性能曲线。
很多控制模型容量的超参数在验证集上都是这样的U型性能曲线,如\figref{fig:chap5_generalization_vs_capacity}。
在\gls{early_stopping}的情况下,我们通过控制拟合训练集的步数来控制模型的有效容量。
大多数超参数的选择必须使用高代价的猜测和检查过程,我们需要在训练开始时猜测一个超参数,然后运行几个步骤检查它的训练效果。
``训练时间''是唯一只要跑一次训练就能尝试很多值的超参数。
通过\gls{early_stopping}自动选择超参数的唯一显著的代价是训练期间要定期评估验证集。
在理想情况下,这可以并行在与主训练过程分离的机器上,或独立的CPU,或独立的GPU上完成。
如果没有这些额外的资源,可以使用比训练集小的验证集或较不频繁地评估验证集来减小评估代价,较粗略地估算取得最佳的训练时间。

另一个\gls{early_stopping}的额外代价是需要保持最佳的参数副本。
这种代价一般是可忽略的,因为可以将它储存在较慢较大的存储器上(例如,在GPU内存中训练,但将最佳参数存储在主存储器或磁盘驱动器上)。
由于最佳参数的写入很少发生而且从不在训练过程中读取,这些偶发的慢写入对总训练时间的影响不大。

\begin{figure}[!htb]
\ifOpenSource
\centerline{\includegraphics{figure.pdf}}
\else
\centerline{\includegraphics{Chapter7/figures/learning_curve_color}}
\fi
\caption{学习曲线显示负对数似然损失如何随时间变化(表示为遍历数据集的训练迭代数,或\firstgls{epochs})。
在这个例子中,我们在MNIST上训练了一个~\gls{maxout}~网络。
我们可以观察到训练目标随时间持续减小,但验证集上的平均损失最终会再次增加,形成不对称的U形曲线。
}
\label{fig:chap7_learning_curve}
\end{figure}

% -- 240 --

\gls{early_stopping}是一种非常不显眼的\gls{regularization}形式,它几乎不需要改变基本训练过程、\gls{objective_function}或一组允许的参数值。
这意味着,无需破坏学习动态就能很容易地使用\gls{early_stopping}。
相对于\gls{weight_decay},必须小心不能使用太多的\gls{weight_decay},以防网络陷入不良\gls{local_minimum}(对应于病态的小权重)。

\gls{early_stopping}可单独使用或与其他的\gls{regularization}策略结合使用。
即使为鼓励更好泛化,使用\gls{regularization}策略改进\gls{objective_function},在训练目标的\gls{local_minimum}达到最好泛化也是非常罕见的。

\gls{early_stopping}需要验证集,这意味着某些训练数据不能被馈送到模型。
为了更好地利用这一额外的数据,我们可以在完成\gls{early_stopping}的首次训练之后,进行额外的训练。
在第二轮,即额外的训练步骤中,所有的训练数据都被包括在内。
有两个基本的策略都可以用于第二轮训练过程。

% -- 241 --

一个策略(\algref{alg:early_stopping_retrain})是再次初始化模型,然后使用所有数据再次训练。
在这个第二轮训练过程中,我们使用第一轮\gls{early_stopping}训练确定的最佳步数。
此过程有一些细微之处。
例如,我们没有办法知道重新训练时,对参数进行相同次数的更新和对数据集进行相同次数的遍历哪一个更好。
由于训练集变大了,在第二轮训练时,每一次遍历数据集将会更多次地更新参数。

另一个策略是保持从第一轮训练获得的参数,然后使用全部的数据\emph{继续}训练。
在这个阶段,已经没有验证集指导我们需要在训练多少步后终止。
取而代之,我们可以监控验证集的平均损失函数,并继续训练,直到它低于\gls{early_stopping}过程终止时的目标值。
此策略避免了重新训练模型的高成本,但表现并没有那么好。
例如,验证集的目标不一定能达到之前的目标值,所以这种策略甚至不能保证终止。
我们会在\algref{alg:early_stopping_continue}中更正式地介绍这个过程。

\gls{early_stopping}对减少训练过程的计算成本也是有用的。
除了由于限制训练的迭代次数而明显减少的计算成本,还带来了\gls{regularization}的益处(不需要添加惩罚项的\gls{cost_function}或计算这种附加项的\gls{gradient})。

\begin{algorithm}[ht]
\caption{使用\gls{early_stopping}确定训练步数,然后在所有数据上训练的元算法。
}
\label{alg:early_stopping_retrain}
\begin{algorithmic}
\STATE 令 $\MX^{(\text{train})}$ 和 $\Vy^{(\text{train})}$ 为训练集。
\STATE 将 $\MX^{(\text{train})}$ 和 $\Vy^{(\text{train})}$ 分别分割为 $(\MX^{(\text{subtrain})}$, $\MX^{(\text{valid})})$ 和 $(\Vy^{(\text{subtrain})}$, $\Vy^{(\text{valid})})$。
\STATE 从随机 $\Vtheta$开始,使用$\MX^{(\text{subtrain})}$ 和 $\Vy^{(\text{subtrain})}$作为训练集,$\MX^{(\text{valid})}$ 和 $\Vy^{(\text{valid})}$ 作为验证集,运行 (\algref{alg:early_stopping})。这将返回最佳训练步数$i^*$。
\STATE 将 $\Vtheta$ 再次设为随机值。
\STATE 在 $\MX^{(\text{train})}$ 和 $\Vy^{(\text{train})}$ 上训练 $i^*$ 步。  
\end{algorithmic}
\end{algorithm}

\begin{algorithm}[ht]
\caption{
使用\gls{early_stopping}确定将会\gls{overfitting}的目标值,然后在所有数据上训练直到再次
达到该值的元算法。
}
\label{alg:early_stopping_continue}
\begin{algorithmic}
\STATE 令 $\MX^{(\text{train})}$ 和 $\Vy^{(\text{train})}$ 为训练集。
\STATE 将 $\MX^{(\text{train})}$ 和 $\Vy^{(\text{train})}$ 分别分割为 $(\MX^{(\text{subtrain})}$, $\MX^{(\text{valid})})$ 和 $(\Vy^{(\text{subtrain})}$, $\Vy^{(\text{valid})})$。
\STATE 从随机 $\Vtheta$开始,使用$\MX^{(\text{subtrain})}$ 和 $\Vy^{(\text{subtrain})}$作为训练集,$\MX^{(\text{valid})}$ 和 $\Vy^{(\text{valid})}$ 作为验证集,运行 (\algref{alg:early_stopping})。这会更新$\Vtheta$。
\STATE $\epsilon \leftarrow J(\Vtheta, \MX^{(\text{subtrain})},\Vy^{(\text{subtrain})})$
\WHILE{$J(\Vtheta, \MX^{(\text{valid})}, \Vy^{(\text{valid})}) > \epsilon$}
\STATE 在 $\MX^{(\text{train})}$ 和 $\Vy^{(\text{train})}$ 上训练 $n$ 步。  
\ENDWHILE
\end{algorithmic}
\end{algorithm}


% -- 242 --

\paragraph{\gls{early_stopping}为何具有\gls{regularization}效果:}
目前为止,我们已经声明\gls{early_stopping}是一种\gls{regularization}策略,但我们只通过展示验证集误差的学习曲线是一个U型曲线来支持这种说法。
\gls{early_stopping}\gls{regularization}模型的真正机制是什么呢? 
\cite{Bishop1995}和~\cite{Sjoberg95}认为\gls{early_stopping}可以将优化过程的参数空间限制在初始参数值$\Vtheta_0$的小邻域内。
更具体地,想象用学习率$\epsilon$进行$\tau$个优化步骤(对应于$\tau$个训练迭代)。
我们可以将$\epsilon \tau$作为有效\gls{capacity}的度量。
假设\gls{gradient}有界,限制迭代的次数和学习速率能够限制从$\Vtheta_0$到达的参数空间的大小,如\figref{fig:chap7_reg_l1_vs_l2_mistake}所示。
在这个意义上,$\epsilon \tau$的效果就好像是\gls{weight_decay}系数的倒数。

\begin{figure}[!htb]
\ifOpenSource
\centerline{\includegraphics{figure.pdf}}
\else
\centerline{\includegraphics[width=0.8\textwidth]{Chapter7/figures/reg_early_stop_vs_l2}}
\fi
\caption{\gls{early_stopping}效果的示意图。
\emph{(左)}实线轮廓线表示负对数似然的轮廓。
虚线表示从原点开始的~\glssymbol{SGD}~所经过的轨迹。 % ??
\gls{early_stopping}的轨迹在较早的点$\tilde \Vw$处停止,而不是停止在最小化代价的点$\Vw^*$处。
\emph{(右)}为了对比,使用$L^2$\gls{regularization}效果的示意图。
虚线圆圈表示$L^2$惩罚的轮廓,$L^2$惩罚使得总代价的最小值比非\gls{regularization}代价的最小值更靠近原点。
}
\label{fig:chap7_reg_l1_vs_l2_mistake}
\end{figure}

事实上,在二次误差的简单\gls{linear_model}和简单的\gls{GD}情况下,我们可以展示\gls{early_stopping}相当于$L^2$\gls{regularization}。

为了与经典$L^2$\gls{regularization}比较,我们只考察唯一的参数是线性权重($\Vtheta = \Vw$)的简单情形。
我们在权重$\Vw$的经验最佳值$\Vw^*$附近以二次近似建模\gls{cost_function} $J$:
\begin{align}
 \hat J(\Vtheta) = J(\Vw^*) + \frac{1}{2}  (\Vw- \Vw^*)^\top \MH  (\Vw - \Vw^*),
\end{align}
其中$\MH$是$J$关于$\Vw$在$\Vw^*$点的\gls{hessian}。
鉴于假设$\Vw^*$是$J(\Vw)$的最小点,我们知道$\MH$为半正定。
在局部\gls{taylor}级数逼近下,\gls{gradient}由下式给出:
\begin{align}
 \nabla_{\Vw} \hat J (\Vw) = \MH (\Vw - \Vw^*).
\end{align}

% -- 243 --

接下来我们研究训练时参数向量的轨迹。
为简化起见,我们将参数向量初始化为原点\footnote{对于\gls{NN},我们需要打破\gls{hidden_unit}间的对称平衡因此不能将所有参数都初始化为$\mathbf{0}$(如\secref{sec:gradient_based_learning}所讨论的)。
然而,对于其他任何初始值$\Vw_{(0)}$该论证都成立},也就是$\Vw^{(0)} = 0$。
我们通过分析$\hat{J}$上的\gls{GD}来研究$J$上近似的\gls{GD}的效果:
\begin{align}
\Vw^{(\tau)} &= \Vw^{(\tau-1)} -\epsilon \nabla_{\Vw} \hat{J}( \Vw^{(\tau-1)} ) \\
&=  \Vw^{(\tau-1)}  - \epsilon  \MH ( \Vw^{(\tau-1)} -  \Vw^* ), \\
\Vw^{(\tau)}  -  \Vw^* &= (\MI - \epsilon  \MH) ( \Vw^{(\tau-1)} -  \Vw^* ).
 \end{align}
现在让我们在$\MH$特征向量的空间中改写表达式,利用$\MH$的特征分解:$\MH = \MQ \VLambda \MQ^\top$,其中$\VLambda$是对角矩阵,$\MQ$是特征向量的一组标准正交基。
\begin{align}
\Vw^{(\tau)}  -  \Vw^* &= (\MI - \epsilon \MQ \VLambda \MQ^\top) ( \Vw^{(\tau-1)} -  \Vw^* ) \\
\MQ^\top (\Vw^{(\tau)}  -  \Vw^*) &= (\MI - \epsilon \VLambda)\MQ^\top ( \Vw^{(\tau-1)} -  \Vw^* )
\end{align}
假定$\Vw^{(0)} = 0$并且$\epsilon$选择得足够小以保证$|1 - \epsilon \lambda_i |<1$,经过$\tau$次参数更新后轨迹如下:
\begin{align} \label{eq:740qw}
\MQ^\top  \Vw^{(\tau)} = [\MI - (\MI - \epsilon \VLambda)^\tau] \MQ^\top  \Vw^* .
\end{align}
现在,\eqnref{eq:713L2}中$\MQ^\top \tilde \Vw$的表达式能被重写为:
\begin{align}
\MQ^\top  \tilde \Vw &= (\VLambda + \alpha \MI)^{-1} \VLambda \MQ^\top  \Vw^*, \\
\MQ^\top  \tilde \Vw &= [\MI - (\VLambda + \alpha \MI)^{-1} \alpha] \MQ^\top  \Vw^*. 
\label{eq:742qw}
\end{align}
比较\eqnref{eq:740qw}和\eqnref{eq:742qw},我们能够发现,如果超参数$\epsilon,\alpha$和$\tau$满足如下:
\begin{align}
(\MI - \epsilon \VLambda)^\tau =  (\VLambda + \alpha \MI)^{-1} \alpha,
\end{align}
那么$L^2~$\gls{regularization}和\gls{early_stopping}可以被看作是等价的(至少在\gls{objective_function}的二次近似下)。
进一步取对数,使用$\log~(1+x)$的级数展开,我们可以得出结论:如果所有$\lambda_i$是小的(即$\epsilon \lambda_i \ll 1$且$\lambda_i / \alpha \ll 1$),那么
\begin{align}
\tau \approx \frac{1}{\epsilon \alpha}, \\
\alpha \approx \frac{1}{\tau \epsilon}.
\end{align}
也就是说,在这些假设下,训练迭代次数$\tau$起着与$L^2$参数成反比的作用,$\tau \epsilon$的倒数与\gls{weight_decay}系数的作用类似。

在大曲率(\gls{objective_function})方向上的参数值受\gls{regularization}影响小于小曲率方向。
当然,在\gls{early_stopping}的情况下,这实际上意味着在大曲率方向的参数比较小曲率方向的参数更早地学习到。
 
本节中的推导表明长度为$\tau$的轨迹结束于$L^2$\gls{regularization}目标的极小点。
当然,\gls{early_stopping}比简单的轨迹长度限制更丰富————\gls{early_stopping}通常涉及监控验证集误差,以便在空间特别好的点处终止轨迹。
因此\gls{early_stopping}比\gls{weight_decay}更具有优势,\gls{early_stopping}能自动确定\gls{regularization}的正确量,而\gls{weight_decay}需要进行多个不同超参数值的训练实验。

% -- 245 --

\section{参数绑定和参数共享}
\label{sec:parameter_tying_and_parameter_sharing}
目前为止,本章讨论对参数添加约束或惩罚时,一直是相对于固定的区域或点。
例如,$L^2$\gls{regularization}(或\gls{weight_decay})对参数偏离零的固定值进行惩罚。
然而,有时我们可能需要其他的方式来表达我们对模型参数适当值的先验知识。
有时候,我们可能无法准确地知道应该使用什么样的参数,但我们根据相关领域和模型结构方面的知识得知模型参数之间应该存在一些相关性。

我们经常想要表达的一种常见依赖是某些参数应当彼此接近。
考虑以下情形:我们有两个模型执行相同的分类任务(具有相同类别),但输入分布稍有不同。
形式地,我们有参数为$\Vw^{(A)}$的模型$A$和参数为$\Vw^{(B)}$的模型$B$。
这两种模型将输入映射到两个不同但相关的输出:$\hat y^{(A)} = f(\Vw^{(A)}, \Vx)$和$\hat y^{(B)} = f(\Vw^{(B)}, \Vx)$。

我们可以想象,这些任务会足够相似(或许具有相似的输入和输出分布),因此我们认为模型参数应彼此靠近:
$\forall i, w_i^{(A)}$应该与$ w_i^{(B)}$接近。
我们可以通过\gls{regularization}利用此信息。
具体来说,我们可以使用以下形式的参数范数惩罚:
$\Omega(\Vw^{(A)}, \Vw^{(B)}) = \norm{\Vw^{(A)}-\Vw^{(B)}}_2^2$。
在这里我们使用$L^2$惩罚,但也可以使用其他选择。

这种方法由\cite{LasserreJ2006}提出,\gls{regularization}一个模型(\gls{supervised}模式下训练的分类器)的参数,使其接近另一个\gls{unsupervised}模式下训练的模型(捕捉观察到的输入数据的分布)的参数。
构造的这种架构使得分类模型中的许多参数能与\gls{unsupervised}模型中对应的参数匹配。

参数范数惩罚是\gls{regularization}参数使其彼此接近的一种方式,而更流行的方法是使用约束:\emph{强迫某些参数相等}。
由于我们将各种模型或模型组件解释为共享唯一的一组参数,这种\gls{regularization}方法通常被称为\firstgls{parameter_sharing}。
和\gls{regularization}参数使其接近(通过范数惩罚)相比,\gls{parameter_sharing}的一个显著优点是,只有参数(唯一一个集合)的子集需要被存储在内存中。
对于某些特定模型,如\gls{CNN},这可能可以显著减少模型所占用的内存。


% -- 246 --
\subsection{\glsentrytext{CNN}}
目前为止,最流行和广泛使用的\gls{parameter_sharing}出现在应用于\gls{CV}的\firstacr{CNN}中。

自然图像有许多统计属性是对转换不变的。
例如,猫的照片即使向右边移了一个像素,仍保持猫的照片。
\glssymbol{CNN}通过在图像多个位置共享参数来考虑这个特性。
相同的特征(具有相同权重的\gls{hidden_unit})在输入的不同位置上计算获得。
这意味着无论猫出现在图像中的第$i$列或$i + 1$列,我们都可以使用相同的猫探测器找到猫。

\gls{parameter_sharing}显著降低了\glssymbol{CNN}模型的参数数量,并显著提高了网络的大小而不需要相应地增加训练数据。
它仍然是将领域知识有效地整合到网络架构的最佳范例之一。

我们将会在\chapref{chap:convolutional_networks}中更详细地讨论\gls{CNN}。

\section{\glsentrytext{sparse}\glsentrytext{representation}}
\label{sec:sparse_representations}
前文所述的\gls{weight_decay}直接惩罚模型参数。
另一种策略是惩罚\gls{NN}中的激活单元,\gls{sparse}化激活单元。
这种策略间接地对模型参数施加了复杂惩罚。

我们已经讨论过(在\secref{sec:l1_regularization}中)$L^1$惩罚如何诱导\gls{sparse}的参数,即许多参数为零(或接近于零)。
另一方面,\gls{representation}的\gls{sparse}性描述了一个\gls{representation}中许多元素是零(或接近零)的情况。
我们可以\gls{linear_regression}的情况下简单说明这种区别:
\begin{align}
\underset{\Vy ~\in~ \SetR^m}{
 \begin{bmatrix}
  18 \\  5 \\ 15 \\ -9 \\ -3
 \end{bmatrix}} = 
 \underset{\MA ~\in~ \SetR^{m \times n}}{
 \begin{bmatrix}
  4 & 0 & 0 & -2 & 0 & 0 \\
  0 & 0 & -1 & 0 & 3 & 0 \\
  0 & 5 & 0 & 0 & 0 & 0 \\
  1 & 0 & 0 & -1 & 0 & -4 \\
  1 & 0 & 0 & 0 & -5 & 0
 \end{bmatrix}} 
  \underset{\Vx ~\in~ \SetR^n}{
  \begin{bmatrix}
 2 \\ 3\\ -2\\ -5 \\ 1 \\ 4
 \end{bmatrix} }\\
 \underset{\Vy ~\in~ \SetR^m}{
 \begin{bmatrix}
  -14 \\  1 \\ 19 \\  2 \\ 23
 \end{bmatrix}} = 
 \underset{\MB ~\in~ \SetR^{m \times n}}{
 \begin{bmatrix}
  3 & -1 & 2 & -5 & 4 & 1 \\
  4 & 2 & -3 & -1 & 1 & 3 \\
  -1 & 5 & 4 & 2 & -3 & -2 \\
  3 & 1 & 2 & -3 & 0 & -3 \\
  -5 & 4 & -2 & 2 & -5 & -1
 \end{bmatrix}} 
  \underset{\Vh ~\in~ \SetR^n}{
  \begin{bmatrix}
 0 \\ 2 \\ 0 \\ 0 \\ -3 \\ 0
 \end{bmatrix} }
\end{align}

% -- 247 --

第一个表达式是参数\gls{sparse}的\gls{linear_regression}模型的例子。
第二个表达式是数据$\Vx$具有\gls{sparse}\gls{representation} $\Vh$的线性回归。
也就是说,$\Vh$是$\Vx$的一个函数,在某种意义上\gls{representation}存在于$\Vx$中的信息,但只是用一个\gls{sparse}向量\gls{representation}。

\gls{representation}的\gls{regularization}可以使用参数\gls{regularization}中同种类型的机制实现。

\gls{representation}的范数惩罚\gls{regularization}是通过向\gls{loss_function} $J$添加对\gls{representation}的范数惩罚来实现的。
我们将这个惩罚记作$\Omega(\Vh)$。
和以前一样,我们将\gls{regularization}后的损失函数记作$\tilde J$:
\begin{align}
 \tilde J(\Vtheta; \MX, \Vy) =  J(\Vtheta; \MX, \Vy)  + \alpha \Omega(\Vh),
\end{align}
其中$\alpha \in [0, \infty]$ 权衡范数惩罚项的相对贡献,越大的$\alpha$对应越多的\gls{regularization}。

正如对参数的$L^1$惩罚诱导参数\gls{sparse}性,对\gls{representation}元素的$L^1$惩罚诱导\gls{sparse}的\gls{representation}:
$\Omega(\Vh) = \norm{\Vh}_1 = \sum_i |h_i|$。
当然$L^1$惩罚是使\gls{representation}\gls{sparse}的方法之一。
其他方法还包括从\gls{representation}上的\ENNAME{Student}-$t$先验导出的惩罚\citep{Olshausen+Field-1996,Bergstra-Phd-2011}和\gls{KL}惩罚\citep{Larochelle+Bengio-2008},这些方法对于将表示中的元素约束于单位区间上特别有用。
\cite{HonglakL2008-small}和\cite{Goodfellow2009}都提供了\gls{regularization}几个样本平均激活的例子,即令$\frac{1}{m}\sum_i \Vh^{(i)}$接近某些目标值(如每项都是$.01$的向量)。

还有一些其他方法通过激活值的硬性约束来获得\gls{representation}\gls{sparse}。
例如,\textbf{正交匹配追踪}(orthogonal matching pursuit)\citep{pati93orthogonal}通过解决以下\gls{constrained_optimization}问题将输入值$\Vx$编码成\gls{representation} $\Vh$
\begin{align}
 \underset{\Vh, \norm{\Vh}_0 < k}{\argmin} \norm{\Vx - \MW \Vh}^2,
\end{align}
其中$\norm{\Vh}_0 $是$\Vh$中非零项的个数。
当$\MW$被约束为正交时,我们可以高效地解决这个问题。
这种方法通常被称为\ENNAME{OMP}-$k$,通过$k$指定允许的非零特征数量。
\cite{Coates2011b}证明\ENNAME{OMP}-$1$可以成为深度架构中非常有效的特征提取器。

% -- 248 --

含有\gls{hidden_unit}的模型在本质上都能变得\gls{sparse}。
在本书中,我们将看到在各种情况下使用\gls{sparse}\gls{regularization}的例子。

\section{\glsentrytext{bagging}和其他\glsentrytext{ensemble}方法}
\label{sec:bagging_and_other_ensemble_methods}
\firstgls{bagging}是通过结合几个模型降低泛化误差的技术\citep{ML:Breiman:bagging}。
主要想法是分别训练几个不同的模型,然后让所有模型表决测试样例的输出。
这是\gls{ML}中常规策略的一个例子,被称为\firstgls{model_averaging}。
采用这种策略的技术被称为\gls{ensemble}方法。

\firstgls{model_averaging}奏效的原因是不同的模型通常不会在测试集上产生完全相同的误差。

假设我们有$k$个回归模型。
假设每个模型在每个例子上的误差是$\epsilon_i$,这个误差服从零均值\gls{variance}为$\SetE[\epsilon_i^2] = v$且\gls{covariance}为$\SetE[\epsilon_i \epsilon_j] = c$的多维正态分布。
通过所有\gls{ensemble}模型的平均预测所得误差是$\frac{1}{k} \sum_i \epsilon_i$。 
\gls{ensemble}预测器平方误差的期望是
\begin{align}
 \SetE \Bigg[\Bigg(\frac{1}{k} \sum_i \epsilon_i \Bigg)^2\Bigg] &= \frac{1}{k^2} 
 \SetE \Bigg[\sum_i \Bigg(\epsilon_i^2 + \sum_{j \neq i} \epsilon_i \epsilon_j\Bigg)\Bigg], \\
&= \frac{1}{k} v + \frac{k-1}{k} c .                             
\end{align}
在误差完全相关即$c=v$的情况下,均方误差减少到$v$,所以\gls{model_averaging}没有任何帮助。
在错误完全不相关即$c =0$的情况下,该\gls{ensemble}平方误差的期望仅为$\frac{1}{k}v$。
这意味着\gls{ensemble}平方误差的期望会随着\gls{ensemble}规模增大而线性减小。
换言之,平均上,\gls{ensemble}至少与它的任何成员表现得一样好,并且如果成员的误差是独立的,\gls{ensemble}将显著地比其成员表现得更好。

不同的\gls{ensemble}方法以不同的方式构建\gls{ensemble}模型。
例如,\gls{ensemble}的每个成员可以使用不同的算法和\gls{objective_function}训练成完全不同的模型。
\gls{bagging}是一种允许重复多次使用同一种模型、训练算法和\gls{objective_function}的方法。

% -- 249 --

具体来说,\gls{bagging}涉及构造$k$个不同的数据集。
每个数据集从原始数据集中重复采样构成,和原始数据集具有相同数量的样例。
这意味着,每个数据集以高概率缺少一些来自原始数据集的例子,还包含若干重复的例子(如果所得训练集与原始数据集大小相同,那所得数据集中大概有原始数据集$2/3$的实例)。
模型$i$在数据集$i$上训练。
每个数据集所含样本的差异导致了训练模型之间的差异。
\figref{fig:chap7_bagging}是一个例子。

\begin{figure}[!htb]
\ifOpenSource
\centerline{\includegraphics{figure.pdf}}
\else
\centerline{\includegraphics{Chapter7/figures/bagging}}
\fi
\caption{描述\gls{bagging}如何工作的草图。
假设我们在上述数据集(包含一个8、一个6和一个9)上训练数字8的检测器。
假设我们制作了两个不同的重采样数据集。
\gls{bagging}训练程序通过有放回采样构建这些数据集。
第一个数据集忽略9并重复8。
在这个数据集上,检测器得知数字顶部有一个环就对应于一个8。
第二个数据集中,我们忽略6并重复9。
在这种情况下,检测器得知数字底部有一个环就对应于一个8。
这些单独的分类规则中的每一个都是不可靠的,但如果我们平均它们的输出,就能得到鲁棒的检测器,只有当8的两个环都存在时才能实现最大置信度。
}
\label{fig:chap7_bagging}
\end{figure}

\gls{NN}能找到足够多的不同的解,意味着他们可以从\gls{model_averaging}中受益(即使所有模型都在同一数据集上训练)。
\gls{NN}中随机初始化的差异、\gls{minibatch}的随机选择、超参数的差异或不同输出的非确定性实现往往足以使得\gls{ensemble}中的不同成员具有部分独立的误差。

% -- 250 --

\gls{model_averaging}是一个减少泛化误差的非常强大可靠的方法。
在科学论文中比较算法的表现时,它通常是不鼓励使用的,因为任何\gls{ML}算法都可以从\gls{model_averaging}中大幅获益(以增加计算和存储为代价)。

\gls{ML}比赛中的取胜算法通常是使用超过几十种\gls{model_averaging}的方法。
最近一个突出的例子是\ENNAME{Netflix Grand Prize}\citep{Koren09}。

不是所有构建\gls{ensemble}的技术都是为了让\gls{ensemble}模型比单一模型更加\gls{regularization}。
例如,一种被称为\textbf{Boosting}的技术\citep{ConfLT:Freund:gametheorie,ConfML:Freund:AdaBoostCompar}构建比单个模型\gls{capacity}更高的\gls{ensemble}模型。
通过向\gls{ensemble}逐步添加\gls{NN},\gls{boosting}已经被应用于构建神经网络的\gls{ensemble}\citep{nips-10:Holger+Yoshua:1998}。
通过逐渐增加\gls{NN}的\gls{hidden_unit},\gls{boosting}也可以将单个神经网络解释为一个\gls{ensemble}。

\section{\glsentrytext{dropout}}
\label{sec:dropout}
\textbf{Dropout}~\citep{Srivastava14}提供了\gls{regularization}一大类模型的方法,计算方便但功能强大。
为了有个初步的近似感受,我们可以认为\gls{dropout}是\gls{ensemble}大量深层\gls{NN}的实用\gls{bagging}方法。
\gls{bagging}涉及训练多个模型,并在每个测试样本上评估多个模型。
当每个模型都是一个很大的\gls{NN}时,这似乎是不切实际的,因为训练和评估这样的网络需要花费很多运行时间和内存。
通常我们只能\gls{ensemble}五至十个神经网络,如\cite{Szegedy-et-al-arxiv2014}\gls{ensemble}了六个神经网络赢得ILSVRC,超过这个数量就会迅速变得难以处理。
\gls{dropout}提供了一种廉价的\gls{bagging}\gls{ensemble}近似,能够训练和评估指数级数量的\gls{NN}。

具体而言,\gls{dropout}训练的\gls{ensemble}包括所有从基础网络除去非输出单元后形成的子网络,如\figref{fig:chap7_subnetworks}所示。
最先进的\gls{NN}基于一系列仿射变换和非线性变换,我们只需将一些单元的输出乘零就能有效地删除一个单元。
这个过程需要对模型(如径向基函数网络,单元的状态和参考值之间存在一定区别)进行一些修改。
为了简单起见,我们在这里提出乘零的简单\gls{dropout}算法,但是它被简单修改后,可以与从网络中移除单元的其他操作结合使用。
\begin{figure}[!htb]
\ifOpenSource
\centerline{\includegraphics{figure.pdf}}
\else
\centerline{\includegraphics{Chapter7/figures/subnetworks}}
\fi
\caption{\gls{dropout}训练由所有子网络组成的\gls{ensemble},其中子网络通过从基本网络中删除非输出单元构建。
我们从具有两个可见单元和两个\gls{hidden_unit}的基本网络开始。
这四个单元有十六个可能的子集。
右图展示了从原始网络中丢弃不同的单元子集而形成的所有十六个子网络。
在这个小例子中,所得到的网络中相当一部分没有输入单元或没有从输入连接到输出的路径。
当层较宽时,丢弃所有从输入到输出的可能路径的概率变小,所以这个问题不太可能在出现层较宽的网络中。}
\label{fig:chap7_subnetworks}
\end{figure}

% -- 251 --

回想一下\gls{bagging}学习,我们定义$k$个不同的模型,从训练集有放回采样构造$k$个不同的数据集,然后在训练集$i$上训练模型$i$。
\gls{dropout}的目标是在指数级数量的\gls{NN}上近似这个过程。
具体来说,在训练中使用\gls{dropout}时,我们会使用基于\gls{minibatch}产生较小步长的学习算法,如\gls{SGD}等。
我们每次在\gls{minibatch}中加载一个样本,然后随机抽样应用于网络中所有输入和\gls{hidden_unit}的不同二值\gls{mask}。
对于每个单元,\gls{mask}是独立采样的。
\gls{mask}值为1的采样概率(导致包含一个单元)是训练开始前一个固定的超参数。
它不是模型当前参数值或输入样本的函数。
通常在每一个\gls{minibatch}训练的神经网络中,一个输入单元被包括的概率为$0.8$,一个\gls{hidden_unit}被包括的概率为$0.5$。
然后,我们运行和之前一样的前向传播、反向传播以及学习更新。
\figref{fig:chap7_dropout_fprop}说明了在\gls{dropout}下的前向传播。
\begin{figure}[!htb]
\ifOpenSource
\centerline{\includegraphics{figure.pdf}}
\else
\centerline{\includegraphics{Chapter7/figures/dropout_fprop}}
\fi
\caption{在使用\gls{dropout}的前馈网络中前向传播的示例。
(顶部)在此示例中,我们使用具有两个输入单元,具有两个\gls{hidden_unit}的\gls{hidden_layer}以及一个输出单元的前馈网络。
(底部)为了执行具有\gls{dropout}的前向传播,我们随机地对向量$\Vmu$进行采样,其中网络中的每个输入或\gls{hidden_unit}对应一项。
$\Vmu$中的每项都是二值的且独立于其他项采样。
超参数的采样概率为$1$,\gls{hidden_layer}的采样概率通常为$0.5$,输入的采样概率通常为$0.8$。
网络中的每个单元乘以相应的\gls{mask},然后正常地继续沿着网络的其余部分前向传播。
这相当于从\figref{fig:chap7_subnetworks}中随机选择一个子网络并沿着前向传播。
}
\label{fig:chap7_dropout_fprop}
\end{figure}
% -- 252 --

更正式地说,假设一个\gls{mask}向量$\Vmu$指定被包括的单元,$J(\Vtheta, \Vmu)$是由参数$\Vtheta$和\gls{mask} $\Vmu$定义的模型代价。
那么\gls{dropout}训练的目标是最小化$\SetE_{\Vmu} J(\Vtheta, \Vmu)$。 
这个期望包含多达指数级的项,但我们可以通过抽样$\Vmu$获得梯度的无偏估计。

\gls{dropout}训练与\gls{bagging}训练不太一样。
在\gls{bagging}的情况下,所有模型都是独立的。
在\gls{dropout}的情况下,所有模型共享参数,其中每个模型继承父\gls{NN}参数的不同子集。
\gls{parameter_sharing}使得在有限可用的内存下表示指数级数量的模型变得可能。
在\gls{bagging}的情况下,每一个模型在其相应训练集上训练到收敛。
在\gls{dropout}的情况下,通常大部分模型都没有显式地被训练,因为通常父\gls{NN}会很大,以致于到宇宙毁灭都不可能采样完所有的子网络。
取而代之的是,在单个步骤中我们训练一小部分的子网络,\gls{parameter_sharing}会使得剩余的子网络也能有好的参数设定。
这些是仅有的区别。
除了这些,\gls{dropout}与\gls{bagging}算法一样。
例如,每个子网络中遇到的训练集确实是有放回采样的原始训练集的一个子集。

\gls{bagging}\gls{ensemble}必须根据所有成员的累积投票做一个预测。
在这种背景下,我们将这个过程称为\firstgls{inference}。
目前为止,我们在介绍\gls{bagging}和\gls{dropout}时没有要求模型具有明确的概率。
现在,我们假定该模型的作用是输出一个概率分布。
在\gls{bagging}的情况下,每个模型$i$产生一个概率分布$p^{(i)}(y \mid \Vx)$。 
\gls{ensemble}的预测由这些分布的算术平均值给出,
\begin{align}
 \frac{1}{k} \sum_{i=1}^k p^{(i)}(y \mid \Vx).
\end{align}

在\gls{dropout}的情况下,通过\gls{mask} $\Vmu$定义每个子模型的概率分布$p(y \mid \Vx, \Vmu)$。
所有\gls{mask}的算术平均值由下式给出
\begin{align}
  \sum_{\Vmu} p(\Vmu) p(y \mid \Vx, \Vmu),
\end{align}
其中$p(\Vmu)$是训练时采样$\Vmu$的概率分布。

% -- 254 --

因为这个求和包含多达指数级的项,除非该模型的结构允许某种形式的简化,否则是不可能计算的。
目前为止,无法得知深度\gls{NN}是否允许某种可行的简化。
相反,我们可以通过采样近似\gls{inference},即平均许多\gls{mask}的输出。
即使是$10-20$个\gls{mask}就足以获得不错的表现。

然而,一个更好的方法能不错地近似整个\gls{ensemble}的预测,且只需一个前向传播的代价。
要做到这一点,我们改用\gls{ensemble}成员预测分布的几何平均而不是算术平均。
\cite{WardeFarley+al-ICLR2014}提出的论点和经验证据表明,在这个情况下几何平均与算术平均表现得差不多。

多个概率分布的几何平均不能保证是一个概率分布。
为了保证结果是一个概率分布,我们要求没有子模型给某一事件分配概率0,并重新标准化所得分布。
通过几何平均直接定义的非标准化概率分布由下式给出
\begin{align}
\tilde{p}_{\text{ensemble}}(y \mid \Vx) = \sqrt[2^d]{\prod_{\Vmu} p(y \mid \Vx, \Vmu)},
\end{align}
其中$d$是可被丢弃的单元数。
这里为简化介绍,我们使用均匀分布的$\Vmu$,但非均匀分布也是可以的。
为了作出预测,我们必须重新标准化\gls{ensemble}:
\begin{align}
p_{\text{ensemble}}(y \mid \Vx)  = \frac{\tilde{p}_{\text{ensemble}}(y \mid \Vx)}
 {\sum_{y'}\tilde{p}_{\text{ensemble}}(y' \mid \Vx) }.
\end{align}
 
涉及\gls{dropout}的一个重要观点\citep{Hinton-et-al-arxiv2012}是,我们可以通过评估模型中$p(y \mid \Vx)$来近似$ p_{\text{ensemble}}$:
该模型具有所有单元,但我们将单元$i$的输出的权重乘以单元$i$的被包含概率。
这个修改的动机是得到从该单元输出的正确期望值。
我们把这种方法称为\firstgls{weight_scaling_inference_rule}。
目前还没有在深度非线性网络上对这种近似推断规则的准确性作任何理论分析,但经验上它表现得很好。

 % -- 255 --
 
因为我们通常使用$\frac{1}{2}$的包含概率,权重比例规则一般相当于在训练结束后将权重除$2$,然后像平常一样使用模型。
实现相同结果的另一种方法是在训练期间将单元的状态乘$2$。
无论哪种方式,我们的目标是确保在测试时一个单元的期望总输入与在训练时该单元的期望总输入是大致相同的(即使近半单位在训练时丢失)。

对许多不具有非线性\gls{hidden_unit}的模型族而言,\gls{weight_scaling_inference_rule}是精确的。
举个简单的例子,考虑\gls{softmax}回归分类,其中由向量$\RVv$表示$n$个输入变量:
\begin{align}
 P(\RSy = \Sy \mid \RVv) = \text{softmax}\big(\MW^\top\RVv + \Vb\big)_y.
\end{align}
我们可以根据二值向量$\Vd$逐元素的乘法将一类子模型进行索引:
\begin{align}
P(\RSy = \Sy \mid \RVv; \Vd) = \text{softmax}\big(\MW^\top(\Vd \odot \RVv) + \Vb \big)_y.
\end{align}
\gls{ensemble}预测器被定义为重新标准化所有\gls{ensemble}成员预测的几何平均:
\begin{align} \label{eq:758pe}
P_{\text{ensemble}}(\RSy = \Sy \mid \RVv)  = \frac{\tilde{P}_{\text{ensemble}}(\RSy = \Sy \mid \RVv)}
 {\sum_{y'}\tilde{P}_{\text{ensemble}}(\RSy = \Sy' \mid \RVv) },
\end{align}
其中
\begin{align}
\tilde{P}_{\text{ensemble}}(\RSy=\Sy \mid \RVv) =
\sqrt[2^n]{\prod_{\Vd \in \{0,1\}^n} P(\RSy = \Sy \mid \RVv; \Vd)}.
\end{align}

为了证明\gls{weight_scaling_inference_rule}是精确的,我们简化$ \tilde{P}_{\text{ensemble}}$:
\begin{align}
\tilde{P}_{\text{ensemble}}(\RSy=\Sy \mid \RVv) =
\sqrt[2^n]{\prod_{\Vd \in \{0,1\}^n} P(\RSy = \Sy \mid \RVv; \Vd)} \\
= \sqrt[2^n]{\prod_{\Vd \in \{0,1\}^n} \text{softmax}(\MW^\top(\Vd \odot \RVv) + \Vb)_y} \\
= \sqrt[2^n]{\prod_{\Vd \in \{0,1\}^n} \frac{\exp (\MW_{y,:}^\top(\Vd \odot \RVv) + \Vb_y)}
{\sum_{y'}\exp (\MW_{y',;}^\top(\Vd \odot \RVv) + \Vb_{y'})}}\\
=  \frac{\sqrt[2^n]{\prod_{\Vd \in \{0,1\}^n}\exp (\MW_{y,:}^\top(\Vd \odot \RVv) + \Vb_y)}}
{ \sqrt[2^n] \prod_{\Vd \in \{0,1\}^n} \sum_{y'}\exp (\MW_{y',:}^\top(\Vd \odot \RVv) + \Vb_{y'})}
\end{align}
由于$\tilde P$将被标准化,我们可以放心地忽略那些相对$y$不变的乘法:
\begin{align}
\tilde{P}_{\text{ensemble}}(\RSy=\Sy \mid \RVv) &\propto 
\sqrt[2^n]{\prod_{\Vd \in \{0,1\}^n} \exp (\MW_{y,:}^\top(\Vd \odot \RVv) + \Vb_y)} \\
& = \exp \Bigg(\frac{1}{2^n} \sum_{\Vd \in \{0,1\}^n} \MW_{y,;}^\top(\Vd \odot \RVv) + \Vb_y \Bigg) \\
& = \exp \Big(\frac{1}{2}\MW_{y,:}^\top \RVv + \Vb_y \Big) .
\end{align}
将其代入\eqnref{eq:758pe},我们得到了一个权重为$\frac{1}{2}\MW$的\gls{softmax}分类器。

% -- 256 --

\gls{weight_scaling_inference_rule}在其他设定下也是精确的,包括条件正态输出的回归网络以及那些隐藏层不包含非线性的深度网络。
然而,\gls{weight_scaling_inference_rule}对具有非线性的深度模型仅仅是一个近似。
虽然这个近似尚未有理论上的分析,但在实践中往往效果很好。
\cite{Goodfellow-et-al-ICML2013}实验发现,在对\gls{ensemble}预测的近似方面,\gls{weight_scaling_inference_rule}比\gls{monte_carlo}近似更好(就分类精度而言)。
即使允许\gls{monte_carlo}近似采样多达1000子网络时也比不过\gls{weight_scaling_inference_rule}。
\cite{gal2015bayesian}发现一些模型可以通过二十个样本和\gls{monte_carlo}近似获得更好的分类精度。
似乎\gls{inference}近似的最佳选择是与问题相关的。

\cite{Srivastava14}显示,\gls{dropout}比其他标准的计算开销小的\gls{regularization}方法(如\gls{weight_decay}、过滤器范数约束和\gls{sparse}激活的\gls{regularization})更有效。
\gls{dropout}也可以与其他形式的\gls{regularization}合并,得到进一步的提升。

计算方便是\gls{dropout}的一个优点。
训练过程中使用\gls{dropout}产生$n$个随机二进制数与状态相乘,每个样本每次更新只需$\CalO(n)$的计算复杂度。
根据实现,也可能需要$\CalO(n)$的存储空间来持续保存这些二进制数(直到反向传播阶段)。
使用训练好的模型\gls{inference}时,计算每个样本的代价与不使用\gls{dropout}是一样的,尽管我们必须在开始运行\gls{inference}前将权重除以2。

% -- 257 --

\gls{dropout}的另一个显著优点是不怎么限制适用的模型或训练过程。
几乎在所有使用\gls{distributed_representation}且可以用\gls{SGD}训练的模型上都表现很好。
包括前馈神经网络、概率模型,如\gls{RBM}\citep{Srivastava14},以及\gls{RNN}\citep{Bayer-et-al-arXiv-2014,Pascanu-et-al-ICLR2014}。
许多效果差不多的其他\gls{regularization}策略对模型结构的限制更严格。

虽然\gls{dropout}在特定模型上每一步的代价是微不足道的,但在一个完整的系统上使用\gls{dropout}的代价可能会是显著的。
因为\gls{dropout}是一个\gls{regularization}技术,它减少了模型的有效容量。
为了抵消这种影响,我们必须增大模型规模。
不出意外的话,使用\gls{dropout}时最佳验证集的误差会低很多,但这是以更大的模型和更多训练算法的迭代次数为代价换来的。
对于非常大的数据集,\gls{regularization}带来的泛化误差减少得很小。
在这些情况下,使用\gls{dropout}和更大模型的计算代价可能超过\gls{regularization}带来的好处。

只有极少的训练样本可用时,\gls{dropout}不会很有效。
在只有不到5000的样本的\ENNAME{Alternative Splicing}数据集上\citep{Xiong2011},贝叶斯神经网络\citep{Neal1996}比\gls{dropout}表现得更好\citep{Srivastava14}。
当有其他未分类的数据可用时,\gls{unsupervised}特征学习也比\gls{dropout}更有优势。


\cite{Wager+al-2013}表明,当\gls{dropout}作用于\gls{linear_regression}时,相当于每个输入特征具有不同\gls{weight_decay}系数的$L^2$\gls{weight_decay}。 每个特征的\gls{weight_decay}系数的大小是由其方差来确定的。
其他\gls{linear_model}也有类似的结果。
而对于深度模型而言,\gls{dropout}与\gls{weight_decay}是不等同的。


使用\gls{dropout}训练时的随机性不是这个方法成功的必要条件。
它仅仅是近似所有子模型总和的一个方法。
\cite{WangManning-ICML2013-small}导出了近似这种边缘分布的解析解。
他们的近似被称为\firstgls{fast_dropout},减小梯度计算中的随机性而获得更快的收敛速度。
这种方法也可以在测试时应用,能够比\gls{weight_scaling_inference_rule}更合理地(但计算也更昂贵)近似所有子网络的平均。
\gls{fast_dropout}在小神经网络上的性能几乎与标准的\gls{dropout}相当,但在大问题上尚未产生显著改善或尚未应用。

% -- 258 --

随机性对实现\gls{dropout}的\gls{regularization}效果不是必要的,同时也不是充分的。
为了证明这一点,\cite{WardeFarley+al-ICLR2014}使用一种被称为\textbf{Dropout Boosting}的方法设计了一个对照实验,具有与传统\gls{dropout}方法完全相同的噪声\gls{mask}, 但缺乏\gls{regularization}效果。
\gls{dropout_boosting}训练整个\gls{ensemble}以最大化训练集上的似然。
从传统\gls{dropout}类似于\gls{bagging}的角度来看,这种方式类似于\gls{boosting}。
如预期一样,和单一模型训练整个网络相比,\gls{dropout_boosting}几乎没有\gls{regularization}效果。
这表明,使用\gls{bagging}解释\gls{dropout}具有使用稳健性噪声解释\gls{dropout}之外的额外价值。
只有当随机抽样的\gls{ensemble}成员相互独立地训练,才能达到\gls{bagging}\gls{ensemble}的\gls{regularization}效果。

\gls{dropout}启发其他以随机方法训练指数量级的共享权重的\gls{ensemble}。
\ENNAME{DropConnect}是
\gls{dropout}的一个特殊情况,其中一个标量权重和单个\gls{hidden_unit}状态之间的每个乘积被认为是可以丢弃的一个单元\citep{Wan+al-ICML2013-small}。
随机\gls{pooling}是构造\gls{CNN}\gls{ensemble}的一种随机化\gls{pooling}的形式(见\secref{sec:pooling}),其中每个卷积网络参与每个特征图的不同空间位置。
目前为止,\gls{dropout}仍然是最广泛使用的隐式\gls{ensemble}方法。

一个关于\gls{dropout}的重要见解是,通过随机行为训练网络并平均多个随机决定进行预测,实现了一种\gls{parameter_sharing}的\gls{bagging}形式。
早些时候,我们将\gls{dropout}描述为通过包括或排除单元形成模型\gls{ensemble}的\gls{bagging}。
然而,这种\gls{parameter_sharing}策略不一定要基于包括和排除。
原则上,任何一种随机的修改都是可接受的。
在实践中,我们必须选择让\gls{NN}能够学习对抗的修改类型。
在理想情况下,我们也应该使用可以快速近似\gls{inference}的模型族。
我们可以认为由向量$\Vmu$参数化的任何形式的修改,是对$\Vmu$所有可能的值训练$p(y \mid \Vx, \Vmu)$的\gls{ensemble}。
注意,这里不要求$\Vmu$具有有限数量的值。
例如,$\Vmu$可以是实值。
\cite{Srivastava14}表明,权重乘以$\Vmu \sim \CalN(\mathbf{1}, \MI)$比基于二值\gls{mask}\gls{dropout}表现得更好。
由于$\SetE[\Vmu] = 1$,标准网络自动实现\gls{ensemble}的近似\gls{inference},而不需要\gls{weight_scaling_inference_rule}。

% -- 259 --

目前为止,我们将\gls{dropout}介绍为一种纯粹高效近似\gls{bagging}的方法。
然而,还有比这更进一步的\gls{dropout}观点。
\gls{dropout}不仅仅是训练一个\gls{bagging}的\gls{ensemble}模型,
并且是共享\gls{hidden_unit}的\gls{ensemble}模型。
这意味着无论其他\gls{hidden_unit}是否在模型中,每个\gls{hidden_unit}必须都能够表现良好。
\gls{hidden_unit}必须准备好进行模型之间的交换和互换。
\cite{Hinton-et-al-arxiv2012-small}由生物学的想法受到启发:有性繁殖涉及到两个不同生物体之间交换基因,进化产生的压力使得基因不仅是良好的而且要准备好不同有机体之间的交换。
这样的基因和这些特点对环境的变化是非常稳健的,因为它们一定会正确适应任何一个有机体或模型不寻常的特性。
因此\gls{dropout}\gls{regularization}使得每个\gls{hidden_unit}不仅要是一个很好的特征,更要在许多情况下是良好的特征。
\cite{WardeFarley+al-ICLR2014}将\gls{dropout}与大\gls{ensemble}的训练相比并得出结论:相比独立模型\gls{ensemble}获得泛化误差改进,\gls{dropout}会带来额外的改进。

\gls{dropout}强大的大部分原因来自施加到\gls{hidden_unit}的\gls{mask}噪声,了解这一事实是重要的。
这可以看作是对输入内容的信息高度智能化、自适应破坏的一种形式,而不是对输入原始值的破坏。
例如,如果模型学得通过鼻检测脸的\gls{hidden_unit} $h_i$,那么丢失$h_i$对应于擦除图像中有鼻子的信息。
模型必须学习另一种$h_i$,要么是鼻子存在的冗余编码,要么是像嘴这样的脸部的另一特征。
传统的噪声注入技术,在输入端加非结构化的噪声不能够随机地从脸部图像中抹去关于鼻子的信息,除非噪声的幅度大到几乎能抹去图像中所有的信息。
破坏提取的特征而不是原始值,让破坏过程充分利用该模型迄今获得的关于输入分布的所有知识。

\gls{dropout}的另一个重要方面是噪声是乘性的。
如果是固定规模的加性噪声,那么加了噪声$\epsilon$的\gls{rectified_linear}\gls{hidden_unit}可以简单地学会使$h_i$变得很大(使增加的噪声$\epsilon$变得不显著)。
乘性噪声不允许这样病态地解决噪声鲁棒性问题。

% -- 260 --

另一种\gls{DL}算法——\gls{batch_normalization},在训练时向\gls{hidden_unit}引入加性和乘性噪声重新参数化模型。
\gls{batch_normalization}的主要目的是改善优化,但噪声具有\gls{regularization}的效果,有时没必要再使用\gls{dropout}。
\gls{batch_normalization}将会在\secref{sec:batch_normalization}中被更详细地讨论。



\section{对抗训练}
\label{sec:adversarial_training}
在许多情况下,\gls{NN}在独立同分布的测试集上进行评估已经达到了人类表现。
因此,我们自然要怀疑这些模型在这些任务上是否获得了真正的人类层次的理解。
为了探索网络对底层任务的理解层次,我们可以探索这个模型错误分类的例子。
\cite{Szegedy-ICLR2014}发现,在精度达到人类水平的\gls{NN}上通过优化过程故意构造数据点,其上的误差率接近\NUMTEXT{100\%},模型在这个输入点$\Vx'$的输出与附近的数据点$\Vx$非常不同。
在许多情况下,$\Vx'$与$\Vx$非常近似,人类观察者不会察觉原始样本和\firstgls{adversarial_example}之间的差异,但是网络会作出非常不同的预测。
见\figref{fig:chap7_panda_577}中的例子。
\begin{figure}[!htb]
\ifOpenSource
\centerline{\includegraphics{figure.pdf}}
\else
\centering
\begin{tabular}{>{\centering\arraybackslash}m{.2\figwidth}m{.5in}>{\centering\arraybackslash}m{.2\figwidth}m{.1in}>{\centering\arraybackslash}m{.2\figwidth}}
    \centering\arraybackslash
%abs max for panda was 138, eps was 1., so relative eps is approximately .007
    \includegraphics[width=.2 \figwidth]{Chapter7/figures/panda_577.png} &%
    \centering\arraybackslash%
$\ +\ .007\ \times$ &%
    \includegraphics[width=.2\figwidth]{Chapter7/figures/nematode_082.png} &%
    $=$ & %
    \includegraphics[width=.2\figwidth]{Chapter7/figures/gibbon_993.png} \\
    $\centering \Vx$     &%
    & $\text{sign} (\nabla_{\Vx} J(\Vtheta, \Vx, y) )$ & & $\Vx + \epsilon \text{sign} (\nabla_{\Vx} J(\Vtheta, \Vx, y) )$ \\
    $y=$``panda'' &                & ``nematode''     &   & ``gibbon'' \\
    w/ 57.7\% confidence &        &   w/ 8.2\% confidence & & w/ 99.3 \% confidence
\end{tabular}    
\fi
\caption[Fast adversarial sample generation]{
在ImageNet上应用GoogLeNet~\citep{Szegedy-et-al-arxiv2014}的\gls{adversarial_example}生成的演示。
通过添加一个不可察觉的小向量(其中元素等于\gls{cost_function}相对于输入的梯度元素的符号),我们可以改变GoogLeNet对此图像的分类结果。
经\citet{Goodfellow-2015-adversarial}许可转载。
}
\label{fig:chap7_panda_577}
\end{figure}

% -- 261 --

\gls{adversarial_example}在很多领域有很多影响,例如计算机安全,这超出了本章的范围。
然而,它们在\gls{regularization}的背景下很有意思,因为我们可以通过\firstgls{adversarial_training}减少原有独立同分布的测试集的错误率——在对抗扰动的训练集样本上训练网络\citep{Szegedy-ICLR2014,Goodfellow-2015-adversarial}。


\cite{Goodfellow-2015-adversarial}表明,这些\gls{adversarial_example}的主要原因之一是过度线性。
\gls{NN}主要是基于线性块构建的。
因此在一些实验中,它们实现的整体函数被证明是高度线性的。
这些线性函数很容易优化。
不幸的是,如果一个线性函数具有许多输入,那么它的值可以非常迅速地改变。
如果我们用$\epsilon$改变每个输入,那么权重为$\Vw$的线性函数可以改变$\epsilon \norm{\Vw}_1$之多,如果$\Vw$是高维的这会是一个非常大的数。
\gls{adversarial_training}通过鼓励网络在训练数据附近的局部区域恒定来限制这一高度敏感的局部线性行为。
这可以被看作是一种明确地向\gls{supervised}\gls{NN}引入局部恒定先验的方法。

对抗训练有助于体现积极\gls{regularization}与大型函数族结合的力量。
纯粹的\gls{linear_model},如\gls{logistic_regression},由于它们被限制为线性而无法抵抗\gls{adversarial_example}。
\gls{NN}能够表示范围广泛的函数,从接近线性到局部近似恒定,从而可以灵活地捕获到训练数据中的线性趋势同时学习抵抗局部扰动。

\gls{adversarial_example}也提供了一种实现\gls{semi_supervised_learning}的方法。
在数据集中没有分配标签的点$\Vx$处,模型自己为其分配一些标签$\hat y$。
模型的标记$\hat y$未必是真正的标签,但如果模型是高品质的,那么$\hat y$提供正确标签的可能性很大。
我们可以搜索一个\gls{adversarial_example} $\Vx'$,导致分类器输出一个标签$y'$且$y' \neq \hat y$。
不使用真正的标签,而是由训练好的模型提供标签产生的\gls{adversarial_example}被称为\firstgls{virtual_adversarial_example}\citep{miyato2015distributional}。
我们可以训练分类器为$\Vx$和$\Vx'$分配相同的标签。
这鼓励分类器学习一个沿着未标签数据所在流形上任意微小变化都很鲁棒的函数。
驱动这种方法的假设是,不同的类通常位于分离的流形上,并且小扰动不会使数据点从一个类的流形跳到另一个类的流形上。

% -- 262 -- 

%%%%%%%%%%%%%%%%%%%%%%%%%%%%%%%%%%%%%%%%%%%%%%
%       Hard to translate
%%%%%%%%%%%%%%%%%%%%%%%%%%%%%%%%%%%%%%%%%%%%%%
\section{\glsentrytext{tangent_distance}、\glsentrytext{tangent_prop}和流形正切分类器}
\label{sec:tangent_distance_tangent_prop_and_manifold_tangent_classifier}
如\secref{sec:manifold_learning}所述,许多\gls{ML}通过假设数据位于低维流形附近来克服维数灾难。

一个利用流形假设的早期尝试是\firstgls{tangent_distance}算法\citep{Simard93-small,Simard98}。
它是一种非参数的最近邻算法,其中使用的度量不是通用的欧几里德距离,而是根据邻近流形关于聚集概率的知识导出的。
这个算法假设我们尝试分类的样本和同一流形上的样本具有相同的类别。
由于分类器应该对局部因素(对应于流形上的移动)的变化保持不变,一种合理的度量是将点$\Vx_1$和$\Vx_2$各自所在流形$M_1$和$M_2$的距离作为点$\Vx_1$和$\Vx_2$之间的最近邻距离。
然而这可能在计算上是困难的(它需要解决一个寻找$M_1$和$M_2$最近点对的优化问题),一种局部合理的廉价替代是使用$\Vx_i$点处切平面近似$M_i$,并测量两条切平面或一个切平面和点之间的距离。
这可以通过求解一个低维线性系统(就流形的维数而言)来实现。
当然,这种算法需要指定那些切向量。

受相关启发,\firstgls{tangent_prop}算法\citep{Simard92-short}(\figref{fig:chap7_mtc_color})训练带有额外惩罚的\gls{NN}分类器,使\gls{NN}的每个输出$f(\Vx)$对已知的变化因素是局部不变的。
这些变化因素对应于沿着的相同样本聚集的流形的移动。
这里实现局部不变性的方法是要求$\nabla_{\Vx} f(\Vx)$与已知流形的切向$\Vv^{(i)}$正交,或者等价地通过\gls{regularization}惩罚$\Omega$使$f$在$\Vx$的$\Vv^{(i)}$方向的导数较小:
\begin{align} \label{eq:767}
 \Omega(f) = \sum_i \Big((\nabla_{\Vx} f(\Vx)^\top \Vv^{(i)}) \Big)^2 .
\end{align}
这个\gls{regularization}项当然可以通过适当的超参数缩放,并且对于大多数\gls{NN},我们需要对许多输出求和(此处为描述简单,$f(\Vx)$为唯一输出)。
与\gls{tangent_distance}算法一样,切向量是从先验知识推导的,通常是从变换(如平移、旋转和缩放图像)的效果获得的形式知识。
\gls{tangent_prop}不仅用于\gls{supervised_learning}\citep{Simard92-short},还在\gls{RL}\citep{Thrun-NIPS1994}中有所应用。
\begin{figure}[!htb]
\ifOpenSource
\centerline{\includegraphics{figure.pdf}}
\else
\centerline{\includegraphics{Chapter7/figures/mtc_color}}
\fi
\caption{\gls{tangent_prop}算法\citep{Simard92-short}和\gls{manifold}正切分类器主要思想的示意图\citep{Dauphin-et-al-NIPS2011-small},它们都\gls{regularization}分类器的输出函数$f(\Vx)$。
每条曲线表示不同类别的\gls{manifold},这里表示嵌入二维空间中的一维\gls{manifold}。
在一条曲线上,我们选择单个点并绘制一个与类别\gls{manifold}相切(平行并接触\gls{manifold})的向量以及与类别\gls{manifold}垂直(与\gls{manifold}正交)的向量。
在多维情况下,可以存在许多切线方向和法线方向。
我们希望分类函数在垂直于\gls{manifold}方向上快速改变,并且在类别\gls{manifold}的方向上保持不变。
\gls{tangent_prop}和\gls{manifold}正切分类器都会\gls{regularization} $f(\Vx)$,使其不随$\Vx$沿\gls{manifold}的移动而剧烈变化。
\gls{tangent_prop}需要用户手动指定正切方向的计算函数(例如指定小平移后的图像保留在相同类别的\gls{manifold}中),而\gls{manifold}正切分类器通过训练\gls{AE}拟合训练数据来估计\gls{manifold}的正切方向 。
我们将在\chapref{chap:autoencoders}中讨论使用\gls{AE}来估计\gls{manifold}。
}
\label{fig:chap7_mtc_color}
\end{figure}
% -- 263 --

\gls{tangent_prop}与数据集增强密切相关。
在这两种情况下,该算法的用户通过指定一组应当不会改变网络输出的转换,将其先验知识编码至算法中。
不同的是在数据集增强的情况下,网络显式地训练正确分类这些施加大量变换后产生的不同输入。
\gls{tangent_prop}不需要显式访问一个新的输入点。
取而代之,它解析地对模型\gls{regularization}从而在指定转换的方向抵抗扰动。
虽然这种解析方法是聪明优雅的,但是它有两个主要的缺点。
首先,模型的\gls{regularization}只能抵抗无穷小的扰动。
显式的数据集增强能抵抗较大的扰动。
其次,我们很难在基于\gls{ReLU}的模型上使用无限小的方法。
这些模型只能通过关闭单元或缩小它们的权重才能缩小它们的导数。
它们不能像\ENNAME{sigmoid}或\ENNAME{tanh}单元一样通过较大权重在高值处饱和以收缩导数。
数据集增强在\gls{ReLU}上工作得很好,因为不同的整流单元会在每一个原始输入的不同转换版本上被激活。

% -- 264 --

\gls{tangent_prop}也和\gls{double_backprop}\citep{DruckerLeCun92}以及\gls{adversarial_training}\citep{Szegedy-ICLR2014,Goodfellow-2015-adversarial}有关联。
\gls{double_backprop}\gls{regularization}使\gls{jacobian}矩阵偏小,而\gls{adversarial_training}找到原输入附近的点,训练模型在这些点上产生与原来输入相同的输出。
\gls{tangent_prop}和手动指定转换的数据集增强都要求模型在输入变化的某些特定的方向上保持不变。
\gls{double_backprop}和\gls{adversarial_training}都要求模型对输入所有方向中的变化(只要该变化较小)都应当保持不变。
正如数据集增强是\gls{tangent_prop}非无限小的版本,\gls{adversarial_training}是\gls{double_backprop}非无限小的版本。

流形正切分类器\citep{Dauphin-et-al-NIPS2011}无需知道切线向量的先验。
我们将在\chapref{chap:autoencoders}看到,\gls{AE}可以估算流形的切向量。
流形正切分类器使用这种技术来避免用户指定切向量。
如\figref{fig:chap14_cifar_cae}所示,这些估计的切向量不仅对图像经典几何变换(如转化、旋转和缩放)保持不变,还必须掌握对特定对象(如正在移动的身体某些部分)保持不变的因素。
因此根据流形正切分类器提出的算法相当简单:
(1)使用\gls{AE}通过\gls{unsupervised_learning}来学习流形的结构,以及(2)如\gls{tangent_prop}(\eqnref{eq:767})一样使用这些切面\gls{regularization}\gls{NN}分类器。

在本章中,我们已经描述了大多数用于\gls{regularization}\gls{NN}的通用策略。
\gls{regularization}是\gls{ML}的中心主题,因此我们将不时在其余各章中重新回顾。
\gls{ML}的另一个中心主题是优化,我们将在下一章描述。

% -- 265 --

% !Mode:: "TeX:UTF-8"
% Translator: Yujun Li 
\chapter{\glsentrytext{deep_model}中的优化}
\label{chap:optimization_for_training_deep_models}
% 267 head
\gls{DL}算法在许多情况下都涉及到优化。
例如,模型中的进行推断(如\,\glssymbol{PCA})涉及到求解优化问题。
我们经常使用解析优化去证明或设计算法。
在\gls{DL}涉及到的诸多优化问题中,最难的是\gls{NN}训练。
甚至是用几百台机器投入几天到几个月来解决单个\gls{NN}训练问题,也是很常见的。
因为这其中的优化问题很重要,代价也很高,因此研究者们开发了一组专门为此设计的优化技术。
本章会介绍\gls{NN}训练中的这些优化技术。

% 267 mid
如果你不熟悉基于梯度优化的基本原则,我们建议回顾\chapref{chap:numerical_computation}。
该章简要概述了一般的\gls{nume_optimization}。


本章主要关注这一类特定的优化问题:寻找\gls{NN}上的一组参数$\Vtheta$,它能显著地降低\gls{cost_function} $J(\Vtheta)$,该\gls{cost_function}通常包括整个\gls{training_set}上的性能评估和额外的\gls{regularization}项。
% 267 mid


首先,我们会介绍在\gls{ML}任务中作为训练算法使用的优化与纯优化有哪些不同。
接下来,我们会介绍导致\gls{NN}优化困难的几个具体挑战。
然后,我们会介绍几个实用算法,包括优化算法本身和初始化参数的策略。
更高级的算法能够在训练中自适应调整\gls{learning_rate},或者使用\gls{cost_function}二阶导数包含的信息。
最后,我们会介绍几个将简单优化算法结合成高级过程的优化策略,以此作为总结。
% 268 head


\section{学习和纯优化有什么不同}
\label{sec:how_learning_differs_from_pure_optimization}
用于\gls{deep_model}训练的优化算法与传统的优化算法在几个方面有所不同。
\gls{ML}通常是间接作用的。
在大多数\gls{ML}问题中,我们关注某些性能度量$P$,其定义于\gls{test_set}上并且可能是不可解的。
因此,我们只是间接地优化$P$。
我们希望通过降低一个不同的\gls{cost_function} $J(\Vtheta)$来提高$P$。
这一点与纯优化不同,纯优化最小化目标$J$本身。
训练\gls{deep_model}的优化算法通常也会包括一些针对\gls{ML}\gls{objective_function}的特定结构进行的特化。
% 268 mid


通常,\gls{cost_function}可写为\gls{training_set}上的平均,如
\begin{equation}
\label{eq:8.1}
    J(\Vtheta) = \SetE_{(\RVx, \RSy) \sim\hat{p}_\text{data}} L(f(\Vx ; \Vtheta), y), % ??
\end{equation}
其中$L$是每个样本的\gls{loss_function},$f(\Vx;\Vtheta)$是输入$\Vx$时所预测的输出,$\hat{p}_{\text{data}}$是\gls{empirical_distribution}。
\gls{supervised_learning}中,$y$是目标输出。
在本章中,我们会介绍不带\gls{regularization}的\gls{supervised}学习,$L$的变量是$f(\Vx;\Vtheta)$和$y$。
不难将这种\gls{supervised_learning}扩展成其他形式,如包括$\Vtheta$或者$\Vx$作为参数,或是去掉参数$y$,以发展不同形式的\gls{regularization}或是\gls{unsupervised_learning}。
% 268 mid


\eqnref{eq:8.1}定义了\gls{training_set}上的\gls{objective_function}。
通常,我们更希望最小化取自\emph{\gls{DGD}}\,$p_{\text{data}}$的期望,而不仅仅是有限\gls{training_set}上的对应\gls{objective_function}:
\begin{equation}
\label{eq:8.2}
    J^*(\Vtheta) = \SetE_{(\RVx, \RSy) \sim p_\text{data}} L(f(\Vx ;\Vtheta),y). % ??
\end{equation}
% 268 end


\subsection{\glsentrytext{empirical_risk_minimization}}
\label{sec:empirical_risk_minimization}
\gls{ML}算法的目标是降低\eqnref{eq:8.2}所示的期望\gls{generalization_error}。
这个数据量被称为\firstgls{risk}。
在这里,我们强调该期望取自真实的\gls{underlying}分布$p_{\text{data}}$。
如果我们知道了真实分布$p_{\text{data}}(\Vx, y)$,那么最小化风险变成了一个可以被优化算法解决的优化问题。
然而,我们遇到的\gls{ML}问题,通常是不知道$p_\text{data}(\Vx, y)$,只知道\gls{training_set}中的样本。
% 269 head


将\gls{ML}问题转化回一个优化问题的最简单方法是最小化\gls{training_set}上的期望损失。
这意味着用\gls{training_set}上的\gls{empirical_distribution}\,$\hat{p}(\Vx,y)$替代真实分布$p(\Vx,y)$。
现在,我们将最小化\firstgls{empirical_risk}:
\begin{equation}
\label{eq:8.3}
    \SetE_{\RVx, \RSy \sim \hat{p}_\text{data}} [L(f(\Vx ; \Vtheta), y)] % ?? \RVx
    = \frac{1}{m} \sum_{i=1}^m L( f(\Vx^{(i)}; \Vtheta), y^{(i)}) ,
\end{equation}
其中$m$表示训练样本的数目。
% 269 mid


基于最小化这种平均\gls{training_error}的训练过程被称为\firstgls{empirical_risk_minimization}。
在这种情况下,\gls{ML}仍然和传统的直接优化很相似。
我们并不直接最优化\gls{risk},而是最优化\gls{empirical_risk},希望也能够很大地降低\gls{risk}。
一系列不同的理论构造了一些条件,使得在这些条件下真实\gls{risk}的期望可以下降不同的量。
% 269 mid


然而,\gls{empirical_risk_minimization}很容易导致\gls{overfitting}。
高\gls{capacity}的模型会简单地记住\gls{training_set}。
在很多情况下,\gls{empirical_risk_minimization}并非真的可行。
最有效的现代优化算法是基于\gls{GD}的,但是很多有用的\gls{loss_function},如$0-1$损失,没有有效的导数(导数要么为零,要么处处未定义)。
这两个问题说明,在\gls{DL}中我们很少使用\gls{empirical_risk_minimization}。
反之,我们会使用一个稍有不同的方法,我们真正优化的目标会更加不同于我们希望优化的目标。
% 269 end


\subsection{\glsentrytext{surrogate_loss_function}和\glsentrytext{early_stopping}}
\label{sec:surrogate_loss_functions_and_early_stopping}
% 269 end
有时,我们真正关心的\gls{loss_function}(比如分类误差)并不能被高效地优化。
例如,即使对于\gls{linear_classifier}而言,精确地最小化$0-1$损失通常是不可解的(复杂度是输入维数的指数级别)\citep{Marcotte-92}。
在这种情况下,我们通常会优化\firstgls{surrogate_loss_function}。
\gls{surrogate_loss_function}作为原目标的代理,还具备一些优点。  
例如,正确类别的负对数似然通常用作$0-1$损失的替代。
负对数似然允许模型估计给定样本的类别的条件概率,如果该模型效果好,那么它能够输出期望最小分类误差所对应的类别。
% 270 head


在某些情况下,\gls{surrogate_loss_function}比原函数学到的更多。
例如,使用对数似然替代函数时,在\gls{training_set}上的$0-1$损失达到$0$之后,\gls{test_set}上的$0-1$损失还能持续下降很长一段时间。
这是因为即使$0-1$损失期望是零时,我们还能拉开不同类别的距离以改进分类器的鲁棒性,获得一个更强壮的、更值得信赖的分类器,从而,相对于简单地最小化\gls{training_set}上的平均$0-1$损失,它能够从训练数据中抽取更多信息。
% 270 head


一般的优化和我们用于训练算法的优化有一个重要不同:训练算法通常不会停止在\gls{local_minimum}。
反之,\gls{ML}通常最小化\gls{surrogate_loss_function},但是在基于\gls{early_stopping}(\secref{sec:early_stopping})的收敛条件满足时停止。
通常,\gls{early_stopping}使用真实\gls{underlying}\gls{loss_function},如\gls{validation_set}上的$0-1$损失,并设计为在\gls{overfitting}发生之前终止。
与纯优化不同的是,\gls{early_stopping}时\gls{surrogate_loss_function}仍然有较大的导数,而纯优化终止时过程已经收敛,导数较小。
% 270 mid


\subsection{\gls{batch}算法和\gls{minibatch}算法}
\label{sec:batch_and_minibatch_algorithms}
\gls{ML}算法和一般优化算法不同的一点是,\gls{ML}算法的\gls{objective_function}通常可以分解为训练样本上的求和。
%\gls{ML}优化算法通常使用整个\gls{cost_function}中的一部分项去更新其参数。
\gls{ML}中的优化算法在计算参数的每一次更新时通常仅使用整个\gls{cost_function}中一部分项来估计\gls{cost_function}的期望值。
% 270 mid

例如,\gls{maximum_likelihood_estimation}问题可以在对数空间中分解成各个样本的总和:
\begin{equation}
\label{eq:8.4}
    \Vtheta_{\text{ML}} = \underset{\Vtheta}{\argmax} \sum_{i=1}^m
    \log p_{\text{model}} (\Vx^{(i)}, y^{(i)}; \Vtheta) .
\end{equation}
% 270 mid


最大化这个总和等价于最大化\gls{training_set}在\gls{empirical_distribution}上的期望:
\begin{equation}
\label{eq:8.5}
    J(\Vtheta) = \SetE_{\RVx, \RSy \sim\hat{p}_\text{data}} 
    \log p_{\text{model}} (\Vx,y ; \Vtheta) .
\end{equation}
%  270 end


优化算法用到的\gls{objective_function}$J$中的大多数属性也是\gls{training_set}上的期望。
例如,最常用的属性是梯度:
\begin{equation}
\label{eq:8.6}
    \nabla_{\Vtheta} J(\Vtheta) = \SetE_{\RVx, \RSy \sim\hat{p}_{\text{data}}} 
    \nabla_{\Vtheta} \log p_{\text{model}} (\Vx,y; \Vtheta) .
\end{equation}
%  271 head


准确计算这个期望的计算代价非常大,因为我们需要在整个数据集上的每个样本上评估模型。
在实践中,我们可以从数据集中随机采样少量的样本,然后计算这些样本上的平均值。
%  271 head


回想一下,$n$个样本均值的\gls{standard_error}(\eqnref{eq:5.46})是$\sigma/\sqrt{n}$,其中$\sigma$是样本值真实的\gls{standard_deviation}。
分母$\sqrt{n}$表明使用更多样本来估计梯度的方法的回报是低于线性的。
比较两个假想的梯度计算,一个基于$100$个样本,另一个基于$10,000$个样本。
后者需要的计算量是前者的$100$倍,但却只降低了$10$倍的均值\gls{standard_error}。
如果能够快速地计算出梯度估计值,而不是缓慢地计算准确值,那么大多数优化算法会收敛地更快(就总的计算量而言,而不是指更新次数)。
%  271 mid


另一个促使我们从小数目样本中获得梯度的统计估计的动机是\gls{training_set}的冗余。
在最坏的情况下,\gls{training_set}中所有的$m$个样本都是彼此相同的拷贝。
基于采样的梯度估计可以使用单个样本计算出正确的梯度,而比原来的做法少花了$m$倍时间。
实践中,我们不太可能真的遇到这种最坏情况,但我们可能会发现大量样本都对梯度做出了非常相似的贡献。

使用整个\gls{training_set}的优化算法被称为\firstgls{batch}或\firstgls{deterministic}梯度算法,因为它们会在一个大\gls{batch}中同时处理所有样本。
这个术语可能有点令人困惑,因为这个词``\gls{batch}''也经常被用来描述\gls{minibatch}\gls{SGD}算法中用到的\gls{minibatch}样本。
通常,术语``\gls{batch}\gls{GD}''指使用全部\gls{training_set},而术语``\gls{batch}''单独出现时指一组样本。
例如,我们普遍使用术语``\gls{batch}大小''表示\gls{minibatch}的大小。
%  271 mid


每次只使用单个样本的优化算法有时被称为\firstgls{stochastic}或者\firstgls{online}算法。
术语``\gls{online}''通常是指从连续产生样本的数据流中抽取样本的情况,而不是从一个固定大小的\gls{training_set}中遍历多次采样的情况。
%  271 end


% 272 head
大多数用于\gls{DL}的算法介于以上两者之间,使用一个以上,而又不是全部的训练样本。
传统上,这些会被称为\firstgls{minibatch}或\firstgls{minibatch_stochastic}方法 ,现在通常将它们简单地称为\firstgls{stochastic}方法。


随机方法的典型示例是\gls{SGD},这将在\secref{sec:stochastic_gradient_descent_chap8}中详细描述。
% 272 mid

\gls{minibatch}的大小通常由以下几个因素决定:
\begin{itemize}
    \item 更大的\gls{batch}会计算更精确的梯度估计,但是回报却是小于线性的。
    
    \item 极小\gls{batch}通常难以充分利用多核架构。
    这促使我们使用一些绝对最小\gls{batch},低于这个值的小\gls{batch}处理不会减少计算时间。
    
    \item 如果\gls{batch}处理中的所有样本可以并行地处理(通常确是如此),那么内存消耗和\gls{batch}大小会正比。
    对于很多硬件设施,这是\gls{batch}大小的限制因素。
    
    \item 在某些硬件上使用特定大小的数组时,运行时间会更少。
    尤其是在使用\,\glssymbol{GPU}\,时,通常使用$2$的幂数作为\gls{batch}大小可以获得更少的运行时间。
    一般,$2$的幂数的取值范围是$32$到$256$,$16$有时在尝试大模型时使用。
% 272 mid    
    \item 
    可能是由于小\gls{batch}在学习过程中加入了\gls{noise},它们会有一些\gls{regularize}效果~\citep{Wilson-2003}。
    \gls{generalization_error}通常在\gls{batch}大小为$1$时最好。
    因为梯度估计的高方差,小\gls{batch}训练需要较小的\gls{learning_rate}以保持稳定性。
    因为降低的\gls{learning_rate}和消耗更多步骤来遍历整个\gls{training_set}都会产生更多的步骤,所以会导致总的运行时间非常大。
\end{itemize}
% 272 mid


不同的算法使用不同的方法从\gls{minibatch}中获取不同的信息。
有些算法对采样误差比其他算法更敏感,这通常有两个可能原因。
一个是它们使用了很难在少量样本上精确估计的信息,另一个是它们以放大采样误差的方式使用了信息。
仅基于梯度$\Vg$的更新方法通常相对鲁棒,并能使用较小的\gls{batch}获得成功,如$100$。
使用\gls{hessian}矩阵$\MH$,计算如$\MH^{-1}\Vg$更新的二阶方法通常需要更大的\gls{batch},如$10,000$。
这些大\gls{batch}需要最小化估计$\MH^{-1}\Vg$的波动。
假设$\MH$被精确估计,但是有\gls{poor_conditioning}数。
乘以$\MH$或是其逆会放大之前存在的误差(这个示例中是指$\Vg$的估计误差)。
即使$\MH$被精确估计,$\Vg$中非常小的变化也会导致更新值$\MH^{-1}\Vg$中非常大的变化。
当然,我们通常只会近似地估计$\MH$,因此相对于我们使用具有较差条件数的操作去估计$\Vg$,更新$\MH^{-1}\Vg$会含有更多的误差。
% 273 head

% 273 head
\gls{minibatch}是随机抽取的这点也很重要。
从一组样本中计算出梯度期望的\gls{unbiased}估计要求这些样本是独立的。
我们也希望两个连续的梯度估计是互相独立的,因此两个连续的\gls{minibatch}样本也应该是彼此独立的。
很多现实的数据集自然排列,从而使得连续的样本之间具有高度相关性。
例如,假设我们有一个很长的血液样本测试结果清单。
清单上的数据有可能是这样获取的,头五个血液样本于不同时间段取自第一个病人,接下来三个血液样本取自第二个病人,再随后的血液样本取自第三个病人,等等。
如果我们从这个清单上顺序抽取样本,那么我们的每个\gls{minibatch}数据的偏差都很大,因为这个\gls{minibatch}很可能只代表着数据集上众多患者中的某一个患者。
在这种数据集中的顺序有很大影响的情况下,很有必要在抽取\gls{minibatch}样本前打乱样本顺序。
对于非常大的数据集,如数据中心含有几十亿样本的数据集,我们每次构建\gls{minibatch}样本时都将样本完全均匀地抽取出来是不太现实的。
幸运的是,实践中通常将样本顺序打乱一次,然后按照这个顺序存储起来就足够了。
之后训练模型时会用到的一组组\gls{minibatch}连续样本是固定的,每个独立的模型每次遍历训练数据时都会重复使用这个顺序。
然而,这种偏离真实随机采样的方法并没有很严重的有害影响。
不以某种方式打乱样本顺序才会极大地降低算法的性能。
% 273 mid


很多\gls{ML}上的优化问题都可以分解成并行地计算不同样本上单独的更新。
换言之,我们在计算\gls{minibatch}样本$\MX$上最小化$J(\MX)$的更新时,同时可以计算其他\gls{minibatch}样本上的更新。
这类\gls{asynchronous}并行分布式方法将在\secref{sec:large_scale_distributed_implementations}中进一步讨论。
% 273 end


\gls{minibatch}\gls{SGD}的一个有趣动机是,只要没有重复使用样本,
它将遵循着真实\emph{\gls{generalization_error}}(\eqnref{eq:8.2})的梯度。
很多\gls{minibatch}\gls{SGD}方法的实现都会打乱数据顺序一次,然后多次遍历数据来更新参数。
第一次遍历时,每个\gls{minibatch}样本都用来计算真实\gls{generalization_error}的\gls{unbiased}估计。
第二次遍历时,估计将会是\gls{biased}的,因为它重新抽取了已经用过的样本,而不是从和原先样本相同的\gls{DGD}中获取新的无偏的样本。
%  274 head


从\gls{online_learning}的情况中我们最容易看出\gls{SGD}最小化的是真实的\gls{generalization_error}的原因。
这时样本或者说(单个样本的)\gls{minibatch}都是从数据\firstgls{stream}中抽取出来的。
换言之,\gls{learner}好像是一个每次看到新样本的人,
每个样本$(\Vx,y)$都来自\gls{DGD}~$p_{\text{data}}(\Vx,y)$,而不是使用大小固定的\gls{training_set}。
这种情况下,样本永远不会重复;每次更新的样本是从分布$p_\text{data}$中采样获得的无偏样本。
%  274 mid


在$\Vx$和$y$是离散时,以上的等价性最容易被看出来。
在这种情况下,\gls{generalization_error}(\eqnref{eq:8.2})可以表示为
\begin{equation}
\label{eq:8.7}
    J^*(\Vtheta) = \sum_{\Vx} \sum_y p_{\text{data}}(\Vx, y) L(f(\Vx; \Vtheta),y),
\end{equation}
上式的准确梯度为
\begin{equation}
\label{eq:8.8}
    \Vg = \nabla_{\Vtheta} J^*(\Vtheta) = \sum_{\Vx} \sum_y p_{\text{data}}
    (\Vx, y) \nabla_{\Vtheta} L(f(\Vx;\Vtheta),y) .
\end{equation}
在\eqnref{eq:8.5}和\eqnref{eq:8.6}中,我们已经在对数似然中看到了同样的结果;现在我们发现这一点在似然之外的其他函数$L$上也是成立的。
在一些关于$p_\text{data}$和$L$的温和假设下,在$\Vx$和$y$是连续时也能得到类似的结果。
%  274 mid


因此,我们可以从\gls{DGD}~$p_\text{data}$抽取\gls{minibatch}样本$\{ \Vx^{(1)}, \dots,\Vx^{(m)} \}$以及对应的目标$y^{(i)}$,然后计算该\gls{minibatch}上损失函数关于对应参数的梯度
\begin{equation}
\label{eq:8.9}
    \hat{\Vg} = \frac{1}{m} \nabla_{\Vtheta} \sum_i L(f(\Vx^{(i)};\Vtheta),y^{(i)} ).
\end{equation}
以此获得\gls{generalization_error}准确梯度的\gls{unbiased}估计。
最后,在方向$\hat{\Vg}$上更新$\Vtheta$也就意味着以(真实)\gls{generalization_error}为目标进行\,\glssymbol{SGD}\,。
%  274 end


当然,这个解释只能用于样本没有重复使用的情况。
然而,除非\gls{training_set}特别大,通常最好是多次遍历\gls{training_set}。
当多次遍历数据集更新时,只有第一遍满足\gls{generalization_error}梯度的\gls{unbiased}估计。
但是,额外的遍历更新当然会由于减小\gls{training_error}而得到足够的好处,以抵消其带来的\gls{training_error}和\gls{test_error}间差距的增加。
% -- 275 head


随着数据集的规模迅速增长,超越了计算能力的增速,
\gls{ML}应用每个样本只使用一次的情况变得越来越常见,甚至是不完整地使用\gls{training_set}。
在使用一个非常大的\gls{training_set}时,\gls{overfitting}不再是问题,而\gls{underfitting}和计算效率变成了主要的顾虑。
读者也可以参考~\cite{bottou-bousquet-2008}中关于训练样本数目增长时,\gls{generalization_error}上计算瓶颈影响的讨论。
% -- 275 mid


\section{\glsentrytext{NN}优化中的挑战}
\label{sec:challenges_in_neural_network_optimization}
优化通常是一个极其困难的任务。
传统的\gls{ML}会小心设计\gls{objective_function}和约束,以确保优化问题是凸的,
从而避免一般优化问题的复杂度。
在训练\gls{NN}时,我们肯定会遇到一般的\gls{nonconvex}情况。
即使是\gls{convex_optimization},也并非没有任何问题。
在这一节中,我们会总结几个训练\gls{deep_model}时会涉及到的主要挑战。
% -- 275 mid


\subsection{\glsentrytext{ill_conditioning}}
\label{sec:ill_conditioning}
在优化凸函数时,会遇到一些挑战。
这其中最突出的是\,\gls{hessian}\,矩阵$\MH$的\gls{ill_conditioning}。
这是\gls{nume_optimization}、\gls{convex_optimization}或其他形式的优化中普遍存在的问题,更多细节请回顾\secref{sec:beyond_the_gradient_jacobian_and_hessian_matrices}。
% -- 275 mid

\gls{ill_conditioning}问题一般被认为存在于\gls{NN}训练过程中。
\gls{ill_conditioning}体现在\gls{SGD}会``卡''在某些地方,此时即使很小的更新步长也会增加\gls{cost_function}。
% -- 275 mid

回顾\eqnref{eq:4.9},\gls{cost_function}的二阶\gls{taylor}级数展开预测\gls{GD}中的$-\epsilon\Vg$步骤会增加
\begin{equation}
    \frac{1}{2} \epsilon^2 \Vg^\top \MH\Vg - \epsilon\Vg^\top\Vg
\end{equation}
到代价中。
当$\frac{1}{2} \epsilon^2 \Vg^\top\MH\Vg$超过$\epsilon\Vg^\top\Vg$时,梯度的\gls{ill_conditioning}会成为问题。
我们可以通过监测平方梯度范数$\Vg^\top\Vg$和$\Vg^\top \MH\Vg$,来判断\gls{ill_conditioning}是否不利于\gls{NN}训练任务。
在很多情况中,梯度范数不会在训练过程中显著缩小,但是$\Vg^\top\MH\Vg$的增长会超过一个数量级。
其结果是尽管梯度很强,学习会变得非常缓慢,因为\gls{learning_rate}必须收缩以弥补更强的\gls{curvature}。
如\figref{fig:chap8_grad_norm_increases}所示,成功训练的\gls{NN}中,梯度显著增加。
% -- 276 head

% -- 276 end
\begin{figure}[!htb]
\ifOpenSource
\centerline{\includegraphics{figure.pdf}}
\else
\centerline{\includegraphics{Chapter8/figures/grad_norm_increases_color}}
\fi
\caption{\gls{GD}通常不会到达任何类型的\gls{critical_points}。
此示例中,在用于对象检测的\gls{convolutional_network}的整个训练期间, 梯度范数持续增加。
\emph{(左)}各个梯度计算的范数如何随时间分布的散点图。
为了方便作图,每\gls{epoch}仅绘制一个梯度范数。 
我们将所有梯度范数的移动平均绘制为实曲线。
梯度范数明显随时间增加,而不是如我们所期望的那样随训练过程收敛到\gls{critical_points}而减小。
\emph{(右)}尽管梯度递增,训练过程却相当成功。 
\gls{validation_set}上的分类误差可以降低到较低水平。
}
\label{fig:chap8_grad_norm_increases}
\end{figure}
% -- 276 end


% -- 276 mid
尽管\gls{ill_conditioning}还存在于除了\gls{NN}训练的其他情况中,有些适用于其他情况的解决\gls{ill_conditioning}的技术并不适用于\gls{NN}。
例如,\gls{newton_method}在解决带有\gls{poor_conditioning}的\,\gls{hessian}\,矩阵的\gls{convex_optimization}问题时,是一个非常优秀的工具,
但是我们将会在以下小节中说明\gls{newton_method}运用到\gls{NN}时需要很大的改动。


% -- 276 mid
\subsection{\glsentrytext{local_minima}}
\label{sec:local_minima}
\gls{convex_optimization}问题的一个突出特点是其可以简化为寻找一个\gls{local_minimum}的问题。
任何一个\gls{local_minimum}都是\gls{global_minimum}。
有些凸函数的底部是一个平坦的区域,而不是单一的\gls{global_minimum},但该平坦区域中的任意点都是一个可以接受的解。
优化一个凸问题时,若发现了任何形式的\gls{critical_points},我们都会知道已经找到了一个不错的可行解。
% -- 277 head


对于\gls{nonconvex}函数时,如\gls{NN},有可能会存在多个\gls{local_minima}。
事实上,几乎所有的\gls{deep_model}基本上都会有非常多的\gls{local_minima}。
然而,我们会发现这并不是主要问题。
% -- 277 mid


由于\firstgls{model_identifiability}问题,\gls{NN}和任意具有多个等效参数化\gls{latent_variable}的模型都会具有多个\gls{local_minima}。
如果一个足够大的\gls{training_set}可以唯一确定一组模型参数,那么该模型被称为\gls{identifiable}。
带有\gls{latent_variable}的模型通常是不\gls{identifiable},因为通过相互交换\gls{latent_variable}我们能得到等价的模型。
例如,考虑\gls{NN}的第一层,我们可以交换单元$i$和单元$j$的传入权重向量、传出权重向量而得到等价的模型。
如果\gls{NN}有$m$层,每层有$n$个单元,那么会有$n!^m$种排列\gls{hidden_unit}的方式。
这种不可辨认性被称为\firstgls{weight_space_symmetry}。
% -- 277 mid


除了\gls{weight_space_symmetry},很多\gls{NN}还有其他导致不可辨认的原因。
例如,在任意\gls{rectified_linear}网络或者~\gls{maxout}~网络中,
我们可以将传入权重和\gls{bias_aff}放缩$\alpha$倍,然后将传出权重放缩$\frac{1}{\alpha}$倍,而保持模型等价。
这意味着,如果\gls{cost_function}不包括如\gls{weight_decay}这种直接依赖于权重而非模型输出的项,那么\gls{rectified_linear}网络或者~\gls{maxout}~网络的每一个\gls{local_minimum}都在等价的\gls{local_minima}的$(m\times n)$维双曲线上。
% -- 277 mid


这些\gls{model_identifiability}问题意味着\gls{NN}\gls{cost_function}具有非常多、甚至不可数无限多的\gls{local_minima}。
然而,所有这些由于不可辨识性问题而产生的\gls{local_minima}都有相同的\gls{cost_function}值。
因此,这些\gls{local_minima}并非是\gls{nonconvex}所带来的问题。
% -- 277 mid


如果\gls{local_minima}相比\gls{global_minimum}拥有很大的\gls{cost},\gls{local_minima}会带来很大的隐患。
我们可以构建没有\gls{hidden_unit}的小规模\gls{NN},其\gls{local_minima}的\gls{cost}比\gls{global_minimum}的\gls{cost}大很多~\citep{Sontag-cs89,Brady89,Gori-pami91}。
如果具有很大\gls{cost}的\gls{local_minima}是常见的,那么这将给基于梯度的优化算法带来极大的问题。
% -- 277 end


对于实际中感兴趣的网络,是否存在大量\gls{cost}很高的\gls{local_minima},优化算法是否会碰到这些\gls{local_minima},都是尚未解决的公开问题。
多年来,大多数从业者认为\gls{local_minima}是困扰\gls{NN}优化的常见问题。
如今,情况有所变化。
这个问题仍然是学术界的热点问题,但是学者们现在猜想,对于足够大的\gls{NN}而言,
大部分\gls{local_minima}都具有很小的\gls{cost_function},我们能不能找到真正的\gls{global_minimum}并不重要,而是需要在参数空间中找到一个\gls{cost}很小(但不是最小)的点~\citep{Saxe-et-al-ICLR13,Dauphin-et-al-NIPS2014-small,GoodfellowOptimization15,Choromanska-et-al-AISTATS2015}。
% 278 head

很多从业者将\gls{NN}优化中的所有困难都归结于\gls{local_minima}。
我们鼓励从业者要仔细分析特定的问题。
一种能够排除\gls{local_minima}是主要问题的检测方法是画出梯度范数随时间的变化。
如果梯度范数没有缩小到一个微小的值,那么该问题既不是\gls{local_minima},也不是其他形式的\gls{critical_points}。
%这种消极的测试可以排除局部极小值是造成问题的原因。
在高维空间中,很难明确证明\gls{local_minima}是导致问题的原因。
许多并非\gls{local_minima}的结构也具有很小的梯度。
% 278 mid

\subsection{高原、\glsentrytext{saddle_points}和其他平坦区域}
\label{sec:plateaus_saddle_points_and_other_flat_regions}
对于很多高维\gls{nonconvex}函数而言,\gls{local_minima}(以及\gls{maxima})事实上都远少于另一类梯度为零的点:\gls{saddle_points}。
\gls{saddle_points}附近的某些点比\gls{saddle_points}有更大的\gls{cost},而其他点则有更小的\gls{cost}。
在\gls{saddle_points}处,\gls{hessian}\,矩阵同时具有正负特征值。
位于正特征值对应的特征向量方向的点比\gls{saddle_points}有更大的\gls{cost},反之,位于负特征值对应的特征向量方向的点有更小的\gls{cost}。
我们可以将\gls{saddle_points}视为\gls{cost_function}某个横截面上的\gls{local_minimum},同时也可以视为\gls{cost_function}某个横截面上的\gls{local_maximum}。
\figref{fig:chap4_saddle_3d_color}\,给了一个示例。
% 278 mid


多类随机函数表现出以下性质:低维空间中,\gls{local_minima}很普遍。
在更高维空间中,\gls{local_minima}很罕见,而\gls{saddle_points}则很常见。
对于这类函数$f:\SetR^n \to \SetR$而言,\gls{saddle_points}和\gls{local_minima}的数目比率的期望随$n$指数级增长。
我们可以从直觉上理解这种现象——\gls{hessian}\,矩阵在\gls{local_minimum}处只有正特征值。
而在\gls{saddle_points}处,\gls{hessian}\,矩阵则同时具有正负特征值。
试想一下,每个特征值的正负号由抛硬币决定。
在一维情况下,很容易抛硬币得到正面朝上一次而获取\gls{local_minimum}。
在$n$-维空间中,要抛掷$n$次硬币都正面朝上的难度是指数级的。 
具体可以参考~\cite{Dauphin-et-al-NIPS2014-small},它回顾了相关的理论工作。
% -- 278 end


很多随机函数一个惊人性质是,当我们到达\gls{cost}较低的区间时,\gls{hessian}\,矩阵的特征值为正的可能性更大。
和抛硬币类比,这意味着如果我们处于低\gls{cost}的\gls{critical_points}时,抛掷硬币正面朝上$n$次的概率更大。
这也意味着,\gls{local_minima}具有低\gls{cost}的可能性比高\gls{cost}要大得多。
具有高\gls{cost}的\gls{critical_points}更有可能是\gls{saddle_points}。
具有极高\gls{cost}的\gls{critical_points}就很可能是\gls{local_maxima}了。
% -- 279 head


以上现象出现在许多种类的随机函数中。
那么是否在\gls{NN}中也有发生呢?
\cite{Baldi89}从理论上证明,不具非线性的浅层\gls{AE}(\chapref{chap:autoencoders}中将介绍的一种将输出训练为输入拷贝的\gls{feedforward_network})只有\gls{global_minima}和\gls{saddle_points},没有\gls{cost}比\gls{global_minima}更大的\gls{local_minima}。
他们还发现这些结果能够扩展到不具非线性的更深的网络上,不过没有证明。
这类网络的输出是其输入的线性函数,但它们仍然有助于分析非线性\gls{NN}模型,因为它们的\gls{loss_function}是关于参数的\gls{nonconvex}函数。
这类网络本质上是多个矩阵组合在一起。
\cite{Saxe-et-al-ICLR13}精确解析了这类网络中完整的学习动态,表明这些模型的学习能够捕捉到许多在训练具有非线性\gls{activation_function}的\gls{deep_model}时观察到的定性特征。
\cite{Dauphin-et-al-NIPS2014-small}通过实验表明,真实的\gls{NN}也存在包含很多高\gls{cost}\gls{saddle_points}的\gls{loss_function}。
\cite{Choromanska-et-al-AISTATS2015}提供了额外的理论论点,表明另一类和\gls{NN}相关的高维随机函数也满足这种情况。

% -- 279 mid

\gls{saddle_points}激增对于训练算法来说有哪些影响呢?
对于只使用梯度信息的一阶优化算法而言,目前情况还不清楚。
\gls{saddle_points}附近的梯度通常会非常小。
另一方面,实验中\gls{GD}似乎可以在许多情况下逃离\gls{saddle_points}。
\cite{GoodfellowOptimization15}可视化了最新\gls{NN}的几个学习轨迹,\figref{fig:chap8_plot_atmu_relu_5}\,给了一个例子。
这些可视化显示,在突出的\gls{saddle_points}附近,\gls{cost_function}都是平坦的,权重都为零。
但是他们也展示了\gls{GD}轨迹能够迅速逸出该区间。
\cite{GoodfellowOptimization15}也主张,应该可以通过分析来表明连续时间的\gls{GD}会逃离而不是吸引到\gls{saddle_points},但对\gls{GD}更现实的使用场景来说,情况或许会有所不同。

% -- 279 mid

% 280 head

\begin{figure}[!htb]
\ifOpenSource
\centerline{\includegraphics{figure.pdf}}
\else
\centerline{\includegraphics{Chapter8/figures/plot_atmu_relu_5}}
\fi
\caption{\gls{NN}\gls{cost_function}的可视化。
这些可视化对应用于真实\gls{object_recognition}和\gls{NLP}任务的\gls{feedforward_neural_network}、\gls{convolutional_network}和\gls{recurrent_network}而言是类似的。
令人惊讶的是,这些可视化通常不会显示出很多明显的障碍。
大约2012年,在\gls{SGD}开始成功训练非常大的模型之前,相比这些投影所显示的\gls{NN}\gls{cost_function}的表面通常被认为有更多的\gls{nonconvex}结构。
该投影所显示的主要障碍是初始参数附近的高\gls{cost}\gls{saddle_points},但如由蓝色路径所示,\glssymbol{SGD}\,训练轨迹能轻易地逃脱该鞍点。
大多数训练时间花费在横穿\gls{cost_function}中相对平坦的峡谷,可能由于梯度中的高噪声、或该区域中~\gls{hessian}~矩阵的\gls{poor_conditioning},或者需要经过间接的弧路径绕过图中可见的高``山'' 。
图经~\citet{GoodfellowOptimization15}许可改编。
}
\label{fig:chap8_plot_atmu_relu_5}
\end{figure}
% 280 head


% -- 279 end
对于\gls{newton_method}而言,\gls{saddle_points}显然是一个问题。
\gls{GD}旨在朝``下坡''移动,而非明确寻求\gls{critical_points}。
而\gls{newton_method}的目标是寻求梯度为零的点。
如果没有适当的修改,\gls{newton_method}就会跳进一个\gls{saddle_points}。
高维空间中\gls{saddle_points}的激增或许解释了在\gls{NN}训练中为什么\gls{second_order_method}无法成功取代\gls{GD}。
\cite{Dauphin-et-al-NIPS2014-small}介绍了二阶优化的\firstgls{saddle_free_newton_method},并表明和传统版本的牛顿法相比有显著改进。
\gls{second_order_method}目前仍然难以扩展到大型\gls{NN},但是如果这类无鞍算法能够扩展的话,还是很有希望的。
% -- 280 mid


除了\gls{minima}和\gls{saddle_points},还存在其他梯度为零的点。
例如\gls{maxima},它的情况从优化的角度看与\gls{saddle_points}很类似,很多算法不会被吸引到\gls{maxima},除了未经修改的\gls{newton_method}。
和\gls{minima}一样,许多种类的随机函数的\gls{maxima}在高维空间中也是指数级稀少。
% -- 280 mid


也可能存在恒值的、宽且平坦的区域。
在这些区域,梯度和\,\gls{hessian}\,矩阵都是零。
这种退化的情形对所有\gls{nume_optimization}算法而言都是主要问题。
在凸问题中,一个宽而平坦的区间肯定包含\gls{global_minima},但是对于一般的优化问题而言,
这样的区域可能会对应着\gls{objective_function}中一个较高的值。
% -- 280 end


\subsection{悬崖和\glsentrytext{explode_gradient}}
\label{sec:cliffs_and_exploding_gradients}
% 281 head
多层\gls{NN}经常存在像悬崖一样的斜率较大区域,如\figref{fig:chap8_cliff}所示。
这是由于几个较大的权重相乘导致的。
遇到斜率极大的悬崖结构时,梯度更新会很大程度地改变参数值,通常会完全跳过这类悬崖结构。
% 281 head


% 281 end
\begin{figure}[!htb]
\ifOpenSource
\centerline{\includegraphics{figure.pdf}}
\else
\centerline{\includegraphics{Chapter8/figures/cliff_color}}
\fi
\caption{高度非线性的\gls{DNN}或\gls{RNN}的\gls{objective_function}通常包含由几个参数连乘而导致的参数空间中尖锐非线性。
这些非线性在某些区域会产生非常大的导数。
当参数接近这样的悬崖区域时,\gls{GD}更新可以使参数弹射得非常远,可能会使大量已完成的优化工作成为无用功。
图经 \citet{Pascanu+al-ICML2013-small}许可改编。}
\label{fig:chap8_cliff}
\end{figure}
% 281 end


不管我们是从上还是从下接近悬崖,情况都很糟糕,但幸运的是我们可以用使用\secref{sec:clipping_gradients}介绍的启发式\firstgls{gradient_clipping}来避免其严重的后果。
其基本想法源自梯度并没有指明最佳步长,只说明了在无限小区域内的最佳方向。
当传统的\gls{GD}算法提议更新很大一步时,启发式\gls{gradient_clipping}会干涉来减小步长,从而使其不太可能走出梯度近似为\gls{steepest}方向的悬崖区域。
悬崖结构在\gls{RNN}的\gls{cost_function}中很常见,因为这类模型会涉及到多个因子的相乘,其中每个因子对应一个\gls{time_step}。
因此,长期时间序列会产生大量相乘。
% 281 mid


% 281 end
\subsection{\glsentrytext{long_term_dependency}}
\label{sec:long_term_dependencies}
当\gls{computational_graph}变得极深时,\gls{NN}优化算法会面临的另外一个难题就是\gls{long_term_dependency}问题——由于变深的结构使模型丧失了学习到先前信息的能力,让优化变得极其困难。 
深层的\gls{computational_graph}不仅存在于\gls{feedforward_network},还存在于之后介绍的\gls{recurrent_network}中(在\chapref{chap:sequence_modeling_recurrent_and_recursive_nets}中描述)。
因为\gls{recurrent_network}要在很长时间序列的各个时刻重复应用相同操作来构建非常深的计算图,并且模型参数共享,这使问题更加凸显。

% 282 head

例如,假设某个\gls{computational_graph}中包含一条反复与矩阵$\MW$相乘的路径。
那么$t$步后,相当于乘以$\MW^t$。
假设$\MW$有特征值分解$\MW = \MV \text{diag}(\Vlambda) \MV^{-1}$。
在这种简单的情况下,很容易看出
\begin{equation}
  \MW^t = (\MV \text{diag}(\Vlambda) \MV^{-1})^t = \MV\text{diag}(\Vlambda)^t  \MV^{-1}.
\end{equation}
当特征值$\lambda_i$的绝对值不在$1$附近时,若绝对值大于$1$则会爆炸;若小于$1$时则会消失。
\firstgls{vanish_explode_gradient}是指该\gls{computational_graph}上的梯度也会因为$\text{diag}(\Vlambda)^t$大幅度变化。
\gls{vanish_gradient}使得我们难以知道参数朝哪个方向移动能够改进\gls{cost_function},而\gls{explode_gradient}会使得学习不稳定。
之前描述的促使我们使用\gls{gradient_clipping}的悬崖结构便是\gls{explode_gradient}现象的一个例子。

% 282 mid

此处描述的在各\gls{time_step}重复与$\MW$相乘非常类似于寻求矩阵$\MW$的最大特征值及对应特征向量的\firstgls{power_method}。
从这个观点来看,$\Vx^\top\MW^t$最终会丢弃$\Vx$中所有与$\MW$的主特征向量正交的成分。

% 282 mid

\gls{recurrent_network}在各\gls{time_step}上使用相同的矩阵$\MW$,而\gls{feedforward_network}并没有。
所以即使使用非常深层的\gls{feedforward_network},也能很大程度上有效地避免\gls{vanish_explode_gradient}~\citep{Sussillo14}。

% 282 mid

在更详细地描述\gls{recurrent_network}之后,我们将会在\secref{sec:the_challenge_of_long_term_dependencies}进一步讨论\gls{recurrent_network}训练中的挑战。

% 282 mid

\subsection{非精确梯度}
\label{sec:inexact_gradients}
大多数优化算法的先决条件都是我们知道精确的梯度或是\,\gls{hessian}\,矩阵。
在实践中,通常这些量会有\gls{noise},甚至是\gls{biased}的估计。
几乎每一个\gls{DL}算法都需要基于采样的估计,至少使用训练样本的\gls{minibatch}来计算梯度。

% 282 end

在其他情况,我们希望最小化的\gls{objective_function}实际上是难以处理的。
当\gls{objective_function}不可解时,通常其梯度也是难以处理的。
在这种情况下,我们只能近似梯度。
这些问题主要出现在第三部分中更高级的模型中。
例如,\gls{contrastive_divergence}是用来近似\gls{BM}中难以处理的对数似然梯度的一种技术。

% 283 head

各种\gls{NN}优化算法的设计都考虑到了梯度估计的缺陷。
我们可以选择比真实\gls{loss_function}更容易估计的\gls{surrogate_loss_function}来避免这个问题。

% 283 mid

\subsection{局部和全局结构间的弱对应}
\label{sec:poor_correspondence_between_local_and_global_structure}
迄今为止,我们讨论的许多问题都是关于\gls{loss_function}在单个点的性质——若$J(\Vtheta)$是当前点$\Vtheta$的\gls{poor_conditioning},或者$\Vtheta$在悬崖中,或者$\Vtheta$是一个下降方向不明显的\gls{saddle_points},那么会很难更新当前步。

% 283 mid

如果该方向在局部改进很大,但并没有指向\gls{cost}低得多的遥远区域,那么我们有可能在单点处克服以上所有困难,但仍然表现不佳。

% 283 mid

\cite{GoodfellowOptimization15}认为大部分训练的运行时间取决于到达解决方案的轨迹长度。 
如\figref{fig:chap8_plot_atmu_relu_5}所示,学习轨迹将花费大量的时间探寻一个围绕山形结构的宽弧。

% 283 mid

大多数优化研究的难点集中于训练是否找到了\gls{global_minimum}、\gls{local_minimum}或是\gls{saddle_points},但在实践中\gls{NN}不会到达任何一种\gls{critical_points}。
\figref{fig:chap8_grad_norm_increases}表明\gls{NN}通常不会到达梯度很小的区域。
甚至,这些\gls{critical_points}不一定存在。
例如,\gls{loss_function} $-\log p(y\mid\Vx;\Vtheta)$ 可以没有\gls{global_minimum},而是当随着训练模型逐渐稳定后,渐近地收敛于某个值。    
对于具有离散的$y$和~\ENNAME{softmax}~分布$p(y\mid\Vx)$的分类器而言,若模型能够正确分类\gls{training_set}上的每个样本,则负对数似然可以无限趋近但不会等于零。
同样地,实值模型$p(y\mid\Vx) = \mathcal{N}(y;f(\Vtheta),\beta^{-1})$的负对数似然会趋向于负无穷——如果$f(\Vtheta)$能够正确预测所有\gls{training_set}中的目标$y$,学习算法会无限制地增加$\beta$。
\figref{fig:chap8_bad_global}给出了一个失败的例子,即使没有\gls{local_minima}和\gls{saddle_points},该例还是不能从局部优化中找到一个良好的\gls{cost_function}值。

% 283 end

% 284 head

\begin{figure}[!htb]
\ifOpenSource
\centerline{\includegraphics{figure.pdf}}
\else
\centerline{\includegraphics{Chapter8/figures/bad_global_color}}
\fi
\caption{如果局部表面没有指向全局解,基于局部下坡移动的优化可能就会失败。
这里我们提供一个例子,说明即使在没有\gls{saddle_points}或\gls{local_minima}的情况下,优化过程会如何失败。
此例中的\gls{cost_function}仅包含朝向低值而不是\gls{minima}的渐近线。
在这种情况下,造成这种困难的主要原因是初始化在``山''的错误一侧,并且无法遍历。
在高维空间中,学习算法通常可以环绕过这样的高山,但是相关的轨迹可能会很长,并且导致过长的训练时间,如\figref{fig:chap8_plot_atmu_relu_5}所示。
}
\label{fig:chap8_bad_global}
\end{figure}

% 284 head

未来的研究需要进一步探索影响学习轨迹长度和更好地表征训练过程的结果。

% 284 mid

许多现有研究方法在求解具有困难全局结构的问题时,旨在寻求良好的初始点,而不是开发非局部范围更新的算法。

% 284 mid

\gls{GD}和基本上所有的可以有效训练\gls{NN}的学习算法,都是基于局部较小更新。
之前的小节主要集中于为何这些局部范围更新的正确方向难以计算。
我们也许能计算\gls{objective_function}的一些性质,如近似的有偏梯度或正确方向估计的方差。
在这些情况下,局部下降有可能定义了通向有效解的足够短的路径,也有可能没有,但真正糟糕的是我们并不能真的做到遵循局部下降的路径。
\gls{objective_function}可能有诸如\gls{poor_conditioning}或不连续梯度的问题,使得梯度为\gls{objective_function}提供较好近似的区间非常小。
在这些情况下,步长为$\epsilon$的局部下降可能定义了到达解的合理的短路经,但是我们只能计算步长为$\delta \ll \epsilon$的局部下降方向。
在这些情况下,局部下降或许能定义通向解的路径,但是该路径包含很多次更新,因此遵循该路径会带来很高的计算代价。
有时,比如说当\gls{objective_function}有一个宽而平的区域,或是我们试图寻求精确的\gls{critical_points}(通常来说后一种情况只发生于显式求解\gls{critical_points}的方法,如\gls{newton_method})时,局部信息不能为我们提供任何指导。
在这些情况下,\gls{local_descent}完全无法定义通向解的路径。
在其他情况下,局部移动可能太过贪心,朝着下坡方向移动,却和所有可行解南辕北辙,如\figref{fig:chap8_bad_global}所示,或者是用舍近求远的方法来求解问题,如\figref{fig:chap8_plot_atmu_relu_5}所示。
目前,我们还不了解这些问题中的哪一个与\gls{NN}优化中的难点最相关,这是研究领域的热点方向。

% 285 head

不管哪个问题最重要,如果存在一个区域,我们遵循\gls{local_descent}便能合理地直接到达某个解,并且我们能够在该良好区域上初始化学习,那么这些问题都可以避免。
最后的这个观点建议在传统优化算法上研究怎样选择更佳的初始化点,以此来实现目标更切实可行。

% 285 mid

\subsection{优化的理论限制}
\label{sec:theoretical_limits_of_optimization}
一些理论结果表明,我们为\gls{NN}设计的任何优化算法都有性能限制\citep{Blum92,JuddBook,wolpert96no}。
通常这些结果不影响\gls{NN}在实践中的应用。

% 285 mid

一些理论结果仅适用于\gls{NN}的单元输出离散值的情况。
然而,大多数\gls{NN}单元输出光滑的连续值,使得局部搜索求解优化可行。
一些理论结果表明,存在某类问题是不可解的,但很难判断一个特定问题是否属于该类。
其他结果表明,寻找给定规模的网络的一个可行解是很困难的,
但在实际情况中,我们通过设置更多参数,使用更大的网络,能轻松找到可接受的解。
此外,在\gls{NN}训练中,我们通常不关注某个函数的精确\gls{minimum},而只关注将其值下降到足够小以获得一个良好的\gls{generalization_error}。
对优化算法是否能完成此目标进行理论分析是非常困难的。
因此,研究优化算法更现实的性能上界仍然是学术界的一个重要目标。

% 285 end

% 286 head

\section{基本算法}
\label{sec:basic_algorithms}
之前我们已经介绍了\gls{GD}(\secref{sec:gradient_based_optimization}),即沿着整个\gls{training_set}的梯度方向下降。
这可以使用\gls{SGD}很大程度地加速,沿着随机挑选的\gls{minibatch}数据的梯度下降方向,就像\secref{sec:stochastic_gradient_descent_chap5}和\secref{sec:batch_and_minibatch_algorithms}中讨论的一样。

% 286 head

\subsection{\glsentrytext{SGD}}
\label{sec:stochastic_gradient_descent_chap8}
\glsacr{SGD}及其变种很可能是一般\gls{ML}中应用最多的优化算法,特别是在\gls{DL}中。
如\secref{sec:batch_and_minibatch_algorithms}中所讨论的,按照\gls{DGD}抽取$m$个\gls{minibatch}(独立同分布的)样本,通过计算它们梯度均值,我们可以得到梯度的\gls{unbiased}估计。

% 286 mid

\algref{alg:sgd}展示了如何沿着这个梯度的估计下降。

% 286 mid

% 286 end

\begin{algorithm}[ht]
\caption{\gls{SGD}(\glssymbol{SGD})在第$k$个训练迭代的更新}
\label{alg:sgd}
\begin{algorithmic}
\REQUIRE \gls{learning_rate} $\epsilon_k$
\REQUIRE 初始参数$\Vtheta$
\WHILE{停止\gls{criterion}未满足}
    \STATE 从\gls{training_set}中采包含$m$个样本$\{ \Vx^{(1)},\dots, \Vx^{(m)}\}$ 的\gls{minibatch},其中$\Vx^{(i)}$对应目标为$\Vy^{(i)}$。
    \STATE 计算梯度估计: $\hat{\Vg} \leftarrow + 
         \frac{1}{m} \nabla_{\Vtheta} \sum_i L(f(\Vx^{(i)};\Vtheta),\Vy^{(i)})$
    \STATE 应用更新:$\Vtheta \leftarrow \Vtheta - \epsilon \hat{\Vg}$
\ENDWHILE
\end{algorithmic}
\end{algorithm}

% 286 end

% 286 mid

\glssymbol{SGD}\,算法中的一个关键参数是\gls{learning_rate}。
之前,我们介绍的\,\glssymbol{SGD}\,使用固定的\gls{learning_rate}。
在实践中,有必要随着时间的推移逐渐降低\gls{learning_rate},因此我们将第$k$步迭代的\gls{learning_rate}记作$\epsilon_k$。

% 286 mid

这是因为\,\glssymbol{SGD}\,中梯度估计引入的噪声源($m$个训练样本的随机采样)并不会在\gls{minimum}处消失。
相比之下,当我们使用\gls{batch}\gls{GD}到达\gls{minimum}时,整个\gls{cost_function}的真实梯度会变得很小,之后为$\mathbf{0}$,因此\gls{batch}\gls{GD}可以使用固定的\gls{learning_rate}。
保证\,\glssymbol{SGD}\,收敛的一个充分条件是
\begin{equation}
\label{eq:8.12}
    \sum_{k=1}^\infty \epsilon_k = \infty,
\end{equation}
且
\begin{equation}
\label{eq:8.13}
    \sum_{k=1}^\infty \epsilon_k^2 < \infty.
\end{equation}

% 287 head

实践中,一般会线性衰减\gls{learning_rate}直到第$\tau$次迭代:
\begin{equation}
\label{eq:8.14}
    \epsilon_k = (1-\alpha) \epsilon_0 + \alpha \epsilon_\tau
\end{equation}
其中$\alpha = \frac{k}{\tau}$。
在$\tau$步迭代之后,一般使$\epsilon$保持常数。

% 287 mid

\gls{learning_rate}可通过试验和误差来选取,通常最好的选择方法是监测\gls{objective_function}值随时间变化的学习曲线。
与其说是科学,这更像是一门艺术,我们应该谨慎地参考关于这个问题的大部分指导。
使用线性策略时,需要选择的参数为$\epsilon_0$,$\epsilon_\tau$,$\tau$。  
通常$\tau$被设为需要反复遍历\gls{training_set}几百次的迭代次数。
通常$\epsilon_\tau$应设为大约$\epsilon_0$的$1\%$。
主要问题是如何设置$\epsilon_0$。
若$\epsilon_0$太大,学习曲线将会剧烈振荡,\gls{cost_function}值通常会明显增加。
温和的振荡是良好的,容易在训练随机\gls{cost_function}(例如使用\,\gls{dropout}\,的\gls{cost_function})时出现。
如果\gls{learning_rate}太小,那么学习过程会很缓慢。
如果初始\gls{learning_rate}太低,那么学习可能会卡在一个相当高的\gls{cost}值。
通常,就总训练时间和最终\gls{cost}值而言,最优的初始\gls{learning_rate}会高于在大约迭代$100$次左右后效果最佳的\gls{learning_rate}。
因此,通常最好是检测最早的几轮迭代,选择一个比在效果上表现最佳的\gls{learning_rate}更大的\gls{learning_rate},但又不能太大导致严重的震荡。  

% 287 mid

\glssymbol{SGD}\,及相关的\gls{minibatch}亦或更广义的基于梯度优化的在线学习算法,一个重要的性质是每一步更新的计算时间不依赖完整训练集样本数目的多寡。
即使训练集样本数目非常大时,它们也能收敛。
对于足够大的数据集,\glssymbol{SGD}\,可能会在处理完整个\gls{training_set}之前就收敛到最终\gls{test_set}误差的某个固定容差范围内。

% 287 mid

% 287 end

研究优化算法的收敛率,一般会衡量\firstgls{excess_error} $J(\Vtheta) - \min_{\Vtheta} J(\Vtheta)$,即当前\gls{cost_function}超出最低可能\gls{cost}的量。
\glssymbol{SGD}\,应用于凸问题时,$k$步迭代后的\gls{excess_error}量级是$O(\frac{1}{\sqrt{k}})$,在强凸情况下是$O(\frac{1}{k})$。
除非假定额外的条件,否则这些界限不能进一步改进。
\gls{batch}\gls{GD}在理论上比\gls{SGD}有更好的收敛率。
然而,Cram\'er-Rao界限~\citep{Cramer-1946,Rao-1945}指出,\gls{generalization_error}的下降速度不会快于$O(\frac{1}{k})$。
\cite{bottou-bousquet-2008-small}因此认为对于\gls{ML}任务,不值得探寻收敛快于$O(\frac{1}{k})$的优化算法——更快的收敛可能对应着\gls{overfitting}。
此外,渐近分析掩盖了\gls{SGD}在少量更新步之后的很多优点。
对于大数据集,\glssymbol{SGD}\,只需非常少量样本计算梯度从而实现初始快速更新,其好处超过了其缓慢的渐近收敛带来的坏处。
本章剩余部分介绍的大多数算法致力于追求实践中有价值的好处,而不在意它的学习率有常数级别的额外误差(在渐进分析上比不上$O(\frac{1}{k})$的额外误差)。
我们也可以在学习过程中逐渐增大\gls{minibatch}的大小,以此权衡\gls{batch}\gls{GD}和\gls{SGD}两者的优点。

% 288 mid

了解\,\glssymbol{SGD}\,更多的信息,请查看~\cite{Bottou98}。

% 288 mid

\subsection{\glsentrytext{momentum}}
\label{sec:momentum}
虽然\gls{SGD}仍然是非常受欢迎的优化方法,但其学习过程有时会很慢。
\gls{momentum}方法~\citep{polyak1964some}旨在加速学习,特别是处理高\gls{curvature}、小但一致的梯度,或是带\gls{noise}的梯度。
\gls{momentum}算法积累了之前梯度指数级衰减的移动平均,并且继续沿该方向移动。
\gls{momentum}的效果如\figref{fig:chap8_momentum}所示。

% 288 mid

% 289 head

\begin{figure}[!htb]
\ifOpenSource
\centerline{\includegraphics{figure.pdf}}
\else
\centerline{\includegraphics{Chapter8/figures/momentum_color}}
\fi
\caption{\gls{momentum}的主要目的是解决两个问题:\gls{hessian}\,矩阵的\gls{poor_conditioning}和随机梯度的方差。 
我们通过此图说明\gls{momentum}如何克服这两个问题的第一个。
等高线描绘了一个二次\gls{loss_function}(具有\gls{poor_conditioning}的\,\gls{hessian}\,矩阵)。
横跨轮廓的红色路径表示\gls{momentum}学习规则所遵循的路径,它使该函数最小化。
我们在该路径的每个步骤画一个箭头,表示\gls{GD}将在该点采取的步骤。
我们可以看到,一个\gls{poor_conditioning}的二次\gls{objective_function}看起来像一个长而窄的山谷或具有陡峭边的峡谷。
\gls{momentum}正确地纵向穿过峡谷,而普通的梯度步骤则会浪费时间在峡谷的窄轴上来回移动。
比较\figref{fig:chap4_poor_conditioning_color},它也显示了没有\gls{momentum}的\gls{GD}的行为。
}
\label{fig:chap8_momentum}
\end{figure}

% 289 head

从形式上看,\gls{momentum}算法引入了变量$\Vv$充当速度角色——它代表参数在参数空间移动的方向和速率。
速度被设为负梯度的指数衰减平均。
名称\firstgls{momentum}来自物理类比,根据牛顿运动定律,负梯度是移动参数空间中粒子的力。
\gls{momentum}在物理学上定义为质量乘以速度。
在\gls{momentum}学习算法中,我们假设是单位质量,因此速度向量$\Vv$也可以看作是粒子的\gls{momentum}。
\gls{hyperparameter}$\alpha\in[0,1)$决定了之前梯度的贡献衰减得有多快。
更新规则如下:
\begin{align}
\Vv & \leftarrow \alpha \Vv - \epsilon \nabla_{\Vtheta} \left( \frac{1}{m} \sum_{i=1}^m  L(\Vf(\Vx^{(i)}; \Vtheta), \Vy^{(i)}   )  \right), \\
\Vtheta & \leftarrow \Vtheta  + \Vv .
\end{align}
速度$\Vv$累积了梯度元素$\nabla_{\Vtheta}( \frac{1}{m} \sum_{i=1}^m L( \Vf(\Vx^{(i)}; \Vtheta), \Vy^{(i)} )  )$。
相对于$\epsilon$,$\alpha$越大,之前梯度对现在方向的影响也越大。
带\gls{momentum}的\,\glssymbol{SGD}\,算法如\algref{alg:momentum}所示。

% 289 end

% 289 mid

\begin{algorithm}[ht]
\caption{使用\gls{momentum}的\gls{SGD}(\glssymbol{SGD})}
\label{alg:momentum}
\begin{algorithmic}
\REQUIRE \gls{learning_rate} $\epsilon$, \gls{momentum}参数 $\alpha$
\REQUIRE 初始参数 $\Vtheta$,初始速度 $\Vv$
\WHILE{没有达到停止\gls{criterion}}
    \STATE 从\gls{training_set}中采包含$m$个样本$\{ \Vx^{(1)},\dots, \Vx^{(m)}\}$ 的\gls{minibatch},对应目标为$\Vy^{(i)}$。
    \STATE 计算梯度估计:$\Vg \leftarrow 
         \frac{1}{m} \nabla_{\Vtheta} \sum_i L(f(\Vx^{(i)};\Vtheta),\Vy^{(i)})$
    \STATE  计算速度更新:$\Vv \leftarrow \alpha \Vv - 
    \epsilon \Vg$
    \STATE 应用更新:$\Vtheta \leftarrow \Vtheta + \Vv$ 
\ENDWHILE
\end{algorithmic}
\end{algorithm}

% 289 mid

% 289 end

之前,步长只是梯度范数乘以\gls{learning_rate}。
现在,步长取决于梯度\emph{序列}的大小和排列。
当许多连续的梯度指向相同的方向时,步长最大。
如果\gls{momentum}算法总是观测到梯度$\Vg$,那么它会在方向$-g$上不停加速,直到达到最终速度,其中步长大小为
\begin{equation}
    \frac{\epsilon \norm{\Vg}}{1-\alpha} .
\end{equation}
因此将\gls{momentum}的\gls{hyperparameter}视为$\frac{1}{1-\alpha}$有助于理解。
例如,$\alpha=0.9$对应着最大速度$10$倍于\gls{GD}算法。

% 290 head

在实践中,$\alpha$的一般取值为$0.5$,$0.9$和$0.99$。
和\gls{learning_rate}一样,$\alpha$也会随着时间不断调整。 
一般初始值是一个较小的值,随后会慢慢变大。
随着时间推移调整$\alpha$没有收缩$\epsilon$重要。

% 290 mid

我们可以将\gls{momentum}算法视为模拟连续时间下牛顿动力学下的粒子。
这种物理类比有助于直觉上理解\gls{momentum}和\gls{GD}算法是如何表现的。

% 290 mid

粒子在任意时间点的位置由$\Vtheta(t)$给定。
粒子会受到净力$\Vf(t)$。
该力会导致粒子加速:
\begin{equation}
    \Vf(t) = \frac{\partial^2}{\partial t^2} \Vtheta(t) .
\end{equation}
与其将其视为位置的二阶\gls{differential_equation},我们不如引入表示粒子在时间$t$处速度的变量$\Vv(t)$,将牛顿动力学重写为一阶\gls{differential_equation}:
\begin{align}
    \Vv(t) &= \frac{\partial}{\partial t} \Vtheta(t) , \\
    \Vf(t) &= \frac{\partial}{\partial t} \Vv(t) .
\end{align}
由此,\gls{momentum}算法包括通过数值模拟求解\gls{differential_equation}。
求解\gls{differential_equation}的一个简单数值方法是欧拉方法,通过在每个梯度方向上小且有限的步来简单模拟该等式定义的动力学。
% 290 mid


这解释了\gls{momentum}更新的基本形式,但具体是哪些力呢?
一个力正比于\gls{cost_function}的负梯度$-\nabla_{\Vtheta} J(\Vtheta)$。
该力推动粒子沿着\gls{cost_function}表面下坡的方向移动。
\gls{GD}算法基于每个梯度简单地更新一步,而使用\gls{momentum}算法的牛顿方案则使用该力改变粒子的速度。
我们可以将粒子视作在冰面上滑行的冰球。
每当它沿着表面最陡的部分下降时,它会沿该方向加速滑行,直到开始向上滑动为止。
% 291 head


另一个力也是必要的。
如果\gls{cost_function}的梯度是唯一的力,那么粒子可能永远不会停下来。
想象一下,假设理想情况下冰面没有摩擦,一个冰球从山谷的一端下滑,上升到另一端,永远来回振荡。
要解决这个问题,我们添加另一个正比于$-\Vv(t)$的力。
在物理术语中,此力对应于粘性阻力,就像粒子必须通过一个抵抗介质,如糖浆。
这会导致粒子随着时间推移逐渐失去能量,最终收敛到\gls{local_minimum}。
% 291 mid


为什么要特别使用$-\Vv(t)$和粘性阻力呢?
部分原因是因为$-\Vv(t)$在数学上的便利——速度的整数幂很容易处理。
然而,其他物理系统具有基于速度的其他整数幂的其他类型的阻力。
例如,颗粒通过空气时会受到正比于速度平方的湍流阻力,而颗粒沿着地面移动时会受到恒定大小的摩擦力。
这些选择都不合适。
湍流阻力,正比于速度的平方,在速度很小时会很弱。
不够强到使粒子停下来。
非零值初始速度的粒子仅受到湍流阻力,会从初始位置永远地移动下去,和初始位置的距离大概正比于$O(\log t)$。
因此我们必须使用速度较低幂次的力。
如果幂次为零,相当于干摩擦,那么力太强了。
当\gls{cost_function}的梯度表示的力很小但非零时,由于摩擦导致的恒力会使得粒子在达到\gls{local_minimum}之前就停下来。
粘性阻力避免了这两个问题——它足够弱,可以使梯度引起的运动直到达到最小,但又足够强,使得坡度不够时可以阻止运动。
% 291 end


\subsection{\glsentrytext{nmomentum}}
\label{sec:nesterov_momentum}
受Nesterov加速梯度算法\citep{Nesterov83b,Nesterov03}启发,\cite{sutskeverimportance}提出了\gls{momentum}算法的一个变种。
这种情况的更新规则如下:
\begin{align}
    \Vv &\leftarrow \alpha\Vv - \epsilon \nabla_{\Vtheta} \left[
    \frac{1}{m} \sum_{i=1}^m L\big( \Vf(\Vx^{(i)}; \Vtheta + \alpha \Vv), \Vy^{(i)} \big)
 \right], \\
    \Vtheta &\leftarrow \Vtheta + \Vv ,
\end{align}
其中参数$\alpha$和$\epsilon$发挥了和标准\gls{momentum}方法中类似的作用。 
\gls{nmomentum}和标准\gls{momentum}之间的区别体现在梯度计算上。
\gls{nmomentum}中,梯度计算在施加当前速度之后。
因此,\gls{nmomentum}可以解释为往标准\gls{momentum}方法中添加了一个\emph{校正因子}。
完整的\,\gls{nmomentum}算法如\algref{alg:nesterov}所示。
% 292 mid


% 292 head
\begin{algorithm}[ht]
\caption{使用\,\gls{nmomentum}的\gls{SGD}(\glssymbol{SGD})}
\label{alg:nesterov}
\begin{algorithmic}
\REQUIRE  \gls{learning_rate} $\epsilon$, \gls{momentum}参数 $\alpha$
\REQUIRE 初始参数 $\Vtheta$,初始速度 $\Vv$
\WHILE{没有达到停止\gls{criterion}}
    \STATE 从\gls{training_set}中采包含$m$个样本$\{ \Vx^{(1)},\dots, \Vx^{(m)}\}$ 的\gls{minibatch},对应目标为$\Vy^{(i)}$。
    \STATE 应用临时更新: $\tilde{\Vtheta} \leftarrow \Vtheta  + \alpha \Vv$
         \STATE 计算梯度(在临时点):$\Vg \leftarrow 
         \frac{1}{m} \nabla_{\tilde{\Vtheta}} \sum_i L(f(\Vx^{(i)};\tilde{\Vtheta}),\Vy^{(i)})$
    \STATE 计算速度更新:$\Vv \leftarrow \alpha \Vv - 
    \epsilon \Vg$
    \STATE 应用更新:$\Vtheta \leftarrow \Vtheta + \Vv$ 
\ENDWHILE
\end{algorithmic}
\end{algorithm}

% 292 head

% 292 mid

在凸\gls{batch}梯度的情况下,\gls{nmomentum}将\gls{excess_error}收敛率从$O(1/k)$($k$步后)改进到$O(1/k^2)$,如~\cite{Nesterov83b}所示。
可惜,在随机梯度的情况下,\gls{nmomentum}没有改进收敛率。

% 292 mid

\section{参数初始化策略}
\label{sec:parameter_initialization_strategies}
有些优化算法本质上是非迭代的,只是求解一个解点。
有些其它优化算法本质上是迭代的,但是应用于这一类的优化问题时,能在可接受的时间内收敛到可接受的解,并且与初始值无关。
\gls{DL}训练算法通常没有这两种奢侈的性质。
\gls{DL}模型的训练算法通常是迭代的,因此要求使用者指定一些开始迭代的初始点。
此外,训练\gls{deep_model}是一个足够困难的问题,以致于大多数算法都很大程度地受到初始化选择的影响。
初始点能够决定算法是否收敛,有些初始点十分不稳定,使得该算法会遭遇数值困难,并完全失败。
当学习收敛时,初始点可以决定学习收敛得多快,以及是否收敛到一个\gls{cost}高或低的点。
此外,差不多\gls{cost}的点可以具有区别极大的\gls{generalization_error},初始点也可以影响\gls{generalization}。

% 293 head

现代的初始化策略是简单的、启发式的。
设定改进的初始化策略是一项困难的任务,因为\gls{NN}优化至今还未被很好地理解。
大多数初始化策略基于在\gls{NN}初始化时实现一些很好的性质。
然而,我们并没有很好地理解这些性质中的哪些会在学习开始进行后的哪些情况下得以保持。
进一步的难点是,有些初始点从优化的观点看或许是有利的,但是从\gls{generalization}的观点看是不利的。
我们对于初始点如何影响\gls{generalization}的理解更是相当原始的,几乎没有提供如何选择初始点的任何指导。

% 293 mid

也许完全确知的唯一特性是初始参数需要在不同单元间``破坏对称性''。
如果具有相同\gls{activation_function}的两个\gls{hidden_unit}连接到相同的输入,那么这些单元必须具有不同的初始参数。
如果它们具有相同的初始参数,然后应用到确定性损失和模型的确定性学习算法将一直以相同的方式更新这两个单元。
即使模型或训练算法能够使用随机性为不同的单元计算不同的更新(例如使用\,\gls{dropout}的训练),通常来说,最好还是初始化每个单元使其和其他单元计算不同的函数。
这或许有助于确保没有输入模式丢失在\gls{forward_propagation}的零空间中,以及没有梯度模式丢失在\gls{backward_propagation}的零空间中。
每个单元计算不同函数的目标促使了参数的随机初始化。
我们可以明确地搜索一大组彼此互不相同的基函数,但这经常会导致明显的计算代价。  
例如,如果我们有和输出一样多的输入,我们可以使用Gram-Schmidt正交化于初始的权重矩阵,保证每个单元计算彼此非常不同的函数。
与之相比,在高维空间上使用高熵分布来随机初始化,计算代价更小并且不太可能分配单元计算彼此相同的函数。

% 293 mid

通常情况下,我们可以为每个单元的\gls{bias_aff}设置启发式挑选的常数,仅随机初始化权重。
额外的参数(例如用于编码预测条件方差的参数)通常和偏置一样设置为启发式选择的常数。

% 293 end

我们几乎总是初始化模型的权重为高斯或均匀分布中随机抽取的值。
高斯或均匀分布的选择似乎不会有很大的差别,但也没有被详尽地研究。
然而,初始分布的大小确实对优化过程的结果和网络\gls{generalize}能力都有很大的影响。

% 294 head

更大的初始权重具有更强的破坏对称性的作用,有助于避免冗余的单元。
它们也有助于避免在每层线性成分的前向或\gls{backward_propagation}中丢失信号——矩阵中更大的值在矩阵乘法中有更大的输出。
如果初始权重太大,那么会在\gls{forward_propagation}或\gls{backward_propagation}中产生爆炸的值。
在循环网络中,很大的权重也可能导致\firstgls{chaos}(对于输入中很小的扰动非常敏感,导致确定性\gls{forward_propagation}过程表现随机)。
在一定程度上,梯度爆炸问题可以通过\gls{gradient_clipping}来缓解(执行\gls{GD}步骤之前设置梯度的阈值)。
较大的权重也会产生使得\gls{activation_function}饱和的值,导致饱和单元的梯度完全丢失。
这些竞争因素决定了权重的理想初始大小。
% 294 mid


关于如何初始化网络,\gls{regularization}和优化有着非常不同的观点。
优化观点建议权重应该足够大以成功传播信息,但是\gls{regularization}希望其小一点。
诸如\gls{SGD}这类对权重较小的增量更新,趋于停止在更靠近初始参数的区域(不管是由于卡在低梯度的区域,还是由于触发了基于\gls{overfitting} 的\gls{early_stopping}准则)的优化算法倾向于最终参数应接近于初始参数。
回顾\secref{sec:early_stopping},在某些模型上,\gls{early_stopping}的\gls{GD}等价于\gls{weight_decay}。
在一般情况下,\gls{early_stopping}的\gls{GD}和\gls{weight_decay}不同,但是提供了一个宽松的类比去考虑初始化的影响。
我们可以将初始化参数$\Vtheta$为$\Vtheta_0$类比于强置均值为$\Vtheta_0$的高斯先验$p(\Vtheta)$。
从这个角度来看,选择$\Vtheta_0$接近$0$是有道理的。
这个先验表明,单元间彼此互不交互比交互更有可能。
只有在目标函数的似然项表达出对交互很强的偏好时,单元才会交互。
另一方面,如果我们初始化$\Vtheta_0$为很大的值,那么我们的先验指定了哪些单元应互相交互,以及它们应如何交互。
% 294 end


有些启发式方法可用于选择权重的初始大小。
一种初始化$m$个输入和$n$输出的全连接层的权重的启发式方法是从分布$U(-\frac{1}{\sqrt{m}}, \frac{1}{\sqrt{m}})$中采样权重,
而~\citet{GlorotAISTATS2010-small}建议使用\firstgls{normalized_initialization}
\begin{equation}
    W_{i,j} \sim U \left(-\sqrt{\frac{6}{m+n}}, \sqrt{\frac{6}{m+n}}\right) .
\end{equation}
后一种启发式方法初始化所有的层,折衷于使其具有相同激活方差和使其具有相同梯度方差之间。
这假设网络是不含非线性的链式矩阵乘法,据此推导得出。
现实的\gls{NN}显然会违反这个假设,但很多设计于线性模型的策略在其非线性对应中的效果也不错。
% 295 head


\cite{Saxe-et-al-ICLR13}推荐初始化为随机正交矩阵,仔细挑选负责每一层非线性缩放或者说\,\textbf{增益}(gain)因子$g$。
他们得到了用于不同类型的非线性\gls{activation_function}的特定缩放因子。
这种初始化方案也是启发于不含非线性的矩阵相乘序列的\gls{deep_network}。
在该模型下,这个初始化方案保证了达到收敛所需的训练迭代总数独立于深度。


增加缩放因子$g$将网络推向网络\gls{forward_propagation}时激活范数增加,\gls{backward_propagation}时梯度范数增加的区域。
\cite{Sussillo14}表明,正确设置缩放因子足以训练深达$1000$层的网络,而不需要使用正交初始化。
这种方法的一个重要观点是,在\gls{feedforward_network}中,激活和梯度会在每一步\gls{forward_propagation}或\gls{backward_propagation}中增加或缩小,遵循随机游走行为。
这是因为\gls{feedforward_network}在每一层使用了不同的权重矩阵。
如果该随机游走调整到保持范数,那么\gls{feedforward_network}能够很大程度地避免相同权重矩阵用于每层的\gls{vanish_explode_gradient},如\secref{sec:long_term_dependencies}所述。
% 295 end


可惜,这些初始权重的最佳准则往往不会带来最佳效果。
这可能有三种不同的原因。
首先,我们可能使用了错误的标准——它实际上并不利于保持整个网络信号的范数。
其次,初始化时强加的性质可能在学习开始进行后不能保持。
最后,该标准可能成功提高了优化速度,但意外地增大了\gls{generalization_error}。
在实践中,我们通常需要将权重范围视为\gls{hyperparameter},其最优值大致接近,但并不完全等于理论预测。
% 296 head


数值范围准则的一个缺点是,设置所有的初始权重具有相同的标准差,例如$\frac{1}{\sqrt{m}}$,会使得层很大时每个单一权重会变得极其小。
\cite{martens2010hessian-small}提出了一种被称为\firstgls{sparse_initialization}的替代方案,每个单元初始化为恰好有$k$个非零权重。
这个想法保持该单元输入的总数量独立于输入数目$m$,而不使单一权重元素的大小随$m$缩小。
稀疏初始化有助于实现单元之间在初始化时更具多样性。
但是,获得较大取值的权重也同时被加了很强的先验。
因为\gls{GD}需要很长时间缩小``不正确''的大值,这个初始化方案可能会导致某些单元出问题,例如\,\gls{maxout}\,单元有几个过滤器,互相之间必须仔细调整。
% 296 mid


计算资源允许的话,将每层权重的初始数值范围设为\gls{hyperparameter}通常是个好主意,使用\secref{sec:automatic_hyperparameter_optimization_algorithms}介绍的\gls{hyperparameter}搜索算法,如随机搜索,挑选这些数值范围。
是否选择使用密集或稀疏初始化也可以设为一个\gls{hyperparameter}。
作为替代,我们可以手动搜索最优初始范围。
一个好的挑选初始数值范围的经验法则是观测单个\gls{minibatch}数据上的激活或梯度的幅度或标准差。
如果权重太小,那么当激活值在\gls{minibatch}上\gls{forward_propagation}于网络时,激活值的幅度会缩小。
通过重复识别具有小得不可接受的激活值的第一层,并提高其权重,最终有可能得到一个初始激活全部合理的网络。
如果学习在这点上仍然很慢,观测梯度的幅度或标准差可能也会有所帮助。
这个过程原则上是自动的,且通常计算量低于基于\gls{validation_set}误差的\gls{hyperparameter}优化,因为它是基于初始模型在单批数据上的行为反馈,而不是在\gls{validation_set}上训练模型的反馈。
由于这个协议很长时间都被启发式使用,最近~\cite{mishkin2015all}更正式地研究了该协议。
% 296 mid


目前为止,我们关注在权重的初始化上。
幸运的是,其他参数的初始化通常更容易。
% 296 end


设置\gls{bias_aff}的方法必须和设置权重的方法协调。
设置\gls{bias_aff}为零通常在大多数权重初始化方案中是可行的。
存在一些我们可能设置\gls{bias_aff}为非零值的情况:
% 297 head

\begin{itemize}
\item 如果\gls{bias_aff}是作为输出单元,那么初始化\gls{bias_aff}以获取正确的输出边缘统计通常是有利的。
要做到这一点,我们假设初始权重足够小,该单元的输出仅由\gls{bias_aff}决定。
这说明设置\gls{bias_aff}为应用于\gls{training_set}上输出边缘统计的\gls{activation_function}的逆。
例如,如果输出是类上的分布,且该分布是高度偏态分布,第$i$类的边缘概率由某个向量$\Vc$的第$i$个元素给定,那么我们可以通过求解方程$\text{softmax}(\Vb)=\Vc$来设置偏置向量$\Vb$。
这不仅适用于分类器,也适用于我们将在第三部分遇到的模型,例如\gls{AE}和\gls{BM}。
这些模型拥有输出类似于输入数据$\Vx$的网络层,初始化这些层的偏置以匹配$\Vx$上的边缘分布将有助于模型学习。
% 297 mid


\item 有时,我们可能想要选择\gls{bias_aff}以避免初始化引起太大饱和。
例如,我们可能会将ReLU的\gls{hidden_unit}设为$0.1$而非$0$,以避免ReLU在初始化时饱和。
尽管这种方法违背不希望偏置具有很强输入的权重初始化准则。
例如,不建议使用随机游走初始化\citep{Sussillo14}。
% 297 mid


\item 有时,一个单元会控制其他单元能否参与到等式中。
在这种情况下,我们有一个单元输出$u$,另一个单元$h\in[0,1]$,那么我们可以将$h$视作门,以决定$uh\approx 1$还是$uh\approx 0$。
在这种情形下,我们希望设置\gls{bias_aff} $h$,使得在初始化的大多数情况下$h\approx 1$。
否则,$u$没有机会学习。
例如,\cite{Jozefowicz-et-al-ICML2015}提议设置~\glssymbol{LSTM}~模型遗忘门的\gls{bias_aff}为$1$,如\secref{sec:the_long_short_term_memory_and_other_gated_rnns}所述。
\end{itemize}
% 297 end


另一种常见类型的参数是方差或精确度参数。
例如,我们用以下模型进行带条件方差估计的\gls{linear_regression}
\begin{equation}
    p(y\mid\Vx) = \mathcal{N} (y \mid \Vw^\top \Vx + b, 1/\beta) ,
\end{equation}
其中$\beta$是精确度参数。
通常我们能安全地初始化方差或精确度参数为$1$。
另一种方法假设初始权重足够接近零,设置偏置可以忽略权重的影响,然后设定偏置以产生输出的正确边缘均值,并将方差参数设置为\gls{training_set}输出的边缘方差。
% 298 head


除了这些初始化模型参数的简单常数或随机方法,还有可能使用\gls{ML}初始化模型参数。
在本书第三部分讨论的一个常用策略是使用相同的输入数据集,用无监督模型训练出来的参数来初始化监督模型。
我们也可以在相关问题上使用监督训练。
即使是在一个不相关的任务上运行监督训练,有时也能得到一个比随机初始化具有更快收敛率的初始值。
这些初始化策略有些能够得到更快的收敛率和更好的\gls{generalization_error},原因在于它们编码了模型初始参数的分布信息。
其他一些策略表现出效果不错的原因则主要在于它们设置参数为正确的数值范围,或是设置不同单元计算互相不同的函数。
% 298 mid


\section{自适应\glsentrytext{learning_rate}算法}
\label{sec:algorithms_with_adaptive_learning_rates}
\gls{NN}研究员早就意识到\gls{learning_rate}肯定是难以设置的\gls{hyperparameter}之一,因为它对模型的性能有显著的影响。
正如我们在\secref{sec:gradient_based_optimization}和\secref{sec:challenges_in_neural_network_optimization}中所探讨的,损失通常高度敏感于参数空间中的某些方向,而不敏感于其他。
\gls{momentum}算法可以在一定程度缓解这些问题,但这样做的代价是引入了另一个\gls{hyperparameter}。
在这种情况下,自然会问有没有其他方法。
如果我们相信方向敏感度在某种程度是轴对齐的,那么每个参数设置不同的\gls{learning_rate},在整个学习过程中自动适应这些\gls{learning_rate}是有道理的。



\textbf{Delta-bar-delta}算法\citep{jacobs1988}是一个早期的在训练时适应模型参数各自\gls{learning_rate}的启发式方法。
该方法基于一个很简单的想法,如果损失对于某个给定模型参数的偏导保持相同的符号,那么\gls{learning_rate}应该增加。如果对于该参数的偏导变化了符号,那么\gls{learning_rate}应减小。
当然,这种方法只能应用于全\gls{batch}优化中。
% 298 mid


最近,提出了一些增量(或者基于\gls{minibatch})的算法来自适应模型参数的\gls{learning_rate}。
这节将简要回顾其中一些算法。
% 298 end


% 299 head
\subsection{\glsentrytext{adagrad}}
\label{sec:adagrad}
\textbf{AdaGrad}算法,如\algref{alg:ada_grad}所示,独立地适应所有模型参数的\gls{learning_rate},按照每个参数的梯度历史值的平方和的平方根成反比缩放每个参数\citep{Duchi+al-2011}。
具有损失最大偏导的参数相应地有一个快速下降的\gls{learning_rate},而具有小偏导的参数在\gls{learning_rate}上有相对较小的下降。
净效果是在参数空间中更为平缓的倾斜方向会取得更大的进步。
% 299 head


在\gls{convex_optimization}背景中,\gls{adagrad} 算法具有一些令人满意的理论性质。
然而,经验上已经发现,对于训练\gls{DNN}模型而言,\emph{从训练开始时}积累梯度平方会导致有效\gls{learning_rate}过早和过量的减小。
\gls{adagrad}\,在某些\gls{DL}模型上效果不错,但不是全部。


\begin{algorithm}[ht]
\caption{AdaGrad算法}
\label{alg:ada_grad}
\begin{algorithmic}
\REQUIRE 全局\gls{learning_rate} $\epsilon$
\REQUIRE 初始参数$\Vtheta$
\REQUIRE 小常数$\delta$,为了数值稳定大约设为$10^{-7}$
\STATE 初始化梯度累积变量$\Vr = 0$
\WHILE{没有达到停止\gls{criterion}}
    \STATE 从\gls{training_set}中采包含$m$个样本$\{ \Vx^{(1)},\dots, \Vx^{(m)}\}$ 的\gls{minibatch},对应目标为$\Vy^{(i)}$。
    \STATE 计算梯度: $\Vg \leftarrow  
         \frac{1}{m} \nabla_{\Vtheta} \sum_i L(f(\Vx^{(i)};\Vtheta),\Vy^{(i)})$ 
    \STATE 累积平方梯度:$\Vr \leftarrow \Vr + \Vg \odot \Vg$
    \STATE 计算更新:$\Delta \Vtheta \leftarrow -
    \frac{\epsilon}{\delta+ \sqrt{\Vr}} \odot\Vg$  \ \  (逐元素地应用除和求平方根)
    \STATE 应用更新:$\Vtheta \leftarrow \Vtheta + \Delta \Vtheta$
\ENDWHILE
\end{algorithmic}
\end{algorithm}


\subsection{RMSProp}
\label{sec:rmsprop}
\textbf{RMSProp}算法\citep{Hinton-ipam2012}修改AdaGrad以在\gls{nonconvex}设定下效果更好,改变梯度积累为指数加权的移动平均。
\gls{adagrad}\,旨在应用于凸问题时快速收敛。
当应用于\gls{nonconvex}函数训练\gls{NN}时,学习轨迹可能穿过了很多不同的结构,最终到达一个局部是凸的碗状的区域。
AdaGrad根据平方梯度的整个历史收缩\gls{learning_rate},可能使得\gls{learning_rate}在达到这样的凸结构前就变得太小了。
RMSProp使用指数衰减平均以丢弃遥远过去的历史,使其能够在找到碗状凸结构后快速收敛,
它就像一个初始化于该碗状结构的AdaGrad算法实例。

% -- 299 --

RMSProp的标准形式如\algref{alg:rms_prop}所示,结合Nesterov动量的形式如\algref{alg:rms_nesterov}所示。
相比于AdaGrad,使用移动平均引入了一个新的\gls{hyperparameter}$\rho$,用来控制移动平均的长度范围。

% -- 300 --

经验上,RMSProp已被证明是一种有效且实用的\gls{DNN}优化算法。
目前它是\gls{DL}从业者经常采用的优化方法之一。


\begin{algorithm}[ht]
\caption{RMSProp算法}
\label{alg:rms_prop}
\begin{algorithmic}
\REQUIRE 全局\gls{learning_rate} $\epsilon$,衰减速率$\rho$
\REQUIRE  初始参数$\Vtheta$
\REQUIRE 小常数$\delta$,通常设为$10^{-6}$(用于被小数除时的数值稳定)
\STATE 初始化累积变量 $\Vr = 0$
\WHILE{没有达到停止\gls{criterion}}
    \STATE 从\gls{training_set}中采包含$m$个样本$\{ \Vx^{(1)},\dots, \Vx^{(m)}\}$ 的\gls{minibatch},对应目标为$\Vy^{(i)}$。
    \STATE 计算梯度:$\Vg \leftarrow  
         \frac{1}{m} \nabla_{\Vtheta} \sum_i L(f(\Vx^{(i)};\Vtheta),\Vy^{(i)})$ 
    \STATE 累积平方梯度:$\Vr \leftarrow \rho
    \Vr + (1-\rho) \Vg \odot \Vg$
    \STATE 计算参数更新:$\Delta \Vtheta =
    -\frac{\epsilon}{\sqrt{\delta + \Vr}} \odot \Vg$  \ \  ($\frac{1}{\sqrt{\delta + \Vr}}$ 逐元素应用)
    \STATE 应用更新:$\Vtheta \leftarrow \Vtheta + \Delta \Vtheta$
\ENDWHILE
\end{algorithmic}
\end{algorithm}

\begin{algorithm}[ht]
\caption{使用Nesterov\,\gls{momentum}的RMSProp算法}
\label{alg:rms_nesterov}
\begin{algorithmic}
\REQUIRE 全局\gls{learning_rate} $\epsilon$,衰减速率$\rho$, \gls{momentum}系数$\alpha$
\REQUIRE 初始参数$\Vtheta$,初始参数$\Vv$
\STATE 初始化累积变量 $\Vr = 0$
\WHILE{没有达到停止\gls{criterion}} % NOTE: do not capitalize the condition
    \STATE 从\gls{training_set}中采包含$m$个样本$\{ \Vx^{(1)},\dots, \Vx^{(m)}\}$ 的\gls{minibatch},对应目标为$\Vy^{(i)}$。
    \STATE 计算临时更新:$\tilde{\Vtheta} \leftarrow \Vtheta + \alpha \Vv$
    \STATE 计算梯度:$\Vg \leftarrow  
         \frac{1}{m} \nabla_{\tilde{\Vtheta}} \sum_i L(f(\Vx^{(i)};\tilde{\Vtheta}),\Vy^{(i)})$ 
    \STATE  累积梯度:$\Vr \leftarrow \rho
    \Vr + (1-\rho) \Vg \odot \Vg$
    \STATE  计算速度更新:$\Vv \leftarrow \alpha \Vv
    -\frac{\epsilon}{\sqrt{\Vr}} \odot \Vg$ \ \  ($\frac{1}{\sqrt{\Vr}}$ 逐元素应用)
    \STATE 应用更新:$\Vtheta \leftarrow \Vtheta + \Vv$
\ENDWHILE
\end{algorithmic}
\end{algorithm}


\subsection{Adam}
\label{sec:adam}
\textbf{Adam}~\citep{kingma2014adam}是另一种\gls{learning_rate}自适应的优化算法,如\algref{alg:adam}所示。
``Adam''这个名字派生自短语``adaptive moments''。
在前述算法背景下,它也许最好被看作结合了RMSProp和\gls{momentum}的具有一些重要区别的变种。
首先,在Adam中,\gls{momentum}直接并入了梯度一阶矩(指数加权)的估计。
将\gls{momentum}加入RMSProp最直观的方法是将\gls{momentum}应用于缩放后的梯度。
结合缩放的\gls{momentum}使用没有明确的理论动机。
其次,Adam包括偏置修正,修正从原点初始化的一阶矩(\gls{momentum}项)和(非中心的)二阶矩的估计(\algref{alg:adam})。
RMSProp也采用了(非中心的)二阶矩估计,然而缺失了修正因子。
因此,不像Adam,RMSProp二阶矩估计可能在训练初期有很高的偏置。
Adam通常被认为对\gls{hyperparameter}的选择相当鲁棒,尽管\gls{learning_rate}有时需要改为与建议的默认值不同的值。

\begin{algorithm}[ht]
\caption{Adam算法}
\label{alg:adam}
\begin{algorithmic}
\REQUIRE 步长 $\epsilon$ (建议默认为: $0.001$)
\REQUIRE 矩估计的指数衰减速率, $\rho_1$ 和 $\rho_2$ 在区间 $[0, 1)$内。
(建议默认为:分别为$0.9$ 和 $0.999$)
\REQUIRE 用于数值稳定的小常数 $\delta$  (建议默认为: $10^{-8}$)
\REQUIRE 初始参数 $\Vtheta$
\STATE 初始化一阶和二阶矩变量 $\Vs = 0 $, $\Vr = 0$
\STATE 初始化\gls{time_step} $t=0$ 
\WHILE{没有达到停止\gls{criterion}}
    \STATE 从\gls{training_set}中采包含$m$个样本$\{ \Vx^{(1)},\dots, \Vx^{(m)}\}$ 的\gls{minibatch},对应目标为$\Vy^{(i)}$。
    \STATE 计算梯度:$\Vg \leftarrow \frac{1}{m} \nabla_{\Vtheta} \sum_i L(f(\Vx^{(i)};\Vtheta),\Vy^{(i)})$ 
    \STATE $t \leftarrow t + 1$
    \STATE 更新有偏一阶矩估计: $\Vs \leftarrow \rho_1 \Vs + (1-\rho_1) \Vg$
    \STATE 更新有偏二阶矩估计:$\Vr \leftarrow \rho_2 \Vr + (1-\rho_2) \Vg \odot \Vg$
    \STATE 修正一阶矩的\gls{bias_sta}:$\hat{\Vs} \leftarrow \frac{\Vs}{1-\rho_1^t}$
    \STATE 修正二阶矩的\gls{bias_sta}:$\hat{\Vr} \leftarrow \frac{\Vr}{1-\rho_2^t}$
    \STATE 计算更新:$\Delta \Vtheta = - \epsilon \frac{\hat{\Vs}}{\sqrt{\hat{\Vr}} + \delta}$ \ \  (逐元素应用操作)
    \STATE 应用更新:$\Vtheta \leftarrow \Vtheta + \Delta \Vtheta$
\ENDWHILE
\end{algorithmic}
\end{algorithm}


% -- 301 --

\subsection{选择正确的优化算法}
\label{sec:choosing_the_right_optimization_algorithms}
在本节中,我们讨论了一系列算法,通过自适应每个模型参数的\gls{learning_rate}以解决优化\gls{deep_model}中的难题。
此时,一个自然的问题是:该选择哪种算法呢?

遗憾的是,目前在这一点上没有达成共识。
\cite{Schaul2014_unittests}展示了许多优化算法在大量学习任务上极具价值的比较。
虽然结果表明,具有自适应\gls{learning_rate}(以RMSProp和AdaDelta为代表)的算法族表现得相当鲁棒,不分伯仲,但没有哪个算法能脱颖而出。

目前,最流行并且使用很高的优化算法包括SGD、具\gls{momentum}的SGD、RMSProp、具\gls{momentum}的RMSProp、AdaDelta和Adam。
此时,选择哪一个算法似乎主要取决于使用者对算法的熟悉程度(以便调节\gls{hyperparameter})。

\section{二阶近似方法}
\label{sec:approximate_second_order_methods}
在本节中,我们会讨论训练\gls{DNN}的二阶方法。
参考\cite{LeCun+98backprop}了解该问题的早期处理方法。
为表述简单起见,我们只考察\gls{objective_function}为经验风险:
\begin{equation}
    J(\Vtheta) = \SetE_{\RVx, \RSy \sim \hat{p}_{\text{data}}(\Vx,y) } [ L(f(\Vx; \Vtheta), y) ] =
\frac{1}{m} \sum_{i=1}^m L(f(\Vx^{(i)}; \Vtheta), y^{(i)}).
\end{equation}
然而,我们在这里讨论的方法很容易扩展到更一般的\gls{objective_function},例如,\chapref{chap:regularization_for_deep_learning}讨论的包括参数正则项的函数。

% -- 302 --

\subsection{\glsentrytext{newton_method}}
\label{sec:newton_method}
在\secref{sec:gradient_based_optimization},我们介绍了二阶梯度方法。
与一阶方法相比,二阶方法使用二阶导数改进了优化。
最广泛使用的二阶方法是\gls{newton_method}。
我们现在更详细地描述\gls{newton_method},重点在其应用于\gls{NN}的训练。

\gls{newton_method}是基于二阶泰勒级数展开在某点$\Vtheta_0$附近来近似$J(\Vtheta)$的优化方法,其忽略了高阶导数:
\begin{equation}
    J(\Vtheta) \approx J(\Vtheta_0) + (\Vtheta - \Vtheta_0)^\top \nabla_{\Vtheta}   
    J(\Vtheta_0) + \frac{1}{2} (\Vtheta - \Vtheta_0)^\top \MH(\Vtheta - \Vtheta_0),
\end{equation}
最速下降法中$\MH$是$J$相对于$\Vtheta$的\,\gls{hessian}\,矩阵在$\Vtheta_0$处的估计。
如果我们再求解这个函数的\gls{critical_points},我们将得到牛顿参数更新规则:
\begin{equation}
    \Vtheta^* = \Vtheta_0 - \MH^{-1}\nabla_{\Vtheta} J(\Vtheta_0).
\label{eq:8.27}
\end{equation}
因此,对于局部的二次函数(具有正定的$\MH$\,),用$\MH^{-1}$重新调整梯度,\gls{newton_method}会直接跳到极小值。
如果\gls{objective_function}是凸的但非二次的(有高阶项),该更新将是迭代的,得到和\gls{newton_method}相关的算法,如\algref{alg:newton}所示。

对于非二次的表面,只要\,\gls{hessian}\,矩阵保持正定,\gls{newton_method}能够迭代地应用。
这意味着一个两步迭代过程。
首先,更新或计算\,\gls{hessian}\,逆(通过更新二阶近似)。
其次,根据\eqnref{eq:8.27}更新参数。

% -- 303 --

在\secref{sec:plateaus_saddle_points_and_other_flat_regions},我们讨论了\gls{newton_method}只适用于\,\gls{hessian}\,矩阵是正定的情况。
在\gls{DL}中,\gls{objective_function}的表面通常\gls{nonconvex}(有很多特征),如\gls{saddle_points}。
因此使用\gls{newton_method}是有问题的。
如果\,\gls{hessian}\,矩阵的特征值并不都是正的,例如,靠近\gls{saddle_points}处,\gls{newton_method}实际上会导致更新朝错误的方向移动。
这种情况可以通过正则化\,\gls{hessian}\,矩阵来避免。
常用的正则化策略包括在\,\gls{hessian}\,矩阵对角线上增加常数$\alpha$。
正则化更新变为
\begin{equation}
    \Vtheta^* = \Vtheta_0 - [ H( f(\Vtheta_0)) + \alpha \MI  ]^{-1} \nabla_{\Vtheta} f(\Vtheta_0).
\end{equation}
这个正则化策略用于\gls{newton_method}的近似,例如Levenberg-Marquardt算法\citep{Levenberg44,Marquardt63},只要\,\gls{hessian}\,矩阵的负特征值仍然相对接近零,效果就会很好。
在\gls{curvature}方向更极端的情况下,$\alpha$的值必须足够大,以抵消负特征值。
然而,如果$\alpha$持续增加,\gls{hessian}\,矩阵会变得由对角矩阵$\alpha \MI$主导,通过\gls{newton_method}所选择的方向会收敛到普通梯度除以$\alpha$。
当很强的负\gls{curvature}存在时,$\alpha$可能需要特别大,以致于\gls{newton_method}比选择合适\gls{learning_rate}的\gls{GD}的步长更小。

\begin{algorithm}[ht]
\caption{目标为$J(\Vtheta)= \frac{1}{m} \sum_{i=1}^m L(f(\Vx^{(i)};\Vtheta), y^{(i)})$的\gls{newton_method}}
\label{alg:newton}
\begin{algorithmic}
\REQUIRE 初始参数$\Vtheta_{0}$
\REQUIRE 包含 $m$个样本的\gls{training_set}
\WHILE{没有达到停止\gls{criterion}} 
      \STATE 计算梯度: $\Vg \leftarrow 
     \frac{1}{m} \nabla_{\Vtheta} \sum_i L(f(\Vx^{(i)};\Vtheta),\Vy^{(i)})$ 
      \STATE 计算\,\gls{hessian}\,矩阵:$\MH \leftarrow  
     \frac{1}{m} \nabla_{\Vtheta}^2 \sum_i L(f(\Vx^{(i)};\Vtheta),\Vy^{(i)})$ 
    \STATE 计算\,\gls{hessian}\,逆:$\MH^{-1}$
    \STATE 计算更新: $\Delta \Vtheta = - \MH^{-1} \Vg$
    \STATE 应用更新:$\Vtheta = \Vtheta+\Delta \Vtheta$
\ENDWHILE
\end{algorithmic}
\end{algorithm}


除了\gls{objective_function}的某些特征带来的挑战,如\gls{saddle_points},\gls{newton_method}用于训练大型\gls{NN}还受限于其显著的计算负担。
\gls{hessian}\,矩阵中元素数目是参数数量的平方,因此,如果参数数目为$k$(甚至是在非常小的\gls{NN}中$k$也可能是百万级别),\gls{newton_method}需要计算$k\times k$矩阵的逆,计算复杂度为$O(k^3)$。
另外,由于参数将每次更新都会改变,\emph{每次训练迭代}都需要计算\,\gls{hessian}\,矩阵的逆。
其结果是,只有参数很少的网络才能在实际中用\gls{newton_method}训练。
在本节的剩余部分,我们将讨论一些试图保持\gls{newton_method}优点,同时避免计算障碍的替代算法。

\subsection{\glsentrytext{CG}法}
\label{sec:conjugate_gradients}
\gls{CG}法是一种通过迭代下降的\firstgls{conjugate_directions}以有效避免\,\gls{hessian}\,矩阵求逆计算的方法。
这种方法的灵感来自于对最速下降方法弱点的仔细研究(详细信息请查看\secref{sec:gradient_based_optimization}),其中线搜索迭代地用于与梯度相关的方向上。
\figref{fig:chap8_steepest_descent_quadratic}说明了该方法在二次碗型目标中如何表现的,是一个相当低效的来回往复,锯齿形模式。
这是因为每一个由梯度给定的线搜索方向,都保证正交于上一个线搜索方向。

\begin{figure}[!htb]
\ifOpenSource
\centerline{\includegraphics{figure.pdf}}
\else
\centerline{\includegraphics{Chapter8/figures/steepest_descent_quadratic_color}}
\fi
\caption{将最速下降法应用于二次代价表面。
在每个步骤,最速下降法沿着由初始点处的梯度定义的线跳到最低代价的点。
这解决了\figref{fig:chap4_poor_conditioning_color}中使用固定\gls{learning_rate}所遇到的一些问题,但即使使用最佳步长,算法仍然朝最优方向曲折前进。
根据定义,在沿着给定方向的目标最小值处,最终点处的梯度与该方向正交。
}
\label{fig:chap8_steepest_descent_quadratic}
\end{figure}


% -- 304 --

假设上一个搜索方向是$\Vd_{t-1}$。   
在极小值处,也就是(上一次)线搜索终止的地方,方向$\Vd_{t-1}$处的方向导数为零:$\nabla_{\Vtheta} J(\Vtheta) \cdot \Vd_{t-1} = 0$。
因为该点的梯度定义了当前的搜索方向,$\Vd_t = \nabla_{\Vtheta} J(\Vtheta)$将不会贡献于方向$\Vd_{t-1}$。
因此方向$\Vd_t$正交于$\Vd_{t-1}$。
最速下降多次迭代中,方向$\Vd_{t-1}$和$\Vd_t$之间的关系如\figref{fig:chap8_steepest_descent_quadratic}所示。
如图展示的,下降正交方向的选择不会保持前一搜索方向上的最小值。
这产生了锯齿形的过程。
在当前梯度方向下降到极小值,我们必须重新最小化之前梯度方向上的目标。
因此,通过遵循每次线搜索结束时的梯度,我们在某种程度上撤销了在之前线搜索的方向上取得的进展。
\gls{CG}法试图解决这个问题。

在\gls{CG}法中,我们寻求一个和先前线搜索方向\firstgls{conjugate}的搜索方向,即它不会撤销该方向上的进展。
在训练迭代$t$时,下一步的搜索方向$\Vd_t$的形式如下:
\begin{equation}
    \Vd_t = \nabla_{\Vtheta} J(\Vtheta) + \beta_t \Vd_{t-1},
\end{equation}
其中,系数$\beta_t$的大小控制我们应沿方向$\Vd_{t-1}$加回多少到当前搜索方向上。

% -- 305 --

如果$\Vd_t^\top \MH \Vd_{t-1} = 0$,其中$\MH$是\,\gls{hessian}\,矩阵,则两个方向$\Vd_t$和$\Vd_{t-1}$被称为共轭的。

适应共轭的直接方法会涉及到$\MH$特征向量的计算以选择$\beta_t$。
这将无法满足我们的开发目标:寻找在大问题比\gls{newton_method}计算更加可行的方法。
我们能否不进行这些计算而得到共轭方向?
幸运的是这个问题的答案是肯定的。

两种用于计算$\beta_t$的流行方法是:
\begin{enumerate}
\item Fletcher-Reeves:
\begin{equation}
    \beta_t = \frac{ \nabla_{\Vtheta} J(\Vtheta_t)^\top \nabla_{\Vtheta} J(\Vtheta_t) }
{ \nabla_{\Vtheta} J(\Vtheta_{t-1})^\top \nabla_{\Vtheta} J(\Vtheta_{t-1}) }
\end{equation}

\item Polak-Ribi\`{e}re:
\begin{equation}
    \beta_t = \frac{ (\nabla_{\Vtheta} J(\Vtheta_t) - \nabla_{\Vtheta} J(\Vtheta_{t-1}))^\top \nabla_{\Vtheta} J(\Vtheta_t) }
{ \nabla_{\Vtheta} J(\Vtheta_{t-1})^\top \nabla_{\Vtheta} J(\Vtheta_{t-1}) }
\end{equation}
\end{enumerate}
对于二次曲面而言,共轭方向确保梯度沿着前一方向幅度不会变大。
因此,我们在前一方向上仍然处于极小值。
其结果是,在$k$-维参数空间中,\gls{CG}法只需要至多$k$次线搜索就能达到极小值。
\gls{CG}法如\algref{alg:cg}所示。

\begin{algorithm}[ht]
\caption{\gls{CG}法}
\label{alg:cg}
\begin{algorithmic}
\REQUIRE 初始参数 $\Vtheta_{0}$
\REQUIRE 包含$m$个样本的\gls{training_set}
\STATE 初始化 $\Vrho_{0} = 0$
\STATE 初始化 $g_0 = 0$
\STATE 初始化 $t = 1$
\WHILE{没有达到停止\gls{criterion}}
    \STATE 初始化梯度 ${\Vg}_{t} = 0$
    \STATE 计算梯度:$\Vg_{t} \leftarrow
         \frac{1}{m}\nabla_{\Vtheta} \sum_i L(f(\Vx^{(i)};\Vtheta),\Vy^{(i)})$ 
    \STATE 计算 $\beta_{t} = \frac{(\Vg_{t}-\Vg_{t-1})^\top \Vg_{t}}{\Vg_{t-1}^\top \Vg_{t-1}}$  (Polak-Ribi\`{e}re)
    \STATE (非线性\gls{CG}法:视情况可重置$\beta_{t}$为零,
           例如  $t$是常数$k$的倍数时,如 $k=5$)
    \STATE 计算搜索方向: $\Vrho_{t} = -\Vg_{t} + \beta_{t} \Vrho_{t-1}$ 
    \STATE 执行\gls{line_search}寻找:$\epsilon^{*} = \arg\!\min_{\epsilon}
    \frac{1}{m} \sum_{i=1}^{m}L(f(\Vx^{(i)};\Vtheta_t + \epsilon \Vrho_t),\Vy^{(i)})$ 
    \STATE (对于真正二次的\gls{cost_function},存在$\epsilon^*$的解析解,而无需显式地搜索)
    \STATE 应用更新:$\Vtheta_{t+1} = \Vtheta_{t}+ \epsilon^{*} \Vrho_{t}$
    \STATE $t \leftarrow t + 1$
\ENDWHILE
\end{algorithmic}
\end{algorithm}


\paragraph{\gls{nonlinear_CG}法:}
目前,我们已经讨论了用于二次\gls{objective_function}的\gls{CG}法。
当然,本章我们主要关注于探索训练\gls{NN}和其他相关\gls{DL}模型的优化方法,其对应的\gls{objective_function}比二次函数复杂得多。
或许令人惊讶,\gls{CG}法在这种情况下仍然是适用的,尽管需要作一些修改。
没有目标是二次的保证,共轭方向也不再保证在以前方向上的目标仍是极小值。
其结果是,\textbf{\gls{nonlinear_CG}法}\,算法会包括一些偶尔的重设,\gls{CG}法沿未修改的梯度重启线搜索。

% -- 306 --

实践者报告在实践中使用\gls{nonlinear_CG}法训练\gls{NN}是合理的,尽管在开始\gls{nonlinear_CG}法前使用\gls{SGD}迭代若干步来初始化效果更好。
另外,尽管(非线性)\gls{CG}法传统上作为批方法,\gls{minibatch}版本已经成功用于训练\gls{NN}~\citep{Le-ICML2011}。
针对神经网路的\gls{CG}法应用早已被提出,例如缩放的\gls{CG}法\citep{Moller}。

\subsection{\glsentrytext{BFGS}}
\label{sec:bfgs}
\textbf{Broyden-Fletcher-Goldfarb-Shanno}(\textbf{\gls{BFGS}})算法具有\gls{newton_method}的一些优点,但没有\gls{newton_method}的计算负担。
在这方面,\gls{BFGS}\,和\gls{CG}法很像。
然而,\gls{BFGS}\,使用了一个更直接的方法近似牛顿更新。回顾牛顿更新由下式给出
\begin{equation}
    \Vtheta^* = \Vtheta_0 - \MH^{-1} \nabla_{\Vtheta} J(\Vtheta_0),
\end{equation}
其中,$\MH$是$J$相对于$\Vtheta$的\,\gls{hessian}\,矩阵在$\Vtheta_0$处的估计。
运用\gls{newton_method}的主要计算难点在于计算\,\gls{hessian}\,逆$\MH^{-1}$。
拟\gls{newton_method}所采用的方法(\gls{BFGS}\,是其中最突出的)是使用矩阵$\MM_t$近似逆,迭代地低秩更新精度以更好地近似$\MH^{-1}$。

% -- 307 --

\gls{BFGS}\,近似的说明和推导出现在很多关于优化的教科书中,包括\cite{Lue84}。

当\,\gls{hessian}\,逆的近似$\MM_t$更新时,下降方向$\Vrho_t$为$\Vrho_t = \MM_t \Vg_t$。
该方向上的线搜索用于决定该方向上的步长$\epsilon^*$。
参数的最后更新为:
\begin{equation}
    \Vtheta_{t+1} = \Vtheta_t + \epsilon^* \Vrho_t.
\end{equation}

和\gls{CG}法相似,\gls{BFGS}\,算法迭代一系列线搜索,其方向含二阶信息。
然而和\gls{CG}法不同的是,该方法的成功并不严重依赖于线搜索找到该方向上和真正极小值很近的一点。
因此,相比于\gls{CG}法,\gls{BFGS}\,的优点是其可以花费较少的时间改进每个线搜索。
在另一方面,\gls{BFGS}\,算法必须存储\,\gls{hessian}\,逆矩阵$\MM$,需要$O(n^2)$的存储空间,使\,\gls{BFGS}\,不适用于大多数具有百万级参数的现代\gls{DL}模型。

\paragraph{存储受限的\,\gls{BFGS}(或\,\gls{LBFGS})}
通过避免存储完整的\,\gls{hessian}\,逆近似$\MM$,\gls{BFGS}\,算法的存储代价可以显著降低。
\gls{LBFGS}\,算法使用和\,\gls{BFGS}\,算法相同的方法计算$\MM$的近似,但开始时假设上一步的$\MM$即$\MM^{(t-1)}$是单位矩阵,而不是每一步都为下一步存储近似。
如果使用精确的线搜索,\gls{LBFGS}\,定义的方向会是相互共轭的。
然而,不同于\gls{CG}法,即使线搜索只是找到近似的极小值点,该过程的效果仍然不错。
这里描述的无存储的\,\gls{LBFGS}\,方法可以拓展为包含\,\gls{hessian}\,矩阵更多的信息,每步存储一些用于更新$\MM$的向量,且每步的存储代价是$O(n)$。

% -- 308 --

\section{优化策略和元算法}
\label{sec:optimization_strategies_and_meta_algorithms}
许多优化技术并非真正的算法,而是一般化的模板,可以特殊化而产生算法,或是并入到很多不同的算法中。

\subsection{\glsentrytext{Batch_normalization}}
\label{sec:batch_normalization}
\gls{batch_normalization}~\citep{Ioffe+Szegedy-2015}是优化\gls{DNN}中最激动人心的最新创新之一。
实际上它并不是一个优化算法,而是一个自适应的重参数化的方法,试图解决训练非常深的模型的困难。

非常深的模型会涉及多个函数或层组合。
在其他层不改变的假设下,梯度用于如何更新每一个参数。
在实践中,我们同时更新所有层。
当我们进行更新时,可能会发生一些意想不到的结果,这是因为许多组合在一起的函数同时改变时,计算更新的假设是其他函数保持不变。
举一个简单的例子,假设我们有一个\gls{DNN},每一层只有一个单元,并且在每个\gls{hidden_layer}不使用\gls{activation_function}:$\hat{y} = xw_1 w_2 w_3 \dots w_l$。
此处,$w_i$表示用于层$i$的权重。层$i$的输出是$h_i = h_{i-1} w_i$。
输出$\hat{y}$是输入$x$的线性函数,但是权重$w_i$的非线性函数。
假设我们的\gls{cost_function} $\hat{y}$上的梯度为$1$,所以我们希望稍稍降低$\hat{y}$。
然后\gls{backward_propagation}算法可以计算梯度$\Vg = \nabla_{\Vw} \hat{y}$。
想想我们在更新$\Vw \leftarrow \Vw - \epsilon \Vg$时会发生什么。
近似$\hat{y}$的一阶泰勒级数会预测$\hat{y}$的值下降$\epsilon \Vg^\top \Vg$。
如果我们希望$\hat{y}$下降$0.1$,那么梯度中的一阶信息表明我们应设置\gls{learning_rate}$\epsilon$为$\frac{0.1}{\Vg^\top \Vg}$。
然而,实际的更新将包括二阶,三阶,直到$l$阶的影响。
$\hat{y}$的更新值为
\begin{equation}
    x(w_1-\epsilon g_1)(w_2-\epsilon g_2)\dots(w_l-\epsilon g_l),
\end{equation}
这个更新中所产生的一个二阶项示例是$\epsilon^2 g_1 g_2 \prod_{i=3}^l w_i$ 。
如果 $\prod_{i=3}^l w_i$很小,那么该项可以忽略不计。而如果层$3$到层$l$的权重都比$1$大时,该项可能会指数级大。
这使得我们很难选择一个合适的\gls{learning_rate},因为某一层中参数更新的效果很大程度上取决于其他所有层。
二阶优化算法通过考虑二阶相互影响来解决这个问题,但我们可以看到,在非常深的网络中,更高阶的相互影响会很显著。
即使是二阶优化算法,计算代价也很高,并且通常需要大量近似,以免真正计算所有的重要二阶相互作用。
因此对于$n>2$的情况,建立$n$阶优化算法似乎是无望的。
那么我们可以做些什么呢?

% -- 309 --

\gls{batch_normalization}提出了一种几乎可以重参数化所有\gls{deep_network}的优雅方法。
重参数化显著减少了多层之间协调更新的问题。
\gls{batch_normalization}可应用于网络的任何输入层或\gls{hidden_layer}。
设$\MH$是需要标准化的某层的\gls{minibatch}\gls{activation_function},排布为设计矩阵,每个样本的激活出现在矩阵的每一行中。
为了标准化$\MH$,我们将其替换为
\begin{equation}
\MH' = \frac{\MH - \Vmu}{\Vsigma},
\end{equation}
其中$\Vmu$是包含每个单元均值的向量,$\Vsigma$是包含每个单元标准差的向量。
此处的算术是基于广播向量$\Vmu$和向量$\Vsigma$应用于矩阵$\MH$的每一行。
在每一行内,运算是逐元素的,因此$H_{i,j}$标准化为减去$\mu_j$再除以$\sigma_j$。
网络的其余部分操作$\MH'$的方式和原网络操作$\MH$的方式一样。

在训练阶段,
\begin{equation}
    \Vmu = \frac{1}{m} \sum_i \MH_{i,:}
\end{equation}
和
\begin{equation}
    \Vsigma = \sqrt{ \delta + \frac{1}{m} \sum_i (\MH - \Vmu)_i^2 },
\end{equation}
其中$\delta$是个很小的正值,比如$10^{-8}$,以强制避免遇到$\sqrt{z}$的梯度在$z=0$处未定义的问题。
至关重要的是,\emph{我们反向传播这些操作},来计算均值和标准差,并应用它们于标准化$\MH$。
这意味着,梯度不会再简单地增加$h_i$的\gls{standard_deviation}或均值;标准化操作会除掉这一操作的影响,归零其在梯度中的元素。
这是\gls{batch_normalization}方法的一个重大创新。
以前的方法添加\gls{cost_function}的惩罚,以鼓励单元标准化激活统计量,或是在每个\gls{GD}步骤之后重新标准化单元统计量。
前者通常会导致不完全的标准化,而后者通常会显著地消耗时间,因为学习算法会反复改变均值和方差而标准化步骤会反复抵消这种变化。
\gls{batch_normalization}重参数化模型,以使一些单元在定义上就总是标准化的,巧妙地回避了这两个问题。

% -- 310 --

在测试阶段,$\Vmu$和$\Vsigma$可以被替换为训练阶段收集的运行均值。
这使得模型可以对单一样本评估,而无需使用定义于整个\gls{minibatch}的$\Vmu$和$\Vsigma$。

回顾例子$\hat{y} = x w_1 w_2 \dots w_l$,我们看到,我们可以通过标准化$h_{l-1}$很大程度地解决了学习这个模型的问题。
假设$x$采样自一个单位高斯分布,
那么$h_{l-1}$也是来自高斯分布,因为从$x$到$h_l$的变换是线性的。
然而,$h_{l-1}$不再有零均值和单位方差。
使用\gls{batch_normalization}后,我们得到的归一化$\hat{h}_{l-1}$恢复了零均值和单位方差的特性。
对于底层的几乎任意更新而言,$\hat{h}_{l-1}$仍然保持着单位高斯分布。
然后输出$\hat{y}$可以学习为一个简单的线性函数$\hat{y} = w_l \hat{h}_{l-1}$。
现在学习这个模型非常简单,因为低层的参数在大多数情况下没有什么影响;它们的输出总是重新标准化为单位高斯分布。
只在少数个例中,低层会有影响。
改变某个低层权重为$0$,可能使输出退化;改变低层权重的符号可能反转$\hat{h}_{l-1}$和$y$之间的关系。
这些情况都是非常罕见的。
没有标准化,几乎每一个更新都会对$h_{l-1}$的统计量有着极端的影响。
因此,\gls{batch_normalization}显著地使得模型更易学习。
在这个示例中,容易学习的代价是使得底层网络没有用。
在我们的线性示例中,较低层不再有任何有害的影响,但它们也不再有任何有益的影响。
这是因为我们已经标准化了一阶和二阶统计量,这是线性网络可以影响的所有因素。
在具有非线性\gls{activation_function}的\gls{DNN}中,较低层可以进行数据的非线性变换,所以它们仍然是有用的。
\gls{batch_normalization}仅标准化每个单元的均值和方差,以稳定化学习,但允许单元和单元之间的关系,以及单个单元的非线性统计量之间的关系发生变化。

由于网络的最后一层能够学习线性变换,实际上我们可能希望移除一层内单元之间的所有线性关系。
事实上,这是~\cite{Desjardins2015}中采用的方法,为\gls{batch_normalization}提供了灵感。
令人遗憾的是,消除所有的线性关联比标准化各个独立单元的均值和\gls{standard_deviation}代价更高,因此\gls{batch_normalization}仍是迄今最实用的方法。

% -- 311 --

标准化一个单元的均值和标准差会降低包含该单元的\gls{NN}的表达能力。
为了保持网络的表现力,通常会将\gls{batch}\gls{hidden_unit}激活$\MH$替换为$\gamma \MH' + \Vbeta$,而不是简单地使用标准化的$\MH'$。
变量$\Vgamma$和$\Vbeta$是允许新变量有任意均值和\gls{standard_deviation}的学习参数。
乍一看,这似乎是无用的——为什么我们将均值设为$0$,然后又引入参数允许它被重设为任意值$\Vbeta$?
答案是新的参数可以表示旧参数作为输入的同一族函数,但是新参数有不同的学习动态。
在旧参数中,$\MH$的均值取决于$\MH$下层中参数的复杂关联。
在新参数中,$\Vgamma \MH' + \Vbeta$的均值仅由$\Vbeta$确定。
新参数很容易通过\gls{GD}来学习。

大多数\gls{NN}层会采取$\phi(\MX\MW+ \Vb)$的形式,其中$\phi$是某个固定的非线性\gls{activation_function},如\gls{rectified_linear}变换。
自然想到我们应该将\gls{batch_normalization}应用于输入$\MX$还是变换后的值$\MX\MW+\Vb$。
\cite{Ioffe+Szegedy-2015}推荐后者。
更具体地,$\MX\MW+\Vb$应替换为$\MX\MW$的标准化形式。
偏置项应被忽略,因为参数$\Vbeta$会加入\gls{batch_normalization}重参数化,它是冗余的。
一层的输入通常是前一层的非线性\gls{activation_function}(如\gls{rectified_linear}函数)的输出。
因此,输入的统计量更符合非高斯分布,而更不服从线性操作的标准化。

\chapref{chap:convolutional_networks}所述的卷积网络,在特征映射中每个空间位置同样地标准化$\Vmu$和$\Vsigma$是很重要的,能使特征映射的统计量在不同的空间位置,仍然保持相同。

\subsection{坐标下降}
\label{sec:coordinate_descent}
在某些情况下,将一个优化问题分解成几个部分,可以更快地解决原问题。
如果我们相对于某个单一变量$x_i$最小化$f(\Vx)$,然后相对于另一个变量$x_j$等等,反复循环所有的变量,我们会保证到达(局部)极小值。
这种做法被称为\firstgls{coordinate_descent},因为我们一次优化一个坐标。
更一般地,\firstgls{block_coordinate_descent}是指对于某个子集的变量同时最小化。
术语``\gls{coordinate_descent}''通常既指块坐标下降,也指严格的单个坐标下降。

% -- 312 --

当优化问题中的不同变量能够清楚地分成相对独立的组,或是当优化一组变量明显比优化所有变量效率更高时,坐标下降最有意义。
例如,考虑\gls{cost_function}
\begin{equation}
    J(\MH, \MW) = \sum_{i,j} |H_{i,j}| + \sum_{i,j} \left( \MX - \MW^\top \MH \right)_{i,j}^2 .
\end{equation}
该函数描述了一种被称为\gls{sparse_coding}的学习问题,其目标是寻求一个权重矩阵$\MW$,可以线性解码激活值矩阵$\MH$以重构\gls{training_set}$\MX$。
\gls{sparse_coding}的大多数应用还涉及到\gls{weight_decay}或$\MW$列范数的约束,以避免极小$\MH$和极大$\MW$的病态解。

函数$J$不是凸的。
然而,我们可以将训练算法的输入分成两个集合:字典参数$\MW$和编码表示$\MH$。
最小化关于这两者之一的任意一组变量的\gls{objective_function}都是凸问题。
因此,块坐标下降允许我们使用高效的\gls{convex_optimization}算法,交替固定$\MH$优化$\MW$和固定$\MW$优化$\MH$。

当一个变量的值很大程度地影响另一个变量的最优值时,坐标下降不是一个很好的方法,如函数$f(\Vx)=(x_1 - x_2)^2+\alpha(x_1^2 + x_2^2)$,其中$\alpha$是正值常数。
第一项鼓励两个变量具有相似的值,而第二项鼓励它们接近零。
解是两者都为零。
\gls{newton_method}可以一步解决这个问题,因为它是一个正定二次问题。
但是,对于小值$\alpha$而言,坐标下降会使进展非常缓慢,因为第一项不允许单个变量变为和其他变量当前值显著不同的值。


\subsection{Polyak平均}
\label{sec:polyak_averaging}
Polyak平均\citep{Polyak+Juditsky-1992}会平均优化算法在参数空间访问轨迹中的几个点。
如果$t$次迭代\gls{GD}访问了点$\Vtheta^{(1)},\dots,\Vtheta^{(t)}$,那么Polyak平均算法的输出是$\hat{\Vtheta}^{(t)} = \frac{1}{t} \sum_i \Vtheta^{(i)}$。
在某些问题中,如\gls{GD}应用于凸问题时,这种方法具有较强的收敛保证。
当应用于\gls{NN}时,其验证更多是启发式的,但在实践中表现良好。
基本想法是,优化算法可能会来回穿过山谷好几次而没经过山谷底部附近的点。
尽管两边所有位置的均值应比较接近谷底。

% -- 313 --

在\gls{nonconvex}问题中,优化轨迹的路径可以非常复杂,并且经过了许多不同的区域。
包括参数空间中遥远过去的点,可能与当前点在\gls{cost_function}上相隔很大的障碍,看上去不像一个有用的行为。
其结果是,当应用Polyak平均于\gls{nonconvex}问题时,通常会使用指数衰减计算平均值:
\begin{equation}
    \hat{\Vtheta}^{(t)} = \alpha \hat{\Vtheta}^{(t-1)} + (1-\alpha) \Vtheta^{(t)} .
\end{equation}

这个计算平均值的方法被用于大量数值应用中。最近的例子请查看\cite{Szegedy-et-al-2015}。

\subsection{监督\glsentrytext{pretraining}}
\label{sec:supervised_pretraining}
有时,如果模型太复杂难以优化,或是如果任务非常困难,直接训练模型来解决特定任务的挑战可能太大。
有时训练一个较简单的模型来求解问题,然后使模型更复杂会更有效。
训练模型来求解一个简化的问题,然后转移到最后的问题,有时也会更有效些。
这些在直接训练目标模型求解目标问题之前,训练简单模型求解简化问题的方法统称为\firstgls{pretraining}。


\firstgls{greedy_algorithm}将问题分解成许多部分,然后独立地在每个部分求解最优值。
令人遗憾的是,结合各个最佳的部分不能保证得到一个最佳的完整解。
然而,贪心算法计算上比求解最优联合解的算法高效得多,并且贪心算法的解在不是最优的情况下,往往也是可以接受的。
贪心算法也可以紧接一个\firstgls{fine_tuning}阶段,联合优化算法搜索全问题的最优解。
使用贪心解初始化联合优化算法,可以极大地加速算法,并提高寻找到的解的质量。

\gls{pretraining}算法,特别是贪心\gls{pretraining},在\gls{DL}中是普遍存在的。
在本节中,我们会具体描述这些将监督学习问题分解成其他简化的监督学习问题的\gls{pretraining}算法。
这种方法被称为\firstgls{greedy_supervised_pretraining}。

% -- 314 --

在\gls{greedy_supervised_pretraining}的原始版本\citep{Bengio-nips-2006-short}中,每个阶段包括一个仅涉及最终\gls{NN}的子集层的监督学习训练任务。
\gls{greedy_supervised_pretraining}的一个例子如\figref{fig:chap8_deep_sup}所示,其中每个附加的\gls{hidden_layer}作为浅层监督\gls{MLP}的一部分\gls{pretraining},以先前训练的\gls{hidden_layer}输出作为输入。
\cite{Simonyan2015}~\gls{pretraining}深度卷积网络(11层权重),然后使用该网络前四层和最后三层初始化更深的网络(多达19层权重),并非一次\gls{pretraining}一层。
非常深的新网络的中间层是随机初始化的。
然后联合训练新网络。
还有一种选择,由\cite{Yu+al-2010}提出,将先前训练\gls{MLP}的\emph{输出},以及原始输入,作为每个附加阶段的输入。

\begin{figure}[!htb]
\ifOpenSource
\centerline{\includegraphics{figure.pdf}}
\else
\centerline{\includegraphics{Chapter8/figures/deep_sup}}
\fi
\caption{一种形式的\gls{greedy_supervised_pretraining}的示意图~\citep{Bengio-nips-2006-small}。
(a)我们从训练一个足够浅的架构开始。
(b)同一个架构的另一描绘。
(c)我们只保留原始网络的输入到隐藏层,并丢弃隐藏到输出层。 
我们将第一层\gls{hidden_layer}的输出作为输入发送到另一监督单隐层\,\glssymbol{MLP}(使用与第一个网络相同的目标训练),从而可以添加第二层\gls{hidden_layer}。 
这可以根据需要重复多层。
(d)所得架构的另一种描绘,可视为\gls{feedforward_network}。 
为了进一步改进优化,我们可以联合地\gls{fine_tuning}所有层(仅在该过程的结束或者该过程的每个阶段)。
}
\label{fig:chap8_deep_sup}
\end{figure}


为什么\gls{greedy_supervised_pretraining}会有帮助呢?
最初由~\cite{Bengio-nips-2006}提出的假说是,其有助于更好地指导深层结构的中间层的学习。
一般情况下,\gls{pretraining}对于优化和\gls{generalization}都是有帮助的。

另一个与监督\gls{pretraining}有关的方法扩展了迁移学习的想法:\cite{yosinski-nips2014}在一组任务上\gls{pretraining}了$8$层权重的深度卷积网络(1000个ImageNet对象类的子集),然而用该网络的前$k$层初始化同样规模的网络。
然后第二个网络的所有层(上层随机初始化)联合训练以执行不同的任务(1000个ImageNet对象类的另一个子集),但训练样本少于第一个任务。
\gls{NN}中另一个和迁移学习相关的方法将在\secref{sec:transfer_learning_and_domain_adaptation}讨论。

另一条相关的工作线是\textbf{FitNets}~\citep{Romero-et-al-ICLR2015-small}方法。
这种方法始于训练深度足够低和宽度足够大(每层单元数),容易训练的网络。
然后,这个网络成为第二个网络(被指定为\,\textbf{学生})的\,\textbf{老师}。
学生网络更深更窄(11至19层),且在正常情况下很难用SGD训练。
训练学生网络不仅需要预测原任务的输出,还需要预测教师网络中间层的值,这样使得训练学生网络变得更容易。
这个额外的任务说明了\gls{hidden_layer}应如何使用,并且能够简化优化问题。
附加参数被引入来从更深的学生网络中间层去回归$5$层教师网络的中间层。
然而,该目标是预测教师网络的中间\gls{hidden_layer},并非预测最终分类目标。
学生网络的低层因而具有两个目标:帮助学生网络的输出完成其目标和预测教师网络的中间层。
尽管一个窄而深的网络似乎比宽而浅的网络更难训练,但窄而深网络的泛化能力可能更好,
并且如果其足够窄,参数足够少,那么其计算代价更小。
没有\gls{hidden_layer}的提示,学生网络在\gls{training_set}和\gls{test_set}上的实验表现都很差。
因而中间层的提示是有助于训练很难训练的网络的方法之一,但是其他优化技术或是架构上的变化也可能解决这个问题。

% -- 316 --

\subsection{设计有助于优化的模型}
\label{sec:designing_models_to_aid_optimization}
改进优化的最好方法并不总是改进优化算法。
相反,\gls{deep_model}中优化的许多改进来自于设计易于优化的模型。

原则上,我们可以使用呈锯齿非单调模式上上下下的\gls{activation_function},但是,这将使优化极为困难。
在实践中,\emph{选择一族容易优化的模型比使用一个强大的优化算法更重要}。
\gls{NN}学习在过去30年的大多数进步主要来自于改变模型族,而非改变优化过程。
1980年代用于训练\gls{NN}的带\gls{momentum}的\gls{SGD},仍然是现代\gls{NN}应用中的前沿算法。

具体来说,现代\gls{NN}的\emph{设计选择}体现在层之间的线性变换,几乎处处可导的\gls{activation_function},和大部分定义域都有明显的梯度。
特别地,创新的模型,如\,\glssymbol{LSTM},\gls{ReLU}和\,\gls{maxout}\,单元都比先前的模型(如基于\,\gls{sigmoid}\,单元的\gls{deep_network})使用更多的线性函数。
这些模型都具有简化优化的性质。
如果线性变换的\,\gls{jacobian}\,具有相对合理的奇异值,那么梯度能够流经很多层。
此外,线性函数在一个方向上一致增加,所以即使模型的输出远离正确值,也可以简单清晰地计算梯度,使其输出方向朝降低\gls{loss_function}的方向移动。
换言之,现代\gls{NN}的设计方案旨在使其\emph{局部}梯度信息合理地对应着移向一个遥远的解。

其他的模型设计策略有助于使优化更简单。
例如,层之间的线性路径或是跳跃连接减少了从较低层参数到输出最短路径的长度,因而缓解了梯度消失的问题\citep{Srivastava-et-al-arxiv2015}。
一个和跳跃连接相关的想法是添加和网络中间\gls{hidden_layer}相连的输出的额外副本,如GoogLeNet~\citep{Szegedy-et-al-arxiv2014}和深度监督网络\citep{Lee-et-al-2014}。
这些``辅助头''被训练来执行和网络顶层主要输出相同的任务,以确保底层网络能够接受较大的梯度。
当训练完成时,辅助头可能被丢弃。
这是之前小节介绍到的\gls{pretraining}策略的替代方法。
以这种方式,我们可以在一个阶段联合训练所有层,而不改变架构,使得中间层(特别是低层)能够通过更短的路径得到一些如何更新的有用信息。
这些信息为底层提供了误差信号。

% -- 317 --

\subsection{\glsentrytext{continuation_method}和\glsentrytext{curriculum_learning}}
\label{sec:continuation_methods_and_curriculum_learning}
正如\secref{sec:poor_correspondence_between_local_and_global_structure}探讨的,许多优化挑战都来自于\gls{cost_function}的全局结构,不能仅通过局部更新方向上更好的估计来解决。
解决这个问题的主要方法是尝试初始化参数到某种区域内,该区域可以通过局部下降很快连接到参数空间中的解。


\firstgls{continuation_method}是一族通过挑选初始点使优化更容易的方法,以确保局部优化花费大部分时间在表现良好的空间。
\gls{continuation_method}的背后想法是构造一系列具有相同参数的\gls{objective_function}。
为了最小化\gls{cost_function} $J(\Vtheta)$,我们构建新的\gls{cost_function} $\{J^{(0)},\dots,J^{(n)}\}$。
这些\gls{cost_function}的难度逐步提高,其中$J^{(0)}$是最容易最小化的,$J^{(n)}$是最难的,真正的\gls{cost_function}驱动整个过程。
当我们说$J^{(i)}$比$J^{(i+1)}$更容易时,是指其在更多的$\Vtheta$空间上表现良好。
随机初始化更有可能落入局部下降可以成功最小化\gls{cost_function}的区域,因为其良好区域更大。
这系列\gls{cost_function}设计为前一个解是下一个的良好初始点。
因此,我们首先解决一个简单的问题,然后改进解以解决逐步变难的问题,直到我们求解真正问题的解。

传统的\gls{continuation_method}(用于\gls{NN}训练之前的\gls{continuation_method})通常基于平滑\gls{objective_function}。
读者可以查看~\cite{Wu-97}了解这类方法的示例,以及一些相关方法的综述。
\gls{continuation_method}也和参数中加入\gls{noise}的模拟退火紧密相关\citep{Kirkpatrick83}。
\gls{continuation_method}在最近几年非常成功。
参考~\cite{Mobahi+Fisher-AAAI2015}了解近期文献的概述,特别是在~\glssymbol{AI}~方面的应用。

% -- 318 --

传统上,\gls{continuation_method}主要用来克服\gls{local_minima}的问题。
具体地,它被设计来在有很多\gls{local_minima}的情况下,求解一个\gls{global_minimum}。
这些连续方法会通过``模糊''原来的\gls{cost_function}来构建更容易的\gls{cost_function}。
这些模糊操作可以是用采样来近似
\begin{equation}
    J^{(i)}(\Vtheta) = \SetE_{\theta' \sim \mathcal{N}(\Vtheta'; \Vtheta, \sigma^{(i)2})} J(\Vtheta')
\end{equation}
这个方法的直觉是有些\gls{nonconvex}函数在模糊后会近似凸的。
在许多情况下,这种模糊保留了关于全局极小值的足够信息,我们可以通过逐步求解模糊更少的问题来求解全局极小值。
这种方法有三种可能失败的方式。
首先,它可能成功地定义了一连串\gls{cost_function},并从开始的一个凸函数起(逐一地)沿着函数链最佳轨迹逼近全局最小值,但可能需要非常多的逐步\gls{cost_function},整个过程的成本仍然很高。
另外,即使\gls{continuation_method}可以适用,NP-hard的优化问题仍然是NP-hard。
其他两种\gls{continuation_method}失败的原因是不实用。
其一,不管如何模糊,函数都没法变成凸的,比如函数$J(\Vtheta) = -\Vtheta^\top \Vtheta$。
其二,函数可能在模糊后是凸的,但模糊函数的最小值可能会追踪到一个局部最小值,而非原始\gls{cost_function}的全局最小值。


尽管\gls{continuation_method}最初用来解决局部最小值的问题,而局部最小值已不再认为是\gls{NN}优化中的主要问题了。
幸运的是,\gls{continuation_method}仍然有所帮助。
\gls{continuation_method}引入的简化\gls{objective_function}能够消除平坦区域,减少梯度估计的方差,提高\,\gls{hessian}\,矩阵的条件数,使局部更新更容易计算,或是改进局部更新方向与朝向全局解方向之间的对应关系。

\cite{Bengio+al-2009}指出被称为\firstgls{curriculum_learning}或者\firstgls{shaping}的方法可以被解释为\gls{continuation_method}。
\gls{curriculum_learning}基于规划学习过程的想法,首先学习简单的概念,然后逐步学习依赖于这些简化概念的复杂概念。
之前这一基本策略被用来加速动物训练过程\citep{Skinner1958,Peterson2004,Krueger+Dayan-2009}和\gls{ML}过程\citep{solomonoff1989system,Elman93,Sanger-1994}。
\cite{Bengio+al-2009}验证这一策略为\gls{continuation_method},通过增加简单样本的影响(通过分配它们较大的系数到\gls{cost_function},或者更频繁地采样),先前的$J^{(i)}$会变得更容易。
实验证明,在大规模的神经语言模型任务上使用\gls{curriculum_learning},可以获得更好的结果。
课程学习已经成功应用于大量的自然语言\citep{Spitkovsky-et-al-HLT2010,collobert2011natural,Mikolov-ASRU-2011,Tu+Honavar-IJCAI2011}和计算机视觉\citep{Kumar+al-2010,Lee+Grauman-CVPR2011,Supancic+Ramanan-CVPR2013}任务上。
\gls{curriculum_learning}被证实为与人类教学方式一致\citep{Khan+Zhu+Mutlu-2011}:
教师刚开始会展示更容易、更典型的示例,然后帮助学习者在不太显然的情况下提炼决策面。
在人类教学上,基于\gls{curriculum_learning}的策略比基于样本均匀采样的策略\emph{更有效},也能提高其他学习策略的效率\citep{Basu+Christensen-AAAI2013}。

% -- 319 --

\gls{curriculum_learning}研究的另一个重要贡献体现在训练循环\gls{NN}捕获长期依赖:
\cite{Zaremba+Sutskever-arxiv2014}发现使用\emph{\gls{stochastic_curriculum}}获得了更好的结果,其中容易和困难的示例混合在一起,随机提供给学习者,更难示例(这些具有长期依赖)的平均比例在逐渐上升。
而使用确定性课程,并没有发现超过基线(完整\gls{training_set}的普通训练)的改进。

现在我们已经介绍了一些基本的\gls{NN}模型,以及如何进行正则化和优化。
在接下来的章节中,我们转向特化的\gls{NN}家族,允许其扩展到能够处理很大规模的数据和具有特殊结构的数据。
在本章中讨论的优化算法在较少改动后或者无需改动,通常就可以直接用于这些特化的架构。

% -- 320 --
 
% !Mode:: "TeX:UTF-8"
% Translator: Kai Li
\chapter{\glsentrytext{convolutional_network}}
\label{chap:convolutional_networks}

\firstgls{convolutional_network}\citep{LeCun89a},也叫做\firstall{CNN},是一种专门用来处理具有类似网格结构的数据的神经网络。
例如时间序列数据(可以认为是在时间轴上有规律地采样形成的一维网格)和图像数据(可以看作是二维的像素网格)。
\gls{convolutional_network}在诸多应用领域都表现优异。
``\gls{CNN}''一词表明该网络使用了\firstgls{convolution}这种数学运算。
卷积是一种特殊的线性运算。
\emph{\gls{convolutional_network}是指那些至少在网络的一层中使用卷积运算来替代一般的矩阵乘法运算的神经网络。}

本章,我们首先说明什么是卷积运算。
接着,我们会解释在神经网络中使用卷积运算的动机。
然后我们会介绍\firstgls{pooling},这是一种几乎所有的\gls{convolutional_network}都会用到的操作。
通常来说,\gls{CNN}中用到的卷积运算和其他领域(例如工程领域以及纯数学领域)中的定义并不完全一致。
我们会对神经网络实践中广泛应用的几种卷积函数的变体进行说明。
我们也会说明如何在多种不同维数的数据上使用卷积运算。
之后我们讨论使得卷积运算更加高效的一些方法。
\gls{convolutional_network}是神经科学原理影响深度学习的典型代表。
我们之后也会讨论这些神经科学的原理,并对\gls{convolutional_network}在深度学习发展史中的作用作出评价。
本章没有涉及如何为你的\gls{convolutional_network}选择合适的结构,因为本章的目标是说明\gls{convolutional_network}提供的各种工具。
\chapref{chap:practical_methodology}将会对如何在具体环境中选择使用相应的工具给出通用的准则。
对于\gls{convolutional_network}结构的研究进展得如此迅速,以至于针对特定基准(benchmark),数月甚至几周就会公开一个新的最优的网络结构,甚至在写这本书时也不好描述究竟哪种结构是最好的。
然而,最好的结构也是由本章所描述的基本部件逐步搭建起来的。

% -- 321 --
 
\section{卷积运算}
\label{sec:the_convolution_operation}

在通常形式中,卷积是对两个实变函数的一种数学运算\footnote{译者注:本书中operation视语境有时翻译成``运算'',有时翻译成``操作''。}。
为了给出卷积的定义,我们从两个可能会用到的函数的例子出发。

假设我们正在用激光传感器追踪一艘宇宙飞船的位置。
我们的激光传感器给出一个单独的输出$x(t)$,表示宇宙飞船在时刻$t$ 的位置。
$x$和$t$都是实值的,这意味着我们可以在任意时刻从传感器中读出飞船的位置。

现在假设我们的传感器受到一定程度的噪声干扰。
为了得到飞船位置的低噪声估计,我们对得到的测量结果进行平均。
显然,时间上越近的测量结果越相关,所以我们采用一种加权平均的方法,对于最近的测量结果赋予更高的权重。
我们可以采用一个加权函数$w(a)$ 来实现,其中$a$表示测量结果距当前时刻的时间间隔。
如果我们对任意时刻都采用这种加权平均的操作,就得到了一个新的对于飞船位置的平滑估计函数$s$:
\begin{equation}
s(t) = \int x(a)w(t-a)da.
\end{equation}

这种运算就叫做\firstgls{convolution}。
卷积运算通常用星号表示:
\begin{equation}
s(t) = (x*w)(t).
\end{equation}

在我们的例子中,$w$必须是一个有效的概率密度函数,否则输出就不再是一个加权平均。
另外,在参数为负值时,$w$的取值必须为0,否则它会预测到未来,这不是我们能够推测得了的。
但这些限制仅仅是对我们这个例子来说。
通常,卷积被定义在满足上述积分式的任意函数上,并且也可能被用于加权平均以外的目的。

在\gls{convolutional_network}的术语中,卷积的第一个参数(在这个例子中,函数$x$)通常叫做\firstgls{input},第二个参数(函数$w$)叫做\firstgls{kernel}。
输出有时被称作\firstgls{feature_map}。

% -- 322 --
 
在本例中,激光传感器在每个瞬间反馈测量结果的想法是不切实际的。
一般地,当我们用计算机处理数据时,时间会被离散化,传感器会定期地反馈数据。
所以在我们的例子中,假设传感器每秒反馈一次测量结果是比较现实的。
这样,时刻$t$只能取整数值。
如果我们假设$x$和$w$都定义在整数时刻$t$上,就可以定义离散形式的卷积:
\begin{equation}
s(t) = (x*w)(t) = \sum_{a = -\infty}^{\infty} x(a)w(t-a).
\end{equation}

在机器学习的应用中,输入通常是多维数组的数据,而核通常是由学习算法优化得到的多维数组的参数。
我们把这些多维数组叫做张量。
因为在输入与核中的每一个元素都必须明确地分开存储,我们通常假设在存储了数值的有限点集以外,这些函数的值都为零。
这意味着在实际操作中,我们可以通过对有限个数组元素的求和来实现无限求和。

最后,我们经常一次在多个维度上进行卷积运算。
例如,如果把一张二维的图像$I$作为输入,我们也许也想要使用一个二维的核$K$:
\begin{equation}
S(i,j) = (I*K)(i,j) = \sum_m \sum_n I(m,n) K(i-m, j-n).
\end{equation}

卷积是可交换的(commutative),我们可以等价地写作:
\begin{equation}
S(i, j) = (K*I)(i,j) = \sum_m \sum_n I(i-m, j-n) K(m, n).
\end{equation}

通常,下面的公式在机器学习库中实现更为简单,因为$m$和$n$的有效取值范围相对较小。

% -- 323 --
 
卷积运算可交换性的出现是因为我们将核相对输入进行了\firstgls{flip},从$m$增大的角度来看,输入的索引在增大,但是核的索引在减小。
我们将核翻转的唯一目的是实现可交换性。
尽管可交换性在证明时很有用,但在神经网络的应用中却不是一个重要的性质。
与之不同的是,许多神经网络库会实现一个相关的函数,称为\firstgls{cross_correlation},和卷积运算几乎一样但是并没有对核进行翻转:
\begin{equation}
S(i, j) = (I*K)(i, j) = \sum_m \sum_n I(i+m, j+n) K(m, n).
\end{equation}
许多机器学习的库实现的是互相关函数但是称之为卷积。
在这本书中我们遵循把两种运算都叫做卷积的这个传统,在与核翻转有关的上下文中,我们会特别指明是否对核进行了翻转。
在机器学习中,学习算法会在核合适的位置学得恰当的值, 所以一个基于核翻转的卷积运算的学习算法所学得的核,是对未进行翻转的算法学得的核的翻转。
单独使用卷积运算在机器学习中是很少见的,卷积经常与其他的函数一起使用,无论卷积运算是否对它的核进行了翻转,这些函数的组合通常是不可交换的。

\figref{fig:chap9_conv_2d}演示了一个在2维张量上的卷积运算(没有对核进行翻转)的例子。
% fig 9.1
\begin{figure}[!htb]
\ifOpenSource
\centerline{\includegraphics{figure.pdf}}
\else
\centerline{\includegraphics{Chapter9/figures/conv_2d}}
\fi
\caption{一个2维卷积的例子(没有对核进行翻转)。
我们限制只对核完全处在图像中的位置进行输出,在一些上下文中称为``\gls{valid}''卷积。
我们用画有箭头的盒子来说明输出张量的左上角元素是如何通过对输入张量相应的左上角区域应用核进行卷积得到的。}
\label{fig:chap9_conv_2d}
\end{figure}

离散卷积可以看作矩阵的乘法,然而,这个矩阵的一些元素被限制为必须和另外一些元素相等。
例如对于单变量的离散卷积,矩阵每一行中的元素都与上一行对应位置平移一个单位的元素相同。
这种矩阵叫做\firstgls{Toeplitz_matrix}。
对于二维情况,卷积对应着一个\firstgls{doubly_block_circulant_matrix}。
%这里不是很理解双重分块循环矩阵
除了这些元素相等的限制以外,卷积通常对应着一个非常稀疏的矩阵(一个几乎所有元素都为零的矩阵)。
这是因为核的大小通常要远小于输入图像的大小。任何一个使用矩阵乘法但是并不依赖矩阵结构的特殊性质的神经网络算法,都适用于卷积运算,并且不需要对神经网络做出大的修改。
典型的\gls{CNN}为了更有效地处理大规模输入,确实使用了一些专门化的技巧,但这些在理论分析方面并不是严格必要的。

% -- 324 --
 
\section{动机}
\label{sec:motivation}

卷积运算通过三个重要的思想来帮助改进机器学习系统:\firstgls{sparse_interactions}、\firstgls{parameter_sharing}、\firstgls{equivariant_representations}。
另外,卷积提供了一种处理大小可变的输入的方法。
我们下面依次介绍这些思想。

传统的神经网络使用矩阵乘法来建立输入与输出的连接关系。%\footnote{译者注:这里可以粗略地理解为输入$\times$参数矩阵=输出。}
其中,参数矩阵中每一个单独的参数都描述了一个输入单元与一个输出单元间的交互。
这意味着每一个输出单元与每一个输入单元都产生交互。
然而,\gls{convolutional_network}具有\firstgls{sparse_interactions}(也叫做\firstgls{sparse_connectivity}或者\firstgls{sparse_weights})的特征。
这是使核的大小远小于输入的大小来达到的。
举个例子,当处理一张图像时,输入的图像可能包含成千上万个像素点,但是我们可以通过只占用几十到上百个像素点的核来检测一些小的有意义的特征,例如图像的边缘。
这意味着我们需要存储的参数更少,不仅减少了模型的存储需求,而且提高了它的统计效率。
这也意味着为了得到输出我们只需要更少的计算量。
这些效率上的提高往往是很显著的。
如果有$m$个输入和$n$个输出,那么矩阵乘法需要$m \times n$个参数并且相应算法的时间复杂度为$O(m\times n)$(对于每一个例子)。
如果我们限制每一个输出拥有的连接数为$k$,那么稀疏的连接方法只需要$k\times n$个参数以及$O(k\times n)$的运行时间。
在很多实际应用中,只需保持$k$比$m$小几个数量级,就能在机器学习的任务中取得好的表现。
\gls{sparse_connectivity}的图形化解释如\figref{fig:chap9_area_of_effect}和\figref{fig:chap9_receptive_field}所示。
在深度\gls{convolutional_network}中,处在网络深层的单元可能与绝大部分输入是\emph{间接}交互的,如\figref{fig:chap9_deep_receptive_field}所示。
这允许网络可以通过只描述\gls{sparse_interactions}的基石来高效地描述多个变量的复杂交互。
% fig 9.2
\begin{figure}[!htb]
\ifOpenSource
\centerline{\includegraphics{figure.pdf}}
\else
\centerline{\includegraphics{Chapter9/figures/area_of_effect}}
\fi
\captionsetup{singlelinecheck=off}
\caption[Caption for LOF]{\gls{sparse_connectivity},对每幅图从下往上看。
我们强调了一个输入单元$x_3$以及在$\Vs$中受该单元影响的输出单元。
\emph{(上)}当$\Vs$是由核宽度为3的卷积产生时,只有三个输出受到$\Vx$的影响\protect\footnotemark。
\emph{(下)}当$\Vs$是由矩阵乘法产生时,连接不再是稀疏的,所以所有的输出都会受到$x_3$的影响。}
\label{fig:chap9_area_of_effect}
\end{figure}
% \footnotetext{译者注:译者认为此处应当是$x_3$。}

% fig 9.3
\begin{figure}[!htb]
\ifOpenSource
\centerline{\includegraphics{figure.pdf}}
\else
\centerline{\includegraphics{Chapter9/figures/receptive_field}}
\fi
\captionsetup{singlelinecheck=off}
\caption[Caption for LOF]{\gls{sparse_connectivity},对每幅图从上往下看。
我们强调了一个输出单元$s_3$以及$\Vx$中影响该单元的输入单元。
这些单元被称为$s_3$的\firstgls{receptive_field}\protect\footnotemark。
\emph{(上)}当$\Vs$是由核宽度为3的卷积产生时,只有三个输入影响$s_3$。
\emph{(下)}当$\Vs$是由矩阵乘法产生时,连接不再是稀疏的,所以所有的输入都会影响$s_3$。}
\label{fig:chap9_receptive_field}
\end{figure}
%\footnotetext{译者注:在生物中称之为``感受野''。}

% fig 9.4
\begin{figure}[!htb]
\ifOpenSource
\centerline{\includegraphics{figure.pdf}}
\else
\centerline{\includegraphics{Chapter9/figures/deep_receptive_field}}
\fi
\caption{处于\gls{convolutional_network}更深的层中的单元,它们的\gls{receptive_field}要比处在浅层的单元的\gls{receptive_field}更大。
如果网络还包含类似\gls{stride}卷积(\figref{fig:chap9_stride_conv})或者\gls{pooling}(\secref{sec:pooling})之类的结构特征,这种效应会加强。
这意味着在\gls{convolutional_network}中尽管\emph{直接}连接都是很稀疏的,但处在更深的层中的单元可以\emph{间接地}连接到全部或者大部分输入图像。}
\label{fig:chap9_deep_receptive_field}
\end{figure}

% -- 325 --
 
\firstgls{parameter_sharing}是指在一个模型的多个函数中使用相同的参数。
在传统的神经网络中,当计算一层的输出时,权重矩阵的每一个元素只使用一次,当它乘以输入的一个元素后就再也不会用到了。
作为\gls{parameter_sharing}的同义词,我们可以说一个网络含有\firstgls{tied_weights},因为用于一个输入的权重也会被绑定在其他的权重上。
在\gls{CNN}中,核的每一个元素都作用在输入的每一位置上(是否考虑边界像素取决于对边界决策的设计)。
卷积运算中的\gls{parameter_sharing}保证了我们只需要学习一个参数集合,而不是对于每一位置都需要学习一个单独的参数集合。
这虽然没有改变\gls{forward_propagation}的运行时间(仍然是$O(k\times n)$),但它显著地把模型的存储需求降低至$k$个参数,并且$k$通常要比$m$小很多个数量级。
因为$m$ 和$n$通常有着大致相同的大小,$k$在实际中相对于$m\times n$是很小的。
因此,卷积在存储需求和统计效率方面极大地优于稠密矩阵的乘法运算。
\figref{fig:chap9_parameter_sharing}演示了\gls{parameter_sharing}是如何实现的。
% fig 9.5
\begin{figure}[!htb]
\ifOpenSource
\centerline{\includegraphics{figure.pdf}}
\else
\centerline{\includegraphics{Chapter9/figures/parameter_sharing}}
\fi
\caption{\gls{parameter_sharing}。
黑色箭头表示在两个不同的模型中使用了特定参数的连接。
\emph{(上)}黑色箭头表示在卷积模型中对3元素核的中间元素的使用。
因为\gls{parameter_sharing},这个单独的参数被用于所有的输入位置。
\emph{(下)}这个单独的黑色箭头表示在全连接模型中对权重矩阵的中间元素的使用。
这个模型没有使用\gls{parameter_sharing},所以参数只使用了一次。}
\label{fig:chap9_parameter_sharing}
\end{figure}
% -- 326 --
 
% -- 327 --
 
作为前两条原则的一个实际例子,\figref{fig:chap9_efficiency_of_edge_detection}说明了\gls{sparse_connectivity}和\gls{parameter_sharing}是如何显著提高线性函数在一张图像上进行边缘检测的效率的。
% fig 9.6
\begin{figure}
\ifOpenSource
\centerline{\includegraphics{figure.pdf}}
\else
\centering    
\subfigure{ \label{fig:chap9_efficiency_of_edge_detection_a}     
\includegraphics[width=0.35\textwidth]{Chapter9/figures/sundance.png}}     
\subfigure{ \label{fig:chap9_efficiency_of_edge_detection_b}     
\includegraphics[width=0.35\textwidth]{Chapter9/figures/edges.png}}     
\fi
\captionsetup{singlelinecheck=off}
\caption{边缘检测的效率。
右边的图像是通过先获得原始图像中的每个像素,然后减去左边相邻像素的值而形成的。
这个操作给出了输入图像中所有垂直方向上的边缘的强度,对目标检测来说是有用的。
两个图像的高度均为280个像素。
输入图像的宽度为320个像素,而输出图像的宽度为319个像素。
这个变换可以通过包含两个元素的卷积核来描述,使用卷积需要$319\times 280\times 3 = 267,960$次浮点运算(每个输出像素需要两次乘法和一次加法)。
为了用矩阵乘法描述相同的变换,需要一个包含$320\times 280\times 319\times 280$个或者说超过80亿个元素的矩阵,这使得卷积对于表示这种变换更有效40亿倍。
直接运行矩阵乘法的算法将执行超过160亿次浮点运算,这使得卷积在计算上大约有60,000倍的效率。
当然,矩阵的大多数元素将为零。
如果我们只存储矩阵的非零元,则矩阵乘法和卷积都需要相同数量的浮点运算来计算。
矩阵仍然需要包含$2\times 319\times 280=178,640$个元素。
将小的局部区域上的相同线性变换应用到整个输入上,卷积是描述这种变换的极其有效的方法。
照片来源:Paula Goodfellow。}   
\label{fig:chap9_efficiency_of_edge_detection}     
\end{figure}
% -- 328 --
 
对于卷积,\gls{parameter_sharing}的特殊形式使得神经网络层具有对平移\firstgls{equivariance}的性质。
如果一个函数满足输入改变,输出也以同样的方式改变这一性质,我们就说它是等变(equivariant)的。
特别地,如果函数$f(x)$与$g(x)$满足$f(g(x))= g(f(x))$,我们就说$f(x)$对于变换$g$具有等变性。
对于卷积来说,如果令$g$是输入的任意平移函数,那么卷积函数对于$g$具有等变性。
举个例子,令$I$表示图像在整数坐标上的亮度函数,$g$表示图像函数的变换函数(把一个图像函数映射到另一个图像函数的函数)使得$I' = g(I)$,其中图像函数$I'$满足$I'(x,y) = I(x-1, y)$。
这个函数把$I$中的每个像素向右移动一个单位。
如果我们先对$I$进行这种变换然后进行卷积操作所得到的结果,与先对$I$进行卷积然后再对输出使用平移函数$g$得到的结果是一样的\footnote{译者注:原文将此处误写成了$I'$。} 。%译者注
当处理时间序列数据时,这意味着通过卷积可以得到一个由输入中出现不同特征的时刻所组成的时间轴。
如果我们把输入中的一个事件向后延时,在输出中仍然会有完全相同的表示,只是时间延后了。
图像与之类似,卷积产生了一个2维映射来表明某些特征在输入中出现的位置。
如果我们移动输入中的对象,它的表示也会在输出中移动同样的量。
当我们知道某个相邻小规模像素的函数在不同的位置都是有用的的时候,前述性质是有用的。
例如在处理图像时,在\gls{convolutional_network}的第一层进行图像的边缘检测是很有用的。
相同的边缘或多或少地散落在图像的各处,所以应当对整个图像进行\gls{parameter_sharing}。
但在某些情况下,我们并不希望对整幅图进行\gls{parameter_sharing}。
例如,在处理已经通过剪裁而使其居中的人脸图像时,我们可能想要提取不同位置上的不同特征(处理人脸上部的部分网络需要去搜寻眉毛,处理人脸下部的部分网络就需要去搜寻下巴了)。

% -- 329 --
 
卷积对其他的一些变换并不是天然等变的,例如对于图像的放缩或者旋转变换,需要其他的一些机制来处理这些变换。

最后,一些不能被传统的由(固定大小的)矩阵乘法定义的神经网络处理的特殊数据,可能通过卷积神经网络来处理,我们将在\secref{sec:data_types}中进行讨论。

\section{\glsentrytext{pooling}}
\label{sec:pooling}

\gls{convolutional_network}中一个典型层包含三级(如\figref{fig:chap9_conv_layer}所示)。
在第一级中,这一层并行地计算多个卷积产生一组线性激活响应。
在第二级中,每一个线性激活响应将会通过一个非线性的激活函数,例如\gls{rectified_linear}激活函数。
这一级有时也被称为\firstgls{detector_stage}。
在第三级中,我们使用\firstgls{pooling_funciton}来进一步调整这一层的输出。
% fig 9.7
\begin{figure}[!htb]
\ifOpenSource
\centerline{\includegraphics{figure.pdf}}
\else
\centerline{\includegraphics{Chapter9/figures/conv_layer}}
\fi
\caption{一个典型\gls{CNN}层的组件。
有两组常用的术语用于描述这些层。
\emph{(左)}在这组术语中,\gls{convolutional_network}被视为少量相对复杂的层,每层具有许多``级''。
在这组术语中,核张量与网络层之间存在一一对应关系。
在本书中,我们通常使用这组术语。
\emph{(右)}在这组术语中,\gls{convolutional_network}被视为更多数量的简单层;每一个处理步骤都被认为是一个独立的层。
这意味着不是每一``层''都有参数。}
\label{fig:chap9_conv_layer}
\end{figure}

\gls{pooling}函数使用某一位置的相邻输出的总体统计特征来代替网络在该位置的输出。
例如,\firstgls{max_pooling}函数\citep{zhou1988computation}给出相邻矩形区域内的最大值。
其他常用的\gls{pooling}函数包括相邻矩形区域内的平均值、$L^2$范数以及基于据中心像素距离的加权平均函数。

% -- 330 --
 
不管采用什么样的\gls{pooling}函数,当输入作出少量平移时,\gls{pooling}能够帮助输入的表示近似\firstgls{invariant}。
对于平移的不变性是指当我们对输入进行少量平移时,经过\gls{pooling}函数后的大多数输出并不会发生改变。
\figref{fig:chap9_max_pool_invariance}用了一个例子来说明这是如何实现的。
\emph{局部平移不变性是一个很有用的性质,尤其是当我们关心某个特征是否出现而不关心它出现的具体位置时}。
例如,当判定一张图像中是否包含人脸时,我们并不需要知道眼睛的精确像素位置,我们只需要知道有一只眼睛在脸的左边,有一只在右边就行了。
但在一些其他领域,保存特征的具体位置却很重要。
例如当我们想要寻找一个由两条边相交而成的拐角时,我们就需要很好地保存边的位置来判定它们是否相交。
% fig 9.8
\begin{figure}[!htb]
\ifOpenSource
\centerline{\includegraphics{figure.pdf}}
\else
\centerline{\includegraphics{Chapter9/figures/max_pool_invariance}}
\fi
\caption{\gls{max_pooling}引入了不变性。
\emph{(上)}卷积层中间输出的视图。
下面一行显示非线性的输出。
上面一行显示\gls{max_pooling}的输出,每个\gls{pool}的宽度为三个像素并且\gls{pooling}区域的\gls{stride}为一个像素。
\emph{(下)}相同网络的视图,不过对输入右移了一个像素。
下面一行的所有值都发生了改变,但上面一行只有一半的值发生了改变,这是因为\gls{max_pooling}单元只对周围的最大值比较敏感,而不是对精确的位置。}
\label{fig:chap9_max_pool_invariance}
\end{figure}

% -- 331 --
 
使用\gls{pooling}可以看作是增加了一个无限强的先验:这一层学得的函数必须具有对少量平移的不变性。
当这个假设成立时,\gls{pooling}可以极大地提高网络的统计效率。

对空间区域进行\gls{pooling}产生了平移不变性,但当我们对分离参数的卷积的输出进行\gls{pooling}时,特征能够学得应该对于哪种变换具有不变性(如\figref{fig:chap9_learned_rotation}所示)。
% fig 9.9
\begin{figure}[!htb]
\ifOpenSource
\centerline{\includegraphics{figure.pdf}}
\else
\centerline{\includegraphics{Chapter9/figures/learned_rotation}}
\fi
\caption{学习不变性的示例。
使用分离的参数学得多个特征,再使用\gls{pooling}单元进行\gls{pooling},可以学得对输入的某些变换的不变性。
这里我们展示了用三个学得的过滤器和一个\gls{max_pooling}单元可以学得对旋转变换的不变性。
这三个过滤器都旨在检测手写的数字5。
每个过滤器尝试匹配稍微不同方向的5。
当输入中出现5时,相应的过滤器会匹配它并且在探测单元中引起大的激活。
然后,无论哪个探测单元被激活,\gls{max_pooling}单元都具有大的激活。
我们在这里演示了网络如何处理两个不同的输入,这导致两个不同的探测单元被激活,然而对\gls{pooling}单元的影响大致相同。
这个原则在\,\gls{maxout}\,网络\citep{Goodfellow-et-al-ICML2013}和其他卷积网络中更有影响。
空间位置上的\gls{max_pooling}对于平移是天然不变的;这种多通道方法只在学习其他变换时是必要的。}
\label{fig:chap9_learned_rotation}
\end{figure}

% -- 332 --
 
因为\gls{pooling}综合了全部邻居的反馈,这使得\gls{pooling}单元少于探测单元成为可能,我们可以通过综合\gls{pooling}区域的$k$个像素的统计特征而不是单个像素来实现。
\figref{fig:chap9_pool_downsample}给出了一个例子。
这种方法提高了网络的计算效率,因为下一层少了约$k$ 倍的输入。
当下一层的参数数目是关于那一层输入大小的函数时(例如当下一层是全连接的基于矩阵乘法的网络层时),这种对于输入规模的减小也可以提高统计效率并且减少对于参数的存储需求。
% fig 9.10
\begin{figure}[!htb]
\ifOpenSource
\centerline{\includegraphics{figure.pdf}}
\else
\centerline{\includegraphics{Chapter9/figures/pool_downsample}}
\fi
\caption{带有\gls{downsampling}的\gls{pooling}。
这里我们使用\gls{max_pooling},\gls{pool}的宽度为三并且\gls{pool}之间的\gls{stride}为二。
这使得表示的大小减少了一半,减轻了下一层的计算和统计负担。
注意到最右边的\gls{pooling}区域尺寸较小,但如果我们不想忽略一些探测单元的话就必须包含这个区域。}
\label{fig:chap9_pool_downsample}
\end{figure}

% -- 333 --
 
在很多任务中,\gls{pooling}对于处理不同大小的输入具有重要作用。
例如我们想对不同大小的图像进行分类时,分类层的输入必须是固定的大小,而这通常通过调整\gls{pooling}区域的偏置大小来实现,这样分类层总是能接收到相同数量的统计特征而不管最初的输入大小了。
例如,最终的\gls{pooling}层可能会输出四组综合统计特征,每组对应着图像的一个象限,而与图像的大小无关。

一些理论工作对于在不同情况下应当使用哪种\gls{pooling}函数给出了一些指导\citep{boureau-icml-10}。
将特征一起动态地\gls{pooling}也是可行的,例如,对于感兴趣特征的位置运行聚类算法\citep{boureau-iccv-11}。
这种方法对于每幅图像产生一个不同的\gls{pooling}区域集合。
另一种方法是先\emph{学习}一个单独的\gls{pooling}结构,再应用到全部的图像中\citep{jia2012beyond}。

\gls{pooling}可能会使得一些利用自顶向下信息的神经网络结构变得复杂,例如\gls{BM}和\gls{AE}。
这些问题将在\chapref{part:deep_learning_research}中当我们遇到这些类型的网络时进一步讨论。
\gls{CBM}中的\gls{pooling}出现在\secref{sec:convolutional_boltzmann_machines}。
一些可微网络中需要的在\gls{pooling}单元上进行的类逆运算将在\secref{sec:convolutional_generative_networks}中讨论。

\figref{fig:chap9_cnn_classifier}给出了一些使用卷积和\gls{pooling}操作的用于分类的完整\gls{convolutional_network}结构的例子。
% fig 9.11
\begin{figure}[!htb]
\ifOpenSource
\centerline{\includegraphics{figure.pdf}}
\else
\centerline{\includegraphics{Chapter9/figures/cnn_classifier}}
\fi
\caption{\gls{convolutional_network}用于分类的结构示例。
本图中使用的具体\gls{stride}和深度并不建议实际使用;它们被设计得非常浅以适合页面。
实际的\gls{convolutional_network}还常常涉及大量的分支,不同于这里为简单起见所使用的链式结构。
\emph{(左)}处理固定大小的图像的\gls{convolutional_network}。
在卷积层和\gls{pooling}层几层交替之后,卷积特征映射的张量被重新变形以展平空间维度。
网络的其余部分是一个普通的前馈网络分类器,如\chapref{chap:deep_feedforward_networks}所述。
\emph{(中)}处理大小可变的图像的\gls{convolutional_network},但仍保持全连接的部分。
该网络使用具有可变大小但是数量固定的\gls{pool}的\gls{pooling}操作,以便向网络的全连接部分提供固定576个单位大小的向量。 
\emph{(右)}没有任何全连接权重层的\gls{convolutional_network}。
相对的,最后的卷积层为每个类输出一个特征映射。
该模型可能会用来学习每个类出现在每个空间位置的可能性的映射。
将特征映射进行平均得到的单个值,提供了顶部softmax分类器的变量。}
\label{fig:chap9_cnn_classifier}
\end{figure}

\section{卷积与\glsentrytext{pooling}作为一种无限强的先验}
\label{sec:convolution_and_pooling_as_an_infinitely_strong_prior}

回忆一下\secref{sec:capacity_overfitting_and_underfitting}中\firstgls{prior_probability_distribution}的概念。
这是一个模型参数的概率分布,它刻画了在我们看到数据之前我们认为什么样的模型是合理的信念。

% -- 334 --
 
先验被认为是强或者弱取决于先验中概率密度的集中程度。
弱先验具有较高的熵值,例如方差很大的\gls{gaussian_distribution}。这样的先验允许数据对于参数的改变具有或多或少的自由性。
强先验具有较低的熵值,例如方差很小的\gls{gaussian_distribution}。这样的先验在决定参数最终取值时起着更加积极的作用。

一个无限强的先验需要对一些参数取某些值的概率置零并且完全禁止对这些参数赋予这些值,无论数据对于这些参数值给出了多大的支持。

我们可以把\gls{convolutional_network}类比成全连接网络,但对于这个全连接网络的权重有一个无限强的先验。
这个无限强的先验是说一个\gls{hidden_unit}的权重必须和它邻居的权重相同,但可以在空间上移动。
这个先验也要求除了那些处在\gls{hidden_unit}的小的空间连续的\gls{receptive_field}内的权重以外,其余的权重都为零。
%这样翻译对吗?其实这里加个图更好
总之,我们可以把卷积的使用当作是对网络中一层的参数引入了一个无限强的\gls{prior_probability_distribution}。
这个先验说明了该层应该学得的函数只包含局部连接关系并且对平移具有等变性。
类似的,使用\gls{pooling}也是一个无限强的先验:每一个单元都具有对少量平移的不变性。

当然,把\gls{CNN}当作一个具有无限强先验的全连接网络来实现会导致极大的计算浪费。
但把\gls{CNN}想成具有无限强先验的全连接网络可以帮助我们更好地洞察\gls{CNN}是如何工作的。

其中一个关键的洞察是卷积和\gls{pooling}可能导致\gls{underfitting}。
与任何其他先验类似,卷积和\gls{pooling}只有当先验的假设合理且正确时才有用。
如果一项任务依赖于保存精确的空间信息,那么在所有的特征上使用\gls{pooling}将会增大训练误差。
一些\gls{convolutional_network}结构~\citep{Szegedy-et-al-arxiv2014}为了既获得具有较高不变性的特征又获得当平移不变性不合理时不会导致\gls{underfitting}的特征,被设计成在一些通道上使用\gls{pooling}而在另一些通道上不使用。
当一项任务涉及到要对输入中相隔较远的信息进行合并时,那么卷积所利用的先验可能就不正确了。

另一个关键洞察是当我们比较卷积模型的统计学习表现时,只能以基准中的其他卷积模型作为比较的对象。
其他不使用卷积的模型即使我们把图像中的所有像素点都置换后依然有可能进行学习。
对于许多图像数据集,还有一些分别的基准,有些是针对那些具有\firstgls{permutation_invariant}并且必须通过学习发现拓扑结构的模型,还有一些是针对模型设计者将空间关系的知识植入了它们的模型。

% -- 336 --

\section{基本卷积函数的变体}
\label{sec:variants_of_the_basic_convolution_function}

当在神经网络的上下文中讨论卷积时,我们通常不是特指数学文献中使用的那种标准的离散卷积运算。
实际应用中的函数略有不同。
这里我们详细讨论一下这些差异,并且对神经网络中用到的函数的一些重要性质进行重点说明。

首先,当我们提到神经网络中的卷积时,我们通常是指由多个并行卷积组成的运算。
这是因为具有单个核的卷积只能提取一种类型的特征,尽管它作用在多个空间位置上。
我们通常希望网络的每一层能够在多个位置提取多种类型的特征。

另外,输入通常也不仅仅是实值的网格,而是由一系列观测数据的向量构成的网格。
例如,一幅彩色图像在每一个像素点都会有红绿蓝三种颜色的亮度。
在多层的\gls{convolutional_network}中,第二层的输入是第一层的输出,通常在每个位置包含多个不同卷积的输出。
当处理图像时,我们通常把卷积的输入输出都看作是3维的张量,其中一个索引用于标明不同的通道(例如红绿蓝),另外两个索引标明在每个通道上的空间坐标。
软件实现通常使用批处理模式,所以实际上会使用4维的张量,第四维索引用于标明批处理中不同的实例,但我们为简明起见这里忽略批处理索引。

因为\gls{convolutional_network}通常使用多通道的卷积,所以即使使用了核翻转, 也不一定保证网络的线性运算是可交换的。
只有当其中的每个运算的输出和输入具有相同的通道数时,这些多通道的运算才是可交换的。

假定我们有一个4维的核张量$\TSK$,它的每一个元素是$\TEK_{i,j,k,l}$,表示输出中处于通道$i$的一个单元和输入中处于通道$j$中的一个单元的连接强度,并且在输出单元和输入单元之间有$k$行$l$列的偏移。
假定我们的输入由观测数据$\TSV$组成,它的每一个元素是$\TEV_{i,j,k}$,表示处在通道$i$中第$j$行第$k$列的值。
假定我们的输出$\TSZ$和输入$\TSV$具有相同的形式。
如果输出$\TSZ$是通过对$\TSK$和$\TSV$进行卷积而不涉及翻转$\TSK$得到的,那么
\begin{equation}
\TEZ_{i,j,k} = \sum_{l,m,n} \TEV_{l, j+m-1, k+n-1} \TEK_{i,l,m,n},
\end{equation}
这里对所有的$l$,$m$和$n$进行求和是对所有(在求和式中)有效的张量索引的值进行求和。
在线性代数中,向量的索引通常从1开始,这就是上述公式中$-1$的由来。
但是像C或Python这类编程语言索引通常从0开始,这使得上述公式可以更加简洁。

% -- 337 --
 
我们有时会希望跳过核中的一些位置来降低计算的开销(相应的代价是提取特征没有先前那么好了)。
我们可以把这一过程看作是对全卷积函数输出的下采样(downsampling)。
如果我们只想在输出的每个方向上每间隔$s$个像素进行采样,那么我们可以定义一个下采样卷积函数$c$使得
\begin{equation}
\TEZ_{i,j,k} = c(\TSK, \TSV, s)_{i,j,k} = \sum_{l,m,n} [\TEV_{l,(j-1)\times s+m, (k-1)\times s +n,}
 \TEK_{i,l,m,n}].
 \label{eq:9.8}
\end{equation}
我们把$s$称为下采样卷积的\firstgls{stride}。
当然也可以对每个移动方向定义不同的步幅。
\figref{fig:chap9_stride_conv}演示了一个实例。
% fig 9.12
\begin{figure}[!htb]
\ifOpenSource
\centerline{\includegraphics{figure.pdf}}
\else
\centerline{\includegraphics{Chapter9/figures/stride_conv}}
\fi
\caption{带有\gls{stride}的卷积。
在这个例子中,我们的\gls{stride}为二。
\emph{(上)}\,在单个操作中实现的\gls{stride}为二的卷积。
\emph{(下)}\,\gls{stride}大于一个像素的卷积在数学上等价于单位\gls{stride}的卷积随后\gls{downsampling}。
显然,涉及\gls{downsampling}的两步法在计算上是浪费的,因为它计算了许多将被丢弃的值。}
\label{fig:chap9_stride_conv}
\end{figure}

在任何\gls{convolutional_network}的实现中都有一个重要性质,那就是能够隐含地对输入$\TSV$用零进行填充(pad)使得它加宽。
如果没有这个性质,\gls{representation}的宽度在每一层就会缩减,缩减的幅度是比核少一个像素这么多。
对输入进行零填充允许我们对核的宽度和输出的大小进行独立的控制。
如果没有零填充,我们就被迫面临二选一的局面,要么选择网络空间宽度的快速缩减,要么选择一个小型的核——这两种情境都会极大得限制网络的表示能力。
\figref{fig:chap9_zero_pad_shrink}给出了一个例子。
% fig 9.13
\begin{figure}[!htb]
\ifOpenSource
\centerline{\includegraphics{figure.pdf}}
\else
\centerline{\includegraphics{Chapter9/figures/zero_pad_shrink}}
\fi
\caption{零填充对网络大小的影响。
考虑一个\gls{convolutional_network},每层有一个宽度为六的核。 
在这个例子中,我们不使用任何\gls{pooling},所以只有卷积操作本身缩小网络的大小。
\emph{(上)}在这个卷积网络中,我们不使用任何隐含的零填充。
这使得表示在每层缩小五个像素。
从十六个像素的输入开始,我们只能有三个卷积层,并且最后一层不能移动核,所以可以说只有两层是真正的卷积层。
可以通过使用较小的核来减缓收缩速率,但是较小的核表示能力不足,并且在这种结构中一些收缩是不可避免的。
\emph{(下)}通过向每层添加五个隐含的零,我们防止了表示随深度收缩。
这允许我们设计一个任意深的卷积网络。}
\label{fig:chap9_zero_pad_shrink}
\end{figure}

有三种零填充设定的情况值得注意。
第一种是无论怎样都不使用零填充的极端情况,并且卷积核只允许访问那些图像中能够完全包含整个核的位置。
在MATLAB的术语中,这称为\firstgls{valid}卷积。
在这种情况下,输出的所有像素都是输入中相同数量像素的函数,这使得输出像素的表示更加规范。
然而,输出的大小在每一层都会缩减。
如果输入的图像宽度是$m$,核的宽度是$k$,那么输出的宽度就会变成$m-k+1$。
如果卷积核非常大的话缩减率会非常显著。
因为缩减数大于0,这限制了网络中能够包含的卷积层的层数。
当层数增加时,网络的空间维度最终会缩减到$1\times 1$,这种情况下增加的层就不可能进行有意义的卷积了。
第二种特殊的情况是只进行足够的零填充来保持输出和输入具有相同的大小。
在MATLAB的术语中,这称为\firstgls{same}卷积。
在这种情况下,只要硬件支持,网络就能包含任意多的卷积层,这是因为卷积运算不改变下一层的结构。。
然而,输入像素中靠近边界的部分相比于中间部分对于输出像素的影响更小。
这可能会导致边界像素存在一定程度的欠表示。
这使得第三种极端情况产生了,在MATLAB中称为\firstgls{full}卷积。
它进行了足够多的零填充使得每个像素在每个方向上恰好被访问了$k$次,最终输出图像的宽度为$m+k-1$。%\footnote{译者注:图\ref{fig:chap9_zero_pad_shrink}给出了\gls{valid}卷积和\gls{same}卷积的例子,这里对\gls{full}卷积略作说明:我们可以认为\gls{full}卷积是在输入的两端各填充$k-1$个零,使得输入的每个像素都恰好被核访问$k$次,最终得到的输出的宽度为$[m+2(k-1)]-k+1=m+k-1$}
在这种情况下,输出像素中靠近边界的部分相比于中间部分是更少像素的函数。
这将导致学得一个在卷积特征映射的所有位置都表现不错的单核更为困难。
通常零填充的最优数量(对于测试集的分类正确率)处于``\gls{valid}卷积''和``\gls{same}卷积''之间的某个位置。

% -- 338 --
 
% -- 339 --

在一些情况下,我们并不是真的想使用卷积,而是想用一些局部连接的网络层\citep{LeCun86,LeCun89a}。
在这种情况下,我们的\gls{MLP}对应的邻接矩阵是相同的,但每一个连接都有它自己的权重,用一个6维的张量$\TSW$来表示。
$\TSW$的索引分别是:输出的通道$i$,输出的行$j$和列$k$,输入的通道$l$,输入的行偏置$m$和列偏置$n$。
局部连接层的线性部分可以表示为
\begin{equation}
\TEZ_{i,j,k} = \sum_{l,m,n} [\TEV_{l, j+m-1, k+n-1} w_{i, j, k, l, m, n}]. %这里应该是$\TEW$?
\end{equation}
这有时也被称为\firstgls{unshared_convolution},因为它和具有一个小核的离散卷积运算很像,但并不横跨位置来共享参数。
图\ref{fig:chap9_local}比较了局部连接、卷积和全连接的区别。
% fig 9.14
\begin{figure}[!htb]
\ifOpenSource
\centerline{\includegraphics{figure.pdf}}
\else
\centerline{\includegraphics{Chapter9/figures/local}}
\fi
\caption{局部连接,卷积和全连接的比较。
\emph{(上)}每一小片(接受域)有两个像素的局部连接层。
每条边用唯一的字母标记,来显示每条边都有自身的权重参数。
\emph{(中)}核宽度为两个像素的卷积层。
该模型与局部连接层具有完全相同的连接。
区别不在于哪些单元相互交互,而在于如何共享参数。
局部连接层没有\gls{parameter_sharing}。
正如用于标记每条边的字母重复出现所指示的,卷积层在整个输入上重复使用相同的两个权重。
\emph{(下)}全连接层类似于局部连接层,它的每条边都有其自身的参数(在该图中用字母明确标记的话就太多了)。 然而,它不具有局部连接层的连接受限的特征。}
\label{fig:chap9_local}
\end{figure}
 
% -- 340 --
 
当我们知道每一个特征都是一小块空间的函数并且相同的特征不会出现在所有的空间上时,局部连接层是很有用的。
例如,如果我们想要辨别一张图片是否是人脸图像时,我们只需要去寻找嘴是否在图像下半部分即可。

使用那些连接被更进一步限制的卷积或者局部连接层也是有用的,例如,限制每一个输出的通道$i$仅仅是输入通道$l$的一部分的函数时。
实现这种情况的一种通用方法是使输出的前$m$个通道仅仅连接到输入的前$n$个通道,输出的接下来的$m$个通道仅仅连接到输入的接下来的$n$个通道,以此类推。
\figref{fig:chap9_conv_groups}给出了一个例子。
对少量通道间的连接进行建模允许网络使用更少的参数,这降低了存储的消耗以及提高了统计效率,并且减少了前向和反向传播所需要的计算量。
这些目标的实现并没有减少\gls{hidden_unit}的数目。
% fig 9.15
\begin{figure}[!htb]
\ifOpenSource
\centerline{\includegraphics{figure.pdf}}
\else
\centerline{\includegraphics{Chapter9/figures/conv_groups}}
\fi
\caption{\gls{convolutional_network}的前两个输出通道只和前两个输入通道相连,随后的两个输出通道只和随后的两个输入通道相连。}
\label{fig:chap9_conv_groups}
\end{figure}

\firstgls{tiled_convolution}\citep{Gregor+LeCun-2010,Le2010}对卷积层和局部连接层进行了折衷。
这里并不是对\emph{每一个}空间位置的权重集合进行学习,我们学习一组核使得当我们在空间移动时它们可以循环利用。
这意味着在近邻的位置上拥有不同的过滤器,就像局部连接层一样,但是对于这些参数的存储需求仅仅会增长常数倍,这个常数就是核的集合的大小,而不是整个输出的特征映射的大小。
\figref{fig:chap9_tiled}对局部连接层、\gls{tiled_convolution}和标准卷积进行了比较。
% fig 9.16
\begin{figure}[!htb]
\ifOpenSource
\centerline{\includegraphics{figure.pdf}}
\else
\centerline{\includegraphics{Chapter9/figures/tiled}}
\fi
\captionsetup{singlelinecheck=off}
\caption[.]{局部连接层、\gls{tiled_convolution}和标准卷积的比较。
当使用相同大小的核时,这三种方法在单元之间具有相同的连接。
此图是对使用两个像素宽的核的说明。
这三种方法之间的区别在于它们如何共享参数。
\emph{(上)}局部连接层根本没有共享参数。
我们对每个连接使用唯一的字母标记,来表明每个连接都有它自身的权重。
\emph{(中)}\gls{tiled_convolution}有$t$个不同的核。
这里我们说明$t=2$的情况。
其中一个核具有标记为``a''和``b''的边,而另一个具有标记为``c''和``d''的边。
每当我们在输出中右移一个像素后,我们使用一个不同的核。
这意味着,与局部连接层类似,输出中的相邻单元具有不同的参数。
与局部连接层不同的是,在我们遍历所有可用的$t$个核之后,我们循环回到了第一个核。
如果两个输出单元间隔$t$个步长的倍数,则它们共享参数。
\emph{(下)}传统卷积等效于$t=1$的\gls{tiled_convolution}。
它只有一个核,并且被应用到各个地方,我们在图中表示为在各处使用具有标记为``a''和``b''的边的核。}
\label{fig:chap9_tiled}
\end{figure}
 
% -- 341 --
 
为了用代数的方法定义\gls{tiled_convolution},令$\TSK$是一个6维的张量\footnote{译者注:原文将$\TSK$误写成了$k$。},其中的两维对应着输出映射中的不同位置。
$\TSK$在这里并没有对输出映射中的每一个位置使用单独的索引,输出的位置在每个方向上在$t$个不同的核组成的集合中进行循环。
如果$t$等于输出的宽度,这就是局部连接层了。
\begin{equation}
\TEZ_{i, j, k} = \sum_{l, m, n} \TEV_{l, j+m-1, k+n-1} \TEK_{i, l, m, n, j\% t +1, k\% t+1},
\end{equation}
这里百分号是取模运算,它的性质包括$t\% t =0, (t+1)\% t = 1$等等。
在每一维上使用不同的$t$可以很容易对这个方程进行扩展。
 
% -- 342 --
  
% -- 343 --
 
局部连接层与\gls{tiled_convolution}层都和\gls{max_pooling}有一些有趣的关联:这些层的探测单元都是由不同的过滤器驱动的。
如果这些过滤器能够学会探测相同隐含特征的不同变换形式,那么\gls{max_pooling}的单元对于学得的变换就具有不变性(如\figref{fig:chap9_learned_rotation}所示)。
卷积层对于平移具有内置的不变性。
 
% -- 344 --
 
实现\gls{convolutional_network}时,通常也需要除卷积以外的其他运算。
为了实现学习,必须在给定输出的梯度时能够计算核的梯度。
在一些简单情况下,这种运算可以通过卷积来实现,但在很多我们感兴趣的情况下,包括\gls{stride}大于1的情况,并不具有这样的性质。

回忆一下卷积是一种线性运算,所以可以表示成矩阵乘法的形式(如果我们首先把输入张量变形为一个扁平的向量)。
其中包含的矩阵是关于卷积核的函数。
这个矩阵是稀疏的并且核的每个元素都复制给矩阵的多个元素。
这种观点能够帮助我们导出实现一个\gls{convolutional_network}所需的很多其他运算。

通过卷积定义的矩阵转置的乘法就是这样一种运算。
这种运算用于在卷积层反向传播误差的导数,所以它在训练多于一个\gls{hidden_layer}的\gls{convolutional_network}时是必要的。
如果我们想要从\gls{hidden_layer}单元重构可视化单元时,同样的运算也是需要的\citep{Simard92-short}。
重构可视化单元是本书第\ref{part:deep_learning_research}部分的模型广泛用到的一种运算,这些模型包括\gls{AE}、\glssymbol{RBM}和稀疏编码等等。
构建这些模型的卷积化的版本都要用到转置化卷积。
类似核梯度运算,这种输入梯度运算在某些情况下可以用卷积来实现,但在一般情况下需要用到第三种运算来实现。%这里不是很懂
必须非常小心地来使这种转置运算和\gls{forward_propagation}过程相协调。
转置运算返回的输出的大小取决于三个方面:零填充的策略、\gls{forward_propagation}运算的\gls{stride}以及\gls{forward_propagation}的输出映射的大小。
在一些情况下,不同大小的输入通过\gls{forward_propagation}过程能够得到相同大小的输出映射,所以必须明确地告知转置运算原始输入的大小。

这三种运算——卷积、从输出到权重的反向传播和从输出到输入的反向传播——对于训练任意深度的前馈\gls{convolutional_network},以及训练带有(基于卷积的转置的)重建函数的\gls{convolutional_network},这三种运算都足以计算它们所需的所有梯度。
对于完全一般的多维、多样例情况下的公式,完整的推导可以参考~\cite{Goodfellow-TR2010}。 
为了直观说明这些公式是如何起作用的,我们这里给出一个二维单个样例的版本。
 
% -- 345 --
 
假设我们想要训练这样一个\gls{convolutional_network},它包含\gls{stride}为$s$的\gls{stride}卷积,该卷积的核为$\TSK$,作用于多通道的图像$\TSV$,定义为$c(\TSK, \TSV, s)$,就像\eqnref{eq:9.8}中一样。
假设我们想要最小化某个损失函数$J(\TSV, \TSK)$。
在\gls{forward_propagation}过程中,我们需要用$c$本身来输出$\TSZ$,然后$\TSZ$传递到网络的其余部分并且被用来计算损失函数$J$。
在反向传播过程中,我们会得到一个张量$\TSG$满足$\TEG_{i, j, k} = \frac{\partial}{\partial \TEZ_{i, j, k}} J(\TSV, \TSK)$。

为了训练网络,我们需要对核中的权重求导。
为了实现这个目的,我们可以使用一个函数
\begin{equation}
g(\TSG, \TSV, s)_{i, j, k, l} = \frac{\partial}{\partial \TEK_{i, j, k, l}} J(\TSV, \TSK) = \sum_{m, n} \TEG_{i, m, n} \TEV_{j, (m-1)\times s+k, (n-1)\times s+l}.
\end{equation}

如果这一层不是网络的底层,我们需要对$\TSV$求梯度来使得误差进一步反向传播。
我们可以使用如下的函数
\begin{eqnarray}
h(\TSK, \TSG, s)_{i, j, k} &=& \frac{\partial }{\partial \TEV_{i, j, k}} J(\TSV, \TSK)\\
&=& \sum_{\substack{l, m\\
                  \text{s.t.}\\
                  (l-1)\times s+m = j}} \sum_{\substack{n, p\\
                                                            \text{s.t.}\\
                                                            (n-1)\times s +p = k}}
            \sum_q \TEK_{q,i,m,p} \TEG_{q, l, n}.
\end{eqnarray}

\chapref{chap:autoencoders}描述的\gls{AE}网络,是一些被训练成把输入拷贝到输出的前馈网络。
一个简单的例子是\,\glssymbol{PCA}\,算法,将输入$\Vx$拷贝到一个近似的重构值$\Vr$,通过函数${\MW}^\top \MW \Vx$来实现。
使用权重矩阵转置的乘法,就像\,\glssymbol{PCA}\,算法这种,在一般的\gls{AE}中是很常见的。
为了使这些模型卷积化,我们可以用函数$h$来实现卷积运算的转置。
假定我们有和$\TSZ$相同形式的隐藏单元$\TSH$,并且我们定义一种重构运算
\begin{equation}
\TSR = h(\TSK, \TSH, s).
\end{equation}

为了训练\gls{AE},我们会得到关于$\TSR$的梯度,表示为一个张量$\TSE$。
为了训练解码器,我们需要获得对于$\TSK$的梯度,这通过$g(\TSH, \TSE, s)$来得到。
为了训练编码器,我们需要获得对于$\TSH$的梯度,这通过$c(\TSK, \TSE, s)$来得到。
通过用$c$和$h$对$g$求微分也是可行的,但这些运算对于任何标准神经网络上的反向传播算法来说都是不需要的。
 
% -- 346 --
 
一般来说,在卷积层从输入到输出的变换中我们不仅仅只用线性运算。
我们一般也会在进行非线性运算前,对每个输出加入一些偏置项。
这样就产生了如何在偏置项中共享参数的问题。
对于局部连接层,很自然地对每个单元都给定它特有的偏置,对于\gls{tiled_convolution},也很自然地用与核一样的平铺模式来共享参数。
对于卷积层来说,通常的做法是在输出的每一个通道上都设置一个偏置,这个偏置在每个卷积映射的所有位置上共享。
然而,如果输入是已知的固定大小,也可以在输出映射的每个位置学习一个单独的偏置。
分离这些偏置可能会稍稍降低模型的统计效率,但同时也允许模型来校正图像中不同位置的统计差异。
例如,当使用隐含的零填充时,图像边缘的探测单元接收到较少的输入,因此需要较大的偏置。

\section{结构化输出}
\label{sec:structured_outputs}

\gls{CNN}可以用于输出高维的结构化对象,而不仅仅是预测分类任务的类标签或回归任务的实数值。
通常这个对象只是一个张量,由标准卷积层产生。
例如,模型可以产生张量$\TSS$,其中$\TES_{i,j,k}$是网络的输入像素$(j, k)$属于类$i$的概率。
这允许模型标记图像中的每个像素,并绘制沿着单个对象轮廓的精确掩模。

经常出现的一个问题是输出平面可能比输入平面要小,如\figref{fig:chap9_zero_pad_shrink}所示。
用于对图像中单个对象分类的常用结构中,网络空间维数的最大减少来源于使用大\gls{stride}的\gls{pooling}层。
为了产生与输入大小相似的输出映射,我们可以完全避免使用\gls{pooling}\citep{jain2007supervised}。
另一种策略是单纯地产生一张低分辨率的标签网格\citep{Pinheiro+Collobert-ICML2014,Pinheiro+Collobert-CVPR2015}。
最后,原则上可以使用具有单位\gls{stride}的\gls{pooling}操作。

对图像逐个像素标记的一种策略是先产生图像标签的原始猜测,然后使用相邻像素之间的交互来修正该原始猜测。
重复这个修正步骤数次对应于在每一步使用相同的卷积,该卷积在深层网络的最后几层之间共享权重\citep{jain2007supervised}。
这使得在层之间共享参数的连续的卷积层所执行的一系列运算,形成了一种特殊的\gls{RNN}~\citep{Pinheiro+Collobert-ICML2014,Pinheiro+Collobert-CVPR2015}。
\figref{fig:chap9_iterative}给出了这样一个循环\gls{convolutional_network}的结构。
% fig 9.17
\begin{figure}[!htb]
\ifOpenSource
\centerline{\includegraphics{figure.pdf}}
\else
\centerline{\includegraphics{Chapter9/figures/iterative}}
\fi
\captionsetup{singlelinecheck=off}
\caption[.]{用于像素标记的\gls{recurrent_convolutional_network}的示例。
输入是图像张量$\TSX$,它的轴对应图像的行、列和通道(红,绿,蓝)。
目标是输出标签张量$\hat{Y}$,它遵循每个像素的标签的概率分布。
该张量的轴对应图像的行、列和不同类别。
\gls{recurrent_network}通过使用$\hat{Y}$的先前估计作为创建新估计的输入,来迭代地改善其估计,而不是单次输出$\hat{Y}$,。
每个更新的估计使用相同的参数,并且估计可以如我们所愿地被改善任意多次。
每一步使用的卷积核张量$\TSU$,是用来计算给定输入图像的隐藏表示的。
核张量$\TSV$用于产生给定隐藏值时标签的估计。
除了第一步之外,核$\TSW$都对$\hat{Y}$进行卷积来提供隐藏层的输入。
在第一步中,此项由零代替。
因为每一步使用相同的参数,所以这是一个\gls{recurrent_network}的例子,如\chapref{chap:sequence_modeling_recurrent_and_recursive_nets}所述。}
\label{fig:chap9_iterative}
\end{figure}

% -- 347 --
 
一旦对每个像素都进行了预测,我们就可以使用各种方法来进一步处理这些预测,以便获得图像在区域上的分割\citep{Briggman-et-al-NIPS2009,Turaga2010,Farabet-et-al-2013}。
一般的想法是假设大片相连的像素倾向于对应着相同的标签。
图模型可以描述相邻像素间的概率关系。
或者,\gls{convolutional_network}可以被训练来最大化地近似图模型的训练目标\citep{Ning-et-al-2005,Thompson-et-al-NIPS2014}。

\section{数据类型}
\label{sec:data_types}

\gls{convolutional_network}使用的数据通常包含多个通道,每个通道是时间上或空间中某一点的不同观测量。
参考表\,\ref{table:data_types}\,来了解具有不同维数和通道数的数据类型的例子。
 
% -- 348 --
\begin{table}[htbp!]
\centering
 \begin{tabular}[t]{l|p{0.4\textwidth}|p{0.4\textwidth}}
& 单通道 & 多通道\\ \hline
1维 & 
音频波形:卷积的轴对应于时间。我们将时间离散化并且在每个时间点测量一次波形的振幅。 &  
骨架动画(skeleton animation)数据:计算机渲染的3D角色动画是通过随时间调整``骨架''的姿势而生成的。 在每个时间点,角色的姿势通过骨架中的每个关节的角度来描述。我们输入到卷积模型的数据的每个通道,表示一个关节关于一个轴的角度。\\ \hline
2维 & 
已经使用\gls{Fourier_transform}预处理过的音频数据:我们可以将音频波形变换成2维张量,不同的行对应不同的频率,不同的列对应不同的时间点。在时间轴上使用卷积对于时间上的移动具有等变性。在频率轴上使用卷积使得模型对于在频率上的移动具有等变性,这使得在不同八度音阶中播放的相同旋律产生相同的表示,但处于网络输出中的不同高度。 & %Fourier transform这里不是很清楚,在频率上使用卷积是做什么的?
彩色图像数据:其中一个通道包含红色像素,另一个包含绿色像素,最后一个包含蓝色像素。在图像的水平轴和竖直轴上移动卷积核,赋予了两个方向上平移等变性。\\ \hline
3维 &
体积数据:这种数据一般来源于医学成像技术,例如CT扫描等。 & 
彩色视频数据:其中一个轴对应着时间,另一个轴对应着视频帧的高度,最后一个对应着视频帧的宽度。\\
\end{tabular}
\caption{用于\gls{convolutional_network}的不同数据格式的示例。}
\label{table:data_types}
\end{table}
% -- 349 --
 
\gls{convolutional_network}用于视频的例子,可以参考~\cite{Chen-Ting-2010}。

到目前为止,我们仅讨论了训练和测试数据中的每个样例都有相同的空间维度的情况。
\gls{convolutional_network}的一个优点是它们还可以处理具有可变的空间尺度的输入。
这些类型的输入不能用传统的基于矩阵乘法的神经网络来表示。
这为\gls{convolutional_network}的使用提供了令人信服的理由,即使当计算开销和过拟合都不是主要问题时。
%\footnote{译者注:传统的基于矩阵乘法的神经网络会面对计算开销和过拟合的问题,即使当计算开销和过拟合都不是主要问题时,我们也有充分的理由来使用卷积网络而不是传统的神经网络,因为卷积网络可以处理可变大小的输入。}

例如,考虑一组图像的集合,其中每个图像具有不同的高度和宽度。
目前还不清楚如何用固定大小的权重矩阵对这样的输入进行建模。
卷积就可以很直接地应用;核依据输入的大小简单地被使用不同次,并且卷积运算的输出也相应地放缩。
卷积可以被视为矩阵乘法;相同的卷积核为每种大小的输入引入了一个不同大小的\gls{doubly_block_circulant_matrix}。
有时,网络的输出允许和输入一样具有可变的大小,例如如果我们想要为输入的每个像素分配一个类标签。
在这种情况下,不需要进一步的设计工作。
在其他情况下,网络必须产生一些固定大小的输出,例如,如果我们想要为整个图像指定单个类标签。
在这种情况下,我们必须进行一些额外的设计步骤,例如插入一个\gls{pooling}层,\gls{pooling}区域的大小要与输入的大小成比例,以便保持固定数量的\gls{pooling}输出。
这种策略的一些例子可以参考\figref{fig:chap9_cnn_classifier}。

注意,使用卷积处理可变尺寸的输入,仅对输入是因为包含对同种事物的不同量的观察(时间上不同长度的记录,空间上不同宽度的观察等)而导致的尺寸变化这种情况才有意义。
如果输入是因为它可以选择性地包括不同种类的观察而具有可变尺寸,使用卷积是不合理的。
例如,如果我们正在处理大学申请,并且我们的特征包括成绩等级和标准化测试分数,但不是每个申请人都进行了标准化测试,则使用相同的权重来对成绩特征和测试分数特征进行卷积是没有意义的。

\section{高效的卷积算法}
\label{sec:efficient_convolution_algorithms}

现代\gls{convolutional_network}的应用通常需要包含超过百万个单元的网络。
利用并行计算资源的强大实现是很关键的,如\secref{sec:large_scale_deep_learning}中所描述的。
然而,在很多情况下,也可以通过选择适当的卷积算法来加速卷积。
 
% -- 350 --
 
卷积等效于使用\gls{Fourier_transform}将输入与核都转换到频域、执行两个信号的逐点相乘,再使用傅立叶逆变换转换回时域。
对于某些问题的规模,这种算法可能比离散卷积的朴素实现更快。

当一个$d$维的核可以表示成$d$个向量(每一维一个向量)的外积时,该核被称为\firstgls{separable}。
当核可分离时,朴素的卷积是低效的。
它等价于组合$d$个一维卷积,每个卷积使用这些向量中的一个。
组合方法显著快于使用它们的外积来执行一个$d$维的卷积。
并且核也只要更少的参数来表示成向量。
如果核在每一维都是$w$个元素宽,那么朴素的多维卷积需要$O(w^d)$的运行时间和参数存储空间,而可分离卷积只需要$O(w\times d)$的运行时间和参数存储空间。
当然,并不是每个卷积都可以表示成这种形式。

设计更快的执行卷积或近似卷积,而不损害模型准确性的方法,是一个活跃的研究领域。 
甚至仅提高\gls{forward_propagation}效率的技术也是有用的,因为在商业环境中,通常部署网络比训练网络还要耗资源。

\section{随机或无监督的特征}
\label{sec:random_or_unsupervised_features}

通常,\gls{convolutional_network}训练中最昂贵的部分是学习特征。 
输出层的计算代价通常相对不高,因为在通过若干层\gls{pooling}之后作为该层输入的特征的数量较少。
当使用梯度下降执行\gls{supervised}训练时,每步梯度计算需要完整地运行整个网络的前向传播和反向传播。
减少\gls{convolutional_network}训练成本的一种方式是使用那些不是由\gls{supervised}方式训练得到的特征。

有三种基本策略可以不通过\gls{supervised}训练而得到卷积核。
其中一种是简单地随机初始化它们。
另一种是手动设计它们,例如设置每个核在一个特定的方向或尺度来检测边缘。
最后,可以使用无监督的标准来学习核。
例如,\cite{Coates2011}将$k$均值聚类算法应用于小图像块,然后使用每个学得的中心作为卷积核。
第\ref{part:deep_learning_research}部分描述了更多的无监督学习方法。
使用无监督的标准来学习特征,允许这些特征的确定与位于网络结构顶层的分类层相分离。
然后只需提取一次全部训练集的特征,构造用于最后一层的新训练集。
假设最后一层类似\gls{logistic_regression}或者\,\glssymbol{SVM},那么学习最后一层通常是凸优化问题。

 
% -- 351 --
 
随机过滤器经常在\gls{convolutional_network}中表现得出乎意料得好~\cite{Jarrett-ICCV2009-small,Saxe-ICML2011,pinto2011scaling,cox2011beyond}。
\cite{Saxe-ICML2011}~说明,由卷积和随后的\gls{pooling}组成的层,当赋予随机权重时,自然地变得具有频率选择性和平移不变性。
他们认为这提供了一种廉价的方法来选择\gls{convolutional_network}的结构:首先通过仅训练最后一层来评估几个\gls{convolutional_network}结构的性能,然后选择最好的结构并使用更昂贵的方法来训练整个网络。

一个中间方法是学习特征,但是使用那种不需要在每个梯度计算步骤中都进行完整的前向和反向传播的方法。
与多层感知机一样,我们使用\gls{greedy_layer_wise_pretraining},单独训练第一层,然后一次性地从第一层提取所有特征,之后用那些特征单独训练第二层,以此类推。
\chapref{chap:optimization_for_training_deep_models}描述了如何实现\gls{supervised}的\gls{greedy_layer_wise_pretraining},第\ref{part:deep_learning_research}部分将此扩展到了无监督的范畴。
卷积模型的\gls{greedy_layer_wise_pretraining}的经典模型是卷积\gls{DBN}\citep{HonglakL2009}。
\gls{convolutional_network}为我们提供了相对于多层感知机更进一步采用预训练策略的机会。
我们可以训练一小块模型,而并非一次训练整个卷积层,就像~\cite{Coates2011}使用$k$均值做的那样。
然后,我们可以用来自这个小块模型的参数来定义卷积层的核。
这意味着使用无监督学习来训练\gls{convolutional_network}\emph{并且在训练的过程中完全不使用卷积}是可能的。
使用这种方法,我们可以训练非常大的模型,并且只在推断期间产生高计算成本\citep{ranzato-cvpr-07-small,Jarrett-ICCV2009-small,koray-nips-10-small,icml2013_coates13}。
这种方法大约在2007到2013年间流行,当时标记的数据集很小,并且计算能力有限。
如今,大多数\gls{convolutional_network}以纯粹\gls{supervised}的方式训练,在每次训练迭代中使用通过整个网络的完整的前向和反向传播。
 
% -- 352 --
 
与其他无监督预训练的方法一样,使用这种方法的一些好处仍然难以说清成因。
可能是无监督预训练可以提供一些相对于\gls{supervised}训练的正则化,或者它可以简单地允许我们训练更大的结构,因为它的学习规则降低了计算成本。

\section{\glsentrytext{convolutional_network}的神经科学基础}
\label{sec:the_neuroscientific_basis_for_convolutional_networks}

\gls{convolutional_network}也许是生物学启发人工智能的最为成功的案例。
虽然\gls{convolutional_network}也受到了许多其他领域的指导,但是神经网络的一些关键设计原则来自于神经科学。

\gls{convolutional_network}的历史始于神经科学实验,这些实验远早于相关计算模型的发展。
为了确定关于哺乳动物视觉系统如何工作的许多最基本的事实,神经生理学家David Hubel和Torsten Wiesel合作多年\citep{Hubel+Wiesel-1959,Hubel62,Hubel+Wiesel-1968}。
他们的成就最终获得了诺贝尔奖。
他们的发现对当代深度学习模型有最大影响的是基于记录猫的单个神经元的活动。
他们观察了猫的脑内神经元如何响应投影在猫前面屏幕上精确位置的图像。
他们的伟大发现是,处于视觉系统较为前面的神经元对非常特定的光模式(例如精确定向的条纹)反应最强烈,但对其他模式几乎完全没有反应。

他们的工作有助于表征大脑功能的许多方面,这些方面超出了本书的范围。
从深度学习的角度来看,我们可以专注于简化的、草图形式的大脑功能视图。

在这个简化的视图中,我们关注被称为V1的大脑的一部分,也称为\firstgls{primary_visual_cortex}。
V1是大脑对视觉输入开始执行显著高级处理的第一个区域。
在该草图视图中,图像是由光到达眼睛并刺激视网膜(眼睛后部的光敏组织)形成的。
视网膜中的神经元对图像执行一些简单的预处理,但是基本不改变它被表示的方式。
然后图像通过视神经和称为\emph{外侧膝状核}的脑部区域。 
这些解剖区域的主要作用是仅仅将信号从眼睛传递到位于头后部的V1。
 
% -- 353 --
 
\gls{convolutional_network}层被设计为描述V1的三个性质:
\begin{enumerate}
  \item V1是以某种具有空间对应关系的方式排布的。
  %这里翻译不好
  它实际上具有二维结构来反映视网膜中的图像结构。
  例如,到达视网膜下半部的光仅影响V1中相应的一半。 
  \gls{convolutional_network}通过用二维映射定义特征的方式来体现该特性。

  \item V1包含许多\firstgls{simple_cells}。
  简单细胞的活动在某种程度上可以概括为在一个小的空间位置感受野内的图像的线性函数。
  \gls{convolutional_network}的检测器单元被设计为模拟简单细胞的这些性质。

  \item V1还包括许多\firstgls{complex_cells}。
  这些细胞响应的特征类似于由简单细胞检测的那些特征,但是复杂细胞对于特征的位置微小偏移具有不变性。 
  这启发了\gls{convolutional_network}的\gls{pooling}单元。
  复杂细胞对于照明中的一些变化也是不变的,不能简单地通过在空间位置上\gls{pooling}来刻画。 
  这些不变性激发了\gls{convolutional_network}中的一些跨通道\gls{pooling}策略,例如maxout单元\citep{Goodfellow-et-al-ICML2013}。
\end{enumerate}

虽然我们最了解V1,但是一般认为相同的基本原理也适用于视觉系统的其他区域。
在我们视觉系统的草图视图中,当我们逐渐深入大脑时,遵循\gls{pooling}的基本探测策略被反复执行。
当我们穿过大脑的多个解剖层时,我们最终会看到一些响应特定概念的细胞,这些细胞对输入的很多种变换都具有不变性。
这些细胞被昵称为``祖母细胞''——这个想法是一个人可能有一个神经元,当看到他祖母的照片时该神经元被激活,无论祖母是出现在照片的左边或右边,无论照片是她的脸部的特写镜头还是她的全身照,也无论她处在光亮还是黑暗中,等等。

这些祖母细胞已经被证明确实存在于人脑中,在一个被称为\emph{内侧颞叶}的区域\citep{quiroga2005invariant}。
研究人员测试了单个神经元是否会响应名人的照片。
他们发现了后来被称为``Halle Berry神经元''的神经元:由Halle Berry的概念激活的单个神经元。
当一个人看到Halle Berry的照片,Halle Berry的图画,甚至包含单词``Halle Berry''的文本时,这个神经元会触发。
当然,这与Halle Berry本人无关;其他神经元会对Bill Clinton,Jennifer Aniston等的出现做出响应。
 
% -- 354 --
 
这些内侧颞叶神经元比现代\gls{convolutional_network}更通用一些,这些网络在读取名称时不会自动联想到识别人或对象。
与\gls{convolutional_network}的最后一层在特征上最接近的类比是称为\emph{颞下皮质}(IT)的脑区。
当查看一个对象时,信息从视网膜经LGN流到V1,然后到V2,V4,之后是IT。
这发生在瞥见对象的前100ms内。
如果允许一个人继续观察对象更多的时间,那么信息将开始回流,因为大脑使用自上而下的反馈来更新较低级脑区中的激活。
然而,如果我们打断人的注视,并且只观察前100ms内的大多数前向激活导致的放电率,那么IT被证明与\gls{convolutional_network}非常相似。
\gls{convolutional_network}可以预测IT放电率,并且在执行对象识别任务时与人类(时间有限的情况)非常类似\citep{dicarlo-tutorial-2013}。

话虽如此,\gls{convolutional_network}和哺乳动物的视觉系统之间还是有许多区别。
这些区别有一些是计算神经科学家所熟知的,但超出了本书的范围。
还有一些区别尚未知晓,因为关于哺乳动物视觉系统如何工作的许多基本问题仍未得到回答。
简要列表如下:
\begin{itemize}
  \item 人眼大部分是非常低的分辨率,除了一个被称为\firstgls{fovea}的小块。
  中央凹仅观察在手臂长度距离内一块拇指大小的区域。
  虽然我们觉得我们可以看到高分辨率的整个场景,但这是由我们的大脑的潜意识部分创建的错觉,因为它缝合了我们瞥见的若干个小区域。
  大多数\gls{convolutional_network}实际上接收大的全分辨率的照片作为输入。
  人类大脑控制几次眼动,称为\firstgls{saccade},以瞥见场景中最显眼的或任务相关的部分。
  将类似的\gls{attention_mechanism}融入深度学习模型是一个活跃的研究方向。
  在深度学习的背景下,\gls{attention_mechanism}对于自然语言处理是最成功的,参考\secref{sec:using_an_attention_mechanism_and_aligning_pieces_of_data}。
  研究者已经研发了几种具有视觉机制的视觉模型,但到目前为止还没有成为主导方法\citep{Larochelle2010,Denil2012}。
  
  \item 人类视觉系统与许多其他感觉相集成,例如听觉,以及像我们的心情和想法一样的因素。
  \gls{convolutional_network}迄今为止纯粹是视觉的。
  
  \item 人类视觉系统不仅仅用于识别对象。
  它能够理解整个场景,包括许多对象,对象和对象之间的关系,以及处理我们的身体与世界交互所需的丰富的三维几何信息。
  \gls{convolutional_network}已经应用于这些问题中的一些,但是这些应用还处于起步阶段。
  
  \item 即使像V1这样简单的大脑区域也受到来自较高级别的反馈的严重影响。
  反馈已经在神经网络模型中被广泛地探索,但还没有被证明提供了引人注目的改进。
  
  \item 虽然前馈IT放电频率刻画了与\gls{convolutional_network}特征很多相同的信息,但是仍不清楚中间计算的相似程度。
  大脑可能使用非常不同的激活和\gls{pooling}函数。
  单个神经元的激活可能不能用单个线性过滤器的响应来很好地表征。
  最近的V1模型涉及对每个神经元的多个二次过滤器\citep{rust:2005}。
  事实上,我们的``简单细胞''和``复杂细胞''的草图图片可能并没有区别;简单细胞和复杂细胞可能是相同种类的细胞,但是它们的``参数''使得它们能够实现从我们所说的``简单''到``复杂''的连续的行为。
\end{itemize}
 
% -- 355 --
 
还值得一提的是,神经科学很少告诉我们该如何\emph{训练}\gls{convolutional_network}。
具有跨多个空间位置的\gls{parameter_sharing}的模型结构,可以追溯到早期关于视觉的联结主义模型\citep{Marr76},但是这些模型没有使用现代的反向传播算法和梯度下降。
例如,\citep{Fukushima80}结合了现代\gls{convolutional_network}的大多数模型结构设计元素,但依赖于层次化的无监督聚类算法。

\cite{Lang+Hinton88}引入反向传播来训练\firstall{TDNNs}。
使用当代术语来说,TDNN是用于时间序列的一维\gls{convolutional_network}。
用于这些模型的反向传播不受任何神经科学观察的启发,并且被一些人认为是生物不可信的。
在基于使用反向传播训练的TDNN成功之后,\cite{LeCun89d}通过将相同的训练算法应用于图像的2维卷积来发展现代\gls{convolutional_network}。

到目前为止,我们已经描述了简单细胞对于某些特征是如何呈现粗略的线性和选择性,复杂细胞是如何更加的非线性,并且对于这些简单细胞特征的某些变换具有不变性,以及在选择性和不变性之间交替放置的层可以产生对非常特定现象作出反应的祖母细胞。
我们还没有精确描述这些单个细胞检测到了什么。
在深度非线性网络中,可能难以理解单个细胞的功能。
第一层中的简单细胞相对更容易分析,因为它们的响应由线性函数驱动。
在人工神经网络中,我们可以直接显示卷积核的图像,来查看卷积层的相应通道是如何响应的。
在生物神经网络中,我们不能访问权重本身。
相反,我们在神经元自身中放置一个电极,在动物视网膜前显示几个白噪声图像样本,并记录这些样本中的每一个是如何导致神经元激活的。
然后,我们可以对这些响应拟合线性模型,以获得近似的神经元权重。
这种方法被称为\firstgls{reverse_correlation}\citep{ringach2004reverse}。
 
% -- 356 --
 
\gls{reverse_correlation}向我们表明,大多数的V1细胞具有由\firstgls{Gabor_function}所描述的权重。
\gls{Gabor_function}描述在图像中的2维点处的权重。我们可以认为图像是2维坐标$I(x,y)$的函数。
类似地,我们可以认为简单细胞是在图像中的一组位置采样,这组位置由一组$x$坐标$\SetX$和一组$y$坐标$\SetY$来定义,并且使用的权重$w(x,y)$也是位置的函数。
从这个观点来看,简单细胞对于图像的响应由下式给出
\begin{equation}
  s(I)=\sum_{x\in \SetX} \sum_{y\in \SetY} w(x, y)I(x,y).
\end{equation}
特别地,$w(x,y)$采用\gls{Gabor_function}的形式:
\begin{equation}
  w(x, y; \alpha, \beta_x, \beta_y, f, \phi, x_0, y_0, \tau) = \alpha \exp(-\beta_x x'^2 - \beta_y y'^2) \cos (fx' + \phi),
\end{equation}
其中
\begin{equation}
  x' = (x-x_0)\cos(\tau) + (y-y_0)\sin(\tau)
\end{equation}
以及
\begin{equation}
  y' = -(x-x_0) \sin(\tau) + (y-y_0)\cos(\tau).
\end{equation}

这里$\alpha, \beta_x, \beta_y, f, \phi, x_0, y_0, \tau$都是控制~\gls{Gabor_function}性质的参数。
\figref{fig:chap9_Gabor_functions}给出了\gls{Gabor_function}在不同参数集上的一些例子。
% fig 9.18
\begin{figure}
\ifOpenSource
\centerline{\includegraphics{figure.pdf}}
\else
\centering    
\subfigure{ \label{fig:chap9_Gabor_functions_a}     
\includegraphics[width=0.3\textwidth]{Chapter9/figures/gabor_coordinate_system.png}}     
\subfigure{ \label{fig:chap9_Gabor_functions_b}     
\includegraphics[width=0.3\textwidth]{Chapter9/figures/gabor_scale.png}}
\subfigure{ \label{fig:chap9_Gabor_functions_c}     
\includegraphics[width=0.3\textwidth]{Chapter9/figures/gabor_sinusoid.png}}     
\fi
\captionsetup{singlelinecheck=off}
\caption{具有各种参数设置的\gls{Gabor_function}。
白色表示绝对值大的正权重,黑色表示绝对值大的负权重,背景灰色对应于零权重。
\emph{(左)}控制坐标系的参数具有不同值的\gls{Gabor_function},这些参数包括:$x_0$、$y_0$和$\gamma$。
在该网格中的每个\gls{Gabor_function}被赋予和它在网格中的位置成比例的$x_0$和$y_0$的值,并且$\tau$被选择为使得每个Gabor过滤器对从网格中心辐射出的方向非常敏感。
对于其他两幅图,$x_0$、$y_0$和$\gamma$固定为零。
\emph{(中)}具有不同高斯比例参数$\beta_x$和$\beta_y$的\gls{Gabor_function}。
当我们从左到右通过网格时,\gls{Gabor_function}被设置为增加宽度(减少$\beta_x$);当我们从上到下通过网格时,\gls{Gabor_function}被设置为为增加高度(减少$\beta_y$)。
对于其他两幅图,$\beta$值固定为图像宽度的1.5倍。
\emph{(右)}具有不同的正弦参数$f$和$\phi$的\gls{Gabor_function}。
当我们从上到下移动时,$f$增加;当我们从左到右移动时,$\phi$增加。对于其他两幅图,$\phi$固定为0,$f$固定为图像宽度的5倍。}     
\label{fig:chap9_Gabor_functions}     
\end{figure}

参数$x_0,y_0$和$\tau$定义坐标系。
我们平移和旋转$x$和$y$来得到$x'$和$y'$。
具体地,简单细胞会响应以点$(x_0, y_0)$为中心的图像特征,并且当我们沿着从水平方向旋转$\tau$弧度的线移动时,简单细胞将响应亮度的变化。
 
% -- 357 --
 
作为$x'$和$y'$的函数,函数$w$会响应当我们沿着$x'$移动时的亮度变化。
它有两个重要的因子:一个是高斯函数,另一个是余弦函数。

高斯因子$ \alpha \exp(-\beta_x x'^2 - \beta_y y'^2)$可以被视为阈值项,用于保证简单细胞仅对接近$x'$和$y'$都为零点处的值响应,换句话说,接近细胞接受域的中心。
尺度因子$\alpha$调整简单细胞响应的总的量级,而$\beta_x$和$\beta_y$控制接受域消退的速度。

余弦因子$ \cos (fx' + \phi)$控制简单细胞如何响应延$x‘$轴的亮度改变。
参数$f$控制余弦的频率,$\phi$控制它的相位偏移。

合在一起,简单细胞的这个草图视图意味着,简单细胞对在特定位置处、特定方向上、特定空间频率的亮度进行响应。
当图像中的光波与细胞的权重具有相同的相位时,简单细胞是最兴奋的。
这种情况发生在当图像亮时,它的权重为正,而图像暗时,它的权重为负。
当光波与权重完全异相时,简单细胞被抑制——当图像较暗时,它的权重为正;较亮时,它的权重为负。
 
% -- 358 --
 
复杂细胞的草图视图是它计算包含两个简单细胞响应的2维向量的$L^2$范数:$c(I)=\sqrt{s_0(I)^2 + s_1(I)^2}$。
一个重要的特殊情况是当$s_1$和$s_0$具有除$\phi$ 以外都相同的参数,并且$\phi$被设置为使得$s_1$与$s_0$相位相差四分之一周期时。
在这种情况下,$s_0$和$s_1$形成\firstgls{quadrature_pair}。
当高斯重新加权的图像$I(x,y)\exp(-\beta_x x'^2 -\beta_y y^2)$包含具有频率$f$、在方向$\tau$上、接近$(x_0, y_0)$的高振幅正弦波时,用先前方法定义的复杂细胞会响应,并且\emph{不管该波的相位偏移}。
换句话说,复杂细胞对于图像在方向$\tau$上的微小平移或者翻转图像(用白色代替黑色,反之亦然)具有不变性。

神经科学和机器学习之间最显著的对应关系,是从视觉上比较机器学习模型学得的特征与使用V1得到的特征。
\cite{Olshausen+Field-1996}说明,一个简单的无监督学习算法,稀疏编码,学习的特征具有与简单细胞类似的感受野。
从那时起,我们发现,当应用于自然图像时,极其多样的统计学习算法都会学习到类\gls{Gabor_function}的特征。这包括大多数深度学习算法,它们在其第一层中学习这些特征。
\figref{fig:chap9_feature_detectors}给出了一些例子。
因为如此众多不同的学习算法会学习到边缘检测器,所以很难仅基于学习算法学得的特征,来断定哪一个特定的学习算法是``正确''的大脑模型(虽然,当应用于自然图像时,如果一个算法\emph{不能}学得某种检测器时,它能够作为一种否定标志)。
这些特征是自然图像的统计结构的重要部分,并且可以通过许多不同的统计建模方法来重新获得。
读者可以参考\citep{hyvarinen-book2009}来获得自然图像统计领域的综述。
% fig 9.19
\begin{figure}
\ifOpenSource
\centerline{\includegraphics{figure.pdf}}
\else
\centering    
\subfigure{ \label{fig:chap9_feature_detectors_a}     
\includegraphics[width=0.4\textwidth]{Chapter9/figures/s3c_filters.png}}    
\subfigure{ \label{fig:chap9_feature_detectors_b}     
\includegraphics[width=0.4\textwidth]{Chapter9/figures/maxout_kernels.png}}  
\fi
\caption{许多机器学习算法在应用于自然图像时,会学习那些用来检测边缘或边缘的特定颜色的特征。 
这些特征检测器使人联想到已知存在于初级视觉皮层中的~\gls{Gabor_function}。   
\emph{(左)}通过应用于小图像块的无监督学习算法(\gls{ss}稀疏编码)学得的权重。%这里不是很懂
\emph{(右)}由完全\gls{supervised}的卷积~\gls{maxout}~网络的第一层学得的卷积核。 相邻的一对过滤器驱动相同的~\gls{maxout_unit}。}     
\label{fig:chap9_feature_detectors}     
\end{figure}

\section{卷积网络与深度学习的历史}
\label{sec:convolutional_networks_and_the_history_of_deep_learning}
 
% -- 359 --
 
\gls{convolutional_network}在深度学习的历史中发挥了重要作用。
它们是将研究大脑获得的深刻理解成功用于机器学习应用的关键例子。
它们也是首批表现良好的深度模型之一,远远早于任意深度模型被认为是可行之前。
\gls{convolutional_network}也是第一个解决重要商业应用的神经网络,并且仍然处于当今深度学习商业应用的前沿。
例如,在20世纪90年代,AT\&T的神经网络研究小组开发了一个用于读取支票的\gls{convolutional_network}\citep{LeCun98-small}。
到90年代末,NEC部署的这个系统已经被用于读取美国10%以上的支票。
后来,微软部署了若干个基于\gls{convolutional_network}的OCR和手写识别系统\citep{simard-03-small}。 
关于\gls{convolutional_network}的这种应用和更现代应用的更多细节,参考\chapref{chap:applications}。
读者可以参考\citep{Lecun_convolutionalnetworks}了解2010年之前的更为深入的\gls{convolutional_network}历史。

\gls{convolutional_network}也被用作在许多比赛中的取胜手段。
当前对深度学习的商业兴趣的热度始于\cite{Krizhevsky-2012-small}赢得了ImageNet对象识别挑战,但是在那之前,\gls{convolutional_network}也已经被用于赢得前些年影响较小的其他机器学习和计算机视觉竞赛了。
 
% -- 360 --
 
\gls{convolutional_network}是第一批能使用反向传播有效训练的深度网络之一。
现在仍不完全清楚为什么\gls{convolutional_network}在一般的反向传播网络被认为已经失败时反而成功了。
这可能可以简单地归结为\gls{convolutional_network}比全连接网络计算效率更高,因此使用它们运行多个实验并调整它们的实现和超参数更容易。
更大的网络也似乎更容易训练。
利用现代硬件,大型全连接的网络在许多任务上也表现得很合理,即使使用过去那些全连接网络被认为不能工作得很好的数据集和当时流行的激活函数时,现在也能执行得很好。
心理可能是神经网络成功的主要阻碍(实践者没有期望神经网络有效,所以他们没有认真努力地使用神经网络)。
无论如何,幸运的是\gls{convolutional_network}在几十年前就表现良好。
在许多方面,它们为余下的深度学习传递火炬,并为一般的神经网络被接受铺平了道路。

\gls{convolutional_network}提供了一种方法来特化神经网络,使其能够处理具有清楚的网格结构拓扑的数据,以及将这样的模型扩展到非常大的规模。 
这种方法在二维图像拓扑上是最成功的。
为了处理一维序列数据,我们接下来转向神经网络框架的另一种强大的特化:\gls{RNN}。

 
% -- 361 --

% !Mode:: "TeX:UTF-8"
% Translator: Shenjian Zhao
\chapter{序列建模:循环和递归网络}
\label{chap:sequence_modeling_recurrent_and_recursive_nets}
\firstgls{RNN}或~\glssymbol{RNN}~\citep{RHW}是一类用于处理序列数据的\gls{NN}。
就像\gls{convolutional_network}是专门用于处理网格化数据$\TSX$(如一个图像)的\gls{NN},\gls{RNN}是专门用于处理序列$\Vx^{(1)}, \dots, \Vx^{(\tau)}$的\gls{NN}。
正如\gls{convolutional_network}可以很容易地扩展到具有很大宽度和高度的图像,以及处理大小可变的图像,\gls{recurrent_network}可以扩展到更长的序列(比不基于序列的特化网络长得多)。
大多数\gls{recurrent_network}也能处理可变长度的序列。 

从多层网络出发到\gls{recurrent_network},我们需要利用上世纪80年代\gls{ML}和统计模型早期思想的优点:在模型的不同部分共享参数。
\gls{parameter_sharing}使得模型能够扩展到不同形式的样本(这里指不同长度的样本)并进行泛化。
如果我们在每个时间点都有一个单独的参数,我们不但不能泛化到训练时没有见过序列长度,也不能在不同序列长度和不同时间位置上共享统计强度。
当信息的特定部分会在序列内多个位置出现时,这样的共享尤为重要。
例如,考虑这两句话:``I went to Nepal in 2009''和``In 2009, I went to Nepal.'' 
如果我们让一个\gls{ML}模型读取这两个句子,并提取叙述者去\ENNAME{Nepal}的年份,无论``2009年''是作为句子的第六个单词还是第二个单词出现,我们都希望模型能认出``2009年''为相关资料片段。
假设我们要训练一个处理固定长度句子的\gls{feedforward_network}。
传统的全连接\gls{feedforward_network}会给每个输入特征分配一个单独的参数,所以需要分别学习句子每个位置的所有语言规则。
相比之下,\gls{RNN}在几个\gls{time_step}内共享相同的权重,不需要分别学习句子每个位置的所有语言规则。

% -- 363 --

一个相关的想法是在1维时间序列上使用卷积。
这种卷积方法是\gls{TDNNs}的基础\citep{Lang+Hinton88,Waibel89b,lang1990time}。
卷积操作允许网络跨时间共享参数,但这种共享是浅层的。
卷积的输出是一个序列,其中输出中的每一项是相邻几项输入的函数。
\gls{parameter_sharing}的概念体现在每个\gls{time_step}中使用的相同卷积核。
\gls{RNN}以不同的方式共享参数。
输出的每一项是前一项的函数。
输出的每一项通过对先前的输出应用相同的更新规则而产生。
这种循环方式导致参数通过很深的\gls{computational_graph}共享。

为简单起见,我们说的~\glssymbol{RNN}~是指在序列上的操作,并且该序列在时刻$t$(从1到$\tau$)包含向量$\Vx^{(t)}$。
在实际情况中,\gls{recurrent_network}通常在序列的\gls{minibatch}上操作,并且\gls{minibatch}的每项具有不同序列长度$\tau$。
我们省略了\gls{minibatch}索引来简化记号。
此外,\gls{time_step}索引不必是字面上现实世界中流逝的时间。
有时,它仅表示序列中的位置。
\glssymbol{RNN}~也可以应用于二维的空间数据(如图像)。
甚至当应用于涉及时间的数据,该网络也可具有时间上朝向过去方向的连接,前提是整个序列在提供给网络之前就已经被完整地观察到。


本章将\gls{computational_graph}的思想扩展到包括循环。
这些周期代表变量自身的值在未来某一\gls{time_step}对自身值的影响。
这样的\gls{computational_graph}允许我们定义\gls{RNN}。
然后,我们描述许多构建、训练和使用\gls{RNN}的不同方式。

本章将简要介绍\gls{RNN},为获取更多详细信息,我们建议读者参考~\cite{Graves-book2012}的著作。

% -- 364 --

\section{\glsentrytext{unfolding}\glsentrytext{computational_graph}}
\label{sec:unfolding_computational_graphs}
\gls{computational_graph}是形式化一组计算结构的方式,如那些涉及将输入和参数映射到输出和损失的计算。
综合的介绍请参考\secref{sec:computational_graphs}。
本节,我们对\firstgls{unfolding}递归或循环计算得到的重复结构进行解释,这些重复结构通常对应于一个事件链。
\firstgls{unfolding}这个计算图将导致深度网络结构中的\gls{parameter_sharing}。

例如,考虑动态系统的经典形式:
\begin{align}
\label{eq:101s}
\Vs^{(t)} = f(\Vs^{(t-1)}; \Vtheta),
\end{align}
其中$ \Vs^{(t)}$称为系统的状态。

$\Vs$在时刻$t$的定义需要参考时刻$t-1$时同样的定义,因此\eqnref{eq:101s}是循环的。

对有限\gls{time_step}~$\tau$, $\tau-1$次应用这个定义可以展开这个图。
例如$\tau = 3$,我们对\eqnref{eq:101s}展开,可以得到:
\begin{align}
 \Vs^{(3)} &= f(\Vs^{(2)}; \Vtheta) \\
 & = f(f(\Vs^{(1)}; \Vtheta) ; \Vtheta).
  \label{eq:103s}
\end{align}

以这种方式重复应用定义,\gls{unfolding}等式,就能得到不涉及循环的表达。
现在我们可以使用传统的有向无环\gls{computational_graph}呈现这样的表达。

\eqnref{eq:101s}和\eqnref{eq:103s}的\gls{unfolding}\gls{computational_graph}如\figref{fig:chap10_unfolded_dynsys}所示。
\begin{figure}[!htb]
\ifOpenSource
\centerline{\includegraphics{figure.pdf}}
\else
\centerline{\includegraphics{Chapter10/figures/unfolded_dynsys}}
\fi
\caption{将\eqnref{eq:101s}描述的经典动态系统表示为展开的计算图。
每个节点表示在某个时刻$t$的状态,并且函数$f$将$t$处的状态映射到$t+1$处的状态。
所有\gls{time_step}都使用相同的参数(用于参数化$f$的相同$\Vtheta$值)。
}
\label{fig:chap10_unfolded_dynsys}
\end{figure}

% -- 365 --

作为另一个例子,让我们考虑由外部信号$\Vx^{(t)}$驱动的动态系统,
\begin{align}
 \label{eq:104s}
 \Vs^{(t)} = f(\Vs^{(t-1)}, \Vx^{(t)} ; \Vtheta),
\end{align}
我们可以看到,当前状态包含了整个过去序列的信息。

\gls{RNN}可以通过许多不同的方式建立。
就像几乎所有函数都可以被认为是\gls{feedforward_network},本质上任何涉及循环的函数都可以被认为是一个\gls{RNN}。

很多\gls{RNN}使用\eqnref{eq:105h}或类似的公式定义\gls{hidden_unit}的值。
为了表明状态是网络的\gls{hidden_unit},我们使用变量$\Vh$代表状态重写\eqnref{eq:104s}:
\begin{align}
 \label{eq:105h}
 \Vh^{(t)} = f(\Vh^{(t-1)}, \Vx^{(t)} ; \Vtheta),
\end{align}
如\figref{fig:chap10_rnn_circuit_unfolded}所示,典型~\glssymbol{RNN}~会增加额外的架构特性,如读取状态信息$\Vh$进行预测的输出层。
\begin{figure}[!htb]
\ifOpenSource
\centerline{\includegraphics{figure.pdf}}
\else
\centerline{\includegraphics{Chapter10/figures/rnn_circuit_unfolded}}
\fi
\caption{没有输出的\gls{recurrent_network}。
此\gls{recurrent_network}只处理来自输入$\Vx$的信息,将其合并到经过时间向前传播的状态$\Vh$。
\emph{(左)}回路原理图。 黑色方块表示单个\gls{time_step}的延迟。
\emph{(右)}同一网络被视为展开的计算图,其中每个节点现在与一个特定的时间实例相关联。
}
\label{fig:chap10_rnn_circuit_unfolded}
\end{figure}

当训练\gls{recurrent_network}根据过去预测未来时,网络通常要学会使用$\Vh^{(t)}$作为过去序列(直到$t$)与任务相关方面的有损摘要。
此摘要一般而言一定是有损的,因为其映射任意长度的序列$(\Vx^{(t)},\Vx^{(t-1)}, \Vx^{(t-2)},\dots,\Vx^{(2)}, \Vx^{(1)})$到一固定长度的向量$\Vh^{(t)}$。
根据不同的训练\gls{criterion},摘要可能选择性地精确保留过去序列的某些方面。
例如,如果在统计语言建模中使用的~\glssymbol{RNN},通常给定前一个词预测下一个词,可能没有必要存储时刻$t$前输入序列中的所有信息;而仅仅存储足够预测句子其余部分的信息。
最苛刻的情况是我们要求$\Vh^{(t)}$足够丰富,并能大致恢复输入序列,如\gls{AE}框架(\chapref{chap:autoencoders})。

% -- 366 --

\eqnref{eq:105h}可以用两种不同的方式来绘制。
一种绘制~\glssymbol{RNN}~的方式是用循环图,对于可能存在于模型的物理实现中的每个组件,都在图中包含其对应的节点。这种方式的一个示例是生物神经网络。
在这种视角下,网络定义了实时操作的回路,如\figref{fig:chap10_rnn_circuit_unfolded}的左侧,其当前状态可以影响其未来的状态。
在本章中,我们使用回路图中的黑色方块表示单个\gls{time_step}延时所发生的交互,即从时刻$t$的状态到时刻$t+1$的状态的变化。
另一种绘制~\glssymbol{RNN}~的方式是用展开的\gls{computational_graph},其中每一个组件由许多不同的变量表示,每个\gls{time_step}一个变量,用来表示组件在该时间点的状态。
每个\gls{time_step}的每个变量绘制为\gls{computational_graph}的一个独立节点,如\figref{fig:chap10_rnn_circuit_unfolded}的右侧。
我们所说的\gls{unfolding}是将左图中的回路映射为右图中包含重复组件的\gls{computational_graph}的操作。
这样,\gls{unfolded_graph}的大小取决于序列长度。

我们可以用一个函数$g^{(t)}$代表经$t$步展开后的循环:
\begin{align}
\label{eq:106}
  \Vh^{(t)} &=g^{(t)}(\Vx^{(t)},\Vx^{(t-1)}, \Vx^{(t-2)},\dots,\Vx^{(2)}, \Vx^{(1)}) \\
  & =  f(\Vh^{(t-1)}, \Vx^{(t)} ; \Vtheta) .
\end{align}
函数$g^{(t)}$将全部的过去序列$(\Vx^{(t)},\Vx^{(t-1)}, \Vx^{(t-2)},\dots,\Vx^{(2)}, \Vx^{(1)})$作为输入来生成当前状态,但是展开的循环架构允许我们将$g^{(t)}$分解为函数$f$的重复应用。
因此,展开过程引入两个主要优点:
\begin{enumerate}
 \item 无论序列的长度,学成的模型始终具有相同的输入大小,因为它指定的是从一种状态到另一种状态的\gls{transition}, 而不是在可变长度的历史状态上操作。
 \item 我们可以在每个\gls{time_step}使用相同参数的\emph{相同}\gls{transition}函数$f$。
\end{enumerate}
这两个因素使得学习在所有\gls{time_step}和所有序列长度上操作单一的模型$f$是可能的,而不需要在所有可能\gls{time_step}学习独立的模型$g^{(t)}$。
学习单一的共享模型允许泛化到没有见过的序列长度(没有出现在训练集中),并且估计模型所需的训练样本远远少于不带\gls{parameter_sharing}的模型。

% -- 367 --

无论是循环图和\gls{unfolded_graph}都有其用途。
循环图简洁。
\gls{unfolded_graph}能够明确描述其中的计算流程。
\gls{unfolded_graph}还通过显式的信息流动路径帮助说明信息在时间上向前(计算输出和损失)和向后(计算\gls{gradient})的思想。

\section{\glsentrytext{RNN}}
\label{sec:recurrent_neural_networks}
基于\secref{sec:unfolding_computational_graphs}中的图展开和\gls{parameter_sharing}的思想,我们可以设计各种\gls{RNN}。
\begin{figure}[!htb]
\ifOpenSource
\centerline{\includegraphics{figure.pdf}}
\else
\centerline{\includegraphics{Chapter10/figures/hidden_recurrence_rnn}}
\fi
\caption{计算\gls{recurrent_network}(将$\Vx$值的输入序列映射到输出值$\Vo$的对应序列)训练损失的计算图。
损失$L$衡量每个$\Vo$与相应的训练目标$\Vy$的距离。
当使用softmax输出时,我们假设$\Vo$是未归一化的对数概率。
损失$L$内部计算$\hat{\Vy} = \text{softmax}(\Vo)$,并将其与目标$\Vy$比较。
\glssymbol{RNN}输入到隐藏的连接由权重矩阵$\MU$参数化,隐藏到隐藏的循环连接由权重矩阵$\MW$参数化以及隐藏到输出的连接由权重矩阵$\MV$参数化。
\eqnref{eq:108a}定义了该模型中的前向传播。
\emph{(左)}使用循环连接绘制的~\glssymbol{RNN}~和它的损失。
\emph{(右)}同一网络被视为展开的计算图,其中每个节点现在与一个特定的时间实例相关联。
}
\label{fig:chap10_hidden_recurrence_rnn}
\end{figure}

\gls{RNN}中一些重要的设计模式包括以下几种:
\begin{enumerate}
 \item 每个\gls{time_step}都有输出,并且\gls{hidden_unit}之间有循环连接的\gls{recurrent_network},如\figref{fig:chap10_hidden_recurrence_rnn}所示。
 \item 每个\gls{time_step}都产生一个输出,只有当前时刻的输出到下个时刻的\gls{hidden_unit}之间有循环连接的\gls{recurrent_network},如\figref{fig:chap10_output_recurrence_rnn}所示。
 \item \gls{hidden_unit}之间存在循环连接,但读取整个序列后产生单个输出的\gls{recurrent_network},如\figref{fig:chap10_single_output_rnn}所示。
\end{enumerate}
\figref{fig:chap10_hidden_recurrence_rnn}是非常具有代表性的例子,我们将会在本章大部分涉及这个例子。

\begin{figure}[!htb]
\ifOpenSource
\centerline{\includegraphics{figure.pdf}}
\else
\centerline{\includegraphics{Chapter10/figures/output_recurrence_rnn}}
\fi
\caption{此类~\glssymbol{RNN}~的唯一循环是从输出到隐藏层的反馈连接。
在每个\gls{time_step}~$t$,输入为$\Vx_t$,\gls{hidden_layer}激活为$\Vh^{(t)}$,输出为$\Vo^{(t)}$,目标为$\Vy^{(t)}$,损失为$L^{(t)}$。
\emph{(左)}回路原理图。
\emph{(右)}展开的计算图。
这样的~\glssymbol{RNN}~没有\figref{fig:chap10_hidden_recurrence_rnn}表示的RNN那样强大(只能表示更小的函数集合)。
\figref{fig:chap10_hidden_recurrence_rnn}中的~\glssymbol{RNN}~可以选择将其想要的关于过去的任何信息放入隐藏表示$\Vh$中并且将$\Vh$传播到未来。
该图中的~\glssymbol{RNN}~被训练为将特定输出值放入$\Vo$中,并且$\Vo$是允许传播到未来的唯一信息。
此处没有从$\Vh$前向传播的直接连接。
之前的$\Vh$仅通过产生的预测间接地连接到当前。
$\Vo$通常缺乏过去的重要信息,除非它非常高维且内容丰富。
这使得该图中的~\glssymbol{RNN}~不那么强大,但是它更容易训练,因为每个\gls{time_step}可以与其他\gls{time_step}分离训练,允许训练期间更多的并行化,如\secref{sec:teacher_forcing_and_networks_with_output_recurrence}所述。
}
\label{fig:chap10_output_recurrence_rnn}
\end{figure}

\begin{figure}[!htb]
\ifOpenSource
\centerline{\includegraphics{figure.pdf}}
\else
\centerline{\includegraphics{Chapter10/figures/single_output_rnn}}
\fi
\caption{关于时间展开的\gls{RNN},在序列结束时具有单个输出。
这样的网络可以用于概括序列并产生用于进一步处理的固定大小的表示。
在结束处可能存在目标(如此处所示),或者通过更下游模块的反向传播来获得输出$\Vo^{(t)}$上的梯度。
}
\label{fig:chap10_single_output_rnn}
\end{figure}

任何图灵可计算的函数都可以通过这样一个有限维的\gls{recurrent_network}计算,在这个意义上\figref{fig:chap10_hidden_recurrence_rnn}和\eqnref{eq:108a}的\gls{RNN}是万能的。
\glssymbol{RNN}~经过若干\gls{time_step}后读取输出,这与由图灵机所用的\gls{time_step}是渐近线性的,与输入长度也是渐近线性的\citep{Siegelmann+Sontag-1991,Siegelmann-1995,Siegelmann+Sontag-1995,Hyotyniemi-1996}。
由图灵机计算的函数是离散的,所以这些结果都是函数的具体实现,而不是近似。
\glssymbol{RNN}~作为图灵机使用时,需要一个二进制序列作为输入,其输出必须离散化以提供二进制输出。
利用单个有限大小的特定~\glssymbol{RNN}~计算在此设置下的所有函数是可能的(~\cite{Siegelmann+Sontag-1995}用了886个单元)。
图灵机的``输入''是要计算函数的详细说明(specification),所以模拟此图灵机的相同网络足以应付所有问题。
用于证明的理论~\glssymbol{RNN}~可以通过激活和权重(由无限精度的有理数表示)来模拟无限堆栈。

% -- 368 --

现在我们研究\figref{fig:chap10_hidden_recurrence_rnn}中~\glssymbol{RNN}~的\gls{forward_propagation}公式。
这个图没有指定\gls{hidden_unit}的激活函数。
我们假设使用双曲正切激活函数。
此外,图中没有明确指定何种形式的输出和\gls{loss_function}。
我们假定输出是离散的,如用于预测词或字符的~\glssymbol{RNN}。
表示离散变量的常规方式是把输出$\Vo$作为每个离散变量可能值的非标准化对数概率。
然后,我们可以应用~\gls{softmax}~后续处理后,获得标准化后概率的输出向量$\hat \Vy$。
\glssymbol{RNN}~从特定的初始状态$\Vh^{(0)}$开始\gls{forward_propagation}。
从$t= 1$到$t = \tau$的每个\gls{time_step},我们应用以下更新方程:
\begin{align}
\label{eq:108a}
 \Va^{(t)} &= \Vb + \MW \Vh^{(t-1)} + \MU \Vx^{(t)}, \\
  \Vh^{(t)} &= \tanh(\Va^{(t)} ), \\
  \Vo^{(t)} &= \Vc + \MV \Vh^{(t)}, \\
  \hat \Vy^{(t)} &= \text{softmax}(\Vo^{(t)}),
\end{align}
其中的参数的\gls{bias_aff}向量$\Vb$和$\Vc$连同权重矩阵$\MU$、$\MV$和$\MW$,分别对应于输入到隐藏、隐藏到输出和隐藏到隐藏的连接。
这个\gls{recurrent_network}将一个输入序列映射到相同长度的输出序列。
与$\Vx$序列配对的$\Vy$的总损失就是所有\gls{time_step}的损失之和。
例如,$L^{(t)}$为给定的$\Vx^{(1)}, \dots, \Vx^{(t)}$后$\Vy^{(t)}$的负对数似然,则
\begin{align} \label{eq:1012L}
 & L\big( \{ \Vx^{(1)}, \dots, \Vx^{(\tau)} \}, \{ \Vy^{(1)}, \dots, \Vy^{(\tau)}  \} \big) \\
 & = \sum_t L^{(t)} \\
 & = - \sum_t \log p_{\text{model}} \big(  y^{(t)} \mid  \{ \Vx^{(1)}, \dots, \Vx^{(t)} \} \big) ,
\end{align}
其中$p_{\text{model}} \big(  y^{(t)} \mid  \{ \Vx^{(1)}, \dots, \Vx^{(t)} \} \big) $需要读取模型输出向量$\hat \Vy^{(t)}$中对应于$y^{(t)}$的项。
关于各个参数计算这个\gls{loss_function}的\gls{gradient}是计算成本很高的操作。
\gls{gradient}计算涉及执行一次\gls{forward_propagation}(如在\figref{fig:chap10_hidden_recurrence_rnn}\gls{unfolded_graph}中从左到右的传播),接着是由右到左的\gls{backward_propagation}。
运行时间是$\CalO(\tau)$,并且不能通过并行化来降低,因为\gls{forward_propagation}图是固有循序的;每个\gls{time_step}只能一前一后地计算。
\gls{forward_propagation}中的各个状态必须保存,直到它们\gls{backward_propagation}中被再次使用,因此内存代价也是$\CalO(\tau)$。
应用于\gls{unfolded_graph}且代价为$\CalO(\tau)$的\gls{backward_propagation}算法称为\firstall{BPTT},将在\secref{sec:computing_the_gradient_in_a_recurrent_neural_network}进一步讨论。
因此\gls{hidden_unit}之间存在循环的网络非常强大但训练代价也很大。
我们是否有其他选择呢?

% -- 371 --

\subsection{\glsentrytext{teacher_forcing}和输出\glsentrytext{recurrent_network}}
\label{sec:teacher_forcing_and_networks_with_output_recurrence}
仅在一个\gls{time_step}的输出和下一个\gls{time_step}的\gls{hidden_unit}间存在循环连接的网络(示于\figref{fig:chap10_output_recurrence_rnn})确实没有那么强大(因为缺乏隐藏到隐藏的循环连接)。
例如,它不能模拟通用图灵机。
因为这个网络缺少隐藏到隐藏的循环,它要求输出单元捕捉用于预测未来的关于过去的所有信息。
因为输出单元明确地训练成匹配训练集的目标,它们不太能捕获关于过去输入历史的必要信息,除非用户知道如何描述系统的全部状态,并将它作为训练目标的一部分。
消除隐藏到隐藏循环的优点在于,任何基于比较时刻$t$的预测和时刻$t$的训练目标的\gls{loss_function}中的所有\gls{time_step}都解耦了。
因此训练可以并行化,即在各时刻$t$分别计算\gls{gradient}。
没有必要先计算前一时刻的输出,因为训练集提供了前一时刻输出的理想值。

由输出反馈到模型而产生循环连接的模型可用\firstgls{teacher_forcing}进行训练。
\gls{teacher_forcing}是从最大似然\gls{criterion}派生出来的一个训练过程,训练模型时,在时刻$t+1$接收真实值$y^{(t)}$作为输入。
我们可以通过检查两个\gls{time_step}的序列看清楚这一点。
条件最大似然\gls{criterion}是
\begin{align}
 &\log p(\Vy^{(1)},\Vy^{(2)} \mid \Vx^{(1)}, \Vx^{(2)} ) \\
 &= \log  p(\Vy^{(2)} \mid \Vy^{(1)}, \Vx^{(1)}, \Vx^{(2)} )  + \log p(\Vy^{(1)} \mid \Vx^{(1)}, \Vx^{(2)}) .
\end{align}

% -- 372 --

在这个例子中,\emph{同时}给定迄今为止的$\Vx$序列和来自训练集的前一$\Vy$值,我们可以看到在时刻$t=2$时,模型被训练为最大化$\Vy^{(2)}$的条件概率。
因此最大似然在训练时指定使用正确的目标值作为反馈,而不是将自己的输出反馈到模型。
如\figref{fig:chap10_teacher_forcing}所示。
\begin{figure}[!htb]
\ifOpenSource
\centerline{\includegraphics{figure.pdf}}
\else
\centerline{\includegraphics{Chapter10/figures/teacher_forcing}}
\fi
\caption{\gls{teacher_forcing}的示意图。
\gls{teacher_forcing}是一种训练技术,适用于输出与下一\gls{time_step}的隐藏状态存在连接的~\glssymbol{RNN}。
\emph{(左)}训练时,我们将训练集中\emph{正确}的输出$\Vy^{(t)}$反馈到$\Vh^{(t+1)}$。
\emph{(右)}当模型部署后,真正的输出通常是未知的。
在这种情况下,我们用模型的输出$\Vo^{(t)}$近似正确的输出$\Vy^{(t)}$,并反馈回模型。
}
\label{fig:chap10_teacher_forcing}
\end{figure}

我们使用\gls{teacher_forcing}的最初动机是为了在缺乏隐藏到隐藏连接的模型中避免\gls{BPTT}。
对于那些存在隐藏到隐藏连接的模型,只要模型一个\gls{time_step}的输出与下一\gls{time_step}计算的值存在连接,\gls{teacher_forcing}仍然可以适用。
只不过,只要\gls{hidden_unit}成为较早\gls{time_step}的函数,\glssymbol{BPTT}~算法就是必要的。
因此训练某些模型时可能要同时使用\gls{teacher_forcing}和~\glssymbol{BPTT}。

% -- 373 --

如果之后网络在\textbf{开环}(open-loop)模式下使用,即网络输出(或输出分布的样本)反馈作为输入,那么完全使用\gls{teacher_forcing}进行训练的缺点就会出现。
在这种情况下,训练期间该网络看到的输入与测试时看到的会有很大的不同。
减轻此问题的一种方法是同时使用\gls{teacher_forcing}和自由运行的输入进行训练,例如在展开循环的输出到输入路径上预测若干个步骤之后的正确目标值。
通过这种方式,网络可以学会考虑在训练时没有接触到的输入条件(如自由运行模式下,自身生成自身),以及将状态映射回使网络几步之后生成正确输出的状态。
另外一种方式\citep{SamyBengio-et-al-arxiv2015}是通过随机选择生成值或真实的数据值作为输入以减小训练时和测试时看到的输入之间的差别。
这种方法利用了\gls{curriculum_learning}策略,逐步使用更多生成值作为输入。

\subsection{计算\glsentrytext{RNN}的\glsentrytext{gradient}}
\label{sec:computing_the_gradient_in_a_recurrent_neural_network}
计算\gls{RNN}的\gls{gradient}是容易的。
我们可以简单地将\secref{sec:general_back_propagation}中的推广\gls{backward_propagation}算法应用于展开的\gls{computational_graph},而不需要特殊化的算法。
由\gls{backward_propagation}计算得到的\gls{gradient},并结合任何通用的基于\gls{gradient}的技术就可以训练~\glssymbol{RNN}。

为了获得~\glssymbol{BPTT}~算法行为的一些直观理解,我们举例说明如何通过~\glssymbol{BPTT}~计算上述\glssymbol{RNN}公式(\eqnref{eq:108a}和\eqnref{eq:1012L})的\gls{gradient}。
\gls{computational_graph}的节点包括参数$\MU,\MV,\MW, \Vb$和$\Vc$,以及以$t$为索引的节点序列$\Vx^{(t)}, \Vh^{(t)},\Vo^{(t)}$和$L^{(t)}$。
对于每一个节点$\TSN$,我们需要基于$\TSN$后面的节点的\gls{gradient},递归地计算\gls{gradient}~$\nabla_{\TSN} L$。
我们从紧接着最终损失的节点开始递归:
\begin{align}
 \frac{\partial L}{\partial L^{(t)}} = 1.
\end{align}
在这个导数中,我们假设输出$\Vo^{(t)}$作为~\gls{softmax}~的参数,我们可以从~\gls{softmax}可以获得关于输出概率的向量$\hat{\Vy}$。
我们也假设损失是迄今为止给定了输入后的真实目标$y^{(t)}$的负对数似然。
对于所有$i,t$,关于\gls{time_step}~$t$输出的\gls{gradient}~$\nabla_{\Vo^{(t)}} L$如下:
\begin{align}
 (\nabla_{\Vo^{(t)}} L)_i =  \frac{\partial L}{\partial o_i^{(t)}} 
 =  \frac{\partial L}{\partial L^{(t)}}  \frac{\partial L^{(t)}}{\partial o_i^{(t)}}  
 = \hat y_i^{(t)} - \mathbf{1}_{i,y^{(t)}}.
\end{align}
我们从序列的末尾开始,反向进行计算。
在最后的\gls{time_step}~$\tau$, $\Vh^{(\tau)}$只有$\Vo^{(\tau)}$作为后续节点,因此这个\gls{gradient}很简单:
\begin{align}
 \nabla_{\Vh^{(\tau)}} L = \MV^\top \nabla_{\Vo^{(\tau)}} L.
\end{align}
然后,我们可以从时刻$t=\tau-1$到$t=1$反向迭代, 通过时间\gls{backward_propagation}\gls{gradient},注意$\Vh^{(t)}(t < \tau)$同时具有$\Vo^{(t)}$和$\Vh^{(t+1)}$两个后续节点。
因此,它的\gls{gradient}由下式计算
\begin{align}
  \nabla_{\Vh^{(t)}} L = \Big( \frac{\partial \Vh^{(t+1)}}{ \partial \Vh^{(t)}}  \Big)^\top(\nabla_{\Vh^{(t+1)}} L) 
  + \Big( \frac{\partial \Vo^{(t)}}{ \partial \Vh^{(t)}}  \Big)^\top (\nabla_{\Vo^{(t)}} L) \\
  = \MW^\top (\nabla_{\Vh^{(t+1)}} L) \text{diag} \Big( 1 - (\Vh^{(t+1)})^2 \Big) 
    + \MV^\top ( \nabla_{\Vo^{(t)}} L ),
\end{align}
其中$\text{diag} \Big( 1 - (\Vh^{(t+1)})^2 \Big) $ 表示包含元素$1 - (h_i^{(t+1)})^2$的对角矩阵。
这是关于时刻$t+1$与\gls{hidden_unit}~$i$关联的双曲正切的\gls{jacobian}。

% -- 374 --

一旦获得了\gls{computational_graph}内部节点的\gls{gradient},我们就可以得到关于参数节点的\gls{gradient}。
因为参数在许多\gls{time_step}共享,我们必须在表示这些变量的微积分操作时谨慎对待。
我们希望实现的等式使用\secref{sec:general_back_propagation}中的{\tt bprop}方法计算\gls{computational_graph}中单一边对\gls{gradient}的贡献。
然而微积分中的$\nabla_{\MW} f$算子,计算$\MW$对于$f$的贡献时将\gls{computational_graph}中的\emph{所有}边都考虑进去了。
为了消除这种歧义,我们定义只在$t$时刻使用的虚拟变量$\MW^{(t)}$作为$\MW$的副本。
然后,我们可以使用$\nabla_{\MW^{(t)}}$表示权重在\gls{time_step}~$t$对\gls{gradient}的贡献。

使用这个表示,关于剩下参数的\gls{gradient}可以由下式给出:
\begin{align}
 \nabla_{\Vc} L &=  \sum_t \Big( \frac{\partial \Vo^{(t)}}{\partial \Vc} \Big)^\top \nabla_{\Vo^{(t)}} L 
 = \sum_t \nabla_{\Vo^{(t)}} L ,\\
 \nabla_{\Vb} L &= \sum_t \Big( \frac{\partial \Vh^{(t)}}{\partial \Vb^{(t)}} \Big)^\top \nabla_{\Vh^{(t)}} L 
 = \sum_t \text{diag} \Big( 1 - \big( \Vh^{(t)} \big)^2 \Big)  \nabla_{\Vh^{(t)}} L  ,\\
 \nabla_{\MV} L &= \sum_t \sum_i \Big( \frac{\partial L} {\partial o_i^{(t)}}\Big) \nabla_{\MV} o_i^{(t)} 
 = \sum_t (\nabla_{\Vo^{(t)}} L) \Vh^{(t)^\top},\\
 \nabla_{\MW} L &= \sum_t \sum_i \Big( \frac{\partial L} {\partial h_i^{(t)}}\Big) 
 \nabla_{\MW^{(t)}} h_i^{(t)} \\
&= \sum_t \text{diag} \Big( 1 - \big( \Vh^{(t)} \big)^2 \Big) ( \nabla_{\Vh^{(t)}} L) \Vh^{(t-1)^\top} ,\\
 \nabla_{\MU} L &= \sum_t \sum_i \Big( \frac{\partial L} {\partial h_i^{(t)}}\Big) 
 \nabla_{\MU^{(t)}} h_i^{(t)} \\
&= \sum_t \text{diag} \Big( 1 - \big( \Vh^{(t)} \big)^2 \Big) ( \nabla_{\Vh^{(t)}} L) \Vx^{(t)^\top} ,
\end{align}
因为\gls{computational_graph}中定义的损失的任何参数都不是训练数据$\Vx^{(t)}$的父节点,所以我们不需要计算关于它的\gls{gradient}。

% -- 375 --

\subsection{作为\glsentrytext{directed_graphical_model}的\glsentrytext{recurrent_network}}
\label{sec:recurrent_networks_as_directed_graphical_models}
目前为止,我们接触的\gls{recurrent_network}例子中损失$L^{(t)}$是训练目标$\Vy^{(t)}$和输出$\Vo^{(t)}$之间的\gls{cross_entropy}。
与\gls{feedforward_network}类似,原则上\gls{recurrent_network}几乎可以使用任何损失。
但必须根据任务来选择损失。
如\gls{feedforward_network},我们通常希望将~\glssymbol{RNN}~的输出解释为一个概率分布,并且我们通常使用与分布相关联的\gls{cross_entropy}来定义损失。
均方误差是与单位高斯分布的输出相关联的\gls{cross_entropy}损失,例如\gls{feedforward_network}中所使用的。

当我们使用一个预测性对数似然的训练目标,如\eqnref{eq:1012L},我们将~\glssymbol{RNN}~训练为能够根据之前的输入估计下一个序列元素$\Vy^{(t)}$的条件分布。
这可能意味着,我们最大化对数似然
\begin{align}
 \log p(\Vy^{(t)} \mid \Vx^{(1)},\dots, \Vx^{(t)}),
\end{align}
或者,如果模型包括来自一个\gls{time_step}的输出到下一个\gls{time_step}的连接,
\begin{align}
 \log p(\Vy^{(t)} \mid \Vx^{(1)},\dots, \Vx^{(t)},\Vy^{(1)},\dots, \Vy^{(t-1)} ).
\end{align}
将整个序列$\Vy$的联合分布分解为一系列单步的概率预测是捕获关于整个序列完整联合分布的一种方法。
当我们不把过去的$\Vy$值反馈给下一步作为预测的条件时,那么\gls{directed_graphical_model}不包含任何从过去$\Vy^{(i)}$到当前$\Vy^{(t)}$的边。
在这种情况下,给定x序列后,输出的y值之间条件独立。%在这种情况下,输出$\Vy$与给定的$\Vx$序列是条件独立的。% 
当我们反馈真实的$\Vy$值(不是它们的预测值,而是真正观测到或生成的值)给网络时,那么\gls{directed_graphical_model}包含所有从过去$\Vy^{(i)}$到当前$\Vy^{(t)}$的边。
\begin{figure}[!htb]
\ifOpenSource
\centerline{\includegraphics{figure.pdf}}
\else
\centerline{\includegraphics{Chapter10/figures/fully_connected_chain}}
\fi
\caption{序列$y^{(1)},y^{(2)},\dots,y^{(t)},\dots$的全连接\gls{graphical_model}。
给定先前的值,每个过去的观察值$y^{(i)}$可以影响一些$y^{(t)}$($t>i$)的条件分布。
当序列中每个元素的输入和参数的数目越来越多,根据此图直接参数化\gls{graphical_model}(如\eqnref{eq:106}中)可能是非常低效的。
\glssymbol{RNN}~可以通过高效的参数化获得相同的全连接,如\figref{fig:chap10_rnn_graphical_model_with_state}所示。
}
\label{fig:chap10_fully_connected_chain}
\end{figure}

% -- 376 --

举一个简单的例子,让我们考虑对标量随机变量序列$ \SetY = \{\RSy^{(1)},\dots,\RSy^{(\tau)}\}$建模的~\glssymbol{RNN},也没有额外的输入$\RSx$。
在\gls{time_step}~$t$的输入仅仅是\gls{time_step}~$t-1$的输出。
该~\glssymbol{RNN}~定义了关于$\RSy$变量的\gls{directed_graphical_model}。
我们使用链式法则(用于条件概率的\eqnref{eq:chap3_chain_rule})参数化这些观察值的联合分布:
\begin{align}
 P(\SetY) = P(\RVy^{(1)},\dots,\RVy^{(\tau)}) = \prod_{t=1}^{\tau}P(\RVy^{(t)} \mid \RVy^{(t-1)},\RVy^{(t-2)},
 \dots,\RVy^{(1)}),
\end{align}
其中当$t=1$时竖杠右侧显然为空。
因此,根据这样一个模型,一组值$\{y^{(1)},\dots,y^{(\tau)} \}$的负对数似然为
\begin{align}
 L = \sum_{t} L^{(t)},
\end{align}
其中
\begin{align}
 L^{(t)} = -\log P(\RSy^{(t)} = y^{(t)} \mid y^{(t-1)},y^{(t-2)}, \dots, y^{(1)}).
\end{align}
\begin{figure}[!htb]
\ifOpenSource
\centerline{\includegraphics{figure.pdf}}
\else
\centerline{\includegraphics{Chapter10/figures/rnn_graphical_model_with_state}}
\fi
\caption{在~\glssymbol{RNN}~\gls{graphical_model}中引入状态变量,尽管它是输入的确定性函数,但它有助于我们根据\eqnref{eq:105h}获得非常高效的参数化。
序列中的每个阶段(对于$\Vh^{(t)}$和$\Vy^{(t)}$)使用相同的结构(每个节点具有相同数量的输入),并且可以与其他阶段共享相同的参数。
}
\label{fig:chap10_rnn_graphical_model_with_state}
\end{figure}

\gls{graphical_model}中的边表示哪些变量直接依赖于其他变量。
许多\gls{graphical_model}的目标是省略不存在强相互作用的边以实现统计和计算的效率。
例如,我们通常可以作\ENNAME{Markov}假设,即\gls{graphical_model}应该只包含从$\{ \RSy^{(t-k)}, \dots, \RSy^{(t-1)}\}$到$\RSy^{(t)}$的边,而不是包含整个过去历史的边。
然而,在一些情况下,我们认为整个过去的输入会对序列的下一个元素有一定影响。
当我们认为$\RSy^{(t)}$的分布可能取决于遥远过去(在某种程度)的$\RSy^{(i)}$的值,且无法通过$\RSy^{(t-1)}$捕获$\RSy^{(i)}$的影响时,\glssymbol{RNN}~将会很有用。

解释~\glssymbol{RNN}~作为\gls{graphical_model}的一种方法是将\glssymbol{RNN}视为定义一个结构为完全图的\gls{graphical_model},且能够表示任何一对$\RSy$值之间的直接联系。
\figref{fig:chap10_fully_connected_chain}是关于$\RSy$值且具有完全图结构的\gls{graphical_model}。
该~\glssymbol{RNN}~完全图的解释基于排除并忽略模型中的\gls{hidden_unit}~$\Vh^{(t)}$。

% -- 377 --

更有趣的是,将\gls{hidden_unit}~$\Vh^{(t)}$视为随机变量,从而产生~\glssymbol{RNN}~的\gls{graphical_model}结构\footnote{给定这些变量的父变量,其条件分布是确定性的。
尽管设计具有这样确定性的\gls{hidden_unit}的\gls{graphical_model}是很少见的,但这是完全合理的。}。
在\gls{graphical_model}中包括隐藏单元预示~\glssymbol{RNN}~能对观测的联合分布提供非常有效的参数化。
假设我们用表格表示法来表示离散值上任意的联合分布,即对每个值可能的赋值分配一个单独条目的数组,该条目表示发生该赋值的概率。
如果$y$可以取$k$个不同的值,表格表示法将有$\CalO(k^\tau)$个参数。
对比~\glssymbol{RNN},由于\gls{parameter_sharing},\glssymbol{RNN}~的参数数目为$\CalO(1)$且是序列长度的函数。 % ??
我们可以调节~\glssymbol{RNN}~的参数数量来控制模型容量,但不用被迫与序列长度成比例。
\eqnref{eq:105h}展示了所述~\glssymbol{RNN}~通过循环应用相同的函数$f$以及在每个\gls{time_step}的相同参数$\Vtheta$,有效地参数化的变量之间的长期联系。
\figref{fig:chap10_rnn_graphical_model_with_state}说明了这个\gls{graphical_model}的解释。
在\gls{graphical_model}中结合$\Vh^{(t)}$节点可以用作过去和未来之间的中间量,从而将它们解耦。
遥远过去的变量$y^{(i)}$可以通过其对$\Vh$的影响来影响变量$y^{(t)}$。
该图的结构表明可以在\gls{time_step}使用相同的条件概率分布有效地参数化模型,并且当观察到全部变量时,可以高效地评估联合分配给所有变量的概率。

% -- 378 --

即便使用高效参数化的\gls{graphical_model},某些操作在计算上仍然具有挑战性。
例如,难以预测序列中缺少的值。

\gls{recurrent_network}为减少的参数数目付出的代价是\emph{优化}参数可能变得困难。

在\gls{recurrent_network}中使用的\gls{parameter_sharing}的前提是相同参数可用于不同\gls{time_step}的假设。
也就是说,假设给定时刻$t$的变量后,时刻$t +1$变量的条件概率分布是\firstgls{stationary},这意味着之前的\gls{time_step}与下个\gls{time_step}之间的关系并不依赖于$t$。
原则上,可以使用$t$作为每个\gls{time_step}的额外输入,并让学习器在发现任何时间依赖性的同时,在不同\gls{time_step}之间尽可能多地共享。
相比在每个$t$使用不同的条件概率分布已经好很多了,但网络将必须在面对新$t$时进行外推。

为了完整描述将~\glssymbol{RNN}~作为\gls{graphical_model}的观点,我们必须描述如何从模型采样。
我们需要执行的主要操作是简单地从每一\gls{time_step}的条件分布采样。
然而,还有一个额外的复杂性。
\glssymbol{RNN}~必须有某种机制来确定序列的长度。
这可以通过多种方式实现。

在当输出是从词汇表获取的符号的情况下,我们可以添加一个对应于序列末端的特殊符号\citep{schmidhuber2012self}。
当产生该符号时,采样过程停止。
在训练集中,我们将该符号作为序列的一个额外成员,即紧跟每个训练样本$\Vx^{(\tau)}$之后。

% -- 379 --

另一种选择是在模型中引入一个额外的~\ENNAME{Bernoulli}~输出,表示在每个\gls{time_step}决定继续生成或停止生成。
相比向词汇表增加一个额外符号,这种方法更通用,因为它适用于任何~\glssymbol{RNN},而不仅仅是输出符号序列的~\glssymbol{RNN}。
例如,它可以应用于一个产生实数序列的~\glssymbol{RNN}。
新的输出单元通常使用~\ENNAME{sigmoid}~单元,并以\gls{cross_entropy}作为损失函数进行训练。
在这种方法中,\ENNAME{sigmoid}~被训练为最大化正确预测的对数似然,即在每个\gls{time_step}序列决定结束或继续。

确定序列长度$\tau$的另一种方法是将一个额外的输出添加到模型并预测整数$\tau$本身。
模型可以采出$\tau$的值,然后采$\tau$步有价值的数据。
这种方法需要在每个\gls{time_step}的循环更新中增加一个额外输入,使得循环更新知道它是否是靠近所产生序列的末尾。
这种额外的输入可以是$\tau$的值,也可以是$\tau - t$即剩下\gls{time_step}的数量。
如果没有这个额外的输入,\glssymbol{RNN}~可能会产生突然结束序列,如一个句子在最终完整前结束。
此方法基于分解
\begin{align}
 P(\Vx^{(1)},\dots, \Vx^{(\tau)}) = P(\tau) P(\Vx^{(1)},\dots,\Vx^{(\tau)} \mid \tau) .
\end{align}
直接预测$\tau$的例子见\cite{Goodfellow+et+al-ICLR2014a}。

\subsection{基于上下文的~\glssymbol{RNN}~序列建模}
\label{sec:modeling_sequences_conditioned_on_context_with_rnns}
上一节描述了没有输入$\Vx$时,关于随机变量序列$y^{(t)}$的\glssymbol{RNN}如何对应于\gls{directed_graphical_model}。
当然,如\eqnref{eq:108a}所示的\glssymbol{RNN}包含一个输入序列$\Vx^{(1)},\Vx^{(2)},\dots,\Vx^{(\tau)}$。
一般情况下,\glssymbol{RNN}~允许将\gls{graphical_model}的观点扩展到不仅代表$y$变量的联合分布也能表示给定$\Vx$后$y$条件分布。
如在\secref{sec:learning_conditional_distributions_with_maximum_likelihood}的\gls{feedforward_network}情形中所讨论的,任何代表变量$P(\Vy;\Vtheta)$的模型都能被解释为代表条件分布$P(\Vy \mid \Vomega)$的模型,其中$\Vomega=\Vtheta$。
我们能像之前一样使用$P(\Vy \mid \Vomega)$代表分布$P(\Vy \mid \Vx)$来扩展这样的模型,但要令$\Vomega$是关于$\Vx$的函数。
在~\glssymbol{RNN}~的情况,这可以通过不同的方式来实现。
此处,我们回顾最常见和最明显的选择。

% -- 380 --

之前,我们已经讨论了将$t =1, \dots, \tau$的向量$\Vx^{(t)}$序列作为输入的~\glssymbol{RNN}。
另一种选择是只使用单个向量$\Vx$作为输入。
当$\Vx$是一个固定大小的向量时,我们可以简单地将其看作产生$\Vy$序列~\glssymbol{RNN}~的额外输入。
将额外输入提供到~\glssymbol{RNN}~的一些常见方法是:
\begin{enumerate}
 \item 在每个时刻作为一个额外输入,或
 \item 作为初始状态$\Vh^{(0)}$,或
 \item 结合两种方式。
\end{enumerate}

第一个也是最常用的方法如\figref{fig:chap10_conditional_rnn}所示。
输入$\Vx$和每个隐藏单元向量$\Vh^{(t)}$之间的相互作用是通过新引入的权重矩阵$\MR$参数化的,这是只包含$y$序列的模型所没有的。
同样的乘积$\Vx^\top\MR$在每个\gls{time_step}作为\gls{hidden_unit}的一个额外输入。
我们可以认为$\Vx$的选择(确定$\Vx^\top\MR$值),是有效地用于每个\gls{hidden_unit}的一个新\gls{bias_aff}参数。
权重与输入保持独立。
我们可以认为这种模型采用了非条件模型的$\Vtheta$,并将$\Vomega$代入$\Vtheta$,其中$\Vomega$内的\gls{bias_aff}参数现在是输入的函数。

\begin{figure}[!htb]
\ifOpenSource
\centerline{\includegraphics{figure.pdf}}
\else
\centerline{\includegraphics{Chapter10/figures/conditional_rnn}}
\fi
\caption{将固定长度的向量$\Vx$映射到序列$\MY$上分布的~\glssymbol{RNN}。
这类~\glssymbol{RNN}~适用于很多任务如图注,其中单个图像作为模型的输入,然后产生描述图像的词序列。
观察到的输出序列的每个元素$\Vy^{(t)}$同时用作输入(对于当前\gls{time_step})和训练期间的目标(对于前一\gls{time_step})。
}
\label{fig:chap10_conditional_rnn}
\end{figure}

% -- 381 --

\glssymbol{RNN}~可以接收向量序列$\Vx^{(t)}$作为输入,而不是仅接收单个向量$\Vx$作为输入。
\eqnref{eq:108a}描述的~\glssymbol{RNN}~对应条件分布$P(\Vy^{(1)}, \dots, \Vy^{(\tau)} \mid \Vx^{(1)}, \dots, \Vx^{(\tau)})$,并在条件独立的假设下这个分布分解为
\begin{align}
 \prod_t P(\Vy^{(t)} \mid \Vx^{(1)}, \dots, \Vx^{(t)}).
\end{align}
为去掉条件独立的假设,我们可以在时刻$t$的输出到时刻$t+1$的\gls{hidden_unit}添加连接,如\figref{fig:chap10_causal_rnn}所示。
该模型就可以代表关于$\Vy$序列的任意概率分布。
这种给定一个序列表示另一个序列分布的模型的还是有一个限制,就是这两个序列的长度必须是相同的。
我们将在\secref{sec:encoder_decoder_sequence_to_sequence_architectures}描述如何消除这种限制。

\begin{figure}[!htb]
\ifOpenSource
\centerline{\includegraphics{figure.pdf}}
\else
\centerline{\includegraphics{Chapter10/figures/causal_rnn}}
\fi
\caption{将可变长度的$\Vx$值序列映射到相同长度的$\Vy$值序列上分布的条件\gls{RNN}。
对比\figref{fig:chap10_hidden_recurrence_rnn},此~\glssymbol{RNN}~包含从前一个输出到当前状态的连接。
这些连接允许此\glssymbol{RNN}对给定$\Vx$的序列后相同长度的$\Vy$序列上的任意分布建模。
\figref{fig:chap10_hidden_recurrence_rnn}的~\glssymbol{RNN}~仅能表示在给定$\Vx$值的情况下,$\Vy$值彼此条件独立的分布。
}
\label{fig:chap10_causal_rnn}
\end{figure}


% -- 382 --

\section{双向~\glssymbol{RNN}}
\label{sec:bidirectional_rnns}
目前为止我们考虑的所有\gls{RNN}有一个``因果''结构,意味着在时刻$t$的状态只能从过去的序列$\Vx^{(1)},\dots,\Vx^{(t-1)}$以及当前的输入$\Vx^{(t)}$捕获信息。
我们还讨论了某些在$\Vy$可用时,允许过去的$\Vy$值信息影响当前状态的模型。

然而,在许多应用中,我们要输出的$\Vy^{(t)}$的预测可能依赖于整个输入序列。
例如,在语音识别中,由于协同发音,当前声音作为音素的正确解释可能取决于未来几个音素,甚至潜在的可能取决于未来的几个词,因为词与附近的词之间存在语言学上的依赖:如果当前的词有两种声学上合理的解释,我们可能要在更远的未来(和过去)寻找信息区分它们。
这在手写识别和许多其他序列到序列学习的任务中也是如此,将会在下一节中描述。

双向\gls{RNN}(或双向~\glssymbol{RNN})为满足这种需要而被发明\citep{Schuster+Paliwal-1997}。
他们在需要双向信息的应用中非常成功\citep{Graves-book2012},如手写识别\citep{Graves-et-al-NIPS2007,Graves+Schmidhuber-2009},语音识别\citep{Graves+Schmidhuber-2005,Graves-et-al-ICASSP2013}以及生物信息学\citep{Baldi-et-al-1999}。

顾名思义,双向~\glssymbol{RNN}~结合时间上从序列起点开始移动的~\glssymbol{RNN}~和另一个时间上从序列末尾开始移动的~\glssymbol{RNN}。
\figref{fig:chap10_bidirectional_rnn}展示了典型的双向~\glssymbol{RNN},其中$\Vh^{(t)}$代表通过时间向前移动的子~\glssymbol{RNN}~的状态,$\Vg^{(t)}$代表通过时间向后移动的子~\glssymbol{RNN}~的状态。
这允许输出单元$\Vo^{(t)}$能够计算同时依赖于过去和未来且对时刻$t$周围的输入值最敏感的表示,而不必指定$t$周围固定大小的窗口(这是\gls{feedforward_network}、\gls{convolutional_network}或具有固定大小的先行缓存器的常规~\glssymbol{RNN}~所必须要做的)。

% -- 383 --

\begin{figure}[!htb]
\ifOpenSource
\centerline{\includegraphics{figure.pdf}}
\else
\centerline{\includegraphics{Chapter10/figures/bidirectional_rnn}}
\fi
\caption{典型的双向\gls{RNN}中的计算,意图学习将输入序列$\Vx$映射到目标序列$\Vy$(在每个步骤$t$具有损失$L^{(t)}$)。
循环性$\Vh$在时间上向前传播信息(向右),而循环性$\Vg$在时间上向后传播信息(向左)。
因此在每个点$t$,输出单元$\Vo^{(t)}$可以受益于输入$\Vh^{(t)}$中关于过去的相关概要以及输入$\Vg^{(t)}$中关于未来的相关概要。
}
\label{fig:chap10_bidirectional_rnn}
\end{figure}

这个想法可以自然地扩展到2维输入,如图像,由\emph{四个}~\glssymbol{RNN}~组成,每一个沿着四个方向中的一个计算:上、下、左、右。
如果~\glssymbol{RNN}~能够学习到承载长期信息,那在2维网格每个点$(i, j)$的输出$O_{i,j}$就能计算一个主要捕捉局部信息但同时又依赖于长期输入的表示。
相比\gls{convolutional_network},应用于图像的~\glssymbol{RNN}~计算成本通常更高,但允许同一特征图的特征之间存在长期横向的相互作用\citep{Visin-et-al-arxiv2015-renet,kalchbrenner2015grid}。
实际上,对于这样的~\glssymbol{RNN},\gls{forward_propagation}公式可以写成表示使用卷积的形式,计算自底向上到每一层的输入(在整合横向相互作用的特征图的循环传播之前)。

% -- 384 --

\section{基于编码-解码的序列到序列架构}
\label{sec:encoder_decoder_sequence_to_sequence_architectures}
我们已经在\figref{fig:chap10_single_output_rnn}看到\glssymbol{RNN}如何将输入序列映射成固定大小的向量,在\figref{fig:chap10_conditional_rnn}中看到\glssymbol{RNN}如何将固定大小的向量映射成一个序列,在\figref{fig:chap10_hidden_recurrence_rnn}、\figref{fig:chap10_output_recurrence_rnn}、\figref{fig:chap10_causal_rnn}和\figref{fig:chap10_bidirectional_rnn}中看到\glssymbol{RNN}如何将一个输入序列映射到等长的输出序列。

本节我们讨论如何训练\glssymbol{RNN},使其将输入序列映射到不一定等长的输出序列。
这在许多场景中都有应用,如语音识别、机器翻译或问答,其中训练集的输入和输出序列的长度通常不相同(虽然它们的长度可能相关)。

我们经常将\glssymbol{RNN}的输入称为``上下文''。
我们希望产生此上下文的表示,$C$。
这个上下文$C$可能是一个概括输入序列$\MX=(\Vx^{(1)},\dots,\Vx^{(n_x)})$的向量或者向量序列。

用于映射可变长度序列到另一可变长度序列最简单的\glssymbol{RNN}架构最初由\cite{cho-al-emnlp14}提出,之后不久由\cite{Sutskever-et-al-NIPS2014}独立开发,并且第一个使用这种方法获得翻译的最好结果。
前一系统是对另一个机器翻译系统产生的建议进行评分,而后者使用独立的\gls{recurrent_network}生成翻译。
这些作者分别将该架构称为编码-解码或序列到序列架构,如\figref{fig:chap10_rnn_encdec}所示。
这个想法非常简单:(1)\firstgls{encoder}或\,\textbf{读取器}\,(reader)或\,\textbf{输入}(input)~\glssymbol{RNN}~处理输入序列。
\gls{encoder}输出上下文$C$(通常是最终隐藏状态的简单函数)。
(2)\firstgls{decoder}或\,\textbf{写入器}(writer)或\,\textbf{输出}(output)~\glssymbol{RNN}~则以固定长度的向量(如\figref{fig:chap10_conditional_rnn})为条件产生输出序列$\MY=(\Vy^{(1)}, \dots, \Vy^{(n_y)})$。
这种架构对比本章前几节提出的架构的创新之处在于长度$n_x$和$n_y$可以彼此不同,而之前的架构约束$n_x = n_y = \tau$。
在序列到序列的架构中,两个~\glssymbol{RNN}~共同训练以最大化$\log P( \Vy^{(1)}, \dots, \Vy^{(n_y)} \mid \Vx^{(1)},\dots,\Vx^{(n_x)} )$(关于训练集中所有$\Vx$和$\Vy$对的平均)。
\gls{encoder}~\glssymbol{RNN}~的最后一个状态$\Vh_{n_x}$通常被当作输入的表示$C$并作为\gls{decoder}~\glssymbol{RNN}~的输入。

% -- 385 --
\begin{figure}[!htb]
\ifOpenSource
\centerline{\includegraphics{figure.pdf}}
\else
\centerline{\includegraphics{Chapter10/figures/rnn_encdec}}
\fi
\caption{在给定输入序列$(\RVx^{(1)},\RVx^{(2)},\dots,\RVx^{(n_x)})$的情况下学习生成输出序列$(\RVy^{(1)},\RVy^{(2)},\dots,\RVy^{(n_y)})$的\gls{encoder}-\gls{decoder}或序列到序列的~\glssymbol{RNN}~架构的示例。 
它由读取输入序列的\gls{encoder}~\glssymbol{RNN}~以及生成输出序列(或计算给定输出序列的概率)的\gls{decoder}~\glssymbol{RNN}~组成。
\gls{encoder}~\glssymbol{RNN}~的最终隐藏状态用于计算一般为固定大小的上下文变量$C$,$C$表示输入序列的语义概要并且作为\gls{decoder}~\glssymbol{RNN}~的输入。
}
\label{fig:chap10_rnn_encdec}
\end{figure}

如果上下文$C$是一个向量,则\gls{decoder}~\glssymbol{RNN}~只是在\secref{sec:modeling_sequences_conditioned_on_context_with_rnns}描述的向量到序列~\glssymbol{RNN}。
正如我们所见,向量到序列~\glssymbol{RNN}~至少有两种接受输入的方法。
输入可以被提供为~\glssymbol{RNN}~的初始状态,或连接到每个\gls{time_step}中的\gls{hidden_unit}。
这两种方式也可以结合。

这里并不强制要求\gls{encoder}与\gls{decoder}的隐藏层具有相同的大小。

此架构的一个明显不足是,\gls{encoder}~\glssymbol{RNN}~输出的上下文$C$的维度太小而难以适当地概括一个长序列。
这种现象由\cite{Bahdanau-et-al-ICLR2015-small}在机器翻译中观察到。
他们提出让$C$成为可变长度的序列,而不是一个固定大小的向量。
此外,他们还引入了将序列$C$的元素和输出序列的元素相关联的\firstgls{attention_mechanism}。
读者可在\secref{sec:using_an_attention_mechanism_and_aligning_pieces_of_data}了解更多细节。

% -- 386 --

\section{深度\glsentrytext{recurrent_network}}
\label{sec:deep_recurrent_networks}
大多数~\glssymbol{RNN}~中的计算可以分解成三块参数及其相关的变换:
\begin{enumerate}
 \item 从输入到隐藏状态,
 \item 从前一隐藏状态到下一隐藏状态,以及
 \item 从隐藏状态到输出。
\end{enumerate}
根据\figref{fig:chap10_hidden_recurrence_rnn}中的~\glssymbol{RNN}~架构,这三个块都与单个权重矩阵相关联。
换句话说,当网络被\gls{unfolding}时,每个块对应一个浅的变换。
能通过深度~\glssymbol{MLP}~内单个层来表示的变换称为浅变换。
通常,这是由学成的仿射变换和一个固定非线性表示组成的变换。

在这些操作中引入深度会有利的吗?
实验证据\citep{Graves-et-al-ICASSP2013,Pascanu-et-al-ICLR2014}强烈暗示理应如此。
实验证据与我们需要足够的深度以执行所需映射的想法一致。
读者可以参考~\cite{Schmidhuber92,ElHihi+Bengio-nips8}或~\cite{Jaeger2007}了解更早的关于深度~\glssymbol{RNN}~的研究。

\cite{Graves-et-al-ICASSP2013}第一个展示了将~\glssymbol{RNN}~的状态分为多层的显著好处,如\figref{fig:chap10_deep}~\emph{(左)}。
我们可以认为,在\figref{fig:chap10_deep}(a)所示层次结构中较低的层起到了将原始输入转化为对更高层的隐藏状态更合适表示的作用。
\cite{Pascanu-et-al-ICLR2014}更进一步提出在上述三个块中各使用一个单独的~\glssymbol{MLP}(可能是深度的),如\figref{fig:chap10_deep}(b)所示。
考虑表示容量,我们建议在这三个步中都分配足够的容量,但增加深度可能会因为优化困难而损害学习效果。
在一般情况下,更容易优化较浅的架构,加入\figref{fig:chap10_deep}(b)的额外深度导致从\gls{time_step}~$t$的变量到\gls{time_step}~$t+1$的最短路径变得更长。
例如,如果具有单个隐藏层的~\glssymbol{MLP}~被用于状态到状态的转换,那么与\figref{fig:chap10_hidden_recurrence_rnn}相比,我们就会加倍任何两个不同\gls{time_step}变量之间最短路径的长度。
然而\cite{Pascanu-et-al-ICLR2014}认为,在隐藏到隐藏的路径中引入\gls{skip_connection}可以缓和这个问题,如\figref{fig:chap10_deep}(c)所示。

% -- 387 --

\begin{figure}[!htb]
\ifOpenSource
\centerline{\includegraphics{figure.pdf}}
\else
\centerline{\includegraphics{Chapter10/figures/deep}}
\fi
\caption{\gls{RNN}可以通过许多方式变得更深\citep{Pascanu-et-al-ICLR2014}。
(a)隐藏循环状态可以被分解为具有层次的组。
(b)可以向输入到隐藏,隐藏到隐藏以及隐藏到输出的部分引入更深的计算(如~\glssymbol{MLP})。
这可以延长链接不同\gls{time_step}的最短路径。
(c)可以引入\gls{skip_connection}来缓解路径延长的效应。
}
\label{fig:chap10_deep}
\end{figure}

\section{递归神经网络}
\label{sec:recursive_neural_networks}
递归神经网络\footnote{我们建议不要将``递归神经网络''缩写为``\glssymbol{RNN}'',以免与``\gls{RNN}''混淆。}代表\gls{recurrent_network}的另一个扩展,它被构造为深的树状结构而不是~\glssymbol{RNN}~的链状结构,因此是不同类型的\gls{computational_graph}。
递归网络的典型\gls{computational_graph}如\figref{fig:chap10_recursive_net}所示。
递归神经网络由~\cite{Pollack90}引入,而~\cite{tr-bottou-2011}描述了这类网络的潜在用途——学习推论。
递归网络已成功地应用于输入是\emph{数据结构}的\gls{NN}~\citep{Frasconi97,Frasconi-1998},如自然语言处理\citep{Socher+al-NIPS2011,Socher+al-EMNLP2011-small,Socher-et-al-EMNLP2013}和计算机视觉\citep{Socher-2011}。

% -- 388 --

递归网络的一个明显优势是,对于具有相同长度$\tau$的序列,深度(通过非线性操作的组合数量来衡量)可以急剧地从$\tau$减小为$\CalO(\log \tau)$,这可能有助于解决\gls{long_term_dependency}。
一个悬而未决的问题是如何以最佳的方式构造树。
一种选择是使用不依赖于数据的树结构,如平衡二叉树。
在某些应用领域,外部方法可以为选择适当的树结构提供借鉴。
例如,处理自然语言的句子时,用于递归网络的树结构可以被固定为句子语法分析树的结构(可以由自然语言语法分析程序提供)\citep{Socher+al-NIPS2011,Socher+al-EMNLP2011-small}。
理想的情况下,人们希望学习器自行发现和推断适合于任意给定输入的树结构,如\citep{tr-bottou-2011}所建议。

% -- 389 --

\begin{figure}[!htb]
\ifOpenSource
\centerline{\includegraphics{figure.pdf}}
\else
\centerline{\includegraphics{Chapter10/figures/recursive_net}}
\fi
\caption{递归网络将循环网络的链状计算图推广到树状计算图。
可变大小的序列$\Vx^{(1)},\Vx^{(2)},\dots,\Vx^{(t)}$可以通过固定的参数集合(权重矩阵$\MU,\MV,\MW$)映射到固定大小的表示(输出$\Vo$)。
该图展示了\gls{supervised_learning}的情况,其中提供了一些与整个序列相关的目标$\Vy$。
}
\label{fig:chap10_recursive_net}
\end{figure}

递归网络想法的变种存在很多可能。
例如,\cite{Frasconi97}和~\cite{Frasconi-1998}将数据与树结构相关联,并将输入和目标与树的单独节点相关联。
由每个节点执行的计算无须是传统的人工神经计算(所有输入的仿射变换后跟一个单调非线性)。
例如,\cite{Socher-et-al-EMNLP2013}提出用张量运算和双线性形式,在这之前人们已经发现当概念是由连续向量(嵌入)表示时,这种方式有利于建模概念之间的联系\citep{Weston+Bengio+Usunier-2010,Bordes-et-al-AISTATS2012-small}。

\section{\glsentrytext{long_term_dependency}的挑战}
\label{sec:the_challenge_of_long_term_dependencies}
学习\gls{recurrent_network}\gls{long_term_dependency}的数学挑战在\secref{sec:long_term_dependencies}中引入。
根本问题是,经过许多阶段传播后的\gls{gradient}倾向于消失(大部分情况)或爆炸(很少,但对优化过程影响很大)。
即使我们假设\gls{recurrent_network}是参数稳定的(可存储记忆,且\gls{gradient}不爆炸),但\gls{long_term_dependency}的困难来自比短期相互作用指数小的权重(涉及许多~\gls{jacobian}~相乘)。
许多资料提供了更深层次的讨论\citep{Hochreiter91-small,Doya93,Bengio1994ITNN,Pascanu+al-ICML2013-small}。
在这一节中,我们会更详细地描述该问题。
其余几节介绍克服这个问题的方法。

% -- 390 --

\gls{recurrent_network}涉及相同函数的多次组合,每个\gls{time_step}一次。
这些组合可以导致极端非线性行为,如\figref{fig:chap10_composition_color}所示。
\begin{figure}[!htb]
\ifOpenSource
\centerline{\includegraphics{figure.pdf}}
\else
\centerline{\includegraphics{Chapter10/figures/composition_color}}
\fi
\caption{重复组合函数。
当组合许多非线性函数(如这里所示的线性tanh层)时,结果是高度非线性的,通常大多数值与微小的导数相关联,也有一些具有大导数的值,以及在增加和减小之间的多次交替。% ??
此处,我们绘制从100维隐藏状态降到单个维度的线性投影,绘制于$y$轴上。
$x$轴是100维空间中沿着随机方向的初始状态的坐标。
因此,我们可以将该图视为高维函数的线性截面。
曲线显示每个\gls{time_step}之后的函数,或者等价地,转换函数被组合一定次数之后。
}
\label{fig:chap10_composition_color}
\end{figure}

特别地,\gls{RNN}所使用的函数组合有点像矩阵乘法。
我们可以把如下的循环联系
\begin{align}
 \Vh^{(t)} = \MW^\top \Vh^{(t-1)}
\end{align}
看做是一个非常简单的、缺少非线性激活函数和输入$\Vx$的\gls{RNN}。
如\secref{sec:long_term_dependencies}描述,这种循环关系本质上描述了幂法。
它可以被简化为
\begin{align}
 \Vh^{(t)} = (\MW^t)^\top \Vh^{(0)},
\end{align}
而当$\MW$符合下列形式的特征分解
\begin{align}
 \MW = \MQ \VLambda \MQ^\top,
\end{align}
其中$\MQ$正交,循环性可进一步简化为
\begin{align}
 \Vh^{(t)} = \MQ^\top \VLambda^t \MQ \Vh^{(0)}.
\end{align}
特征值提升到$t$次后,导致幅值不到一的特征值衰减到零,而幅值大于一的就会激增。
任何不与最大特征向量对齐的$\Vh^{(0)}$的部分将最终被丢弃。


这个问题是针对\gls{recurrent_network}的。
在标量情况下,想象多次乘一个权重$w$。
该乘积$w^t$消失还是爆炸取决于$w$的幅值。
然而,如果每个时刻使用不同权重$w^{(t)}$的非\gls{recurrent_network},情况就不同了。
如果初始状态给定为$1$,那么时刻$t$的状态可以由$\prod_t w^{(t)}$给出。
假设$w^{(t)}$的值是随机生成的,各自独立,且有$0$均值$v$方差。
乘积的方差就为$\CalO(v^n)$。
为了获得某个期望的方差$v^*$,我们可以选择让单个权重的方差为$v=\sqrt[n]{v^*}$。
因此,非常深的\gls{feedforward_network}通过精心设计的比例可以避免\gls{gradient}消失和爆炸问题,如~\cite{Sussillo14}所主张的。

% -- 391 --

\glssymbol{RNN}~\gls{gradient}消失和爆炸问题是由不同研究人员独立发现\citep{Hochreiter91-small,Bengio_icnn93,Bengio1994ITNN}。
有人可能会希望通过简单地停留在\gls{gradient}不消失或爆炸的参数空间来避免这个问题。
不幸的是,为了储存记忆并对小扰动具有鲁棒性,\glssymbol{RNN}~必须进入参数空间中的\gls{gradient}消失区域\citep{Bengio_icnn93,Bengio1994ITNN}。
具体来说,每当模型能够表示\gls{long_term_dependency}时,长期相互作用的\gls{gradient}幅值就会变得指数级小(相比短期相互作用的\gls{gradient}幅值)。
这并不意味着这是不可能学习的,由于\gls{long_term_dependency}关系的信号很容易被短期相关性产生的最小波动隐藏,因而学习\gls{long_term_dependency}可能需要很长的时间。
实践中,\cite{Bengio1994ITNN}的实验表明,当我们增加了需要捕获的依赖关系的跨度,基于\gls{gradient}的优化变得越来越困难,\glssymbol{SGD}~在长度仅为10或20的序列上成功训练传统~\glssymbol{RNN}~的概率迅速变为0。

将\gls{recurrent_network}作为动力系统更深入探讨的资料见~\cite{Doya93,Bengio1994ITNN,Siegelmann+Sontag-1995}及~
\cite{Pascanu-et-al-ICML2013}的回顾。
本章的其余部分将讨论目前已经提出的降低学习\gls{long_term_dependency}(在某些情况下,允许一个~\glssymbol{RNN}~学习横跨数百步的依赖)难度的不同方法,但学习\gls{long_term_dependency}的问题仍是\gls{DL}中的一个主要挑战。


\section{\glsentrytext{ESN}}
\label{sec:echo_state_networks}
从$\Vh^{(t-1)}$到$\Vh^{(t)}$的循环权重映射以及从$\Vx^{(t)}$到$\Vh^{(t)}$的输入权重映射是\gls{recurrent_network}中最难学习的参数。
研究者\citep{Jaeger-NIPS2002,Maass-et-al-2002,Jaeger+Haas-2004,Jaeger-2007}提出避免这种困难的方法是设定循环\gls{hidden_unit},使其能很好地捕捉过去输入历史,并且\emph{只学习输出权重}。
\firstgls{ESN}或~\glssymbol{ESN}~\citep{Jaeger+Haas-2004,Jaeger-2007},以及\firstgls{liquid_state_machines}\citep{Maass-et-al-2002}分别独立地提出了这种想法。
后者是类似的,只不过它使用脉冲神经元(二值输出)而不是~\glssymbol{ESN}~中的连续\gls{hidden_unit}。
\glssymbol{ESN}~和\gls{liquid_state_machines}都被称为\firstgls{reservoir_computing}\citep{Lukosevicius+Jaeger-2009},因为\gls{hidden_unit}形成了可能捕获输入历史不同方面的临时特征池。

% -- 392 --

\gls{reservoir_computing}\gls{recurrent_network}类似于\gls{kernel_machines},这是思考它们的一种方式:它们将任意长度的序列(到时刻$t$的输入历史)映射为一个长度固定的向量(循环状态$\Vh^{(t)}$),之后可以施加一个线性预测算子(通常是一个\gls{linear_regression})以解决感兴趣的问题。
训练\gls{criterion}就可以很容易地设计为输出权重的凸函数。
例如,如果输出是从\gls{hidden_unit}到输出目标的\gls{linear_regression},训练\gls{criterion}就是\gls{mean_squared_error},由于是凸的就可以用简单的学习算法可靠地解决\citep{Jaeger-NIPS2002}。

因此,重要的问题是:我们如何设置输入和循环权重才能让一组丰富的历史可以在\gls{RNN}的状态中表示?
\gls{reservoir_computing}研究给出的答案是将\gls{recurrent_network}视为动态系统,并设定让动态系统接近稳定边缘的输入和循环权重。

最初的想法是使状态到状态转换函数的~\gls{jacobian}~矩阵的特征值接近1。
如\secref{sec:long_term_dependencies}解释,\gls{recurrent_network}的一个重要特征就是~\gls{jacobian}~矩阵的特征值谱$\MJ^{(t)} = \frac{\partial s^{(t)}}{\partial s^{(t-1)}}$。
特别重要的是$\MJ^{(t)}$的\firstgls{spectral_radius},定义为特征值的最大绝对值。

为了解\gls{spectral_radius}的影响,可以考虑\gls{backward_propagation}中~\gls{jacobian}~矩阵$\MJ$不随$t$改变的简单情况。
例如当网络是纯线性时,会发生这种情况。
假设$\MJ$特征值$\lambda$对应的特征向量为$\Vv$。
考虑当我们通过时间向后传播\gls{gradient}向量时会发生什么。
如果刚开始的\gls{gradient}向量为$\Vg$,然后经过\gls{backward_propagation}的一个步骤后,我们将得到$\MJ \Vg$,$n$步之后我们会得到$\MJ^n \Vg$。
现在考虑如果我们向后传播扰动版本的$\Vg$会发生什么。
如果我们刚开始是$\Vg + \delta \Vv$,一步之后,我们会得到$\MJ(\Vg + \delta \Vv)$。
$n$步之后,我们将得到$\MJ^n(\Vg + \delta \Vv)$。
由此我们可以看出,由$\Vg$开始的\gls{backward_propagation}和由$\Vg+\delta \Vv$开始的\gls{backward_propagation},$n$步之后偏离$\delta \MJ^n \Vv$。
如果$\Vv$选择为$\MJ$特征值$\lambda$对应的一个单位特征向量,那么在每一步乘~\gls{jacobian}~矩阵只是简单地缩放。
\gls{backward_propagation}的两次执行分离的距离为$\delta | \lambda |^n$。
当$\Vv$对应于最大特征值$|\lambda|$,初始扰动为$\delta$时这个扰动达到可能的最宽分离。

当$ | \lambda | > 1$,偏差$\delta | \lambda |^n$就会指数增长。
当$ | \lambda | < 1$,偏差就会变得指数小。

% -- 393 --

当然,这个例子假定~\gls{jacobian}~矩阵在每个\gls{time_step}是相同的,即对应于没有非线性的\gls{recurrent_network}。
当非线性存在时,非线性的导数将在许多\gls{time_step}后接近零,并有助于防止因过大的\gls{spectral_radius}而导致的爆炸。
事实上,关于\gls{ESN}的最近工作提倡使用远大于1的\gls{spectral_radius}~\citep{yildiz2012re,jaeger2012long}。

我们上面说的这些关于用重复矩阵相乘进行\gls{backward_propagation}的\gls{recurrent_network}的事实同样适用于没有非线性的正向传播的网络,其状态为$\Vh^{(t+1)} = \Vh^{(t)\top} \MW $。

如果线性映射$\MW^\top$在$L^2$范数的测度下总是缩小$\Vh$,那么我们说这个映射是\firstgls{contractive}的。
当\gls{spectral_radius}小于一,则从$\Vh^{(t)}$到$\Vh^{(t+1)}$的映射是\gls{contractive}的,因此小变化在每个\gls{time_step}后变得更小。
当我们使用有限精度(如32位整数)来存储状态向量时,必然会使得网络忘掉过去的信息。

\gls{jacobian}~矩阵告诉我们$\Vh^{(t)}$一个微小的变化如何向前一步传播,或等价的,$\Vh^{(t+1)}$的\gls{gradient}如何向后一步传播。
需要注意的是,$\MW$和$\MJ$都不需要是对称的(尽管它们是实方阵),因此它们可能有复的特征值和特征向量,其中虚数分量对应于潜在的振荡行为(如果迭代地应用同一~\gls{jacobian})。
即使$\Vh^{(t)}$或$\Vh^{(t)}$中有趣的小变化在\gls{backward_propagation}中是实值的,它们仍可以用这样的复数基表示。
重要的是,当向量乘以矩阵时,这些复数基的系数幅值(复数的绝对值)会发生什么变化。
幅值大于1的特征值对应于放大(如果反复应用则指数增长)或收缩(如果反复应用则指数减小)。

非线性映射情况时,\gls{jacobian}~会在每一步任意变化。
因此,动态量变得更加复杂。
然而,一个小的初始变化多步之后仍然会变成一个大的变化。
纯线性和非线性情况的一个不同之处在于使用压缩非线性(如$\tanh$)可以使循环动态量有界。
注意,即使\gls{forward_propagation}动态量有界,\gls{backward_propagation}的动态量仍然可能无界,例如,当$\tanh$序列都在它们状态中间的线性部分,并且由\gls{spectral_radius}大于1的权重矩阵连接。
然而,所有$\tanh$单元同时位于它们的线性激活点是非常罕见的。

% -- 394 --

\gls{ESN}的策略是简单地固定权重使其具有一定的\gls{spectral_radius}如3,其中信息通过时间前向传播,但会由于饱和非线性单元(如$\tanh$)的稳定作用而不会爆炸。

最近,已经有研究表明,用于设置~\glssymbol{ESN}~权重的技术可以用来\emph{初始化}完全可训练的\gls{recurrent_network}的权重(通过时间\gls{backward_propagation}来训练隐藏到隐藏的循环权重),帮助学习\gls{long_term_dependency}~\citep{Sutskever-thesis2012,sutskeverimportance}。
在这种设定下,结合\secref{sec:parameter_initialization_strategies}中稀疏初始化的方案,设置$1.2$的初始\gls{spectral_radius}表现不错。

\section{\glsentrytext{leaky_unit}和其他多时间尺度的策略}
\label{sec:leaky_units_and_other_strategiesfor_multiple_time_scales}
处理\gls{long_term_dependency}的一种方法是设计工作在多个时间尺度的模型,使模型的某些部分在细粒度时间尺度上操作并能处理小细节,而其他部分在粗时间尺度上操作并能把遥远过去的信息更有效地传递过来。
存在多种同时构建粗细时间尺度的策略。
这些策略包括在时间轴增加\gls{skip_connection},``\gls{leaky_unit}''使用不同时间常数整合信号,以及去除一些用于建模细粒度时间尺度的连接。

\subsection{时间维度的\glsentrytext{skip_connection}}
\label{sec:adding_skip_connections_through_time}
增加从遥远过去的变量到目前变量的直接连接是得到粗时间尺度的一种方法。
使用这样\gls{skip_connection}的想法可以追溯到\cite{Lin-ieeetnn96},紧接是向\gls{feedforward_network}引入延迟的想法\citep{Lang+Hinton88}。
在普通的\gls{recurrent_network}中,循环从时刻$t$的单元连接到时刻$t+1$单元。
构造较长的延迟\gls{recurrent_network}是可能的\citep{Bengio91}。

正如我们在\secref{sec:long_term_dependencies}看到,\gls{gradient}可能\emph{关于时间步数}呈指数消失或爆炸。
\citep{Lin-ieeetnn96}引入了$d$延时的循环连接以减轻这个问题。
现在导数指数减小的速度与$\frac{\tau}{d}$相关而不是$\tau$。
既然同时存在延迟和单步连接,\gls{gradient}仍可能成$t$指数爆炸。
这允许学习算法捕获更长的依赖性,但不是所有的\gls{long_term_dependency}都能在这种方式下良好地表示。

% -- 395 --

\subsection{\glsentrytext{leaky_unit}和一系列不同时间尺度}
\label{sec:leaky_units_and_a_spectrum_of_different_time_scales}
获得导数乘积接近1的另一方式是设置\emph{线性}自连接单元,并且这些连接的权重接近1。

我们对某些$v$值应用更新$\mu^{(t)} \gets \alpha \mu^{(t-1)} + (1-\alpha) v^{(t)}$累积一个滑动平均值$\mu^{(t)}$,其中$\alpha$是一个从$ \mu^{(t-1)}$到$ \mu^{(t)}$线性自连接的例子。
当$\alpha$接近1时,滑动平均值能记住过去很长一段时间的信息,而当$\alpha$接近0,关于过去的信息被迅速丢弃。
线性自连接的\gls{hidden_unit}可以模拟滑动平均的行为。
这种\gls{hidden_unit}称为\firstgls{leaky_unit}。

$d$~\gls{time_step}的\gls{skip_connection}可以确保单元总能被$d$个\gls{time_step}前的那个值影响。
使用权重接近1的线性自连接是确保该单元可以访问过去值的不同方式。
线性自连接通过调节实值$\alpha$(而不是调整整数值的跳跃长度)更平滑灵活地调整这种效果。

这个想法由~\cite{Mozer-nips92}和~\cite{ElHihi+Bengio-nips8}提出。
在\gls{ESN}中,\gls{leaky_unit}也被发现很有用\citep{JaegerLPS07}。

我们可以通过两种基本策略设置\gls{leaky_unit}使用的时间常数。
一种策略是手动将其固定为常数,例如在初始化时从某些分布采样它们的值。
另一种策略是使时间常数成为自由变量,并学习出来。
在不同时间尺度使用这样的\gls{leaky_unit}似乎能帮助学习\gls{long_term_dependency}~\citep{Mozer-nips92,Pascanu+al-ICML2013-small}。

\subsection{删除连接}
\label{sec:removing_connections}
处理\gls{long_term_dependency}另一种方法是在多个时间尺度组织~\glssymbol{RNN}~状态的想法\citep{ElHihi+Bengio-nips8},信息在较慢的时间尺度上更容易长距离流动。

这个想法与之前讨论的时间维度上的\gls{skip_connection}不同,因为它涉及主动\emph{删除}长度为一的连接并用更长的连接替换它们。
以这种方式修改的单元被迫在长时间尺度上运作。
而通过时间\gls{skip_connection}是\emph{添加}边。
收到这种新连接的单元,可能会学习在长时间尺度上运作,但也可能选择专注于自己其他的短期连接。

% -- 396 --

强制一组循环单元在不同时间尺度上运作有不同的方式。
一种选择是使循环单元变成\gls{leaky_unit},但不同的单元组关联不同的固定时间尺度。
这由~\cite{Mozer-nips92}提出,并被成功应用于~\cite{Pascanu+al-ICML2013-small}。
另一种选择是使显式且离散的更新发生在不同的时间,不同的单元组有不同的频率。
这是~\cite{ElHihi+Bengio-nips8}和~\cite{Koutnik-et-al-ICML2014}的方法。
它在一些基准数据集上表现不错。

\section{\glsentrytext{LSTM}和其他\glsentrytext{gated_rnn}}
\label{sec:the_long_short_term_memory_and_other_gated_rnns}
本文撰写之时,实际应用中最有效的序列模型称为\firstgls{gated_rnn}。
包括基于\firstgls{LSTM}和基于\firstgls{gated_recurrent_unit}的网络。

像\gls{leaky_unit}一样,\gls{gated_rnn}~想法也是基于生成通过时间的路径,其中导数既不消失也不发生爆炸。
\gls{leaky_unit}通过手动选择常量的连接权重或参数化的连接权重来达到这一目的。
\gls{gated_rnn}~将其推广为在每个\gls{time_step}都可能改变的连接权重。

\gls{leaky_unit}允许网络在较长持续时间内\emph{积累}信息(诸如用于特定特征或类的线索)。
然而,一旦该信息被使用,让\gls{NN}\emph{遗忘}旧的状态可能是有用的。
例如,如果一个序列是由子序列组成,我们希望\gls{leaky_unit}能在各子序列内积累线索,我们需要将状态设置为0以忘记旧状态的机制。
我们希望\gls{NN}学会决定何时清除状态,而不是手动决定。
这就是\gls{gated_rnn}~要做的事。

\subsection{\glssymbol{LSTM}}
\label{sec:lstm}
引入自循环的巧妙构思,以产生\gls{gradient}长时间持续流动的路径是初始\firstall{LSTM}模型的核心贡献\citep{Hochreiter+Schmidhuber-1997}。
其中一个关键扩展是使自循环的权重视上下文而定,而不是固定的\citep{Gers-et-al-2000}。
\gls{gated}此自循环(由另一个\gls{hidden_unit}控制)的权重,累积的时间尺度可以动态地改变。
在这种情况下,即使是具有固定参数的~\glssymbol{LSTM},累积的时间尺度也可以因输入序列而改变,因为时间常数是模型本身的输出。
\glssymbol{LSTM}~已经在许多应用中取得重大成功,如无约束手写识别\citep{Graves-et-al-2009}、语音识别\citep{Graves-et-al-ICASSP2013,Graves+Jaitly-ICML2014}、手写生成\citep{Graves-arxiv2013}、机器翻译\citep{Sutskever-et-al-NIPS2014}、为图像生成标题\citep{Kiros-et-al-arxiv2014,Vinyals-et-al-arxiv2014,Xu-et-al-ICML2015}和解析\citep{Vinyals2014}。

% -- 397 --

\begin{figure}[!htb]
\ifOpenSource
\centerline{\includegraphics{figure.pdf}}
\else
\centerline{\includegraphics{Chapter10/figures/lstm}}
\fi
\caption{\glssymbol{LSTM}~\gls{recurrent_network}``细胞''的框图。
细胞彼此循环连接,代替一般\gls{recurrent_network}中普通的\gls{hidden_unit}。
这里使用常规的人工神经元计算输入特征。
如果sigmoid输入门允许,它的值可以累加到状态。
状态单元具有线性自循环,其权重由\gls{forget_gate}控制。
细胞的输出可以被输出门关闭。
所有\gls{gated}单元都具有sigmoid非线性,而输入单元可具有任意的压缩非线性。
状态单元也可以用作\gls{gated}单元的额外输入。
黑色方块表示单个\gls{time_step}的延迟。
}
\label{fig:chap10_lstm}
\end{figure}

\glssymbol{LSTM}~块如\figref{fig:chap10_lstm}所示。
在浅\gls{recurrent_network}的架构下,相应的\gls{forward_propagation}公式如下。
更深的架构也被成功应用\citep{Graves-et-al-ICASSP2013,Pascanu-et-al-ICLR2014}。
\glssymbol{LSTM}~\gls{recurrent_network}除了外部的~\glssymbol{RNN}~循环外,还具有内部的``\glssymbol{LSTM}~细胞''循环(自环),因此~\glssymbol{LSTM}~不是简单地向输入和循环单元的仿射变换之后施加一个逐元素的非线性。
与普通的\gls{recurrent_network}类似,每个单元有相同的输入和输出,但也有更多的参数和控制信息流动的\gls{gated}单元系统。
最重要的组成部分是状态单元$s_i^{(t)}$,与前一节讨论的\gls{leaky_unit}有类似的线性自环。
然而,此处自环的权重(或相关联的时间常数)由\firstgls{forget_gate}~$f_i^{(t)}$控制(时刻$t$和细胞$i$),由~\ENNAME{sigmoid}~单元将权重设置为0和1之间的值:
\begin{align}
 f_i^{(t)} = \sigma \Big( b_i^f + \sum_j U_{i,j}^f x_j^{(t)} + \sum_j W_{i,j}^f h_j^{(t-1)} \Big),
\end{align}
其中$\Vx^{(t)}$是当前输入向量,$\Vh^{t}$是当前隐藏层向量,$\Vh^{t}$包含所有~\glssymbol{LSTM}~细胞的输出。 
$\Vb^f, \MU^f, \MW^f$分别是\gls{forget_gate}的\gls{bias_aff}、输入权重和循环权重。
因此~\glssymbol{LSTM}~细胞内部状态以如下方式更新,其中有一个条件的自环权重$f_i^{(t)}$:
\begin{align}
 s_i^{(t)} = f_i^{(t)}  s_i^{(t-1)} +  g_i^{(t)}
 \sigma \Big( b_i + \sum_j U_{i,j} x_j^{(t)} + \sum_j W_{i,j} h_j^{(t-1)} \Big),
\end{align}
其中$\Vb, \MU, \MW$分别是~\glssymbol{LSTM}~细胞中的\gls{bias_aff}、输入权重和循环权重。
\textbf{外部输入门}(external input gate)单元$g_i^{(t)}$以类似\gls{forget_gate}(使用\ENNAME{sigmoid}获得一个0和1之间的值)的方式更新,但有自身的参数:
\begin{align}
 g_i^{(t)} = \sigma \Big( b_i^g + \sum_j U_{i,j}^g x_j^{(t)} + \sum_j W_{i,j}^g h_j^{(t-1)} \Big).
\end{align}
\glssymbol{LSTM}~细胞的输出$h_i^{(t)}$也可以由\textbf{输出门}(output gate)~$q_i^{(t)}$关闭(使用\ENNAME{sigmoid}单元作为\gls{gated}):
\begin{align}
 h_i^{(t)} &= \text{tanh}\big( s_i^{(t)} \big) q_i^{(t)}, \\
 q_i^{(t)} &= \sigma \Big( b_i^o + \sum_j U_{i,j}^o x_j^{(t)} + \sum_j W_{i,j}^o h_j^{(t-1)} \Big),
\end{align}
其中$\Vb^o, \MU^o, \MW^o$分别是输出门的\gls{bias_aff}、输入权重和循环权重。
在某些变体中,可以选择使用细胞状态$s_i^{(t)}$作为额外的输入(及其权重),输入到第$i$个单元的三个门,如\figref{fig:chap10_lstm}所示。
这将需要三个额外的参数。

% -- 399 --

\glssymbol{LSTM}~网络比简单的循环架构更易于学习\gls{long_term_dependency},先是用于测试\gls{long_term_dependency}学习能力的人工数据集\citep{Bengio-trnn94,Hochreiter+Schmidhuber-1997,chapter-gradient-flow-2001},然后是在具有挑战性的序列处理任务上获得最先进的表现\citep{Graves-book2012,Graves-arxiv2013,Sutskever-et-al-NIPS2014}。
\glssymbol{LSTM}~的变体和替代也已经被研究和使用,这将在下文进行讨论。

\subsection{其他\glsentrytext{gated_rnn}}
\label{sec:other_gated_rnns}
\glssymbol{LSTM}~架构中哪些部分是真正必须的?
还可以设计哪些其他成功架构允许网络动态地控制不同单元的时间尺度和遗忘行为?

最近关于\gls{gated_rnn}~的工作给出了这些问题的某些答案,其单元也被称为\gls{gated_recurrent_unit}或~\glssymbol{gated_recurrent_unit}~\citep{cho-al-arxiv14-ssst8,Chung-et-al-NIPSDL2014-small,chung-icml15-gated,Jozefowicz-et-al-ICML2015,Chrupala-et-al-arxiv2015}。
与~\glssymbol{LSTM}~的主要区别是,单个\gls{gated}单元同时控制遗忘因子和更新状态单元的决定。
更新公式如下:
\begin{align}
 h_i^{(t)} = u_i^{(t-1)} h_i^{(t-1)} + (1 - u_i^{(t-1)}) \sigma 
 \Big( b_i + \sum_j U_{i,j} x_j^{(t)} + \sum_j W_{i,j} r_j^{(t-1)} h_j^{(t-1)} \Big),
\end{align}
其中$\Vu$代表``更新''门,$\Vr$表示``复位''门。
它们的值就如通常所定义的:
\begin{align}
 u_i^{(t)} = \sigma \Big( b_i^u + \sum_j U_{i,j}^u x_j^{(t)} + \sum_j W_{i,j}^u h_j^{(t)} \Big),
\end{align}
和
\begin{align}
 r_i^{(t)} = \sigma \Big( b_i^r + \sum_j U_{i,j}^r x_j^{(t)} + \sum_j W_{i,j}^r h_j^{(t)} \Big).
\end{align}
复位和更新门能独立地``忽略''状态向量的一部分。
更新门像条件渗漏累积器一样可以线性\gls{gated}任意维度,从而选择将它复制(在~\ENNAME{sigmoid}~的一个极端)或完全由新的``目标状态''值(朝向渗漏累积器的收敛方向)替换并完全忽略它(在另一个极端)。
复位门控制当前状态中哪些部分用于计算下一个目标状态,在过去状态和未来状态之间引入了附加的非线性效应。

% -- 400 --

围绕这一主题可以设计更多的变种。
例如复位门(或\gls{forget_gate})的输出可以在多个\gls{hidden_unit}间共享。
或者,全局门的乘积(覆盖一整组的单元,例如整一层)和一个局部门(每单元)可用于结合全局控制和局部控制。
然而,一些调查发现这些~\glssymbol{LSTM}~和GRU架构的变种,在广泛的任务中难以明显地同时击败这两个原始架构\citep{Greff-et-al-arxiv2015,Jozefowicz-et-al-ICML2015}。
\cite{Greff-et-al-arxiv2015}发现其中的关键因素是\gls{forget_gate},而~\cite{Jozefowicz-et-al-ICML2015}发现向~\glssymbol{LSTM}~\gls{forget_gate}加入1的\gls{bias_aff}(由~\cite{Gers-et-al-2000}提倡)能让~\glssymbol{LSTM}~变得与已探索的最佳变种一样健壮。

\section{优化\glsentrytext{long_term_dependency}}
\label{sec:optimization_for_long_term_dependencies}
我们已经在\secref{sec:long_term_dependencies}和\secref{sec:the_challenge_of_long_term_dependencies}中描述过在许多\gls{time_step}上优化~\glssymbol{RNN}~时发生的\gls{gradient}消失和爆炸的问题。

由~\cite{Martens+Sutskever-ICML2011}提出了一个有趣的想法是,二阶导数可能在一阶导数消失的同时消失。
二阶优化算法可以大致被理解为将一阶导数除以二阶导数(在更高维数,由\gls{gradient}乘以~\gls{hessian}~的逆)。
如果二阶导数与一阶导数以类似的速率收缩,那么一阶和二阶导数的比率可保持相对恒定。
不幸的是,二阶方法有许多缺点,包括高的计算成本、需要一个大的\gls{minibatch}、并且倾向于被吸引到鞍点。
\cite{Martens+Sutskever-ICML2011}发现采用二阶方法的不错结果。
之后,\cite{sutskeverimportance}发现使用较简单的方法可以达到类似的结果,例如经过谨慎初始化的~\ENNAME{Nesterov}~动量法。
更详细的内容参考~\cite{Sutskever-thesis2012}。
应用于~\glssymbol{LSTM}~时,这两种方法在很大程度上会被单纯的~\glssymbol{SGD}(甚至没有动量)取代。
这是\gls{ML}中一个延续的主题,设计一个易于优化模型通常比设计出更加强大的优化算法更容易。

% -- 401 --

\subsection{\glsentrytext{clipping_gradient}}
\label{sec:clipping_gradients}
如\secref{sec:cliffs_and_exploding_gradients}讨论,强非线性函数(如由许多\gls{time_step}计算的\gls{recurrent_network})往往倾向于非常大或非常小幅度的\gls{gradient}。
如\figref{fig:chap8_cliff}和\figref{fig:chap10_cliff_clipping_color}所示,我们可以看到,\gls{objective_function}(作为参数的函数)存在一个伴随``悬崖''的``地形'':宽且相当平坦区域被\gls{objective_function}变化快的小区域隔开,形成了一种悬崖。

这导致的困难是,当参数\gls{gradient}非常大时,\gls{gradient}下降的参数更新可以将参数抛出很远,进入\gls{objective_function}较大的区域,到达当前解所作的努力变成了无用功。
\gls{gradient}告诉我们,围绕当前参数的无穷小区域内最速下降的方向。
这个无穷小区域之外,\gls{cost_function}可能开始拐头向上。
更新必须被选择为足够小,以避免沿向上的曲面走太远。
我们通常使用衰减速度足够慢的学习率,使连续的步骤具有大致相同的学习率。
适合于一个相对线性的地形部分的步长经常在下一步进入地形中更加弯曲的部分时变得不适合,会导致上坡运动。

\begin{figure}[!htb]
\ifOpenSource
\centerline{\includegraphics{figure.pdf}}
\else
\centerline{\includegraphics{Chapter10/figures/cliff_clipping_color}}
\fi
\caption{\gls{gradient_clipping}在有两个参数$\Vw$和$\Vb$的\gls{recurrent_network}中的效果示例。
\gls{gradient_clipping}可以使\gls{GD}在极陡峭的悬崖附近更合理地执行。
这些陡峭的悬崖通常发生在\gls{recurrent_network}中,位于\gls{recurrent_network}近似线性的附近。
悬崖在\gls{time_step}的数量上呈指数地陡峭,因为对于每个\gls{time_step},权重矩阵都自乘一次。
\emph{(左)}没有\gls{gradient_clipping}的\gls{GD}越过这个小峡谷的底部,然后从悬崖面接收非常大的梯度。
大梯度灾难性地将参数推到图的轴外。
\emph{(右)}使用\gls{gradient_clipping}的\gls{GD}对悬崖的反应更温和。
当它上升到悬崖面时,步长受到限制,使得它不会被推出靠近解的陡峭区域。
经~\citet{Pascanu+al-ICML2013-small}许可改编此图。
}
\label{fig:chap10_cliff_clipping_color}
\end{figure}

% -- 402 --

一个简单的解决方案已被从业者使用多年:\firstgls{clipping_gradient}。
此想法有不同实例\citep{Mikolov-thesis-2012,Pascanu+al-ICML2013-small}。
一种选择是在参数更新之前,\emph{逐元素}地截断\gls{minibatch}产生的参数\gls{gradient}~\citep{Mikolov-thesis-2012}。
另一种是在参数更新之前\emph{截断\gls{gradient}~$\Vg$的范数$\norm{ \Vg }$}~\citep{Pascanu+al-ICML2013-small}:
\begin{align}
 \text{if}~ \norm{\Vg} &> v \\
 \Vg &\gets \frac{\Vg v}{\norm{g}},
\end{align}
其中$v$是范数上界,$\Vg$用来更新参数。
因为所有参数(包括不同的参数组,如权重和\gls{bias_aff})的\gls{gradient}被单个缩放因子联合重整化,所以后一方法具有的优点是保证了每个步骤仍然是在\gls{gradient}方向上的,但实验表明两种形式类似。
虽然参数更新与真实\gls{gradient}具有相同的方向\gls{gradient},经过\gls{gradient}范数截断,参数更新的向量范数现在变得有界。
这种有界\gls{gradient}能避免执行\gls{gradient}爆炸时的有害一步。
事实上,当\gls{gradient}大小高于阈值时,即使是采取简单的\emph{随机步骤}往往工作得几乎一样好。
如果爆炸非常严重,\gls{gradient}数值上为{\tt Inf}或{\tt Nan}(无穷大或不是一个数字),则可以采取大小为$v$的随机一步,通常会离开数值不稳定的状态。
截断每\gls{minibatch}\gls{gradient}范数不会改变单个\gls{minibatch}的\gls{gradient}方向。
然而,对许多\gls{minibatch}使用范数截断\gls{gradient}后的平均值不等同于截断真实\gls{gradient}(使用所有的实例所形成的\gls{gradient})的范数。
大导数范数的样本,和与之一起出现在同一\gls{minibatch}的样本,其对最终方向的贡献将被截断。
不像传统\gls{minibatch}\gls{gradient}下降,其中真实\gls{gradient}的方向是等于所有\gls{minibatch}\gls{gradient}的平均。
换句话说,传统的\gls{SGD}使用\gls{gradient}的无偏估计,而使用范数截断的\gls{gradient}下降引入了经验上是有用的启发式\gls{bias_aff}。
而对于逐元素截断,更新的方向与真实\gls{gradient}或\gls{minibatch}的\gls{gradient}不再对齐,但是它仍然是一个下降方向。
还有学者提出\citep{Graves-arxiv2013}(相对于隐藏单元)截断\gls{backward_propagation}\gls{gradient},但没有公布与这些变种之间的比较; 我们推测,所有这些方法表现类似。

% -- 403 --

\subsection{引导信息流的正则化}
\label{sec:regularizing_to_encourage_information_flow}
\gls{gradient_clipping}有助于处理爆炸的\gls{gradient},但它无助于消失的\gls{gradient}。
为了解决消失的\gls{gradient}问题并更好地捕获\gls{long_term_dependency},我们讨论了如下想法:在展开循环架构的\gls{computational_graph}中,沿着与弧边相关联的\gls{gradient}乘积接近1的部分创建路径。
在\secref{sec:the_long_short_term_memory_and_other_gated_rnns}中已经讨论过,实现这一点的一种方法是使用~\glssymbol{LSTM}~以及其他自循环和\gls{gated}机制。
另一个想法是\gls{regularization}或约束参数,以引导``信息流''。
特别是即使\gls{loss_function}只对序列尾部的输出作惩罚,我们也希望\gls{gradient}向量$\nabla_{\Vh^{(t)}} L$在\gls{backward_propagation}时能维持其幅度。
形式上,我们要使
\begin{align}
 (\nabla_{\Vh^{(t)}} L) \frac{\partial \Vh^{(t)}}{\partial \Vh^{(t-1)}}
\end{align}
与
\begin{align}
\nabla_{\Vh^{(t)}} L 
\end{align}
一样大。
在这个目标下,\citet{Pascanu+al-ICML2013-small}提出以下正则项:
\begin{align}
 \Omega = \sum_t \Bigg(  \frac{
 \norm{ (\nabla_{\Vh^{(t)}} L) \frac{\partial \Vh^{(t)}}{\partial \Vh^{(t-1)}}}}
 {\norm{\nabla_{\Vh^{(t)}} L}} -1 \Bigg)^2.
\end{align}
计算这一\gls{gradient}的正则项可能会出现困难,但~\cite{Pascanu+al-ICML2013-small}提出可以将后向传播向量$\nabla_{\Vh^{(t)}} L$考虑为恒值作为近似(为了计算正则化的目的,没有必要通过它们向后传播)。
使用该正则项的实验表明,如果与标准的启发式截断(处理梯度爆炸)相结合,该正则项可以显著地增加~\glssymbol{RNN}~可以学习的依赖的跨度。
因为这种方法将~\glssymbol{RNN}~的动态保持在爆炸\gls{gradient}的边缘,\gls{gradient_clipping}特别重要。
如果没有\gls{gradient_clipping},\gls{gradient}爆炸将阻碍学习的成功。

这种方法的一个主要弱点是,在处理数据冗余的任务时如\gls{language_model},它并不像~\glssymbol{LSTM}~一样有效。

% -- 404 --

\section{外显记忆}
\label{sec:explicit_memory}
智能需要知识并且可以通过学习获取知识,这已促使大型深度架构的发展。
然而,知识是不同的并且种类繁多。
有些知识是隐含的、潜意识的并且难以用语言表达——比如怎么行走或狗与猫的样子有什么不同。
其他知识可以是明确的、可陈述的以及可以相对简单地使用词语表达——每天常识性的知识,如``猫是一种动物'',或者为实现自己当前目标所需知道的非常具体的事实,如``与销售团队会议在141室于下午3:00开始''。


\gls{NN}擅长存储隐性知识,但是他们很难记住事实。
被存储在\gls{NN}参数中之前,\gls{SGD}需要多次提供相同的输入,即使如此,该输入也不会被特别精确地存储。
\citet{Graves-et-al-arxiv2014}推测这是因为\gls{NN}缺乏\textbf{工作存储}(working memory)系统,即类似人类为实现一些目标而明确保存和操作相关信息片段的系统。
这种外显记忆组件将使我们的系统不仅能够快速``故意''地存储和检索具体的事实,也能利用他们循序推论。
\gls{NN}处理序列信息的需要,改变了每个步骤向网络注入输入的方式,长期以来推理能力被认为是重要的,而不是对输入做出自动的、直观的反应\citep{hinton1990mapping} 。


为了解决这一难题,\citet{Weston2014}引入了\firstgls{memory_network},其中包括一组可以通过寻址机制来访问的记忆单元。
\gls{memory_network}原本需要监督信号指示他们如何使用自己的记忆单元。
\citet{Graves-et-al-arxiv2014}引入的\firstgls{NTM},不需要明确的监督指示采取哪些行动而能学习从记忆单元读写任意内容,并通过使用基于内容的软注意机制(~见\citet{Bahdanau-et-al-ICLR2015-small}和\secref{sec:using_an_attention_mechanism_and_aligning_pieces_of_data}),允许端到端的训练。
这种软寻址机制已成为其他允许基于\gls{gradient}优化的模拟算法机制的相关架构的标准\citep{Sukhbaatar2015,Joulin+Mikolov-2015,Kumar-et-al-2015,Vinyals2015,Grefenstette-et-al-NIPS2015}。

每个记忆单元可以被认为是~\glssymbol{LSTM}~和GRU中记忆单元的扩展。
不同的是,网络输出一个内部状态来选择从哪个单元读取或写入,正如数字计算机读取或写入到特定地址的内存访问。

% -- 405 --

产生确切整数地址的函数很难优化。
为了缓解这一问题,\glssymbol{NTM}~实际同时从多个记忆单元写入或读取。
读取时,它们采取许多单元的加权平均值。
写入时,他们对多个单元修改不同的数值。
用于这些操作的系数被选择为集中在少数单元,如通过~\ENNAME{softmax}~函数产生它们。
使用这些具有非零导数的权重允许控制访问存储器的函数能使用梯度下降法优化。
关于这些系数的\gls{gradient}指示着其中每个参数是应该增加还是减少,但\gls{gradient}通常只在接收大系数的存储器地址上具有大的值。

这些记忆单元通常扩充为包含向量,而不是由~\glssymbol{LSTM}~或GRU存储单元所存储的单个标量。
增加记忆单元大小的原因有两个。
原因之一是,我们已经增加了访问记忆单元的成本。
我们为产生用于许多单元的系数付出计算成本,但我们预期这些系数聚集在周围小数目的单元。
通过读取向量值,而不是一个标量,我们可以抵消部分成本。 
使用向量值的记忆单元的另一个原因是,它们允许\textbf{基于内容的寻址}(content-based addressing),其中从一个单元读或写的权重是该单元的函数。
如果我们能够产生一个模式来匹配某些但并非所有元素,向量值单元允许我们检索一个完整向量值的记忆。
这类似于人们能够通过几个歌词回忆起一首歌曲的方式。
我们可以认为基于内容的读取指令是说,``检索一首副歌歌词中带有'我们都住在黄色潜水艇'的歌''。
当我们要检索的对象很大时,基于内容的寻址更为有用——如果歌曲的每一个字母被存储在单独的记忆单元中,我们将无法通过这种方式找到他们。
通过比较,\textbf{基于位置的寻址}(location-based addressing)不允许引用存储器的内容。
我们可以认为基于位置的读取指令是说``检索347档的歌的歌词''。
当存储单元哪怕很小时,基于位置的寻址常常也可以是完全合理的机制。

如果一个存储单元的内容在大多数\gls{time_step}上会被复制(不被忘记),则它包含的信息可以在时间上向前传播,\gls{gradient}随时间向后传播,而不会消失或爆炸。

\begin{figure}[!htb]
\ifOpenSource
\centerline{\includegraphics{figure.pdf}}
\else
\centerline{\includegraphics{Chapter10/figures/memory_network}}
\fi
\caption{具有外显记忆网络的示意图,具备\gls{NTM}的一些关键设计元素。
在此图中,我们将模型的``\gls{representation}''部分(``任务网络'',这里是底部的循环网络)与存储事实的模型(记忆单元的集合)的``存储器''部分区分开。
任务网络学习``控制''存储器,决定从哪读取以及在哪写入(通过读取和写入机制,由指向读取和写入地址的粗箭头指示)。
}
\label{fig:chap10_memory_network}
\end{figure}

外显记忆的方法在\figref{fig:chap10_memory_network}说明,其中我们可以看到一个``任务神经网络''搭配了一个存储器。
虽然这一任务神经网络可以是前馈或循环的,但整个系统是一个\gls{recurrent_network}。
任务网络可以选择读取或写入特定的存储器地址。
外显记忆似乎允许模型学习普通~\glssymbol{RNN}~或~\glssymbol{LSTM} \glssymbol{RNN}~不能学习的任务。
这种优点的一个原因可能是因为信息和\gls{gradient}可以(分别在时间上向前和向后)在非常长的持续时间内传播。

% -- 406 --

作为存储器单元的加权平均值\gls{backward_propagation}的替代,我们可以将存储器寻址系数解释为概率,并依据此概率随机从一个单元读取\citep{Zaremba+Sutskever-arxiv2015}。
优化离散决策的模型需要专门的优化算法,这将在\secref{sec:back_propagating_through_discrete_stochastic_operations}中描述。
目前为止,训练这些做离散决策的随机架构,仍比训练进行软判决的确定性算法更难。

无论是软(允许\gls{backward_propagation})或随机硬性的,用于选择一个地址的机制与先前在机器翻译的背景下引入的\gls{attention_mechanism}形式相同\citep{Bahdanau-et-al-ICLR2015-small},这在\secref{sec:using_an_attention_mechanism_and_aligning_pieces_of_data}中也有讨论。
甚至更早之前,\gls{attention_mechanism}的想法就被引入了\gls{NN},在手写生成的情况下\citep{Graves-arxiv2013},有一个被约束为通过序列只向前移动的\gls{attention_mechanism}。
在机器翻译和\gls{memory_network}的情况下,每个步骤中关注的焦点可以移动到一个完全不同的地方(相比之前的步骤)。

\gls{RNN}提供了将\gls{DL}扩展到序列数据的一种方法。
它们是我们的\gls{DL}工具箱中最后一个主要的工具。
现在我们的讨论将转移到如何选择和使用这些工具,以及如何在真实世界的任务中应用这些工具。

% !Mode:: "TeX:UTF-8"
% Translator: Yujun Li 
\chapter{实践方法论}
\label{chap:practical_methodology}
要成功地使用深度学习技术,仅仅知道存在哪些算法和解释他们为何有效的原理是不够的。
一个优秀的机器学习实践者还需要知道如何针对具体应用挑选一个合适的算法以及如何监控,并根据实验反馈改进机器学习系统。
在\gls{ML}系统的日常开发中,实践者需要决定是否收集更多的数据、增加或减少模型\gls{capacity}、添加或删除\gls{regularization}项、改进模型的优化、改进模型的近似推断或调试模型的软件实现。
尝试这些操作都需要大量时间,因此确定正确做法,而不盲目猜测尤为重要的。


本书的大部分内容都是关于不同的\gls{ML}模型、训练算法和\gls{objective_function}。
这可能给人一种印象——成为\gls{ML}专家的最重要因素是了解各种各样的\gls{ML}技术,并熟悉各种不同的数学。
在实践中,正确使用一个普通算法通常比草率地使用一个不清楚的算法效果更好。
正确应用一个算法需要掌握一些相当简单的方法论。
本章的许多建议都来自~\cite{ng-lecture-advice}。


我们建议参考以下几个实践设计流程:
\begin{itemize}
\item 确定目标——使用什么样的\gls{error_metric},并为此\gls{error_metric}指定目标值。
这些目标和\gls{error_metric}取决于该应用旨在解决的问题。
% -- 409 end


\item 尽快建立一个\gls{end_to_end}工作流程,包括估计合适的\gls{performance_metrics}。
% 410 head

\item 搭建系统,并确定性能瓶颈。
检查哪个部分的性能差于预期,以及是否是因为\gls{overfitting}、\gls{underfitting},或者数据或软件缺陷造成的。

\item 根据具体观察反复地进行增量式的改动,如收集新数据、调整\gls{hyperparameter}或改进算法。
\end{itemize}


我们将使用街景地址号码\gls{transcription_system}~\citep{Goodfellow+et+al-ICLR2014a}作为一个运行示例。
该应用的目标是将建筑物添加到谷歌地图。
街景车拍摄建筑物,并记录与每张建筑照片相关的GPS坐标。
\gls{convolutional_network}识别每张照片上的地址号码,由谷歌地图数据库在正确的位置添加该地址。
这个商业应用是一个很好的示例,它的开发流程遵循我们倡导的设计方法。

% 410 mid
我们现在描述这个过程中的每一个步骤。



\section{\glsentrytext{performance_metrics}}
\label{sec:performance_metrics}
确定目标,即使用什么\gls{error_metric},是必要的第一步,因为\gls{error_metric}将指导接下来的所有工作。
同时我们也应该了解大概能得到什么级别的目标性能。
% 410 mid


值得注意的是对于大多数应用而言,不可能实现绝对零误差。
即使你有无限的训练数据,并且恢复了真正的\gls{PD},\gls{bayes_error}仍定义了能达到的最小\gls{error_rate}。
这是因为输入特征可能无法包含输出变量的完整信息,或是因为系统可能本质上是随机的。
当然我们还会受限于有限的训练数据。

% 410 end
训练数据的数量会因为各种原因受到限制。
当目标是打造现实世界中最好的产品或服务时,我们通常需要收集更多的数据,但必须确定进一步减少误差的价值,并与收集更多数据的成本做权衡。
数据收集会耗费时间、金钱,或带来人体痛苦(例如,收集人体医疗测试数据)。
科研中,目标通常是在某个确定\gls{benchmarks}下探讨哪个算法更好,一般会固定\gls{training_set},不允许收集更多的数据。
% 411 head


如何确定合理的性能期望?
在学术界,通常我们可以根据先前公布的\gls{benchmarks}结果来估计预期\gls{error_rate}。
在现实世界中,一个应用的\gls{error_rate}有必要是安全的、具有成本效益的或吸引消费者的。
一旦你确定了想要达到的\gls{error_rate},那么你的设计将由如何达到这个\gls{error_rate}来指导。
% 411 mid


除了需要考虑\gls{performance_metrics}之外,另一个需要考虑的是度量的选择。
我们有几种不同的\gls{performance_metrics},可以用来度量一个含有\gls{ML}组件的完整应用的有效性。
这些\gls{performance_metrics}通常不同于训练模型的\gls{cost_function}。 
如\secref{sec:the_performance_measure_p}所述,我们通常会度量一个系统的\gls{accuracy},或等价地,\gls{error_rate}。


然而,许多应用需要更高级的度量。
% 411 mid

有时,一种错误可能会比另一种错误更严重。
例如,垃圾邮件检测系统会有两种错误:将正常邮件错误地归为垃圾邮件,将垃圾邮件错误地归为正常邮件。
阻止正常消息比允许可疑消息通过糟糕得多。
我们希望度量某种形式的总\gls{cost},其中拦截正常邮件比允许垃圾邮件通过的\gls{cost}更高,而不是度量垃圾邮件分类的\gls{error_rate}。
% 411 mid


有时,我们需要训练检测某些罕见事件的二元分类器。
例如,我们可能会为一种罕见疾病设计医疗测试。
假设每一百万人中只有一人患病。
我们只需要让分类器一直报告没有患者,就能轻易地在检测任务上实现$99.9999\%$的正确率。
显然,正确率很难描述这种系统的性能。
解决这个问题的方法是度量\firstgls{precision}和\firstgls{recall}。
\gls{precision}是模型报告的检测是正确的比率,而\gls{recall}则是真实事件被检测到的比率。
检测器永远报告没有患者,会得到一个完美的\gls{precision},但\gls{recall}为零。
而报告每个人都是患者的检测器会得到一个完美的\gls{recall},但是\gls{precision}会等于人群中患有该病的比例(在我们的例子是$0.0001\%$,每一百万人只有一人患病)。
当使用\gls{precision}和\gls{recall}时,我们通常会画\firstgls{prcurve},$y$轴表示\gls{precision},$x$轴表示\gls{recall}。
如果检测到的事件发生了,那么分类器会返回一个较高的得分。
例如,我们将\gls{feedforward_network}设计为检测一种疾病,估计一个医疗结果由特征$\Vx$表示的人患病的概率为$\hat{y} = P(y=1\mid\Vx)$。
每当这个得分超过某个阈值时,我们报告检测结果。
通过调整阈值,我们能权衡\gls{precision}和\gls{recall}。
在很多情况下,我们希望用一个数而不是曲线来概括分类器的性能。
要做到这一点,我们可以将\gls{precision} $p$和\gls{recall} $r$转换为\firstgls{fscore}
\begin{equation}
	F = \frac{2pr}{p+r}.
\end{equation}
另一种方法是报告\,\gls{prcurve}下方的总面积。
% -- 412 head


在一些应用中,\gls{ML}系统可能会拒绝做出判断。
如果\gls{ML}算法能够估计所作判断的置信度,这将会非常有用,特别是在错误判断会导致严重危害,而人工操作员能够偶尔接管的情况下。
街景\gls{transcription_system}可以作为这种情况的一个示例。
这个任务是识别照片上的地址号码,将照片拍摄地点对应到地图上的地址。%??  翻译的很有问题
如果地图是不精确的,那么地图的价值会严重下降。
因此只在转录正确的情况下添加地址十分重要。
如果\gls{ML}系统认为它不太能像人一样正确地转录,那么最好办法当然是让人来转录照片。
当然,只有当\gls{ML}系统能够大量降低需要人工操作处理的图片时,它才是有用的。
在这种情况下,一种自然的\gls{performance_measures}是\firstgls{coverage}。
\gls{coverage}是\gls{ML}系统能够产生响应的样本所占的比率。
我们权衡\gls{coverage}和\gls{precision}。
一个系统可以通过拒绝处理任意样本的方式来达到$100\%$的\gls{precision},但是\gls{coverage}降到了$0\%$。
对于街景任务,该项目的目标是达到人类级别的转录\gls{precision},同时保持$95\%$的\gls{coverage}。
在这项任务中,人类级别的性能是$98\%$的\gls{precision}。
% 412 end

还有许多其他的\gls{performance_measures}。
例如,我们可以度量点击率、收集用户满意度调查等等。
许多专业的应用领域也有特定的标准。
% 412 end

最重要的是首先要确定改进哪个\gls{performance_measures},然后专心提高\gls{performance_measures}。
如果没有明确的目标,那么我们很难判断\gls{ML}系统上的改动是否有所改进。

% -- 412 end

\section{默认的\glsentrytext{baseline}模型}
\label{sec:default_baseline_models}
确定\gls{performance_metrics}和目标后,任何实际应用的下一步是尽快建立一个合理的\gls{end_to_end}系统。
本节给出了一些关于在不同情况下使用哪种算法作为第一个\gls{baseline}方法推荐。
在本节中,我们提供了关于不同情况下使用哪种算法作为第一\gls{baseline}方法的推荐。
值得注意的是,\gls{DL}研究进展迅速,所以本书出版后很快可能会有更好的默认算法。
% 413 head 

根据问题的复杂性,项目开始时可能无需使用\gls{DL}。
如果只需正确地选择几个线性权重就可能解决问题,那么项目可以开始于一个简单的统计模型,如\gls{logistic_regression}。


如果问题属于``\glssymbol{AI}-完全''类的,如\gls{object_recognition}、\gls{SR}、\gls{machine_translation}等等,那么项目开始于一个合适的\gls{DL}模型,效果会比较好。
% 413 mid 


首先,根据数据的结构选择一类合适的模型。
如果项目是以固定大小的向量作为输入的\gls{supervised_learning},那么可以使用全连接的\gls{feedforward_network}。
如果输入有已知的拓扑结构(例如,输入是图像),那么可以使用\gls{convolutional_network}。
在这些情况下,刚开始可以使用某些\gls{piecewise}线性单元(\glssymbol{ReLU}\,或者其扩展,如\,\glssymbol{leaky_ReLU}、\glssymbol{PReLU}\,和\,\glssymbol{maxout})。
如果输入或输出是一个序列,可以使用\gls{gated_recurrent_net}(\glssymbol{LSTM}\,或\,\glssymbol{gated_recurrent_unit})。
% 413 mid 

具有衰减\gls{learning_rate}以及\gls{momentum}的\,\glssymbol{SGD}\,是优化算法一个合理的选择
(流行的衰减方法有,衰减到固定最低\gls{learning_rate}的线性衰减、指数衰减,或每次发生验证错误停滞时将\gls{learning_rate}降低$2-10$倍,这些衰减方法在不同问题上好坏不一)。
另一个非常合理的选择是Adam算法。
\gls{batch_normalization}对优化性能有着显著的影响,特别是对\gls{convolutional_network}和具有~\gls{sigmoid}~非线性函数的网络而言。
虽然在最初的\gls{baseline}中忽略\gls{batch_normalization}是合理的,然而当优化似乎出现问题时,应该立刻使用\gls{batch_normalization}。
% 413 mid 


除非\gls{training_set}包含数千万以及更多的样本,否则项目应该在一开始就包含一些温和的\gls{regularization}。 
\gls{early_stopping}也被普遍采用。
\gls{dropout}~也是一个很容易实现,且兼容很多模型和训练算法的出色\gls{regularizer}。
\gls{batch_normalization}有时也能降低\gls{generalization_error},此时可以省略~\gls{dropout}~步骤,因为用于标准化变量的统计量估计本身就存在\gls{noise}。 %?? 还是有问题
% -- 413 --  end


如果我们的任务和另一个被广泛研究的任务相似,那么通过复制先前研究中已知性能良好的模型和算法,可能会得到很好的效果。
甚至可以从该任务中复制一个训练好的模型。
例如,通常会使用在ImageNet上训练好的\gls{convolutional_network}的特征来解决其他\gls{CV}任务\citep{girshickregion}。
% 414 head


一个常见问题是项目开始时是否使用\gls{unsupervised_learning},我们将在第三部分进一步探讨这个问题。
 这个问题和特定领域有关。
在某些领域,比如\gls{NLP},能够大大受益于\gls{unsupervised_learning}技术,如学习\gls{unsupervised}\gls{word_embeddings}。
在其他领域,如\gls{CV},除非是在\gls{semi_supervised}的设定下(\gls{labeled}样本数量很少)\citep{Kingma-et-al-NIPS2014,Rasmus-et-al-arxiv2015},目前\gls{unsupervised_learning}并没有带来益处。
如果应用所在环境中,\gls{unsupervised_learning}被认为是很重要的,那么将其包含在第一个\gls{end_to_end}\gls{baseline}中。
否则,只有在解决\gls{unsupervised}问题时,才会第一次尝试时使用\gls{unsupervised_learning}。
在发现初始\gls{baseline}\gls{overfitting}的时候,我们可以尝试加入\gls{unsupervised_learning}。
% 414 mid


\section{决定是否收集更多数据}
\label{sec:determining_whether_to_gather_more_data}

在建立第一个\gls{end_to_end}系统后,就可以度量算法性能并决定如何改进算法。
许多\gls{ML}新手都忍不住尝试很多不同的算法来进行改进。
然而,收集更多的数据往往比改进学习算法要有用得多。
% 414 mid


怎样判断是否要收集更多的数据?
首先,确定\gls{training_set}上的性能是否可接受。
如果模型在\gls{training_set}上的性能就很差,学习算法都不能在\gls{training_set}上学习出良好的模型,那么就没必要收集更多的数据。
反之,可以尝试增加更多的网络层或每层增加更多的\gls{hidden_unit},以增加模型的规模。
此外,也可以尝试调整\gls{learning_rate}等\gls{hyperparameter}的措施来改进学习算法。
如果更大的模型和仔细调试的优化算法效果不佳,那么问题可能源自训练数据的\emph{质量}。
数据可能含太多\gls{noise},或是可能不包含预测输出所需的正确输入。
这意味着我们需要重新开始,收集更干净的数据或是收集特征更丰富的数据集。
% -- 414 end


如果\gls{training_set}上的性能是可接受的,那么我们开始度量\gls{test_set}上的性能。
如果\gls{test_set}上的性能也是可以接受的,那么就顺利完成了。
如果\gls{test_set}上的性能比\gls{training_set}的要差得多,那么收集更多的数据是最有效的解决方案之一。
这时主要的考虑是收集更多数据的代价和可行性,其他方法降低\gls{test_error}的代价和可行性,和增加数据数量能否显著提升\gls{test_set}性能。
在拥有百万甚至上亿用户的大型网络公司,收集大型数据集是可行的,并且这样做的成本可能比其他方法要少很多,所以答案几乎总是收集更多的训练数据。
例如,收集大型\gls{labeled}数据集是解决\gls{object_recognition}问题的主要因素之一。
在其他情况下,如医疗应用,收集更多的数据可能代价很高或者不可行。
一个可以替代的简单方法是降低模型大小或是改进\gls{regularization}(调整\gls{hyperparameter},如\gls{weight_decay}系数,或是加入\gls{regularization}策略,如\,\gls{dropout})。
如果调整\gls{regularization}\gls{hyperparameter}后,\gls{training_set}性能和\gls{test_set}性能之间的差距还是不可接受,那么收集更多的数据是可取的。
% 415 mid


在决定是否收集更多的数据时,也需要确定收集多少数据。
如\figref{fig:chap5_training_size_grows}所示,绘制曲线显示\gls{training_set}规模和\gls{generalization_error}之间的关系是很有帮助的。
根据走势延伸曲线,可以预测还需要多少训练数据来达到一定的性能。
通常,加入总数目一小部分的样本不会对\gls{generalization_error}产生显著的影响。
因此,建议在对数尺度上考虑\gls{training_set}的大小,例如在后续的实验中倍增样本数目。
% 415 mid


如果收集更多的数据是不可行的,那么改进\gls{generalization_error}的唯一方法是改进学习算法本身。
这属于研究领域,并非对应用实践者的建议。

\section{选择\glsentrytext{hyperparameter}}
\label{sec:selecting_hyperparameters}
大部分\gls{DL}算法都有许多\gls{hyperparameter}来控制不同方面的算法表现。
有些\gls{hyperparameter}会影响算法运行的时间和存储成本。
有些\gls{hyperparameter}会影响学习到的模型质量,以及在新输入上推断正确结果的能力。
% -- 415 -end


有两种选择\gls{hyperparameter}的基本方法:手动选择和自动选择。
手动选择\gls{hyperparameter}需要了解\gls{hyperparameter}做了些什么,以及\gls{ML}模型如何才能取得良好的\gls{generalization}。
自动选择\gls{hyperparameter}算法大大减少了解这些想法的需要,但它们往往需要更高的计算成本。
% 416 head


\subsection{手动调整\glsentrytext{hyperparameter}}
\label{sec:manual_hyperparameter_tuning}
手动设置\gls{hyperparameter},我们必须了解\gls{hyperparameter}、\gls{training_error}、\gls{generalization_error}和计算资源(内存和运行时间)之间的关系。
这需要切实了解一个学习算法有效\gls{capacity}的基础概念,如\chapref{chap:machine_learning_basics}所描述的。
% 416 mid


手动搜索\gls{hyperparameter}的目标通常是最小化受限于运行时间和内存预算的\gls{generalization_error}。
我们不去探讨如何确定各种\gls{hyperparameter}对运行时间和内存的影响,因为这高度依赖于平台。


手动搜索\gls{hyperparameter}的主要目标是调整模型的有效\gls{capacity}以匹配任务的复杂性。
有效\gls{capacity}受限于三个因素:模型的表示\gls{capacity}、学习算法成功最小化训练模型\gls{cost_function}的能力以及\gls{cost_function}和训练过程\gls{regularize}模型的程度。
具有更多网络层,每层有更多\gls{hidden_unit}的模型具有较高的表示能力——能够表示更复杂的函数。
然而,如果训练算法不能找到某个合适的函数来最小化训练\gls{cost},或是\gls{regularization}项(如\gls{weight_decay})排除了这些合适的函数,那么即使模型的表达能力较高,也不能学习出合适的函数。%??  还是太乱

% 416 mid


当\gls{generalization_error}以某个\gls{hyperparameter}为变量,作为函数绘制出来时,通常会表现为U形曲线,如\figref{fig:chap5_generalization_vs_capacity}所示。
在某个极端情况下,\gls{hyperparameter}对应着低\gls{capacity},并且\gls{generalization_error}由于\gls{training_error}较大而很高。
这便是\gls{underfitting}的情况。
另一种极端情况,\gls{hyperparameter}对应着高\gls{capacity},并且\gls{generalization_error}由于\gls{training_error}和\gls{test_error}之间的差距较大而很高。
最优的模型\gls{capacity}位于曲线中间的某个位置,能够达到最低可能的\gls{generalization_error},由某个中等的泛化差距(generalization gap)和某个中等的\gls{training_error}相加构成。
% -- 416 end



对于某些\gls{hyperparameter},当\gls{hyperparameter}数值太大时,会发生\gls{overfitting}。
例如中间层\gls{hidden_unit}的数量,增加数量能提高模型的\gls{capacity},容易发生\gls{overfitting}。%??  原文不符 
对于某些\gls{hyperparameter},当\gls{hyperparameter}数值太小时,也会发生\gls{overfitting}。
例如,最小的\gls{weight_decay}系数允许为零,此时学习算法具有最大的有效\gls{capacity},反而容易\gls{overfitting}。%??  原文不符   这一段好像偏差较大 
% 417 head


并非每个\gls{hyperparameter}都能对应着完整的U形曲线。
很多\gls{hyperparameter}是离散的,如中间层单元数目或是~\gls{maxout_unit}中线性元件的数目,这种情况只能沿曲线探索一些点。
有些\gls{hyperparameter}是二值的。
通常这些\gls{hyperparameter}用来指定是否使用学习算法中的一些可选部分,如预处理步骤减去均值并除以标准差来标准化输入特征。
这些\gls{hyperparameter}只能探索曲线上的两点。
其他一些\gls{hyperparameter}可能会有最小值或最大值,限制其探索曲线的某些部分。%??    
例如,\gls{weight_decay}系数最小是零。
这意味着,如果\gls{weight_decay}系数为零时模型\gls{underfitting},那么我们将无法通过修改\gls{weight_decay}系数探索\gls{overfitting}区域。
换言之,有些\gls{hyperparameter}只能减少模型\gls{capacity}。
% 417 mid


\gls{learning_rate}可能是最重要的\gls{hyperparameter}。
如果你只有时间调整一个\gls{hyperparameter},那就调整\gls{learning_rate}。
相比其他\gls{hyperparameter},它以一种更复杂的方式控制模型的有效\gls{capacity}——当\gls{learning_rate}适合优化问题时,模型的有效\gls{capacity}最高,此时\gls{learning_rate}是\emph{正确}的,既不是特别大也不是特别小。
\gls{learning_rate}关于\emph{\gls{training_error}}具有U形曲线,如\figref{fig:chap11_lr}所示。
当\gls{learning_rate}过大时,\gls{GD}可能会不经意地增加而非减少\gls{training_error}。
在理想化的二次情况下,如果\gls{learning_rate}是最佳值的两倍大时,会发生这种情况\citep{LeCun+98backprop-small}。
当\gls{learning_rate}太小,训练不仅慢,还有可能永久停留在一个很高的\gls{training_error}。
关于这种效应,我们知之甚少(不会发生于一个凸\gls{loss_function}中)。
% 417 end

\begin{figure}[!htb]
\ifOpenSource
\centerline{\includegraphics{figure.pdf}}
\else
\centerline{\includegraphics{Chapter11/figures/lr_color}}
\fi
\caption{\gls{training_error}和\gls{learning_rate}之间的典型关系。
注意当\gls{learning_rate}大于最优值时误差会有显著的提升。
此图针对固定的训练时间,越小的\gls{learning_rate}有时候可以以一个正比于\gls{learning_rate}减小量的因素来减慢训练过程。
\gls{generalization_error}也会得到类似的曲线,或者由于\gls{learning_rate}过大或过小引起的正则化作用而变得复杂化。
由于一个糟糕的优化从某种程度上说可以避免\gls{overfitting},即使是\gls{training_error}相同的点也会拥有完全不同的\gls{generalization_error}。}
\label{fig:chap11_lr}
\end{figure}

% 417 end
调整\gls{learning_rate}外的其他参数时,需要同时监测\gls{training_error}和\gls{test_error},以判断模型是否\gls{overfitting}或\gls{underfitting},然后适当调整其\gls{capacity}。


如果\gls{training_set}错误率大于目标\gls{error_rate},那么只能增加模型\gls{capacity}以改进模型。
如果没有使用\gls{regularization},并且确信优化算法正确运行,那么有必要添加更多的网络层或\gls{hidden_unit}。
然而,令人遗憾的是,这增加了模型的计算代价。
% 418 head


如果\gls{test_set}错误率大于目标\gls{error_rate},那么可以采取两个方法。
\gls{test_error}是\gls{training_error}和\gls{test_error}之间差距与\gls{training_error}的总和。
寻找最佳的\gls{test_error}需要权衡这些数值。
当\gls{training_error}较小(因此\gls{capacity}较大),\gls{test_error}主要取决于\gls{training_error}和\gls{test_error}之间的差距时,通常神经网络效果最好。
此时目标是缩小这一差距,使\gls{training_error}的增长速率不快于差距减小的速率。
要减少这个差距,我们可以改变\gls{regularization}\gls{hyperparameter},以减少有效的模型\gls{capacity},如添加~\gls{dropout}~或\gls{weight_decay}策略。
通常,最佳性能来自\gls{regularization}得很好的大规模模型,比如使用~\gls{dropout}~的神经网络。
% 418 mid


大部分\gls{hyperparameter}可以通过推理其是否增加或减少模型\gls{capacity}来设置。
部分示例如表\ref{tab:hyperparameter_effect}所示。


% 418 end
手动调整\gls{hyperparameter}时,不要忘记最终目标:提升\gls{test_set}性能。
加入\gls{regularization}只是实现这个目标的一种方法。
只要\gls{training_error}低,随时都可以通过收集更多的训练数据来减少\gls{generalization_error}。
实践中能够确保学习有效的暴力方法就是不断提高模型\gls{capacity}和\gls{training_set}的大小,直到解决问题。
这种做法增加了训练和推断的计算代价,所以只有在拥有足够资源时才是可行的。
原则上,这种做法可能会因为优化难度提高而失败,但对于许多问题而言,优化似乎并没有成为一个显著的障碍,当然,前提是选择了合适的模型。
% 419 end


% -- 419
\begin{table}
\centering
\small
\begin{tabular}{p{2.5cm}|p{1.5cm}|p{4.0cm}|p{4.0cm}}
\gls{hyperparameter} & \gls{capacity}何时增加 & 原因  & 注意事项 \\
\hline
\gls{hidden_unit}数量 &  增加          & 增加\gls{hidden_unit}数量会增加模型的\gls{representation}能力。 & 几乎模型每个操作所需的时间和内存代价都会随\gls{hidden_unit}数量的增加而增加。\\
\hline
\gls{learning_rate} & 调至最优 & 不正确的学习速率,不管是太高还是太低都会由于优化失败而导致低有效\gls{capacity}的模型。
 & \\
\hline
卷积核宽度 & 增加 & 增加卷积核宽度会增加模型的参数数量。&
较宽的卷积核导致较窄的输出尺寸,除非使用隐式零填充减少此影响,否则会降低模型\gls{capacity}。 
较宽的卷积核需要更多的内存存储参数,并会增加运行时间,但较窄的输出会降低内存代价。
\\
\hline
隐式零填充 & 增加 & 在卷积之前隐式添加零能保持较大尺寸的\gls{representation}。&
大多数操作的时间和内存代价会增加。\\
\hline
\gls{weight_decay}系数 & 降低 & 降低\gls{weight_decay}系数使得模型参数可以自由地变大。
 & \\
\hline
\gls{dropout}\,比率 & 降低 & 较少地丢弃单元可以更多地让单元彼此``协力''来适应\gls{training_set}。
 & \\
\end{tabular}
\caption{各种\gls{hyperparameter}对模型\gls{capacity}的影响。}
\label{tab:hyperparameter_effect}
\index{Dropout}
\index{Weight decay}
\end{table}
% -- 419




\subsection{自动\glsentrytext{hyperparameter}优化算法}
\label{sec:automatic_hyperparameter_optimization_algorithms}
理想的学习算法应该是只需要输入一个数据集,就可以输出学习的函数,而不需要手动调整\gls{hyperparameter}。
一些流行的学习算法,如\gls{logistic_regression}和\gls{SVM},流行的部分原因是这类算法只有一到两个\gls{hyperparameter}需要调整,它们也能表现出不错的性能。
有些情况下,所需调整的\gls{hyperparameter}数量较少时,神经网络可以表现出不错的性能;但\gls{hyperparameter}数量有几十甚至更多时,效果会提升得更加明显。
当使用者有一个很好的初始值,例如由在相同类型的应用和架构上具有经验的人确定初始值,或者使用者在相似问题上具有几个月甚至几年的神经网络\gls{hyperparameter}调整经验,那么手动调整\gls{hyperparameter}能有很好的效果。%??  由在相同类型的应用和架构上具有经验的人确定初始  这句话好像不对吧? --没觉得有问题,你觉得是什么了
然而,对于很多应用而言,这些起点都不可用。
在这些情况下,自动算法可以找到合适的\gls{hyperparameter}。
% 420 head


如果我们仔细想想使用者搜索学习算法合适\gls{hyperparameter}的方式,我们会意识到这其实是一种优化:
我们在试图寻找\gls{hyperparameter}来优化\gls{objective_function},例如验证误差,有时还会有一些约束(如训练时间,内存或识别时间的预算)。
因此,原则上有可能开发出封装学习算法的\firstgls{hyperparameter_optimization}算法,并选择其\gls{hyperparameter},从而使用者不需要指定学习算法的\gls{hyperparameter}。
令人遗憾的是,\gls{hyperparameter}优化算法往往有自己的\gls{hyperparameter},如学习算法的每个\gls{hyperparameter}应该被探索的值的范围。
然而,这些次级\gls{hyperparameter}通常很容易选择,这是说,相同的次级\gls{hyperparameter}能够很多不同的问题上具有良好的性能。
% 420 mid


\subsection{\glsentrytext{grid_search}}
\label{sec:grid_search}
当有三个或更少的\gls{hyperparameter}时,常见的\gls{hyperparameter}搜索方法是\firstgls{grid_search}。
对于每个\gls{hyperparameter},使用者选择一个较小的有限值集去探索。
然后,这些\gls{hyperparameter}笛卡尔乘积得到一组组\gls{hyperparameter},\gls{grid_search}使用每组\gls{hyperparameter}训练模型。
挑选\gls{validation_set}误差最小的\gls{hyperparameter}作为最好的\gls{hyperparameter}。
如\figref{fig:chap11_grid_vs_random}所示\gls{hyperparameter}值的网格。
% -- 420 end


\begin{figure}[!htb]
\ifOpenSource
\centerline{\includegraphics{figure.pdf}}
\else
\begin{tabular}{cc}
\includegraphics[width=0.35\textwidth]{Chapter11/figures/grid} &
\includegraphics[width=0.35\textwidth]{Chapter11/figures/random}
\end{tabular}
\fi
\caption{\gls{grid_search}和\gls{random_search}的比较。
为了方便地说明,我们只展示两个\gls{hyperparameter}的例子,但是我们关注的问题中\gls{hyperparameter}个数通常会更多。
\emph{(左)}为了实现\gls{grid_search},我们为每个\gls{hyperparameter}提供了一个值的集合。
搜索算法对每一种在这些集合的交叉积中的\gls{hyperparameter}组合进行训练。
\emph{(右)}为了实现\gls{random_search},我们给联合\gls{hyperparameter}赋予了一个概率分布。
通常\gls{hyperparameter}之间是相互独立的。
常见的这种分布的选择是均匀分布或者是对数均匀(从对数均匀分布中抽样,就是对从均匀分布中抽取的样本进行指数运算)的。
然后这些搜索算法从联合的\gls{hyperparameter}空间中采样,然后运行每一个样本。
\gls{grid_search}和\gls{random_search}都运行了\gls{validation_set}上的误差并返回了最优的解。
这个图说明了通常只有一个\gls{hyperparameter}对结果有着重要的影响。
在这个例子中,只有水平轴上的\gls{hyperparameter}对结果有重要的作用。
\gls{grid_search}将大量的计算浪费在了指数量级的对结果无影响的\gls{hyperparameter}中,相比之下\gls{random_search}几乎每次测试都测试了对结果有影响的每个\gls{hyperparameter}的独一无二的值。
此图经~\citet{Bergstra+Bengio-LW2011}允许转载。}
\label{fig:chap11_grid_vs_random}
\end{figure}


应该如何选择搜索集合的范围呢?
在\gls{hyperparameter}是数值(有序)的情况下,每个列表的最小和最大的元素可以基于先前相似实验的经验保守地挑选出来,以确保最优解非常可能在所选范围内。
通常,\gls{grid_search}大约会在\firstgls{logarithmic_scale}下挑选合适的值,例如,一个\gls{learning_rate}的取值集合是$\{0.1,0.01,10^{-3},10^{-4},10^{-5}\}$,或者\gls{hidden_unit}数目的取值集合$\{50,100,200,500,1000,2000\}$。
% 421 head


通常重复进行\gls{grid_search}时,效果会最好。
例如,假设我们在集合$\{-1,0,1\}$上\gls{grid_search}\gls{hyperparameter} $\alpha$。
如果找到的最佳值是$1$,那么说明我们低估了最优值$\alpha$所在的范围,应该改变搜索格点,例如在集合$\{1,2,3\}$中搜索。
如果最佳值是$0$,那么我们不妨通过细化搜索范围以改进估计,在集合$\{-0.1,0,0.1\}$上进行\gls{grid_search}。
% -- 422 head


\gls{grid_search}带来的一个明显问题是,计算代价会随着\gls{hyperparameter}数量呈指数级增长。
如果有$m$个\gls{hyperparameter},每个最多取$n$个值,那么训练和估计所需的试验数将是$O(n^m)$。
我们可以并行地进行实验,并且并行要求十分宽松(进行不同搜索的机器之间几乎没有必要进行通信)。
令人遗憾的是,由于\gls{grid_search}指数级增长的计算代价,即使是并行,我们也无法提供令人满意的搜索规模。


\subsection{\glsentrytext{random_search}}
\label{sec:random_search}
幸运的是,有一个替代\gls{grid_search}的方法,并且编程简单,使用更方便,能更快地收敛到\gls{hyperparameter}的良好取值:\gls{random_search}~\citep{Bergstra+Bengio-2012-small}。
% 422


\gls{random_search}过程如下。
首先,我们为每个\gls{hyperparameter}定义一个边缘分布,例如,\gls{bernoulli_distribution}或\gls{categorical_distribution}(分别对应着二元\gls{hyperparameter}或离散\gls{hyperparameter}),或者\gls{logarithmic_scale}上的均匀分布(对应着正实值\gls{hyperparameter})。
例如,
\begin{align}
	\texttt{log\_learning\_rate} &\sim u(-1, -5), \\
	\texttt{learning\_rate} &= 10^{\texttt{log\_learning\_rate}},
\end{align}
其中,$u(a,b)$表示区间$(a,b)$上均匀采样的样本。
类似地,$\texttt{log\_number\_of\_hidden\_units}$可以从$u(\log(50), \log(2000))$上采样。
% 422 mid


与\gls{grid_search}不同,我们\emph{不需要离散化}\gls{hyperparameter}的值。这允许我们在一个更大的集合上进行搜索,而不产生额外的计算代价。%?? bin 不知道如何翻译bin,你有什么好的翻译没 
实际上,如\figref{fig:chap11_grid_vs_random}所示,当有几个\gls{hyperparameter}对\gls{performance_metrics}没有显著影响时,\gls{random_search}相比于\gls{grid_search}指数级地高效。
\cite{Bergstra+Bengio-2012-small}进行了详细的研究并发现相比于\gls{grid_search}, \gls{random_search}能够更快地减小\gls{validation_set}误差(就每个模型运行的试验数而言)。
% 422 end

与\gls{grid_search}一样,我们通常会重复运行不同版本的\gls{random_search},以基于前一次运行的结果改进下一次搜索。
% 422 end


\gls{random_search}能比\gls{grid_search}更快地找到良好\gls{hyperparameter}的原因是,没有浪费的实验,不像\gls{grid_search}有时会对一个\gls{hyperparameter}的两个不同值(给定其他\gls{hyperparameter}值不变)给出相同结果。
在\gls{grid_search}中,其他\gls{hyperparameter}将在这两次实验中拥有相同的值,而在\gls{random_search}中,它们通常会具有不同的值。
因此,如果这两个值的变化所对应的\gls{validation_set}误差没有明显区别的话,\gls{grid_search}会没有必要地重复两个等价的实验,而\gls{random_search}仍然会对其他\gls{hyperparameter}进行两次独立地探索。

% 423 head

\subsection{基于模型的\glsentrytext{hyperparameter}优化}
\label{sec:model_based_hyperparameter_optimization}
\gls{hyperparameter}搜索问题可以转化为一个优化问题。
决策变量是\gls{hyperparameter}。
优化的\gls{cost}是\gls{hyperparameter}训练出来的模型在\gls{validation_set}上的误差。
在简化的设定下,可以计算\gls{validation_set}上可导误差函数关于\gls{hyperparameter}的梯度,然后我们遵循这个梯度更新\citep{bengio:1999:snowbird,bengio-hyper-NC00,maclaurin2015gradient}。
令人遗憾的是,在大多数实际设定中,这个梯度是不可用的。
这可能是因为其高额的计算代价和存储成本,也可能是因为\gls{validation_set}误差在\gls{hyperparameter}上本质上不可导,例如\gls{hyperparameter}是离散值的情况。
% 423 mid


为了弥补梯度的缺失,我们可以对\gls{validation_set}误差建模,然后通过优化该模型来提出新的\gls{hyperparameter}猜想。
大部分基于模型的\gls{hyperparameter}搜索算法,都是使用贝叶斯回归模型来估计每个\gls{hyperparameter}的\gls{validation_set}误差期望和该期望的不确定性。
因此,优化涉及到探索(探索高度不确定的\gls{hyperparameter},可能带来显著的效果提升,也可能效果很差)和使用(使用已经确信效果不错的\gls{hyperparameter}——通常是先前见过的非常熟悉的\gls{hyperparameter})之间的权衡。
关于\gls{hyperparameter}优化的最前沿方法还包括Spearmint~\citep{Snoek+al-NIPS2012-small},TPE~\citep{Bergstra+al-NIPS2011}和SMAC~\citep{hutter+hoos+leyton+brown:2011}。


% 423 end
目前,我们无法明确确定,贝叶斯\gls{hyperparameter}优化是否是一个能够实现更好\gls{DL}结果或是能够事半功倍的成熟工具。
贝叶斯\gls{hyperparameter}优化有时表现得像人类专家,能够在有些问题上取得很好的效果,但有时又会在某些问题上发生灾难性的失误。
看看它是否适用于一个特定的问题是值得尝试的,但目前该方法还不够成熟或可靠。
就像所说的那样,\gls{hyperparameter}优化是一个重要的研究领域,通常主要受\gls{DL}所需驱动,但是它不仅能贡献于整个\gls{ML}领域,还能贡献于一般的工程学。
% 424 head


大部分\gls{hyperparameter}优化算法比\gls{random_search}更复杂,并且具有一个共同的缺点,在它们能够从实验中提取任何信息之前,它们需要运行完整的训练实验。
相比于人类实践者手动搜索,考虑实验早期可以收集的信息量,这种方法是相当低效的,因为手动搜索通常可以很早判断出某组\gls{hyperparameter}是否是完全病态的。
\cite{swersky2014freeze}提出了一个可以维护多个实验的早期版本算法。
在不同的时间点,\gls{hyperparameter}优化算法可以选择开启一个新实验,``冻结''正在运行但希望不大的实验,或是``解冻''并恢复早期被冻结的,但现在根据更多信息后又有希望的实验。


% 424
\section{调试策略}
\label{sec:debugging_strategies}
当一个\gls{ML}系统效果不好时,通常很难判断效果不好的原因是算法本身,还是算法实现错误。
由于各种原因,\gls{ML}系统很难调试。


% 424
在大多数情况下,我们不能提前知道算法的行为。
事实上,使用\gls{ML}的整个出发点是,它会发现一些我们自己无法发现的有用行为。
如果我们在一个\emph{新}的分类任务上训练一个神经网络,它达到$5\%$的\gls{test_error},我们没法直接知道这是期望的结果,还是次优的结果。


另一个难点是,大部分\gls{ML}模型有多个自适应的部分。
如果一个部分失效了,其他部分仍然可以自适应,并获得大致可接受的性能。
例如,假设我们正在训练多层神经网络,其中参数为权重$\MW$和\gls{bias_aff} $\Vb$。
进一步假设,我们单独手动实现了每个参数的\gls{GD}规则。
而我们在偏置更新时犯了一个错误:
\begin{equation}
	\Vb \leftarrow \Vb - \alpha,
\end{equation}
其中$\alpha$是\gls{learning_rate}。
这个错误更新没有使用梯度。
它会导致偏置在整个学习中不断变为负值,对于一个学习算法来说这显然是错误的。 
然而只是检查模型输出的话,该错误可能并不是显而易见的。
根据输入的分布,权重可能可以自适应地补偿负的偏置。
% -- 425 head


大部分神经网络的调试策略都是解决这两个难题的一个或两个。
我们可以设计一种足够简单的情况,能够提前得到正确结果,判断模型预测是否与之相符;我们也可以设计一个测试,独立检查神经网络实现的各个部分。


一些重要的调试检测如下所列。

\emph{可视化计算中模型的行为}:%??  in action 
当训练模型检测图像中的对象时,查看一些模型检测到部分重叠的图像。
在训练语音生成模型时,试听一些生成的语音样本。
这似乎是显而易见的,但在实际中很容易只注意量化\gls{performance_metrics},如\gls{accuracy}或对数似然。
直接观察\gls{ML}模型运行其任务,有助于确定其达到的量化性能数据是否看上去合理。
错误评估模型性能可能是最具破坏性的错误之一,因为它们会使你在系统出问题时误以为系统运行良好。
% 425 mid


\emph{可视化最严重的错误}:
大多数模型能够输出运行任务时的某种置信度量。
例如,基于\,\gls{softmax}\,输出层的分类器给每个类分配一个概率。
因此,分配给最有可能的类的概率给出了模型在其分类决定上的置信估计值。
通常,相比于正确预测的概率最大似然训练会略有高估。
但是由于实际上模型的较小概率指示结果对应着正确的标签的可能也较小,因此它们在一定意义上还是有些用的。
通过查看\gls{training_set}中很难正确建模的样本,通常可以发现该数据预处理或者标记方式的问题。
例如,街景\gls{transcription_system}原本有个问题是,地址号码检测系统会将图像裁剪得过于紧密,而省略掉了一些数字。
然后转录网络会给这些图像的正确答案分配非常低的概率。
将图像排序,确定置信度最高的错误,显示系统的裁剪有问题。
修改检测系统裁剪更宽的图像,使整个系统获得了更好的性能,即便转录网络需要能够处理地址号码中位置和范围更大变化的情况。
% 425 end


\emph{根据训练和\gls{test_error}检测软件}:
我们往往很难确定底层软件是否是正确实现。
训练和\gls{test_error}能够提供一些线索。
如果\gls{training_error}较低,但是\gls{test_error}较高,那么很有可能训练过程是在正常运行,但模型由于算法原因\gls{overfitting}了。
另一种可能是,\gls{test_error}没有被正确地度量,可能是由于训练后保存模型再重载去度量\gls{test_set}时出现问题,或者是因为测试数据和训练数据预处理的方式不同。
如果训练和\gls{test_error}都很高,那么很难确定是软件错误,还是由于算法原因模型\gls{underfitting}。
这种情况需要进一步的测试,如下面所述。

% -- 426 head

\emph{拟合极小的数据集}:
当\gls{training_set}上有很大的误差时,我们需要确定问题是真正的\gls{underfitting},还是软件错误。
通常,即使是小模型也可以保证很好地拟合一个足够小的数据集。
例如,只有一个样本的分类数据可以通过正确设置输出层的偏置来拟合。
通常,如果不能训练一个分类器来正确标注一个单独的样本,或不能训练一个\gls{AE}来成功地精准再现一个单独的样本,或不能训练一个生成模型来一致地生成一个单独的样本,那么很有可能是由于软件错误阻止\gls{training_set}上的成功优化。
此测试可以扩展到只有少量样本的小数据集上。

% 426 mid

\emph{比较反向传播导数和数值导数}:
如果读者正在使用一个需要实现梯度计算的软件框架,或者在添加一个新操作到求导库中,必须定义它的\texttt{bprop}方法,那么常见的错误原因是没能正确地实现梯度表达。
验证这些求导正确性的一种方法是比较实现的自动求导和通过\firstgls{finite_difference}计算的导数。
因为
\begin{equation}
	f'(x) = \lim_{\epsilon \to 0} \frac{f(x+\epsilon) - f(x)}{\epsilon},
\end{equation}
我们可以使用小的、有限的$\epsilon$近似导数:
\begin{equation}
	f'(x) \approx \frac{f(x+\epsilon) - f(x)}{\epsilon}.
\end{equation}
我们可以使用\firstgls{centered_difference}提高近似的\gls{accuracy}:
\begin{equation}
	f'(x) \approx \frac{ f(x+\frac{1}{2}\epsilon) - f(x-\frac{1}{2}\epsilon) }{\epsilon}.
\end{equation}
扰动大小$\epsilon$必须足够大,以确保该扰动不会由于数值计算的有限\gls{precision}问题产生舍入误差。

% 426 end

通常,我们会测试向量值函数$g:\SetR^m \to \SetR^n$的梯度或\,\gls{jacobian}\,矩阵。
令人遗憾的是,\gls{finite_difference}只允许我们每次计算一个导数。
我们可以使用\gls{finite_difference} $mn$次评估$g$的所有偏导数,也可以将该测试应用于一个新函数(在函数$g$的输入输出都加上随机投影)。%??  后面这句好像不对?  g的输入输出都使用随机投影的
例如,我们可以将导数实现的测试用于函数$f(x) = \Vu^T g(\Vv x)$,其中$\Vu$和$\Vv$是随机向量。%??  这句也不对
正确计算$f'(x)$要求能够正确地通过$g$反向传播,但是使用\gls{finite_difference}能够高效地计算,因为$f$只有一个输入和一个输出。
通常,一个好的方法是在多个$\Vu$值和$\Vv$值上重复这个测试,可以减少测试忽略了垂直于随机投影的错误的几率。%??  很难

% 427 head

如果我们可以在复数上进行数值计算,那么使用复数作为函数的输入会有非常高效的数值方法估算梯度\citep{Squire+Trapp-1998}。
该方法基于如下观察
\begin{align}
	f(x + i\epsilon) &= f(x) + i\epsilon f'(x) + O(\epsilon^2) ,\\
	\text{real}( f(x+i\epsilon) ) &= f(x) + O(\epsilon^2), \quad \text{image}( \frac{f(x+i\epsilon)}{ \epsilon } ) = f'(x) + O(\epsilon^2),
\end{align}
其中$i=\sqrt{-1}$。
和上面的实值情况不同,这里不存在消除影响,因为我们对$f$在不同点上计算差分。
因此我们可以使用很小的$\epsilon$,比如$\epsilon = 10^{-150}$,其中误差$O(\epsilon^2)$对所有实用目标都是微不足道的。

% 427 mid

\emph{监控激活函数值和梯度的直方图}:
可视化神经网络在大量训练迭代后(也许是一个\gls{epoch})收集到的激活函数值和梯度的统计量往往是有用的。
\gls{hidden_unit}的预激活值可以告诉我们该单元是否饱和,或者它们饱和的频率如何。
例如,对于整流器,它们多久关一次?是否有单元一直关闭?
对于双曲正切单元而言,预激活绝对值的平均值可以告诉我们该单元的饱和程度。
在深度网络中,传播梯度的快速增长或快速消失,可能会阻碍优化过程。
最后,比较参数梯度和参数的量级也是有帮助的。
正如\citep{Bottou-DLSS2015}所建议的,我们希望参数在一个\gls{minibatch}更新中变化的幅度是参数量值$1\%$这样的级别,而不是$50\%$或者$0.001\%$(这会导致参数移动得太慢)。
也有可能是某些参数以良好的步长移动,而另一些停滞。
如果数据是稀疏的(比如自然语言),有些参数可能很少更新,检测它们变化时应该记住这一点。

% 428 head

最后,许多\gls{DL}算法为每一步产生的结果提供了某种保证。
例如,在第三部分,我们将看到一些使用代数解决优化问题的近似推断算法。
通常,这些可以通过测试它们的每个保证来调试。
某些优化算法提供的保证包括,\gls{objective_function}值在算法的迭代步中不会增加,某些变量的导数在算法的每一步中都是零,所有变量的梯度在收敛时会变为零。
通常,由于舍入误差,这些条件不会在数字计算机上完全成立,因此调试测试应该包含一些容差参数。

% 428 mid

\section{示例:多位数字识别}
\label{sec:example_multi_digit_number_recognition}
为了\gls{end_to_end}说明如何在实践中应用我们的设计方法论,我们从设计\gls{DL}组件的视角出发,简单地介绍下街景\gls{transcription_system}。
显然,整个系统的许多其他组件,如街景车、数据库设施等等,也是极其重要的。

% 428 mid

从\gls{ML}任务的视角出发,首先这个过程要采集数据。
街景车收集原始数据,然后操作员手动提供标签。
转录任务开始前有大量的数据处理工作,包括在转录前使用其他\gls{ML}技术\emph{探测}房屋号码。

% 428 mid

转录项目开始于选择\gls{performance_metrics}和对这些度量的期望值。
一个重要的总原则是度量的选择要符合项目的业务目标。
因为地图只有是高\gls{accuracy}时才有用,所以为这个项目设置高\gls{accuracy}的要求非常重要。
具体地,目标是达到人类水平,$98\%$的\gls{accuracy}。
这种程度的\gls{accuracy}并不是总能达到。
为了达到这个级别的\gls{accuracy},街景\gls{transcription_system}牺牲了\gls{coverage}。
因此在保持\gls{accuracy} $98\%$的情况下,\gls{coverage}成了这个项目优化的主要\gls{performance_metrics}。
随着\gls{convolutional_network}的改进,我们能够降低网络拒绝转录输入的置信度阈值,最终超出了\gls{coverage} $95\%$的目标。

% -- 428 end

在选择量化目标后,我们推荐方法的下一步是要快速建立一个合理的\gls{baseline}系统。
对于视觉任务而言,\gls{baseline}系统是带有\gls{ReLU}的\gls{convolutional_network}。
转录项目开始于一个这样的模型。
当时,使用\gls{convolutional_network}输出预测序列并不常见。
开始时,我们使用一个尽可能简单的\gls{baseline}模型,该模型输出层的第一个实现包含$n$个不同的~\gls{softmax_unit}来预测$n$个字符的序列。
我们使用与训练分类任务相同的方式来训练这些~\gls{softmax_unit},独立地训练每个~\gls{softmax_unit}。

% 429 head

我们建议反复细化这些\gls{baseline},并测试每个变化是否都有改进。
街景\gls{transcription_system}的第一个变化受激励于\gls{coverage}指标的理论理解和数据结构。
具体地,当输出序列的概率低于某个值$t$即$p(\Vy\mid\Vx) < t$时,网络拒绝为输入$\Vx$分类。
最初,$p(\Vy\mid\Vx)$的定义是临时的,简单地将所有~\gls{softmax}输出乘在一起。
这促使我们发展能够真正计算出合理对数似然的特定输出层和\gls{cost_function}。%??  后来是否该去掉?
这种方法使得样本拒绝机制更有效。


此时,\gls{coverage}仍低于$90\%$,但该方法没有明显的理论问题了。
因此,我们的方法论建议综合\gls{training_set}和\gls{test_set}性能,以确定问题是否是\gls{underfitting}或\gls{overfitting}。
在这种情况下,训练和\gls{test_set}误差几乎是一样的。
事实上,这个项目进行得如此顺利的主要原因是有数以千万计的\gls{labeled}样本数据集可用。
因为训练和\gls{test_set}的误差是如此相似,这表明要么是这个问题\gls{underfitting},要么是训练数据的问题。
我们推荐的调试策略之一是可视化模型最糟糕的错误。
在这种情况下,这意味着可视化 不正确而模型给了最高置信度的\gls{training_set}转录结果。
结果显示,主要是输入图像裁剪得太紧,有些和地址相关的数字被裁剪操作除去了。
例如,地址``1849''的图片可能裁切得太紧,只剩下``849''是可见的。
如果我们花费几周时间改进确定裁剪区域的地址号码检测系统的\gls{accuracy},或许也可以解决这个问题。%??  
与之不同,项目团队采取了更实际的办法,简单地系统性扩大裁剪区域的宽度,使其大于地址号码检测系统预测的区域宽度。%??    ’大于‘ 不对吧, 加个 使其
这种单一改变将\gls{transcription_system}的\gls{coverage}提高了$10$个百分点。

% -- 429 end

最后,性能提升的最后几个百分点来自调整\gls{hyperparameter}。
这主要包括在保持一些计算代价限制的同时加大模型的规模。
因为\gls{training_error}和\gls{test_error}保持几乎相等,所以明确表明性能不足是由\gls{underfitting}造成的,数据集本身也存在一些问题。


总体来说,转录项目是非常成功的,可以比人工速度更快、代价更低地转录数以亿计的地址。

我们希望本章中介绍的设计原则能带来其他更多类似的成功。

% -- 430 --

% !Mode:: "TeX:UTF-8"
% Translator: Tianfan Fu: 12.1~12.3 Shenjian Zhao: 12.4~12.5
\chapter{应用}
\label{chap:applications}

在本章中,我们将介绍如何使用\gls{DL}来解决\gls{CV}、\gls{SR}、\gls{NLP}以及其他商业领域中的应用。
首先我们将讨论在许多最重要的~\glssymbol{AI}~应用中所需的大规模\gls{NN}的实现。
接着,我们将回顾\gls{DL}已经成功应用的几个特定领域。
尽管\gls{DL}的一个目标是设计能够处理各种任务的算法,然而截止目前\gls{DL}的应用仍然需要一定程度的特化。
例如,\gls{CV}中的任务对每一个样本都需要处理大量的输入特征(像素)。
\gls{NLP}任务的每一个输入特征都需要对大量的可能值(词汇表中的词)建模。
% 431

\section{大规模\glsentrytext{DL}}
\label{sec:large_scale_deep_learning}
% 431 

\gls{DL}的基本思想基于\gls{connectionism}:尽管\gls{ML}模型中单个生物性的神经元或者说是单个特征不是智能的,但是大量的神经元或者特征作用在一起往往能够表现出智能。
我们必须着重强调神经元数量必须\emph{很大}这个事实。
相比20世纪80年代,如今\gls{NN}的精度以及处理任务的复杂度都有一定提升,其中一个关键的因素就是网络规模的巨大提升
。
正如我们在\secref{sec:increasing_model_sizes}中看到的一样,在过去的三十年内,网络规模是以指数级的速度递增的。
然而如今的\gls{ANN}的规模也仅仅和昆虫的神经系统差不多。

由于规模的大小对于神经网络来说至关重要,因此深度学习需要高性能的硬件设施和软件实现。
%大规模\gls{NN}的必要性,所以\gls{DL}需要高性能的硬件设施和软件实现。
% 431

\subsection{快速的CPU实现}
\label{sec:fast_cpu_implementations}

传统的\gls{NN}是用单台机器的CPU来训练的。
如今,这种做法通常被视为是不可取的。
现在,我们通常使用~\glssymbol{GPU}~或者许多台机器的CPU连接在一起进行计算。
在使用这种昂贵配置之前,为论证CPU无法承担\gls{NN}所需的巨大计算量,研究者们付出了巨大的努力。
% 432 head


描述如何实现高效的数值CPU代码已经超出了本书的讨论范围,但是我们在这里还是要强调通过设计一些特定的CPU上的操作可以大大提升效率。
例如,在2011年,最好的CPU在训练\gls{NN}时使用\gls{fixed_point_arithmetic}能够比\gls{float_point_arithmetic}跑得更快。
通过调整\gls{fixed_point_arithmetic}的实现方式,\citet{Vanhoucke-et-al-2011}获得了3倍于一个强\gls{float_point_arithmetic}系统的速度。
% 相对一个很强的\gls{float_point_arithmetic}系统3倍的加速
因为各个新型CPU都有各自不同的特性,所以也有时候采用\gls{float_point_arithmetic}实现会更快。
% 一条重要的准则就是通过特殊设计的数值运算可以获得巨大的回报。
一条重要的准则就是,通过特殊设计的数值运算,我们可以获得巨大的回报。
除了选择\gls{fixed_point_arithmetic}或者\gls{float_point_arithmetic}以外,其他的策略还包括了如通过优化数据结构避免高速缓存缺失、使用向量指令等。
%还包括其他的策略,如通过优化数据结构避免高速缓存缺失、使用向量指令等。
\gls{ML}的研究者们一般会忽略这些实现的细节,但是如果模型的性能限制了模型能达到的规模,那该模型的准确性就要受到影响。
% 432

\subsection{GPU 实现}
\label{sec:gpu_implementations}

许多现代\gls{NN}的实现基于\firstall{GPU}。
\glsacr{GPU}最初是为图形应用而开发的专用硬件组件。
%是一种特殊设计的硬件,设计的原始目的是为了处理图形应用。
视频游戏系统的消费市场刺激了图形处理硬件的发展。
它为视频游戏所设计的特性也可以使\gls{NN}的计算受益。
% 432  

视频游戏的渲染要求许多操作能够快速并行地执行。
环境和角色模型通过一系列顶点的3D坐标确定。
为了将大量的3D坐标转化为2D显示器上的坐标,显卡必须并行地对许多顶点执行矩阵乘法与除法。
%显卡必须快速实现矩阵乘法或者除法。
之后,显卡必须并行地在每个像素上执行诸多计算,来确定每个像素点的颜色。
在这两种情况下,计算都是非常简单的,并且不涉及CPU通常遇到的复杂的分支运算。
例如,同一个刚体内的每个顶点都会乘上相同的矩阵;也就是说,不需要通过{\tt if}语句来判断确定每个顶点需要乘哪个矩阵。
各个计算过程之间也是完全相互独立的,因此能够实现并行操作。
计算过程还涉及处理大量内存缓冲以及描述每一个需要被渲染的对象的纹理(颜色模式)的位图信息。
总的来说,这使显卡设计为拥有高度并行特性以及很高的内存带宽,同时也付出了一些代价,如相比传统的CPU更慢的时钟速度以及更弱的处理分支运算的能力。
% p 433


与上述的实时图形算法相比,\gls{NN}算法所需要的性能特性是相同的。
\gls{NN}算法通常涉及大量参数、激活值、梯度值的缓冲区,其中每个值在每一次训练迭代中都要被完全更新。
这些缓冲太大,会超出传统的桌面计算机的高速缓存(cache),所以内存带宽通常会成为主要瓶颈。
相比CPU,\glssymbol{GPU}~一个显著的优势是其极高的内存带宽。
\gls{NN}的训练算法通常并不涉及大量的分支运算与复杂的控制指令,所以更适合在~\glssymbol{GPU}~硬件上训练。
由于\gls{NN}能够被分为多个单独的``神经元'',并且独立于同一层内其他神经元进行处理,所以\gls{NN}可以从~\glssymbol{GPU}~的并行特性中受益匪浅。
% 433


\glssymbol{GPU}~硬件最初专为图形任务而设计。
随着时间的推移,\glssymbol{GPU}~也变得更灵活,允许定制的子程序处理转化顶点坐标或者计算像素颜色的任务。
原则上,\glssymbol{GPU}~不要求这些像素值实际基于渲染任务。
只要将计算的输出值作为像素值写入缓冲区,\glssymbol{GPU}~就可以用于科学计算。
\citet{Steinkrau2005}在GPU上实现了一个两层全连接的\gls{NN},并获得了相对基于CPU的基准方法三倍的加速。
不久以后,\citet{chellapilla:inria-00112631}也论证了相同的技术可以用来加速\gls{supervised}\gls{convolutional_network}的训练。
% 433


在\gls{GP_GPU}~发布以后,使用显卡训练\gls{NN}的热度开始爆炸性地增长。
这种\gls{GP_GPU}~可以执行任意的代码,而并非仅仅渲染子程序。
NVIDIA的CUDA编程语言使得我们可以用一种像C一样的语言实现任意代码。
由于相对简便的编程模型,强大的并行能力以及巨大的内存带宽,\gls{GP_GPU}~为我们提供了训练\gls{NN}的理想平台。
在它发布以后不久,这个平台就迅速被\gls{DL}的研究者们所采纳~\citep{RainaICML09-small,Ciresan-2010}。



如何在\gls{GP_GPU}~上写高效的代码依然是一个难题。
在~\glssymbol{GPU}~上获得良好表现所需的技术与CPU上的技术非常不同。
比如说,基于CPU的良好代码通常被设计为尽可能从高速缓存中读取更多的信息。
然而在~\glssymbol{GPU}~中,大多数可写内存位置并不会被高速缓存,所以计算某个值两次往往会比计算一次然后从内存中读取更快。
\glssymbol{GPU}~代码是天生多线程的,不同线程之间必须仔细协调好。
例如,如果能够把数据\firstgls{coalesced}起来,那么涉及内存的操作一般会更快。
当几个线程同时需要读/写一个值时,像这样的\gls{coalesced}会作为一次内存操作出现。
不同的~\glssymbol{GPU}~可能采用不同的\gls{coalesced}读/写数据的方式。
通常来说,如果在$n$个线程中,线程$i$访问的是第$i+j$处的内存,其中$j$是$2$的某个幂的倍数,那么内存操作就易于\gls{coalesced}。
具体的设定在不同的~\glssymbol{GPU}~型号中有所区别。
\glssymbol{GPU}~另一个常见的设定是使一个组中的所有线程都同时执行同一指令。
这意味着~\glssymbol{GPU}~难以执行分支操作。
线程被分为一个个称作~\textbf{warp}~的小组。
在一个~\gls{warp}~中的每一个线程在每一个循环中执行同一指令,所以当同一个~\gls{warp}~中的不同线程需要执行不同的指令时,需要使用串行而非并行的方式。
% 434 


由于实现高效~\glssymbol{GPU}~代码的困难性,研究人员应该组织好他们的工作流程,避免对每一个新的模型或算法都编写新的~\glssymbol{GPU}~代码。
通常来讲,人们会选择建立一个包含高效操作(如卷积和矩阵乘法)的软件库解决这个问题,然后再从库中调用所需要的操作确定模型。
%例如,\gls{ML}库Pylearn2 \citep{pylearn2_arxiv_2013}通过调用Theano \citep{bergstra+al:2010-scipy-short,Bastien-2012}和cuda-convnet \citep{Krizhevsky2010tr}提供的高性能操作,囊括了许多\gls{ML}算法。
例如,\gls{ML}库Pylearn2 \citep{pylearn2_arxiv_2013}将其所有的\gls{ML}算法都通过调用Theano~\citep{bergstra+al:2010-scipy-short,Bastien-2012}和cuda-convnet~\citep{Krizhevsky2010tr} 所提供的高性能操作来指定。
这种分解方法还可以简化对多种硬件的支持。
例如,同一个Theano程序可以在CPU或者~\glssymbol{GPU}~上运行,而不需要改变调用Theano的方式。
其他库如Tensorflow~\citep{tensorflow}和Torch~\citep{Torch-2011}也提供了类似的功能。
% 434 end


\subsection{大规模的分布式实现}
\label{sec:large_scale_distributed_implementations}

在许多情况下,单个机器的计算资源是有限的。
因此,我们希望把训练或者推断的任务分摊到多个机器上进行。
% 434 end

分布式的推断是容易实现的,因为每一个输入的样本都可以在单独的机器上运行。
这也被称为\firstgls{data_parallelism}。

同样地,\firstgls{model_parallelism}也是可行的,其中多个机器共同运行一个数据点,每一个机器负责模型的一个部分。
对于推断和训练,这都是可行的。
% 435



在训练过程中,\gls{data_parallelism}某种程度上来说更加困难。
对于\gls{SGD}的单步来说,我们可以增加\gls{minibatch}的大小,但是从优化性能的角度来说,我们得到的回报通常并不会线性增长。
使用多个机器并行地计算多个\gls{GD}步骤是一个更好的选择。
不幸的是,\gls{GD}的标准定义完全是一个串行的过程:
第$t$步的梯度是第$t-1$步所得参数的函数。
% 435


这个问题可以使用\firstgls{ASGD}\citep{BenDucVin01-small,Recht-et-al-NIPS2011}解决。
在这个方法中,几个处理器的核共用存有参数的内存。
每一个核在无锁情况下读取这些参数并计算对应的梯度,然后在无锁状态下增量更新这些参数。
%这种方法减少了每一个\gls{GD}所获得的平均提升,因为一些核把其他的核所更新的参数(写)覆盖了。
由于一些核把其他的核所更新的参数覆盖了,因此这种方法减少了每一步\gls{GD}所获得的平均提升。
但因为更新步数的速率增加,总体上还是加快了学习过程。
\citet{Dean-et-al-NIPS2012}率先尝试了无锁的\gls{GD}方法的多机器实现,其中参数是由\firstgls{parameter_server}管理而非存储在共用的内存中。
分布式的异步\gls{GD}方法目前依然是训练大型\gls{DNN}的基本策略,并被工业界很多\gls{ML}组所使用\citep{chilimbi2014project,Wu-et-al-arXiv2015}。
学术界的\gls{DL}研究者们通常无法负担那么大规模的分布式学习系统,但是仍有一些研究关注于如何在校园环境中使用相对廉价的硬件系统构造分布式网络~\citep{icml2013_coates13}。
% 435


\subsection{\glsentrytext{model_compression}}
\label{sec:model_compression}
% 435 

在许多商业应用的\gls{ML}模型中,一个时间和内存开销较小的推断算法比一个时间和内存开销较小的训练算法要更为重要。
对于那些不需要个性化设计的应用来说,我们只需要一次性的训练模型,然后它就可以被成千上万的用户使用。
在许多情况下,相比开发者,终端用户的可用资源往往更有限。
例如,开发者们可能会使用巨大的计算机集群训练一个\gls{SR}的网络,然后将其部署到移动手机上。


减少推断所需开销的一个关键策略是\firstgls{model_compression}~\citep{bucilua2006model}。
\gls{model_compression}的基本思想是用一个更小的模型取代原始耗时的模型,从而使得存储与评估模型所需的内存与运行时间更少。



当原始模型的规模主要由防止\gls{overfitting}的需求驱动时,\gls{model_compression}就可以起到作用。
在许多情况下,拥有最小\gls{generalization}误差的模型往往是多个独立训练而成的模型的\gls{ensemble}。
评估所有$n$个\gls{ensemble}成员的成本很高。
有时候,当单个模型很大(例如,如果它使用~\gls{dropout}~正则化)时,其\gls{generalization}能力也会很好。
% 436 

这些巨大的模型能够学习到某个函数$f(\Vx)$,但选用的参数数量超过了任务所需的参数数量。
%仅仅训练样本数是有限的,所以网络的规模是受限的。
只是因为训练样本数是有限的,所以模型的规模才变得必要。
只要我们拟合了这个函数$f(\Vx)$,我们就可以通过将$f$作用于随机采样点$x$来生成有无穷多训练样本的训练集。
%我们就可以生成一个拥有了无穷多训练样本的训练集,只需将$f$作用于任意生成的$\Vx$。
然后,我们使用这些样本训练一个新的更小的模型,使其能够在这些点上拟合$f(\Vx)$。
为了更加充分地利用了这个新的小模型的\gls{capacity},最好从类似于真实测试数据(之后将提供给模型)的分布中采样$\Vx$。
这个过程可以通过损坏训练样本或者从原始训练数据训练的生成模型中采样完成。

此外,我们还可以仅在原始训练数据上训练一个更小的模型,但只是为了复制模型的其他特征,比如在不正确的类上的后验分布\citep{Hinton-dark-2014,hinton2015distilling}。
% 436 

\subsection{\glsentrytext{dynamic_structure}}
\label{sec:dynamic_structure}

一般来说,加速数据处理系统的一种策略是构造一个系统,这个系统用\firstgls{dynamic_structure}描述图中处理输入的所需计算过程。
在给定一个输入的情况中,数据处理系统可以动态地决定运行神经网络系统的哪一部分。
单个神经网络内部同样也存在\gls{dynamic_structure},给定输入信息,决定特征(\gls{hidden_unit})哪一部分用于计算。
这种神经网络中的\gls{dynamic_structure}有时被称为\firstgls{conditional_computation}\citep{Bengio-chapterSLSP-2013,bengio2013estimating}。
由于模型结构许多部分可能只跟输入的一小部分有关,只计算那些需要的特征可以起到加速的目的。
% 436 

\gls{dynamic_structure}计算是一种基础的计算机科学方法,广泛应用于软件工程项目。
应用于神经网络的最简单的\gls{dynamic_structure}基于决定神经网络(或者其他\gls{ML}模型)中的哪些子集需要应用于特定的输入。
% 437 

在分类器中加速推断的可行策略是使用\firstgls{cascade}的分类器。
当目标是检测罕见对象(或事件)是否存在时,可以应用\gls{cascade}策略。
要确定对象是否存在,我们必须使用具有高\gls{capacity}、运行成本高的复杂分类器。 
然而,因为对象是罕见的,我们通常可以使用更少的计算拒绝不包含对象的输入。
在这些情况下,我们可以训练一序列分类器。
序列中的第一个分类器具有低\gls{capacity},训练为具有高\gls{recall}。
换句话说,他们被训练为确保对象存在时,我们不会错误地拒绝输入。
最后一个分类器被训练为具有高精度。
在测试时,我们按照顺序运行分类器进行推断,一旦\gls{cascade}中的任何一个拒绝它,就选择抛弃。
总的来说,这允许我们使用高\gls{capacity}模型以较高的置信度验证对象的存在,而又不会强制我们为每个样本付出完全推断的成本。
有两种不同的方式可以使得\gls{cascade}实现高\gls{capacity}。
一种方法是使\gls{cascade}中靠后的成员单独具有高\gls{capacity}。
在这种情况下,由于系统中的一些个体成员具有高\gls{capacity},因此系统作为一个整体显然也具有高\gls{capacity}。
还可以使用另一种\gls{cascade},其中每个单独的模型具有低\gls{capacity},但是由于许多小型模型的组合,整个系统具有高\gls{capacity}。
\citet{Viola01}使用\gls{cascade}的增强\gls{decision_tree}实现了适合在手持数字相机中使用的快速并且鲁棒的面部检测器。
本质上,它们的分类器使用滑动窗口方法来定位面部。
分类器会检查许多的窗口,如果这些窗口内不包含面部则被拒绝。
\gls{cascade}的另一个版本使用靠前的模型来实现一种硬\gls{attention_mechanism}:\gls{cascade}的先遣成员定位对象,\gls{cascade}的后续成员在给定对象位置的情况下执行进一步处理。
例如,Google使用两步\gls{cascade}从街景视图图像中转换地址编号:首先使用一个\gls{ML}模型查找地址编号,然后使用另一个\gls{ML}模型将其转录\citep{Goodfellow+et+al-ICLR2014a}。
% 437 

\gls{decision_tree}本身是\gls{dynamic_structure}的一个例子,因为树中的每个节点决定应该使用哪个子树来评估输入。
一个结合\gls{DL}和\gls{dynamic_structure}的简单方法是训练一个\gls{decision_tree},其中每个节点使用神经网络做出决策\citep{guo1992classification},虽然这种方法没有实现加速推断计算的目标。
% 437 end



类似的,我们可以使用称为\firstgls{gater}的神经网络来选择在给定当前输入的情况下将使用几个\firstgls{expert_network}中的哪一个来计算输出。
这个想法的第一个版本被称为\firstgls{mixture_of_experts}\citep{Nowlan90,Jacobs-nc91},其中\gls{gater}为每个专家输出一个概率或权重(通过非线性的\gls{softmax}获得),并且最终输出由各个专家输出的加权组合获得。
在这种情况下,使用\gls{gater}不会降低计算成本,但如果每个样本的\gls{gater}选择单个专家,我们就会获得一个特殊的\firstgls{hard_mixture_of_experts}\,\citep{collobert:2001:rr01-12,collobert:2002},这可以加速推断和训练。
当\gls{gater}决策的数量很小时,这个策略效果会很好,因为它不是组合的。
但是当我们想要选择不同的单元或参数子集时,不可能使用``软开关'',因为它需要枚举(和计算输出)所有的\gls{gater}配置。
为了解决这个问题,许多工作探索了几种方法来训练组合的\gls{gater}。
\citet{bengio2013estimating}提出使用\gls{gater}概率梯度的若干估计器,而~\citet{Bacon-et-al-RLDM2015,BengioE-et-al-arXiv2015}使用\gls{RL}技术(\firstgls{policy_gradient})来学习一种条件的~\gls{dropout}~形式(作用于\gls{hidden_unit}块),减少了实际的计算成本,而不会对近似的质量产生负面影响。
% 438


另一种\gls{dynamic_structure}是开关,其中隐藏单元可以根据具体情况从不同单元接收输入。
这种动态路由方法可以理解为\firstgls{attention_mechanism} \citep{Olshausen1993}。
目前为止,硬性开关的使用在大规模应用中还没有被证明是有效的。
较为先进的方法一般采用对许多可能的输入使用加权平均,因此不能完全得到\gls{dynamic_structure}所带来的计算益处。
先进的\gls{attention_mechanism}将在\secref{sec:using_an_attention_mechanism_and_aligning_pieces_of_data}中描述。
% 438



使用动态结构化系统的主要障碍是由于系统针对不同输入的不同代码分支导致的并行度降低。
这意味着网络中只有很少的操作可以被描述为对样本\gls{minibatch}的矩阵乘法或批量卷积。
我们可以写更多的专用子程序,用不同的核对样本做卷积,或者通过不同的权重列来乘以设计矩阵的每一行。
不幸的是,这些专用的子程序难以高效地实现。
由于缺乏高速缓存的一致性,CPU实现会十分缓慢。
此外,由于缺乏\gls{coalesced}的内存操作以及~\gls{warp}~成员使用不同分支时需要串行化操作,\glssymbol{GPU}~的实现也会很慢。
在一些情况下,我们可以通过将走同一个分支的样本分到同一组,从而可以同时处理这些样本组的方式来缓解这些问题。
在离线环境中,这是最小化处理固定量样本所需时间的一项可接受的策略。
然而在实时系统中,样本必须连续处理,对工作负载进行分区可能会导致负载均衡问题。
例如,如果我们分配一台机器处理\gls{cascade}中的第一步,另一台机器处理\gls{cascade}中的最后一步,那么第一台机器将倾向于过载,最后一个机器倾向于欠载。
如果每个机器被分配以实现神经\gls{decision_tree}的不同节点,也会出现类似的问题。
% 439


\subsection{深度网络的专用硬件实现}
\label{sec:specialized_hardware_implementations_of_deep_networks}
%439

自从早期的神经网络研究以来,硬件设计者已经致力于可以加速\gls{NN}算法的训练和/或推断的专用硬件实现。
读者可以查看早期和更近的专用硬件深度网络的评论\citep{Lindsey+Lindblad-1994,Beiu-et-al-2003,Misra+Saha-2010}。
%439


不同形式的专用硬件\citep{Graf+Jackel-1989,Mead+Ismail-2012,Kim-et-al-2009,Pham-et-al-2012,Chen-et-al-IEEE2014,Chen-et-al-ACM2014}的研究已经持续了好几十年,比如\firstall{ASIC}的数字(基于数字的二进制表示),模拟\citep{Graf+Jackel-1989,Mead+Ismail-2012}(基于以电压或电流表示连续值的物理实现)和混合实现(组合数字和模拟组件)。
近年来更灵活的\firstall{FPGA}实现(其中电路的具体细节可以在制造完成后写入芯片)也得到了长足发展。
% 439 mid



虽然CPU和~\glssymbol{GPU}~上的软件实现通常使用32或64位的精度来表示浮点数,但是长期以来使用较低的精度在更短的时间内完成推断也是可行的\citep{Holt-et-al-1991,Holi+Hwang-1993,Presley-et-al-1994,Simard+Graf-NIPS1994,Wawrzynek-et-al-IEEE1996,Savich-et-al-2007}。
这已成为近年来更迫切的问题,因为\gls{DL}在工业产品中越来越受欢迎,并且由于更快的硬件产生的巨大影响已经通过~\glssymbol{GPU}~的使用得到了证明。
激励当前对深度网络专用硬件研究的另一个因素是单个CPU或~\glssymbol{GPU}~核心的进展速度已经减慢,并且最近计算速度的改进来自于核心的并行化(无论CPU还是~\glssymbol{GPU})。
这与20世纪90年代的情况(上一个神经网络时代)不同,当时\gls{NN}的硬件实现(从开始到芯片可用可能需要两年)跟不上快速进展和价格低廉的通用CPU的脚步。
因此,在针对诸如手机等低功率设备开发新的硬件设计,并且想要用于\gls{DL}的一般公众应用(例如,具有语音、\gls{CV}或自然语言功能的设施)等时,研究专用硬件能够进一步推动其发展。
% 440 head


最近对基于\gls{BP}\gls{NN}的低精度实现的工作\citep{Vanhoucke-et-al-2011,Courbariaux-et-al-ICLR2015workshop,Gupta-et-al-2015}表明,8和16位之间的精度足以满足使用或训练基于\gls{BP}的\gls{DNN}的要求。
显而易见的是,在训练期间需要比在推断时更高的精度,并且数字某些形式的动态定点表示能够减少每个数需要的存储空间。
传统的定点数被限制在了一个固定范围之内(其对应于浮点表示中的给定指数)。
而动态定点表示在一组数字 (例如一个层中的所有权重) 之间共享该范围。
使用定点代替浮点表示并且每个数使用较少的比特能够减少执行乘法所需的硬件表面积、功率需求和计算时间。
而乘法是使用或训练\gls{BP}的现代深度网络中要求最高的操作。



% 440 new
\section{\glsentrytext{CV}}
\label{sec:computer_vision}

一直以来,\gls{CV}就是\gls{DL}应用中几个最活跃的研究方向之一。
因为视觉是一个对人类以及许多动物毫不费力,但对计算机却充满挑战的任务\citep{ballard1983parallel}。
\gls{DL}中许多流行的标准基准任务包括\gls{object_recognition}以及光学字符识别。
% 440  


\gls{CV}是一个非常广阔的发展领域,其中包括多种多样的处理图片的方式以及应用方向。
\gls{CV}的应用广泛:从复现人类视觉能力(比如识别人脸)到创造全新的视觉能力。
举个后者的例子,近期一个新的\gls{CV}应用是从视频中可视物体的振动中识别相应的声波\citep{Davis2014VisualMic}。
大多数\gls{CV}领域的\gls{DL}研究未曾关注过这样一个奇异的应用,它扩展了图像的范围,而不是仅仅关注于\gls{AI}中较小的核心目标——复制人类的能力。
无论是报告图像中存在哪个物体,还是给图像中每个对象添加包围它的边框,
或从图像中转录符号序列,或给图像中的每个像素标记它所属对象的标识,
大多数\gls{CV}中的\gls{DL}往往用于\gls{object_recognition}或者某种形式的检测。
由于\gls{generative_model}已经是\gls{DL}研究的指导原则,因此还有大量图像合成工作使用了深度模型。
尽管图像合成(``\emph{无中生有}'')通常不包括在\gls{CV}内,但是能够进行图像合成的模型常常用于图像恢复,即修复图像中的缺陷或从图像中移除对象这样的\gls{CV}任务。
% 441 

\subsection{预处理}
\label{sec:preprocessing}
% 441 

由于原始输入往往以\gls{DL}架构难以表示的形式出现,许多应用领域需要复杂精细的预处理。
\gls{CV}通常只需要相对少的这种预处理。
图像应该被标准化,从而使得它们的像素都在相同并且合理的范围内,比如$[0,1]$或者$[-1,1]$。
将$[0,1]$中的图像与$[0,255]$中的图像混合通常会导致失败。
将图像格式化为具有相同的比例严格上说是唯一一种必要的预处理。
许多\gls{CV}架构需要标准尺寸的图像,因此必须裁剪或缩放图像以适应该尺寸。
然而,严格地说即使是这种重新调整比例的操作并不总是必要的。
一些卷积模型接受可变大小的输入并动态地调整它们的\gls{pooling}区域大小以保持输出大小恒定\citep{Waibel89b}。
其他卷积模型具有可变大小的输出,其尺寸随输入自动缩放,例如对图像中的每个像素进行\gls{denoise}或标注的模型\citep{Hadsell-RSS-07}。
% 441 

\gls{dataset_augmentation}可以被看作是一种只对训练集做预处理的方式。
\gls{dataset_augmentation}是减少大多数\gls{CV}模型\gls{generalization}误差的一种极好方法。
在测试时可用的一个相关想法是将同一输入的许多不同版本传给模型(例如,在稍微不同的位置处裁剪的相同图像),并且在模型的不同实例上决定模型的输出。
后一个想法可以被理解为集成方法,并且有助于减少\gls{generalization}误差。
% 441 

其他种类的预处理需要同时应用于训练集和测试集,其目的是将每个样本置于更规范的形式,以便减少模型需要考虑的变化量。
减少数据中的变化量既能够减少\gls{generalization}误差,也能够减小拟合训练集所需模型的大小。
更简单的任务可以通过更小的模型来解决,而更简单的解决方案\gls{generalization}能力一般更好。
这种类型的预处理通常被设计为去除输入数据中的某种可变性,这对于人工设计者来说是容易描述的,并且人工设计者能够保证不受到任务影响。
当使用大型数据集和大型模型训练时,这种预处理通常是不必要的,并且最好就让模型去学习哪些变化性应该保留。
例如,用于分类ImageNet的AlexNet系统仅具有一个预处理步骤:对每个像素减去训练样本的平均值\citep{Krizhevsky-2012}。
% 442 

\subsubsection{对比度归一化}
\label{sec:contrast_normalization}
% 442 


在许多任务中,对比度是能够安全移除的最为明显的变化源之一。
简单地说,对比度指的是图像中亮像素和暗像素之间差异的大小。
量化图像对比度有许多方式。
在\gls{DL}中,对比度通常指的是图像或图像区域中像素的\gls{standard_deviation}。
假设我们有一个张量表示的图像$\TSX \in\SetR^{r\times c\times 3}$,其中$\TEX_{i,j,1}$表示第$i$行第$j$列红色的强度,$\TEX_{i,j,2}$对应的是绿色的强度,$\TEX_{i,j,3}$对应的是蓝色的强度。
然后整个图像的对比度可以表示如下:
\begin{align}
\label{eqn:121}
\sqrt{\frac{1}{3rc}\sum_{i=1}^{r} \sum_{j=1}^{c}\sum_{k=1}^{3} (\TEX_{i,j,k} - \bar{\TSX} )^2},
\end{align}
其中$\bar{\TSX}$是整个图片的平均强度,满足
\begin{align}
\label{eqn:122}
 \bar{\TSX} =  \frac{1}{3rc}\sum_{i=1}^{r} \sum_{j=1}^{c}\sum_{k=1}^{3}\TEX_{i,j,k}.
\end{align}


\firstall{GCN}旨在通过从每个图像中减去其平均值,然后重新缩放其使得其像素上的\gls{standard_deviation}等于某个常数$s$来防止图像具有变化的对比度。
这种方法非常复杂,因为没有缩放因子可以改变零对比度图像(所有像素都具有相等强度的图像)的对比度。
具有非常低但非零对比度的图像通常几乎没有信息内容。
在这种情况下除以真实\gls{standard_deviation}通常仅能放大传感器噪声或压缩伪像。
这种现象启发我们引入一个小的正的\gls{regularization}参数$\lambda$来平衡估计的\gls{standard_deviation}。
或者,我们至少可以约束分母使其大于等于$\epsilon$。
给定一个输入图像$\TSX$,\gls{GCN}产生输出图像$\TSX'$,定义为
\begin{align}
\label{eqn:123}
\TEX'_{i,j,k} = s\frac{\TEX_{i,j,k} - \bar{\TEX}}{\max\{ \epsilon, \sqrt{ \lambda + {\frac{1}{3rc}\sum_{i=1}^{r}\sum_{j=1}^{c}\sum_{k=1}^{3}} (\TEX_{i,j,k} - \bar{\TEX} )^2} \}}.
\end{align}
% 442 end


从大图像中剪切感兴趣的对象所组成的数据集不太可能包含任何强度几乎恒定的图像。
在这些情况下,通过设置$\lambda = 0$来忽略小分母问题是安全的,并且在非常罕见的情况下为了避免除以$0$,通过将$\epsilon$设置为一个非常小的值比如说$10^{-8}$。
这也是~\citet{Goodfellow+al-arxiv-2013}在CIFAR-10数据集上所使用的方法。
随机剪裁的小图像更可能具有几乎恒定的强度,使得激进的\gls{regularization}更有用。
在处理从CIFAR-10数据中随机选择的小区域时,\citet{Coates2011}使用$\epsilon = 0, \lambda = 10$。
% 443

尺度参数$s$通常可以设置为$1$(如~\citet{Coates2011}所采用的),或选择使所有样本上每个像素的\gls{standard_deviation}接近$1$(如\citet{Goodfellow+al-arxiv-2013}所采用的)。
% 443 mid


\eqnref{eqn:123}中的\gls{standard_deviation}仅仅是对图片$L^2$范数的重新缩放(假设图像的平均值已经被移除)。
我们更偏向于根据\gls{standard_deviation}而不是$L^2$范数来定义~\glssymbol{GCN},因为\gls{standard_deviation}包括除以像素数量这一步,从而基于\gls{standard_deviation}的~\glssymbol{GCN}~能够使用与图像大小无关的固定的$s$。
然而,观察到$L^2$范数与\gls{standard_deviation}成比例,能够帮助我们建立一种直观的感觉。
我们可以把~\glssymbol{GCN}~理解成到球壳的一种映射。
\figref{fig:gcn_sphere_color}对此有所说明。
这可能是一个有用的属性,因为\gls{NN}往往能更好地响应空间方向,而不是精确的位置。
响应相同方向上的多个距离需要具有共线权重向量但具有不同偏置的\gls{hidden_unit}。
这样的情况对于学习算法来说可能是困难的。
此外,许多浅层的图模型把多个分离的模式表示在一条线上会出现问题。
\glssymbol{GCN}~采用一个样本一个方向\footnote{译者:所有样本相似的距离}而不是不同的方向和距离来避免这些问题。
% 444 head

\begin{figure}[!htb]
\ifOpenSource
\centerline{\includegraphics{figure.pdf}}
\else
	\centerline{\includegraphics{Chapter12/figures/gcn_sphere_color}}
\fi
	\caption{\glssymbol{GCN}~将样本投影到一个球上。
\emph{(左)}原始的输入数据可能拥有任意的范数。
\emph{(中)}$\lambda=0$时候的~\glssymbol{GCN}~可以完美地将所有的非零样本投影到球上。
这里我们令$s=1$,$\epsilon = 10^{-8}$。
由于我们使用的~\glssymbol{GCN}~是基于归一化\gls{standard_deviation}而不是$L^2$范数,所得到的球并不是单位球。
\emph{(右)}$\lambda>0$的\gls{regularization}~\glssymbol{GCN}~将样本投影到球上,但是并没有完全地丢弃其范数中变化。
$s$和$\epsilon$的取值与之前一样。}
\label{fig:gcn_sphere_color}
\end{figure}
% 443 tail


% 444 head
与直觉相反的是,存在被称为~\textbf{sphering}~的预处理操作,并且它不同于~\glssymbol{GCN}。
\gls{sphering}~并不会使数据位于球形壳上,而是将主成分重新缩放以具有相等方差,使得~\glssymbol{PCA}~使用的多变量正态分布具有球形等高线。 
\gls{sphering}~通常被称为\firstgls{whitening}。
% 444 head


\gls{GCN}常常不能突出我们想要突出的图像特征,例如边缘和角。
如果我们有一个场景,包含了一个大的黑暗区域和一个大的明亮的区域(例如一个城市广场有一半的区域处于建筑物的阴影之中),
则\gls{GCN}将确保暗区域的亮度与亮区域的亮度之间存在大的差异。
然而,它不能确保暗区内的边缘突出。
% 444

这催生了\firstall{LCN} 。
\gls{LCN}确保对比度在每个小窗口上被归一化,而不是作为整体在图像上被归一化。
关于\gls{LCN}和\gls{GCN}的比较可以参考\figref{fig:122}。
% 444
% src0    gray0, gcn0?   lcn0  
%    src1?  gcn1?  gray1?     lcn1  
\begin{figure}[!htb]
\ifOpenSource
\centerline{\includegraphics{figure.pdf}}
\else
    \centering
    \begin{tabular}{ccc}
        \includegraphics[width=.3\figwidth]{Chapter12/figures/gray0.jpg} &
        \includegraphics[width=.3\figwidth]{Chapter12/figures/gcn0.jpg} &
        \includegraphics[width=.3\figwidth]{Chapter12/figures/lcn0.jpg} \\
        \includegraphics[width=.3\figwidth]{Chapter12/figures/gray1.jpg} &   % ?? may be problem
        \includegraphics[width=.3\figwidth]{Chapter12/figures/gcn1.jpg} &
        \includegraphics[width=.3\figwidth]{Chapter12/figures/lcn1.jpg}\\
        Input image & GCN & LCN
    \end{tabular}
\fi
	\caption{\gls{GCN}和\gls{LCN}的比较。直观上说,\gls{GCN}的效果很巧妙。它使得所有的图片的尺度都差不多,这减轻了学习算法处理多个尺度的负担。\gls{LCN}更多地改变了图像,丢弃了所有相同强度的区域。这使得模型能够只关注于边缘。较好的纹理区域,如第二行的屋子,可能会由于归一化核的过高带宽而丢失一些细节。}
	\label{fig:122}
\end{figure}

\gls{LCN}的各种定义都是可行的。
在所有情况下,我们可以通过减去邻近像素的平均值并除以邻近像素的\gls{standard_deviation}来修改每个像素。
在一些情况下,计算的是以当前要修改的像素为中心的矩形窗口中所有像素的平均值和\gls{standard_deviation}~\citep{Pinto08}。
在其他情况下,使用的则是以要修改的像素为中心的高斯权重的加权平均和加权\gls{standard_deviation}。
在彩色图像的情况下,一些策略单独处理不同的颜色通道,而其他策略组合来自不同通道的信息以使每个像素归一化\citep{sermanet-icpr-12}。
% 444

\gls{LCN}通常可以通过使用可分离卷积(参考\secref{sec:efficient_convolution_algorithms})来计算\gls{feature_map}的局部平均值和局部\gls{standard_deviation},然后在不同的\gls{feature_map}上使用逐元素的减法和除法来实现。
% 444

\gls{LCN}是可微分的操作,并且还可以作为一种非线性作用应用于网络隐藏层,以及应用于输入的预处理操作。
% 444

与\gls{GCN}一样,我们通常需要\gls{regularization}\gls{LCN}来避免出现除以零的情况。
事实上,因为\gls{LCN}通常作用于较小的窗口,所以\gls{regularization}更加重要。
较小的窗口更可能包含彼此几乎相同的值,因此更可能具有零\gls{standard_deviation}。
% 445


\subsection{\glsentrytext{dataset_augmentation}}
\label{sec:dataset_augmentation_chap12}
如\secref{sec:dataset_augmentation_chap7}中讲到的一样,我们很容易通过增加训练集的额外副本来增加训练集的大小,进而改进分类器的\gls{generalization}能力。
这些额外副本可以通过对原始图像进行一些变化来生成,但是并不改变其类别。
\gls{object_recognition}这个分类任务特别适合于这种形式的\gls{dataset_augmentation},因为类别信息对于许多变换是不变的,而我们可以简单地对输入应用诸多几何变换。
如前所述,分类器可以受益于随机平移或者旋转,以及某些情况下输入的翻转,以此增强数据集。
在专门的\gls{CV}应用中,存在很多更高级的用以\gls{dataset_augmentation}的变换。
这些方案包括图像中颜色的随机扰动~\citep{Krizhevsky-2012},以及对输入的非线性几何变形~\citep{LeCun98-small}。
% 445




\section{\glsentrytext{SR}}
\label{sec:speech_recognition}
% 446

\gls{SR}任务在于将一段包括了自然语言发音的声学信号映射到对应说话人意图表达的词序列上。
令$\MX=(\Vx^{(1)},\Vx^{(2)},\ldots,\Vx^{(T)})$表示语音的输入向量(传统做法以$20$ms为一帧分割信号)。
许多\gls{SR}的系统通过特殊的手工设计方法预处理输入信号,从而提取特征,但是某些\gls{DL}系统\citep{jaitly2011learning}直接从原始输入中学习特征。
令$\Vy=(y_{1},y_{2},\ldots,y_{N})$表示目标的输出序列(通常是一个词或者字符的序列)。
\firstall{ASR}任务指的是构造一个函数$f^*_{\text{ASR}}$,使得它能够在给定声学序列$\MX$的情况下计算最有可能的语言序列$\Vy$:
\begin{align}
\label{eqn:124}
f^*_{\text{ASR}}(\MX) =  \underset{\Vy}{\arg\max}  P^*(\RVy \mid \RMX = \MX),
\end{align}
其中$P^*$是给定输入值$\MX$时对应目标$\Vy$的真实条件分布。
% 446

从20世纪80年代直到约2009-2012年,最先进的\gls{SR}系统是\firstall{HMM}和\firstall{GMM}的结合。
\glssymbol{GMM}~对声学特征和\firstgls{phoneme}之间的关系建模\citep{Bahl87},\glssymbol{HMM}~对\gls{phoneme}序列建模。
\glssymbol{GMM}-\glssymbol{HMM}~模型将语音信号视作由如下过程生成:首先,一个~\glssymbol{HMM}~生成了一个\gls{phoneme}的序列以及离散的子\gls{phoneme}状态(比如每一个\gls{phoneme}的开始,中间,结尾),然后~\glssymbol{GMM}~把每一个离散的状态转化为一个简短的声音信号。
尽管直到最近~\glssymbol{GMM}-\glssymbol{HMM}~一直在~\glssymbol{ASR}~中占据主导地位,\gls{SR}仍然是\gls{NN}所成功应用的第一个领域。
从20世纪80年代末期到90年代初期,大量\gls{SR}系统使用了\gls{NN}~\citep{Bourlard-cspla89,Waibel89b,Robinson+Fallside91,Bengio91z,Bengio92c,Konig96}。
当时,基于\gls{NN}的~\glssymbol{ASR}~的表现和~\glssymbol{GMM}-\glssymbol{HMM}~系统的表现差不多。
比如说,\citet{Robinson+Fallside91}在TIMIT数据集\citep{garofolo1993darpa}(有$39$个区分的\gls{phoneme}% ?? error
)上达到了$26$\%的\gls{phoneme}错误率,这个结果优于或者说是可以与基于~\glssymbol{HMM}~的结果相比。
从那时起,TIMIT成为了\gls{phoneme}识别的一个基准数据集,在\gls{SR}中的作用就和MNIST在\gls{object_recognition}中的作用差不多。
然而,由于\gls{SR}软件系统中复杂的工程因素以及在基于~\glssymbol{GMM}-\glssymbol{HMM}~的系统中已经付出的巨大努力,工业界并没有看到迫切转向\gls{NN}的需求。
结果,直到21世纪00年代末期,学术界和工业界的研究者们更多的是用\gls{NN}为~\glssymbol{GMM}-\glssymbol{HMM}~系统学习一些额外的特征。
% 447


之后,随着\emph{更大更深}的模型以及更大的数据集的出现,通过使用\gls{NN}代替~\glssymbol{GMM}~来实现将声学特征转化为\gls{phoneme}(或者子\gls{phoneme}状态)的过程可以大大地提高识别的精度。
从2009年开始,\gls{SR}的研究者们将一种\gls{unsupervised_learning}的\gls{DL}方法应用于\gls{SR}。
这种\gls{DL}方法基于训练一个被称作是\gls{RBM}的无向概率模型,从而对输入数据建模。
 \gls{RBM}将会在第三部分中描述。
 为了完成\gls{SR}任务,\gls{unsupervised}的\gls{pretraining}被用来构造一个\gls{deep_feedforward_network},这个\gls{NN}每一层都是通过训练\gls{RBM}来初始化的。
 这些网络的输入是从一个固定规格的输入窗(以当前帧为中心)的谱声学表示抽取,预测了当前帧所对应的~\glssymbol{HMM}~状态的条件概率。
 训练一个这样的\gls{NN}能够可以显著提高在TIMIT数据集上的识别率~\citep{mohamed2009deep,Mohamed+Dahl+Hinton-2012},并将\gls{phoneme}级别的错误率从大约$26$\%降到了$20.7$\%。
关于这个模型成功原因的详细分析可以参考\citet{mohamed2012understanding}。
对于基本的电话识别工作流程的一个扩展工作是添加说话人自适应相关特征\citep{mohamed2011deep}的方法,这可以进一步地降低错误率。
紧接着的工作则将结构从\gls{phoneme}识别(TIMIT所主要关注的)转向了大规模词汇语音识别\citep{Dahl2012},这不仅包含了识别\gls{phoneme},还包括了识别大规模词汇的序列。
\gls{SR}上的深度网络从最初的使用\gls{RBM}进行\gls{pretraining}发展到了使用诸如\gls{ReLU}和~\gls{dropout}~这样的技术~\citep{Zeiler+al-ICASSP-2013,Dahl-et-al-ICASSP2013}。
从那时开始,工业界的几个语音研究组开始寻求与学术圈的研究者之间的合作。
\citet{Hinton-et-al-2012}描述了这些合作所带来的突破性进展,这些技术现在被广泛应用在产品中,比如移动手机端。
% 447

随后,当研究组使用了越来越大的带标签的数据集,加入了各种初始化,训练方法以及调试\gls{DNN}的结构之后,
他们发现这种\gls{unsupervised}的\gls{pretraining}方式是没有必要的,或者说不能带来任何显著的改进。
% 447

用\gls{SR}中词错误率来衡量,在\gls{SR}性能上的这些突破是史无前例的(大约$30$\%的提高)。
在这之前的长达十年左右的时间内,尽管数据集的规模是随时间增长的(见~\citet{Deng+Yu-2014}的图2.4),但基于~\glssymbol{GMM}-\glssymbol{HMM}~的系统的传统技术已经停滞不前了。
这也导致了\gls{SR}领域快速地转向\gls{DL}的研究。
在大约的两年时间内,工业界的大多数的\gls{SR}产品都包含了\gls{DNN},这种成功也激发了~\glssymbol{ASR}领~域对\gls{DL}算法和结构的一波新的研究浪潮,并且影响至今。
% 448

其中的一个创新点是\gls{convolutional_network}的应用~\citep{Sainath-et-al-ICASSP2013}。
\gls{convolutional_network}在时域与频域上复用了权重,改进了之前的仅在时域上复用权重的\gls{TDNNs}。
这种新的二维的卷积模型并不是将输入的频谱当作一个长的向量,而是当成是一个图像,其中一个轴对应着时间,另一个轴对应的是谱分量的频率。
% 448

完全抛弃~\glssymbol{HMM}~并转向研究\gls{end_to_end}\gls{DL}\gls{SR}系统是至今仍然活跃的另一个重要推动。
这个领域第一个主要的突破是~\citet{Graves-et-al-ICASSP2013},其中训练了一个深度的\gls{LSTM}\gls{RNN}(见\secref{sec:the_long_short_term_memory_and_other_gated_rnns}),使用了帧到\gls{phoneme}对其的~\glssymbol{MAP}~推断,就像~\citet{LeCun98-small}以及CTC框架~\citep{Graves-et-al-2006,Graves-book2012}中一样。
一个深度\gls{RNN}~\citep{Graves-et-al-ICASSP2013}每个\gls{time_step}的各层都有状态变量,使得它的\gls{unfolded_graph}具有两种不同深度:一种是普通的根据层的堆叠衡量的深度,另一种根据时间\gls{unfolding}衡量的深度。
这个工作把TIMIT数据集上\gls{phoneme}的错误率记录降到了的新低$17.7$\%。
关于应用于其他领域的深度\gls{RNN}的变种可以参考~\citet{Pascanu-et-al-ICLR2014,Chung-et-al-NIPSDL2014-small}。
% 448

另一个\gls{end_to_end}\gls{DL}\gls{SR}方向的最新进展是让系统学习如何将\firstgls{phonetic}层级的信息与\firstgls{acoustic}层级的信息``对齐''~\citep{Chorowski-et-al-arxiv2014,llu_is2015b}。
% 448


% Translator: Shenjian Zhao
\section{\glsentrytext{NLP}}
\label{sec: natural_language_processing}

\firstgls{NLP}让计算机能够使用人类语言,例如英语或法语。
为了让简单的程序能够高效明确地解析,计算机程序通常读取和发出特殊化的语言。
而自然的语言通常是模糊的,并且可能不遵循严格的形式描述。
\gls{NLP}中的应用如机器翻译,学习者需要读取一种人类语言的句子,并用另一种人类语言发出相同意思的句子。
许多~\glssymbol{NLP}~应用程序基于\gls{language_model},\gls{language_model}定义了关于自然语言中的字、字符或字节序列的概率分布。

% -- 448 --

与本章讨论的其他应用一样,非常通用的\gls{NN}技术可以成功地应用于\gls{NLP}。
然而,为了实现卓越的性能并扩展到大型应用程序,一些领域特定的策略也很重要。
为了构建自然语言的有效模型,通常必须使用专门处理序列数据的技术。
在很多情况下,我们将自然语言视为一系列词,而不是单个字符或字节序列。
因为可能的词总数非常大,基于词的\gls{language_model}必须在极高维度和稀疏的离散空间上操作。
为使这种空间上的模型在计算和统计意义上都高效,研究者已经开发了几种策略。

\subsection{\glsentrytext{n_gram}}
\label{sec:n_grams}

\firstgls{language_model}定义了自然语言中\gls{token}序列的概率分布。
根据模型的设计,\gls{token}可以是词、字符、甚至是字节。
\gls{token}总是离散的实体。
最早成功的\gls{language_model}基于固定长度序列的\gls{token}模型,称为~\gls{n_gram}。
一个~\gls{n_gram}~是一个包含$n$个\gls{token}的序列。


基于~\gls{n_gram}~的模型定义一个条件概率——给定前$n-1$个\gls{token}后,第$n$个\gls{token}出现的条件概率。
该模型使用这些条件分布的乘积定义更长(长度大于$n$)序列的概率分布:
\begin{align}
P(x_1, \dots, x_\tau) = P(x_1, \dots, x_{n-1}) \prod_{t=n}^\tau P(x_t \mid x_{t-n+1}, \dots, x_{t-1} ).
\end{align}
这个分解可以由概率的链式法则证明。
初始序列 $P(x_1, \dots, x_{n-1})$的概率分布可以通过带有较小$n$值的不同模型建模。

训练~\gls{n_gram}~模型是简单的,因为\gls{maximum_likelihood_estimation}可以通过简单地统计每个可能的~\gls{n_gram}~在训练集中出现的次数来获得。                                   
几十年来,基于~\gls{n_gram}~的模型都是统计\gls{language_model}的核心模块~\citep{Jelinek+Mercer80,Katz87,Chen+Goodman99}。

对于小的$n$值,模型有特定的名称:$n=1$称为\firstgls{unigram},$n=2$称为\firstgls{bigram}及$n=3$称为\firstgls{trigram}。
这些名称源于相应数字的拉丁前缀和希腊后缀``-gram'',gram的意思是所写之物。

% -- 449 --

通常我们同时训练~\gls{n_gram}~模型和$n-1$ gram模型。 
这使得下式可以简单地通过查找两个存储的概率来计算。
\begin{align}
\label{eq:ml-ngram}
P(x_t \mid x_{t-n+1}, \dots, x_{t-1}) = \frac{P_n(x_{t-n+1}, \dots, x_t)} { P_{n-1}( x_{t-n+1}, \dots, x_{t-1}) }
\end{align}
为了在$P_n$中精确地再现\gls{inference},我们训练$P_{n-1}$时必须省略每个序列最后一个字符。

举个例子,我们演示三元模型如何计算句子``{\tt THE DOG RAN AWAY}.''的概率。
句子的第一个词不能通过上述条件概率的公式计算,因为句子的开头没有上下文。
取而代之,在句子的开头我们必须使用词的边缘概率。
因此我们计算$P_3({\tt THE\ DOG\ RAN})$。
最后,可以使用条件分布$P({\tt AWAY} \mid {\tt DOG\ RAN})$(典型情况)来预测最后一个词。
将这与\eqnref{eq:ml-ngram}放在一起,我们得到:
\begin{align}
P({\tt THE\ DOG\ RAN\ AWAY}) = P_3({\tt THE\ DOG\ RAN}) P_3({\tt DOG\ RAN\ AWAY}) / P_2({\tt DOG\ RAN}).
\end{align}

\gls{n_gram}~模型最大似然的基本限制是,在许多情况下从训练集计数估计得到的$P_n$很可能为零(即使元组$(x_{t-n+1},  \dots, x_{t})$可能出现在测试集中)。
这可能会导致两种不同的灾难性后果。
当$P_{n-1}$为零时,该比率是未定义的,因此模型甚至不能产生有意义的输出。
当$P_{n-1}$非零而$P_n$为零时,测试样本的对数似然为 $-\infty$。
为避免这种灾难性的后果,大多数~\gls{n_gram}~模型采用某种形式的\firstgls{smoothing}。
\gls{smoothing}技术将概率质量从观察到的元组转移到类似的未观察到的元组。
见~\citet{Chen+Goodman99}的综述和实验对比。
其中一种基本技术基于向所有可能的下一个符号值添加非零概率质量。
这个方法可以被看作带有一个计数参数符合均匀或~\ENNAME{Dirichlet}~分布的先验的贝叶斯\gls{inference}。
另一个非常流行的想法是包含高阶和低阶~\gls{n_gram}~模型的混合模型,其中高阶模型提供更多的\gls{capacity},而低阶模型尽可能地避免零计数。
如果上下文$x_{t-n+k}, \ldots, x_{t-1}$的频率太小而不能使用高阶模型,\textbf{回退方法}(back-off methods)就查找低阶~\gls{n_gram} 。
更正式地说,它们通过上下文$x_{t-n+k}, \ldots, x_{t-1}$估计$x_t$上的分布,并增加$k$直到找到足够可靠的估计。

% -- \450 --

经典的~\gls{n_gram}~模型特别容易引起\gls{curse_of_dimensionality}。
因为存在$|\SetV|^n$可能的~\gls{n_gram},而且 $|\SetV|$ 通常很大。
即使有大量训练数据和适当的$n$,大多数~\gls{n_gram}~也不会出现在训练集中。
看待经典~\gls{n_gram}~模型的一种观点是把它看作执行最近邻查询。
换句话说,它可以被视为局部\gls{nonparametric}预测器,类似于$k$-最近邻。
这些极端局部预测器面临的统计问题已经在\secref{sec:local_constancy_and_smoothness_regularization}中描述过。
\gls{language_model}的问题甚至比普通模型更严重,因为任何两个不同的词在\gls{one_hot}向量空间中的距离彼此相同。
因此,难以大量利用来自任何``邻居''的信息 —— 只有重复相同上下文的训练样本对局部泛化有用。
为了克服这些问题,\gls{language_model}必须能够在一个词和其他语义相似的词之间共享知识。

为了提高~\gls{n_gram}~模型的统计效率,\textbf{基于类的语言模型}(class-based language model)~\citep{Brown92,Ney+Kneser93,Niesler98}引入词类别的概念,然后属于同一类别的词共享词之间的统计强度。
这个想法使用了聚类算法,基于它们与其他词同时出现的频率,将该组词分成集群或类。
随后,模型可以在条件竖杠的右侧使用词类ID而不是单个词ID。
使用混合或回退(fallback)策略将词模型和类模型组成复合模型也是可能的。
尽管词类提供了在序列之间泛化的方式,但其中一些词被相同类的另一个替换,导致该\gls{representation}丢失了很多信息。

\subsection{\glsentrytext{NLM}}
\label{sec:neural_language_models}

\firstall{NLM}是一类用来克服\gls{curse_of_dimensionality}的\gls{language_model},它使用词的\gls{distributed_representation}对自然语言序列建模~\citep{BenDucVin01-small}。
不同于基于类的~\gls{n_gram}~模型,\gls{NLM}在能够识别两个相似的词,并且不丧失将每个词编码为彼此不同的能力。
\gls{NLM}共享一个词(及其上下文)和其他类似词(和上下文之间)的统计强度。
模型为每个词学习的\gls{distributed_representation},允许模型处理具有类似共同特征的词来实现这种共享。
例如,如果词{\tt dog}和词{\tt cat}映射到具有许多共同属性的表示,则包含词{\tt cat}的句子可以影响模型对包含词{\tt dog}的句子做出的预测,反之亦然。
因为这样的属性很多,所以存在许多泛化的方式,可以将信息从每个训练语句传递到指数数量的语义相关语句。
\gls{curse_of_dimensionality}需要模型泛化到指数多的句子(指数相对句子长度而言)。
该模型通过将每个训练句子与指数数量的类似句子相关联克服这个问题。

% -- 451 --

我们有时将这些词\gls{representation}称为\firstgls{word_embedding}。
在这个解释下,我们将原始符号视为维度等于词表大小的空间中的点。
词\gls{representation}将这些点嵌入到较低维的特征空间中。
在原始空间中,每个词由一个\gls{one_hot}向量表示,因此每对词彼此之间的欧氏距离都是$\sqrt{2}$。
在嵌入空间中,经常出现在类似上下文(或共享由模型学习的一些``特征''的任何词对)中的词彼此接近。
这通常导致具有相似含义的词变得邻近。
\figref{fig:chap12_word_embeddings_color}放大了学到的\gls{word_embedding}空间的特定区域,我们可以看到语义上相似的词如何映射到彼此接近的表示。

\begin{figure}[htp]
\centering
\ifOpenSource
\centerline{\includegraphics{figure.pdf}}
\else
\includegraphics{Chapter12/figures/word_embeddings_color.pdf}
\fi
\caption{从\gls{NMT}模型获得的\gls{word_embedding}的二维可视化\citep{Bahdanau-et-al-ICLR2015-small}。
此图在语义相关词的特定区域放大,它们具有彼此接近的嵌入向量。
国家在左图,数字在右图。
注意,这些嵌入是为了可视化才表示为2维。
在实际应用中,嵌入通常具有更高的维度并且可以同时捕获词之间多种相似性。
}
\label{fig:chap12_word_embeddings_color}
\end{figure}

其他领域的\gls{NN}也可以定义嵌入。
例如,\gls{convolutional_network}的隐藏层提供``图像嵌入''。
因为自然语言最初不在实值向量空间上,所以~\glssymbol{NLP}~从业者通常对嵌入的这个想法更感兴趣。
隐藏层在表示数据的方式上提供了更显著的质上的变化。

% -- 452 --

使用\gls{distributed_representation}来改进\gls{NLP}模型的基本思想不必局限于\gls{NN}。
它还可以用于\gls{graphical_model},其中\gls{distributed_representation}是多个\gls{latent_variable}的形式\citep{Mnih+Hinton-2007}。

\subsection{高维输出}
\label{sec:high_dimensional_outputs}

在许多自然语言应用中,我们通常希望我们的模型产生词(而不是字符)作为输出的基本单位。
对于大词汇表,由于词汇量很大,以在所有可能的词上进行选择的形式表示输出分布的计算成本可能非常高。
在许多应用中,$\SetV$包含数十万词。
表示这种分布的朴素方法是应用一个仿射变换,将隐藏表示转换到输出空间,然后应用~\ENNAME{softmax}~函数。
假设我们的词汇表$\SetV$大小为$| \SetV |$。
因为其输出维数为$| \SetV |$,描述该仿射变换线性分量的权重矩阵非常大。
这造成了表示该矩阵的高存储成本,以及与之相乘的高计算成本。
因为~\ENNAME{softmax}~要在所有$| \SetV |$输出之间归一化,所以在训练时以及测试时执行全矩阵乘法是必要的 ——我们不能仅计算与正确输出的权重向量的点积。
因此,输出层的高计算成本在训练期间(计算似然性及其梯度)和测试期间(计算所有或所选词的概率)都有出现。
对于专门的\gls{loss_function},可以有效地计算梯度 \citep{Vincent2015},但是应用于传统~\ENNAME{softmax}~输出层的标准\gls{cross_entropy}损失时会出现许多困难。

假设$\Vh$是用于预测输出概率$\hat \Vy$的顶部隐藏层。
如果我们使用学到的权重$\MW$和学到的\gls{bias_aff} $\Vb$参数化从$\Vh$到$\hat \Vy$的变换,则仿射~\ENNAME{softmax}~输出层执行以下计算:
\begin{align}
\label{eq:softmax-over-words}
  a_i &= b_i + \sum_j  W_{ij} h_j \;\;\; \forall i \in \{1,\ldots,|\SetV|\}, \\
  \hat{y}_i &= \frac{e^{a_i}}{\sum_{i'=1}^{|\SetV|} e^{a_{i'}}}.
\end{align}
如果$\Vh$包含$n_h$个元素,则上述操作复杂度是 $O(|\SetV| n_h)$。
在$n_h$为数千和$| \SetV |$数十万的情况下,这个操作占据了\gls{NLM}的计算开销的一大部分。

% -- 453 --

\subsubsection{使用\glsentrytext{shortlist}}
第一个\gls{NLM}~\citep{BenDucVin01-small,Bengio-nnlm2003-small}通过将词汇量限制为10,000或20,000来减轻大词汇表上~\ENNAME{softmax}~的高成本。
\citet{Schwenk+Gauvain2002}和 \citet{Schwenk-2007}在这种方法的基础上建立新的方式,将词汇表$\SetV$分为最常见词汇(由\gls{NN}处理)的\firstgls{shortlist}~$\SetL$和较稀有词汇的尾列表$\SetT = \SetV \backslash \SetL$(由\gls{n_gram}模型处理)。
为了组合这两个预测,\gls{NN}还必须预测在上下文$C$之后出现的词位于尾列表的概率。
我们可以添加额外的~\ENNAME{sigmoid}~输出单元估计 $P(i \in \SetT \mid C)$实现这个预测。
用上这个额外输出,我们就可以估计$\SetV$中所有词的概率分布,如下:
\begin{align}
 P(y=i\mid C)  =& 1_{i \in \SetL} P(y=i\mid C, i \in \SetL) (1 - P(i \in \SetT\mid C)) \nonumber \\
     & + 1_{i \in \SetT} P(y=i\mid C, i \in \SetT) P(i \in \SetT\mid C),
\end{align}
其中$P(y=i\mid C, i \in \SetL)$由\gls{NLM}提供$P(y=i\mid C, i \in \SetT)$由~\gls{n_gram}~模型提供。
稍作修改,这种方法也可以在\gls{NLM}的~\ENNAME{softmax}~层中使用额外的输出值,而不是单独的~\ENNAME{sigmoid}~单元。

\gls{shortlist}方法的一个明显缺点是,\gls{NLM}的潜在泛化优势仅限于最常用的词,而在这部分词上,\gls{NLM}大概相对来说是用处最小的。
这个缺点引发了处理高维输出替代方法的探索,如下所述。

\subsubsection{分层Softmax}
减少大词汇表$\SetV$上高维输出层计算负担的经典方法~\citep{Goodman2001}是分层地分解概率。
$|\SetV|$因子可以降低到$\log |\SetV|$一样低,而无需执行与$|\SetV|$成比例数量(并且也与隐藏单元数量$n_h$成比例)的计算。
\citet{BengioTR1215}和~\citet{Morin+Bengio-2005-small} 将这种因子分解方法引入\gls{NLM}中。

% -- 454 --

我们可以认为这种层次结构是先建立词的类别,然后是词类别的类别,然后是词类别的类别的类别等等。
这些嵌套类别构成一棵树,其叶子为词。
在平衡树中,树的深度为$\log |\SetV|$。
选择一个词的概率是由路径(从树根到包含该词叶子的路径)上的每个节点通向该词分支概率的乘积给出。
\figref{fig:chap12_word_hierarchy}是一个简单的例子。
\citet{Mnih+Hinton-2009}也描述了使用多个路径来识别单个词的方法,以便更好地建模具有多个含义的词。
此时计算词的概率则涉及在导向该词的所有路径上进行求和。
\begin{figure}[htp]
\centering
\ifOpenSource
\centerline{\includegraphics{figure.pdf}}
\else
\includegraphics{Chapter12/figures/word_hierarchy.pdf}
\fi
% Magic incantation to allow math in caption
\captionsetup{singlelinecheck=off}
% Part of the incantation is that we must define the [] argument
\caption[.]{词类别简单层次结构的示意图,其中8个词$w_0,\dots,w_7$组织成三级层次结构。
树的叶子表示实际特定的词。
内部节点表示词的组别。
任何节点都可以通过二值决策序列($0$=左,$1$=右)索引,从根到达节点。
超类$(0)$包含类$(0,0)$和$(0,1)$,其中分别包含词$\{w_0,w_1\}$和$\{w_2,w_3\}$的集合,类似地超类$(1)$包含类$(1,0)$和$(1,1)$,分别包含词$\{w_4,w_5\}$和$\{w_6,w_7\}$。
如果树充分平衡,则最大深度(二值决策的数量)与词数$| \SetV |$的对数同阶:从$| \SetV |$个词中选一个词只需执行$\CalO(\log |\SetV|)$次操作(从根开始的路径上的每个节点一次操作)。
在该示例中,我们乘三次概率就能计算词$y$的概率,这三次概率与从根到节点$y$的路径上每个节点向左或向右的二值决策相关联。
令$b_i(y)$为遍历树移向$y$时的第$i$个二值决策。
对输出$\RSy$进行采样的概率可以通过条件概率的链式法则分解为条件概率的乘积,其中每个节点由这些位的前缀索引。
例如,节点$(1,0)$对应于前缀$(b_0(w_4)=1, b_1(w_4)=0)$,并且$w_4$的概率可以如下分解:
\begin{align}
  P(\RSy = w_4) &= P(\RSb_0 = 1, \RSb_1 = 0, \RSb_2 = 0) \\
    &= P(\RSb_0 = 1) P(\RSb_1 = 0 \mid \RSb_0 = 1) P(\RSb_2 = 0 \mid \RSb_0 = 1, \RSb_1 = 0). 
\end{align}
}
\label{fig:chap12_word_hierarchy}
\end{figure}

为了预测树的每个节点所需的条件概率,我们通常在树的每个节点处使用\gls{logistic_regression}模型,并且为所有这些模型提供与输入相同的上下文$C$。
因为正确的输出编码在训练集中,我们可以使用\gls{supervised_learning}训练\gls{logistic_regression}模型。
我们通常使用标准\gls{cross_entropy}损失,对应于最大化正确判断序列的对数似然。

因为可以高效地计算输出对数似然(低至$\log |\SetV|$而不是$ |\SetV|$),所以也可以高效地计算梯度。
这不仅包括关于输出参数的梯度,而且还包括关于隐藏层激活的梯度。

优化树结构从而最小化期望的计算量是可能的,但通常不实际。
给定词的相对频率,信息理论的工具可以指定如何选择最佳的二进制编码。
为此,我们可以构造树,使得与词相关联的位数量近似等于该词频率的对数。
然而在实践中,节省计算通常事倍功半,因为输出概率的计算仅是\gls{NLM}中总计算的一部分。
例如,假设有$l$个全连接的宽度为$n_h$的隐藏层。
令$n_b$是识别一个词所需比特数的加权平均值,其加权由这些词的频率给出。
在这个例子中,计算隐藏激活所需的操作数增长为$O(ln_h^2)$,而输出计算增长为$O(n_h n_b)$。
只要$ n_b \leq l n_h$,相比于收缩$n_b$,我们可以通过收缩$n_h$减少更多的计算量。
事实上,$n_b$通常很小。
因为词汇表的大小很少超过一百万而$\log_ 2(10^6) \approx 20$,所以可以将$n_b$减小到大约20,但$n_h$通常大得多,大约为$10^3$或更大。
与其仔细优化分支因子为$2$的树,我们不如定义深度为2和分支因子为$\sqrt{|\SetT|}$的树。
这样的树对应于简单定义一组互斥的词类。
基于深度为$2$的树的简单方法可以获得层级策略大部分的计算益处。

% -- 455 --

一个仍然有点开放的问题是如何最好地定义这些词类,或者说如何定义一般的词层次结构。
早期工作使用现有的层次结构\citep{Morin+Bengio-2005-small} ,但理想情况下也可以与\gls{NLM}联合学习层次结构。
学习层次结构很困难。
对数似然的精确优化似乎难以解决,因为词层次的选择是离散的,不适于基于梯度的优化。
然而,我们可以使用离散优化来近似地最优化词类的分割。

分层~\ENNAME{softmax}~的一个重要优点是,它在训练期间和测试期间(如果在测试时我们想计算特定词的概率)都带来了计算上的好处。

当然即使使用分层~\ENNAME{softmax},计算所有$|\SetV|$个词概率的成本仍是很高的。
另一个重要的操作是在给定上下文中选择最可能的词。
不幸的是,树结构不能为这个问题提供高效精确的解决方案。

分层~\ENNAME{softmax}~的一个缺点是在实践中倾向于表现得比基于采样的方法差,基于采样的方法我们将在下文描述。
这可能是因为词类选择得不好。

\subsubsection{\glsentrytext{importance_sampling}}
\label{sec:importance_sampling_chap12}
加速\gls{NLM}训练的一种方式是,避免明确地计算所有未出现在下一位置的词对梯度的贡献。
每个不正确的词在此模型下具有低概率。
枚举所有这些词的计算成本可能会很高。
相反,我们可以仅采样词的子集。
使用\eqnref{eq:softmax-over-words}中引入的符号,梯度可以写成如下形式:
\begin{align}
 \frac{\partial \log P(y \mid C)}{\partial \theta} &= \frac{\partial \log {\rm softmax}_y(\Va)}{\partial \theta} \\ 
  &= \frac{\partial}{\partial \theta} \log \frac{e^{a_y}}{\sum_i e^{a_i}} \\ 
 &= \frac{\partial}{\partial \theta} (a_y - \log \sum_i e^{a_i}) \\ 
 &= \frac{\partial a_y}{\partial \theta}  - \sum_i P(y = i \mid C) \frac{\partial a_i}{\partial \theta},
\end{align}
其中$\Va$是~\ENNAME{presoftmax}~激活(或得分)向量,每个词对应一个元素。
第一项是\textbf{正相}(positive phase)项,推动$a_y$向上;而第二项是\textbf{负相}(negative phase)项,对于所有$i$以权重$P(i \mid C)$推动$a_i$向下。
由于负相项是期望值,我们可以通过\gls{monte_carlo}采样估计。
然而,这将需要从模型本身采样。
从模型中采样需要对词汇表中所有的$i$计算$P(i \mid C)$,这正是我们试图避免的。

我们可以从另一个分布中采样,而不是从模型中采样,这个分布称为\firstgls{proposal_distribution}(记为$q$),并通过适当的权重校正从错误分布采样引入的\gls{bias_sta} \citep{Bengio+Senecal-2003-small,Bengio+Senecal-2008}。
这是一种称为\firstgls{importance_sampling}的更通用技术的应用,我们将在\secref{sec:importance_sampling_chap12}中更详细地描述。
不幸的是,即使精确\gls{importance_sampling}也不一定有效,因为我们需要计算权重$p_i / q_i$,其中的$p_i = P(i \mid C)$只能在计算所有得分$a_i$后才能计算。
这个应用采取的解决方案称为\gls{biased_importance_sampling},其中重要性权重被归一化至加和为1。
当对负词$n_i$进行采样时,相关联的梯度被加权为:
\begin{align}
  w_i = \frac{p_{n_i} / q_{n_i}}{\sum_{j=1}^N p_{n_j} / q_{n_j}}.
\end{align}
这些权重用于对来自$q$的$m$个负样本给出适当的重要性,以估计负相项对梯度的贡献:
\begin{align}
  \sum_{i=1}^{|\SetV|} P(i \mid C) \frac{\partial a_i}{\partial \theta}  \approx \frac{1}{m} \sum_{i=1}^m w_i \frac{\partial a_{n_i}}{\partial \theta}.
  \end{align}
  \gls{unigram}或\gls{bigram}分布可以很好地作为\gls{proposal_distribution} $q$。
从数据估计这种分布的参数是很容易的。
在估计参数之后,也可以非常高效地从这样的分布采样。

\firstgls{importance_sampling}不仅可以加速具有较大~\ENNAME{softmax}~输出的模型。
更一般地,它可以加速具有大稀疏输出层的训练,其中输出是稀疏向量而不是$n$选$1$。
其中一个例子是\firstgls{bag_of_words}。
\gls{bag_of_words}具有稀疏向量$\Vv$,其中$v_i$表示词汇表中的词$i$存不存在文档中。
或者,$v_i$可以指示词$i$出现的次数。
由于各种原因,训练产生这种稀疏向量的\gls{ML}模型的成本可能很高。
在学习的早期,模型可能不会真的使输出真正稀疏。
此外,将输出的每个元素与目标的每个元素进行比较,可能是描述训练的\gls{loss_function}最自然的方式。
这意味着稀疏输出并不一定能带来计算上的好处,因为模型可能会选择使大多数输出非零,而所有这些非零值都需要与相应的训练目标进行比较( 即使训练目标是零)。
\citet{Dauphin2011-small} 证明可以使用\gls{importance_sampling}加速这种模型。
高效算法最小化``正词''(在目标中非零的那些词)和相等数量的``负词''的重构损失。
负词是被随机选取的,如使用启发式采样更可能被误判的词。
该启发式过采样引入的偏差则可以使用重要性权重校正。

% -- 458 --

在所有这些情况下,输出层梯度估计的计算复杂度被减少为与负样本数量成比例,而不是与输出向量的大小成比例。


\subsubsection{噪声对比估计和排名损失}


\label{sec:combining_neural_language_models_with_n_grams}
为减少训练大词汇表的\gls{NLM}的计算成本,研究者也提出了其他基于采样的方法。
早期的例子是 \citet{Collobert+Weston-ICML2008}提出的排名损失,将\gls{NLM}每个词的输出视为一个得分,并试图使正确词的得分$a_y$比其他词的得分$a_i$排名更高。提出的排名损失则是
\begin{align} 
 L = \sum_i \max(0,1-a_y+a_i).
\end{align} 
如果观察到词的得分$a_y$远超过负词的得分$a_i$(相差大于1),则第$i$项梯度为零。
这个\gls{criterion}的一个问题是它不提供估计的条件概率,条件概率在很多应用中是有用的,包括语音识别和文本生成(包括诸如翻译的条件文本生成任务)。

最近用于\gls{NLM}的训练目标是噪声对比估计,将在\secref{sec:noise_contrastive_estimation}中介绍。
这种方法已成功应用于\gls{NLM}~\citep{Mnih+Teh-ICML2012,Mnih2013}。

% -- 459 --

\subsection{结合\glsentrytext{n_gram}和\glsentrytext{NLM}}
\gls{n_gram}~模型相对\gls{NN}的主要优点是~\gls{n_gram}~模型具有更高的模型\gls{capacity}(通过存储非常多的元组的频率),并且处理样本只需非常少的计算量(通过查找只匹配当前上下文的几个元组)。
如果我们使用哈希表或树来访问计数,那么用于~\gls{n_gram}~的计算量几乎与\gls{capacity}无关。
相比之下,将\gls{NN}的参数数目加倍通常也大致加倍计算时间。
当然,避免每次计算时使用所有参数的模型是一个例外。
嵌入层每次只索引单个嵌入,所以我们可以增加词汇量,而不会增加每个样本的计算时间。
一些其他模型,例如平铺\gls{convolutional_network},可以在减少\gls{parameter_sharing}程度的同时添加参数以保持相同的计算量。然而,基于矩阵乘法的典型\gls{NN}层需要与参数数量成比例的计算量。

因此,增加\gls{capacity}的一种简单方法是将两种方法结合,由\gls{NLM}和~\gls{n_gram}~\gls{language_model}组成\gls{ensemble}~\citep{BenDucVin01-small,Bengio-nnlm2003-small}。

对于任何\gls{ensemble},如果\gls{ensemble}成员产生的错误相互独立,这种技术可以减少测试误差。
\gls{ensemble}学习领域提供了许多方法来组合\gls{ensemble}成员的预测,包括平均权重的加权和针对验证集选择权重。
\citet{Mikolov-Interspeech-2011} 扩展了\gls{ensemble},不是仅包括两个模型,而是包括大量模型。
我们也可以将\gls{NN}与最大熵模型配对并联合训练\citep{Mikolov-ASRU-2011}。
该方法可以被视为训练具有一组额外输入的\gls{NN},额外输入直接连接到输出并且不连接到模型的任何其他部分。
额外输入是输入上下文中特定~\gls{n_gram}~是否存在的指示器,因此这些变量是非常高维且非常稀疏的。

模型\gls{capacity}的增加是巨大的 (架构的新部分包含高达$| sV |^n$个参数 ),但是处理输入所需的额外计算量是很小的(因为额外输入非常稀疏)。

\subsection{\glsentrytext{NMT}}
\label{sec:neural_machine_translation}

机器翻译以一种自然语言读取句子并产生等同含义的另一种语言的句子。
机器翻译系统通常涉及许多组件。
在高层次,通常有一个组件负责提出许多候选翻译。
由于语言之间的差异,这些翻译中的许多翻译是不符合语法的。
例如,许多语言在名词后放置形容词,因此直接翻译成英语时,它们会产生诸如``apple red''的短语。
提议机制提出建议翻译的许多变体,理想情况下应包括``red apple''。
翻译系统的第二个组件(\gls{language_model})评估提议的翻译,并可以评估``red apple''比``apple red''更好。

% -- 460 --

最早的机器翻译\gls{NN}探索中已经纳入了\gls{encoder}和\gls{decoder}的想法(Allen 1987; Chrisman 1991; Forcada
and Ñeco 1997),而翻译中\gls{NN}的第一个大规模有竞争力的用途是通过\gls{NLM}升级翻译系统的\gls{language_model}~\citep{Schwenk-et-al-IWSLT2006,Schwenk-2010}。
之前,大多数机器翻译系统在该组件使用~\gls{n_gram}~模型。
机器翻译中基于~\gls{n_gram}~的模型不仅包括传统的回退~\gls{n_gram}~模型~\citep{Jelinek+Mercer80,Katz87,Chen+Goodman99},而且包括\textbf{最大熵语言模型}(maximum entropy language models)~\citep{MaxEnt96},其中给定上下文中常见的词,affine-softmax层预测下一个词。

传统\gls{language_model}仅仅报告自然语言句子的概率。
因为机器翻译涉及给定输入句子产生输出句子,所以将自然\gls{language_model}扩展为条件的是有意义的。
如\secref{sec:learning_conditional_distributions_with_maximum_likelihood}所述可以直接地扩展一个模型,该模型定义某些变量的边缘分布,以便在给定上下文$C$($C$可以是单个变量或变量列表)的情况下定义该变量的条件分布。
\citet{Devlin-et-al-ACL2014}在一些统计机器翻译的基准中击败了最先进的技术,他给定源语言中的短语$\RSs_1, \RSs_2,\ldots, \RSs_k$后使用~\glssymbol{MLP}~对目标语言的短语$\RSt_1, \RSt_2,\ldots, \RSt_k$进行评分。
这个~\glssymbol{MLP}~估计$P(\RSt_1, \RSt_2,\ldots, \RSt_k \mid \RSs_1, \RSs_2,\ldots, \RSs_k)$。
这个~\glssymbol{MLP}~的估计替代了条件~\gls{n_gram}~模型提供的估计。

基于~\glssymbol{MLP}~方法的缺点是需要将序列预处理为固定长度。
为了使翻译更加灵活,我们希望模型允许可变的输入长度和输出长度。
\glssymbol{RNN}~具备这种能力。
\secref{sec:modeling_sequences_conditioned_on_context_with_rnns}描述了给定某些输入后,表示(输出)序列上的条件分布的~\glssymbol{RNN}~的几种构造方法,并且\secref{sec:encoder_decoder_sequence_to_sequence_architectures}描述了当输入是序列时如何实现这种条件分布。
在所有情况下,一个模型首先读取输入序列并产生概括输入序列的数据结构。我们称这个概括为``上下文''~$C$。
上下文$C$可以是向量列表,也可以是向量或张量。
读取输入以产生$C$的模型可以是~\glssymbol{RNN}~\citep{Cho-et-al-EMNLP2014,Sutskever-et-al-NIPS2014,Jean-et-al-arxiv2014}或\gls{convolutional_network}~\citep{Kalchbrenner+Blunsom-EMNLP2013}。
另一个模型(通常是RNN),则读取上下文$C$并且生成目标语言的句子。
在\figref{fig:chap12_encoder_decoder_architecture}中展示了这种用于机器翻译的\gls{encoder}-\gls{decoder}框架的总体思想。

% -- 461 --

\begin{figure}[htp]
\ifOpenSource
\centerline{\includegraphics{figure.pdf}}
\else
\centerline{\includegraphics{Chapter12/figures/encoder_decoder_architecture.pdf}}
\fi
\caption{\gls{encoder}-\gls{decoder}架构在直观\gls{representation}(例如词序列或图像)和语义\gls{representation}之间来回映射。
通过将一种模态数据的\gls{encoder}的输出(例如从法语句子到捕获句子含义的隐藏\gls{representation}的\gls{encoder}映射)作为另一模态数据的\gls{decoder}的输入(如\gls{decoder}将捕获句子含义的隐藏\gls{representation}映射到英语),我们可以训练将一种模态转换到另一种模态的系统。
这个想法已经成功应用于很多领域,不仅仅是机器翻译,还包括为图像生成标题。
}
\label{fig:chap12_encoder_decoder_architecture}
\end{figure}

为生成以源句为条件的整句,模型必须具有表示整个源句的方式。 
早期模型只能表示单个词或短语。
从\gls{representation_learning}的观点来看,具有相同含义的句子具有类似\gls{representation}是有用的,无论它们是以源语言还是以目标语言书写。
研究者首先使用卷积和~\glssymbol{RNN}~的组合探索该策略~\citep{Kalchbrenner+Blunsom-EMNLP2013}。
后来的工作介绍了使用~\glssymbol{RNN}~对所提议的翻译进行打分\citep{Cho-et-al-EMNLP2014}或生成翻译句子\citep{Sutskever-et-al-NIPS2014}。
\cite{Jean-et-al-arxiv2014}将这些模型扩展到更大的词汇表。

% -- 462 --

\subsubsection{使用\glsentrytext{attention_mechanism}并对齐数据片段}
\label{sec:using_an_attention_mechanism_and_aligning_pieces_of_data}
使用固定大小的表示概括非常长的句子(例如60个词)的所有语义细节是非常困难的。 
这需要使用足够大的RNN,并且用足够长时间训练得很好才能实现,如 \citet{Cho-et-al-EMNLP2014}和\citet{Sutskever-et-al-NIPS2014}所表明的。
然而,更高效的方法是先读取整个句子或段落(以获得正在表达的上下文和焦点),然后一次翻译一个词,每次聚焦于输入句子的不同部分来收集产生下一个输出词所需的语义细节。
这正是~\citet{Bahdanau-et-al-ICLR2015-small}第一次引入的想法。
\figref{fig:chap12_attention}中展示了\gls{attention_mechanism},其中每个\gls{time_step}关注输入序列的特定部分。

 \begin{figure}[htp]
\ifOpenSource
\centerline{\includegraphics{figure.pdf}}
\else
\centerline{\includegraphics{Chapter12/figures/attention.pdf}}
\fi
\caption{由~\citet{Bahdanau-et-al-ICLR2015-small}引入的现代\gls{attention_mechanism},本质上是加权平均。
\gls{attention_mechanism}对具有权重$\alpha^{(t)}$的特征向量$\Vh^{(t)}$进行加权平均形成上下文向量$\Vc$。
在一些应用中,特征向量$\Vh$是\gls{NN}的\gls{hidden_unit},但它们也可以是模型的原始输入。
权重$\alpha^{(t)}$由模型本身产生。
它们通常是区间$[0,1]$中的值,并且旨在仅仅集中在单个$\Vh^{(t)}$周围,使得加权平均近似模拟精确地读取一个特定\gls{time_step}的特征向量。
权重$\alpha^{(t)}$通常由模型另一部分发出的相关性得分应用softmax函数后产生。
\gls{attention_mechanism}在计算上需要比直接索引期望的$\Vh^{(t)}$付出更高的代价,但直接索引不能使用\gls{GD}训练。
基于加权平均的\gls{attention_mechanism}是平滑、可微的近似,可以使用现有优化算法训练。
}
\label{fig:chap12_attention}
\end{figure}

我们可以认为基于\gls{attention_mechanism}的系统有三个组件:
\begin{itemize}
 \item   读取器\emph{读取}原始数据(例如源语句中的源词)并将其转换为\gls{distributed_representation},其中一个特征向量与一个词的位置相关联。
 \item 用于存储读取器输出的特征向量的列表。这可以被理解为一个包含事实序列的\emph{存储器},之后可以从中检索,而检索不必以相同的顺序发生,也不必访问全部。
 \item 最后一个程序\emph{利用}存储器的内容顺序地执行任务,每个\gls{time_step}聚焦于某个存储器元素的内容(或几个,具有不同权重)。
\end{itemize}
第三组件可以生成翻译结果语句。

% -- 463 --

当用一种语言书写的句子中的词与另一种语言的翻译语句中的相应词对齐时,可以使对应的\gls{word_embedding}相关联。
早期的工作表明,我们可以学习将一种语言中的\gls{word_embedding}与另一种语言中的\gls{word_embedding}相关联的翻译矩阵\citep{Kocisky-et-al-ACL2014},与传统的基于短语表中频率计数的方法相比,可以产生较低的对齐错误率。
更早的工作\citep{Klementiev-et-al-COLING2012}也对跨语言词向量进行了研究。 
这种方法存在很多的扩展。
例如,允许在更大数据集上训练的更高效的跨语言对齐~\citep{Gouws-et-al-arxiv2014} 。

\subsection{历史展望}
\label{sec:historical_perspective_chap12}

在对\gls{back_propagation}的第一次探索中,\citet{Rumelhart86b-small}等人提出了\gls{distributed_representation}符号的思想,其中符号对应于家族成员的标识,而\gls{NN}捕获家族成员之间的关系,训练样本形成三元组如(Colin,Mother,Victoria)。
\gls{NN}的第一层学习每个族成员的表示。例如,Colin的特征可能代表Colin所在的族树,他所在树的分支,他来自哪一代等等。
我们可以将\gls{NN}认为是计算一组学到的规则,这些规则将属性关联在一起从而可以获得期望的预测结果。
模型于是可以进行预测,例如推断谁是Colin的母亲。

\cite{Deerwester90}将符号嵌入的想法扩展到对词的嵌入。
这些嵌入使用SVD学习。 
后来,嵌入通过\gls{NN}学习。

\gls{NLP}的历史是由流行表示(对模型输入不同方式的表示)的变化为标志的。
在早期对符号和词建模的工作之后,\gls{NN}在\glssymbol{NLP}上一些最早的应用\citep{Miikkulainen91,Schmidhuber96}将输入表示为字符序列。

\citet{BenDucVin01-small} 将焦点重新引到对词建模并引入\gls{NLM},能产生可解释的\gls{word_embedding}。
这些神经模型已经从一开始在一小组符号上定义的表示(20世纪80年代)扩展到现代应用中的数百万字(包括专有名词和拼写错误)。
这种计算扩展的努力导致了\secref{sec:high_dimensional_outputs}中描述的技术发明。

% -- 464 --

最初,使用词作为\gls{language_model}的基本单元改进了语言建模的性能 \citep{BenDucVin01-small}。
而今,新技术不断推动基于字符 \citep{Sutskever-et-al-ICML2011})和基于词的模型一起向前发展,最近的工作 \citep{gillick2015multilingual}甚至建模Unicode字符的单个字节。

\gls{NLM}背后的思想已经扩展到多个\gls{NLP}应用,如解析\citep{Henderson-NAACL2003,Henderson-ACL2004,Collobert-AISTATS2011}、词性标注、语义角色标注、分块等,有时使用共享\gls{word_embedding}的单一多任务学习架构\citep{Collobert+Weston-ICML2008,collobert2011natural}。

随着t-SNE~\gls{dimensionality_reduction}算法的发展\citep{VanDerMaaten08-small}以及Joseph Turian在2009年引入的专用于可视化词嵌入的应用,用于分析\gls{language_model}嵌入的二维可视化成为一种流行的工具。

\section{其他应用}
\label{sec:other_applications}

在本节中,我们介绍\gls{DL}一些其他类型的应用,它们与上面讨论的标准对象识别、语音识别和\gls{NLP}任务不同。
本书的第三部分将扩大这个范围,甚至进一步扩展到仍是目前主要研究领域的任务。


\subsection{\glsentrytext{recommender_system}}
\label{sec:recommender_systems}
信息技术部门中\gls{ML}的主要应用之一是向潜在用户或客户推荐项目。
这可以分为两种主要的应用:在线广告和项目建议(通常这些建议的目的仍然是为了销售产品)。
两者都依赖于预测用户和项目之间的关联, 一旦向该用户展示了广告或推荐了该产品,\gls{recommender_system}要么预测一些行为的概率(用户购买产品或该行为的一些代替)或预期增益(其可取决于产品的价值)。
目前,互联网的资金主要来自于各种形式的在线广告。
经济的主要部分依靠网上购物。 
包括Amazon和eBay在内的公司都使用了\gls{ML}(包括\gls{DL})推荐他们的产品。
有时,项目不是实际出售的产品,而是如选择在社交网络新闻信息流上显示的帖子、推荐观看的电影、推荐笑话、推荐专家建议、匹配视频游戏的玩家或匹配约会的人。

% -- 465 --
通常,这种关联问题可以作为\gls{supervised_learning}问题来处理:给出一些关于项目和关于用户的信息,预测感兴趣的行为(用户点击广告、输入评级、点击``喜欢''按钮、 购买产品,在产品上花钱、花时间访问产品页面等)。
通常这最终会归结到回归问题(预测一些条件期望值)或概率分类问题(预测一些离散事件的条件概率)。

早期\gls{recommender_system}的工作依赖于这些预测输入的最小信息:用户ID和项目ID。
在这种情况下,唯一的泛化方式依赖于不同用户或不同项目的目标变量值之间的模式相似性。
假设用户1和用户2都喜欢项目A,B和C.
由此,我们可以推断出用户1和用户2具有类似的口味。
如果用户1喜欢项目D,那么这可以强烈提示用户2也喜欢D。
基于此原理的算法称为\firstgls{collaborative_filtering}。
非参数方法(例如基于估计偏好模式之间相似性的最近邻方法)和参数方法都可能用来解决这个问题。
参数方法通常依赖于为每个用户和每个项目学习\gls{distributed_representation}(也称为嵌入)。
目标变量的双线性预测(例如评级)是一种简单的参数方法,这种方法非常成功,通常被认为是最先进系统的组成部分。
通过用户嵌入和项目嵌入之间的点积(可能需要使用仅依赖于用户ID或项目ID的常数来校正)获得预测。
令$\hat{\MR}$是包含我们预测的矩阵,$\MA$矩阵行中是用户嵌入,$\MB$矩阵列中具有项目嵌入。
令$\Vb$和$\Vc$是分别包含针对每个用户(表示用户平常坏脾气或积极的程度)以及每个项目(表示其大体受欢迎程度)的\gls{bias_aff}向量。
因此,双线性预测如下获得:
\begin{equation}
\label{eq:bilinear-prediction}
 \hat{R}_{u,i} = b_u + c_i + \sum_j A_{u,j} B_{j,i}.
\end{equation}
通常,人们希望最小化预测评级$\hat{R}_{u,i}$ 和实际评级${R}_{u,i}$ 之间的平方误差。
当用户嵌入和项目嵌入首次缩小到低维度(两个或三个)时,它们就可以方便地可视化,或者可以将用户或项目彼此进行比较(就像\gls{word_embedding})。
获得这些嵌入的一种方式是对实际目标(例如评级)的矩阵$\MR$进行\gls{SVD}。
这对应于将$\MR = \MU \MD \MV'$(或归一化的变体)分解为两个因子的乘积,低秩矩阵 $\MA=\MU \MD$ 和 $\MB = \MV'$。
\glssymbol{SVD}的一个问题是它以任意方式处理缺失条目,如同它们对应于目标值0。
相反,我们希望避免为缺失条目做出的预测付出任何代价。
幸运的是,观察到的评级的平方误差总和也可以使用基于梯度的优化最小化。
\glssymbol{SVD}~和\eqnref{eq:bilinear-prediction}中的双线性预测在~\ENNAME{Netflix}~奖竞赛中(目的是仅基于大量匿名用户的之前评级预测电影的评级)表现得非常好\citep{bennett2007netflix}。
许多\gls{ML}专家参加了2006年和2009年之间的这场比赛。
它提高了使用先进\gls{ML}的\gls{recommender_system}的研究水平,并改进了\gls{recommender_system}。
即使简单的双线性预测或~\glssymbol{SVD}~本身并没有赢得比赛,但它是大多数竞争对手提出的整体模型中一个组成部分,包括胜者\citep{BigChaos-Netflix2009,Koren09}。

% -- 466 --

除了这些具有\gls{distributed_representation}的双线性模型之外,第一次用于\gls{collaborative_filtering}的\gls{NN}之一是基于~\glssymbol{RBM}~的无向概率模型~\citep{Salakhutdinov-2007-short}。
\glssymbol{RBM}~是~\ENNAME{Netflix}~比赛获胜方法的一个重要组成部分\citep{BigChaos-Netflix2009,Koren09}。
\gls{NN}社群中也已经探索了对评级矩阵进行因子分解的更高级变体\citep{Salakhutdinov2008-small}。

然而,\gls{collaborative_filtering}系统有一个基本限制:当引入新项目或新用户时,缺乏评级历史意味着无法评估其与其他项目或用户的相似性,或者说无法评估新的用户和现有项目的联系。
这被称为冷启动推荐问题。
解决冷启动推荐问题的一般方式是引入单个用户和项目的额外信息。
例如,该额外信息可以是用户简要信息或每个项目的特征。
使用这种信息的系统被称为\,\textbf{基于内容的推荐系统}(content-based recommender system)。
从丰富的用户特征或项目特征集到嵌入的映射可以通过\gls{DL}架构学习\citep{Huang-et-al-2013,Elkahky-et-al-2015}。

专用的\gls{DL}架构,如\gls{convolutional_network}已经应用于从丰富内容中提取特征,如提取用于音乐推荐的音乐音轨\citep{vandenOord-et-al-NIPS2013}。
在该工作中,\gls{convolutional_network}将声学特征作为输入并计算相关歌曲的嵌入。
该歌曲嵌入和用户嵌入之间的点积则可以预测用户是否将收听该歌曲。

% -- 467 --

\subsubsection{\glsentrytext{exploration}与\glsentrytext{exploitation}}
当向用户推荐时,会产生超出普通\gls{supervised_learning}范围的问题,并进入\gls{RL}的领域。
理论上,许多推荐问题最准确的描述是\gls{contextual_bandit}\citep{Langford+Zhang-NIPS2008,Lu-et-al-2010}。
问题是,当我们使用\gls{recommender_system}收集数据时,我们得到是一个有偏且不完整的用户偏好观:我们只能看到用户对推荐给他们项目的反应,而不是其他项目。
此外,在某些情况下,我们可能无法获得未向其进行推荐的用户的任何信息(例如,在广告竞价中,可能是广告的建议价格低于最低价格阈值,或者没有赢得竞价,因此广告不会显示)。
更重要的是,我们不知道推荐任何其他项目会产生什么结果。
这就像训练一个分类器,为每个训练样本$\Vx$挑选一个类别$\hat y$(通常是基于模型最高概率的类别),然后只能获得该类别正确与否的反馈。
显然,每个样本传达的信息少于监督的情况(其中真实标签$y$是可直接访问的),因此需要更多的样本。
更糟糕的是,如果我们不够小心,即使收集越来越多的数据,我们得到的系统可能会继续选择错误的决定,因为正确的决定最初只有很低的概率:直到学习者选择正确的决定之前,该系统都无法学习正确的决定。
这类似于\gls{RL}的情况,其中仅观察到所选动作的奖励。
一般来说,\gls{RL}会涉及许多动作和许多奖励的序列。
\gls{bandit}~情景是\gls{RL}的特殊情况,其中学习者仅采取单一动作并接收单个奖励。
\gls{bandit}~问题在学习者知道哪个奖励与哪个动作相关联的时候,是更容易的。
在一般的\gls{RL}场景中,高奖励或低奖励可能是由最近的动作或很久以前的动作引起的。
术语\firstgls{contextual_bandit}指的是在一些输入变量可以通知决定的上下文中采取动作的情况。
例如,我们至少知道用户身份,并且我们要选择一个项目。
从上下文到动作的映射也称为\firstgls{policy}。
学习者和数据分布(现在取决于学习者的动作)之间的反馈循环是\gls{RL}和\gls{bandit}研究的中心问题。

% -- 468 --

\gls{RL}需要权衡\firstgls{exploration}与\firstgls{exploitation}。
\gls{exploitation}指的是从目前学到的最好\gls{policy}采取动作,也就是我们所知的将获得高奖励的动作。
\firstgls{exploration}是指采取行动以获得更多的训练数据。
如果我们知道给定上下文$\Vx$,动作$a$给予我们1的奖励,但我们不知道这是否是最好的奖励。
我们可能想利用我们目前的\gls{policy},并继续采取行动$a$相对肯定地获得1的奖励。
然而,我们也可能想通过尝试动作$a'$来探索。
我们不知道尝试动作$a'$会发生什么。
我们希望得到2的奖励,但有获得0奖励的风险。
无论如何,我们至少获得了一些知识。

\firstgls{exploration}可以以许多方式实现,从覆盖可能动作的整个空间的随机动作到基于模型的方法(基于预期回报和模型对该回报不确定性的量来计算动作的选择)。

许多因素决定了我们喜欢\gls{exploration}或\gls{exploitation}的程度。
最突出的因素之一是我们感兴趣的时间尺度。
如果代理只有短暂的时间积累奖励,那么我们喜欢更多的\gls{exploitation}。
如果代理有很长时间积累奖励,那么我们开始更多的\gls{exploration},以便使用更多的知识更有效地规划未来的动作。

\gls{supervised_learning}在\gls{exploration}或\gls{exploitation}之间没有权衡,因为监督信号总是指定哪个输出对于每个输入是正确的。
我们总是知道标签是最好的输出,没有必要尝试不同的输出来确定是否优于模型当前的输出 。

除了权衡\gls{exploration}和\gls{exploitation}之外,\gls{RL}背景下出现的另一个困难是难以评估和比较不同的\gls{policy}。
\gls{RL}包括学习者和环境之间的相互作用。
这个反馈回路意味着使用固定的测试集输入评估学习者的表现不是直接的。
\gls{policy}本身确定将看到哪些输入。
\citet{Dudik-2011} 提出了评估\gls{contextual_bandit}的技术。

% -- 469 --

\subsection{知识表示、推理和回答}
\label{sec:knowledge_representation_reasoning and_question_answering}
因为使用符号嵌入\citep{Rumelhart86b-small}和词嵌入\citep{Deerwester90,BenDucVin01-small},\gls{DL}方法在\gls{language_model}、机器翻译和\gls{NLP}方面非常成功。
这些嵌入表示关于单个词或概念的语义知识。
研究前沿是为短语或词和事实之间的关系开发嵌入。
搜索引擎已经使用\gls{ML}来实现这一目的,但是要改进这些更高级的表示还有许多工作要做。

\subsubsection{知识、联系和回答}
一个有趣的研究方向是确定如何训练\gls{distributed_representation}才能捕获两个实体之间的\firstgls{relation}。

数学中,\gls{binary_relation}是一组有序的对象对。
集合中的对具有这种关系,而那些不在集合中的对则没有。
例如,我们可以在实体集$\{ 1, 2, 3 \}$上定义关系``小于''来定义有序对的集合$\SetS = \{ (1, 2), (1, 3), (2, 3) \}$。
一旦这个关系被定义,我们可以像动词一样使用它。
因为$(1, 2) \in \SetS$,我们说1小于2。
因为$(2, 1) \not \in \SetS$,我们不能说2小于1。
当然,彼此相关的实体不必是数字。
我们可以定义关系{\tt is\_a\_type\_of}包含如{\tt(狗,哺乳动物)}的元组。

在~\glssymbol{AI}~的背景下,我们将\gls{relation}看作句法上简单且高度结构化的语言。
\gls{relation}起到动词的作用,而\gls{relation}的两个参数发挥着主体和客体的作用。
这些句子是一个三元组\gls{token}的形式:
\begin{align}
({\rm subject}, {\rm verb}, {\rm object})
\end{align}
其值是
\begin{align}
  ({\rm entity}_i, {\rm relation}_j, {\rm entity}_k).
\end{align}

我们还可以定义\firstgls{attribute},类似于\gls{relation}的概念,但只需要一个参数:
\begin{align}
  ( {\rm entity}_i, {\rm attribute}_j ).
\end{align}
例如,我们可以定义{\tt has\_fur} 属性,并将其应用于像{\tt 狗}这样的实体。

许多应用中需要表示\gls{relation}和推理。
我们如何在\gls{NN}中做到这一点?

\gls{ML}模型当然需要训练数据。
我们可以推断非结构化自然语言组成的训练数据集中实体之间的\gls{relation},也可以使用明确定义\gls{relation}的结构化数据库。 
这些数据库的共同结构是\gls{relational_database},它存储这种相同类型的信息,虽然没有格式化为三元\gls{token}的句子。
当数据库旨在将日常生活中常识或关于应用领域的专业知识传达给\gls{AI}系统时,我们将这种数据库称为\gls{knowledge_base}。
\gls{knowledge_base}包括一般的像{\tt Freebase}、{\tt OpenCyc}、 {\tt WordNet}、 {\tt Wikibase}\footnote{分别可以在如下网址获取: \url{freebase.com}, \url{cyc.com/opencyc},
\url{wordnet.princeton.edu}, \url{wikiba.se}}~等等,和专业的知识库,如GeneOntology\footnote{\url{geneontology.org}}。
实体和\gls{relation}的\gls{representation}可以将\gls{knowledge_base}中的每个三元组作为训练样本来学习,并且以最大化捕获它们的联合分布为训练目标\citep{Bordes-et-al-LSML2013}。

除了训练数据,我们还需定义训练的模型族。
一种常见的方法是将\gls{NLM}扩展到模型实体和\gls{relation}。
\gls{NLM}学习提供每个词\gls{distributed_representation}的向量。
他们还通过学习这些向量的函数来学习词之间的相互作用,例如哪些词可能出现在词序列之后。
我们可以学习每个关系的嵌入向量将这种方法扩展到实体和\gls{relation}。
事实上,建模语言和通过\gls{relation}编码建模知识的联系非常接近,研究人员可以\emph{同时}使用\gls{knowledge_base}\emph{和}自然语言句子训练这样的实体表示\citep{bordes-aaai-2011,Bordes-et-al-AISTATS2012-small,Wang-et-al-EMNLP2014},或组合来自多个\gls{relational_database}的数据~\citep{Bordes-et-al-NIPS2013}。
可能与这种模型相关联的特定参数化有许多种。
早期关于学习实体间\gls{relation}的工作~\citep{Paccanaro2000}假定高度受限的参数形式(``线性关系嵌入''),通常对\gls{relation}使用与实体形式不同的表示。
例如,\citet{Paccanaro2000}和~\citet{bordes-aaai-2011} 用向量表示实体而矩阵表示\gls{relation},其思想是\gls{relation}在实体上相当于运算符。
或者,关系可以被认为是任何其他实体\citep{Bordes-et-al-AISTATS2012-small},允许我们关于关系作声明,但是更灵活的是将它们结合在一起并建模联合分布的机制。

这种模型的实际短期应用是\firstgls{link_prediction}:预测\gls{knowledge_graph}中缺失的弧。
这是基于旧事实推广新事实的一种形式。
目前存在的大多数\gls{knowledge_base}都是通过人力劳动构建的,这往往使\gls{knowledge_base}缺失许多并且可能是大多数真正的关系。
请查看~\citet{Wang-et-al-AAAI2014}、\citet{Lin-et-al-AAAI2015}和~\citet{Garcia-Duran-et-al-arxiv2015}中这样应用的例子。

我们很难评估\gls{link_prediction}任务上模型的性能,因为我们的数据集只有正样本(已知是真实的事实)。
如果模型提出了不在数据集中的事实,我们不确定模型是犯了错误还是发现了一个新的以前未知的事实。
度量基于测试模型如何将已知真实事实的留存集合与不太可能为真的其他事实相比较,因此有些不精确。
构造感兴趣的负样本(可能为假的事实)的常见方式是从真实事实开始,并创建该事实的损坏版本,例如用随机选择的不同实体替换\gls{relation}中的一个实体 。
通用的测试精度(10$\%$度量)计算模型在该事实的所有损坏版本的前10$\%$中选择``正确''事实的次数。

\gls{knowledge_base}和\gls{distributed_representation}的另一个应用是\firstgls{word_sense_disambiguation} \citep{Navigli+Verlardi-2005,Bordes-et-al-AISTATS2012-small},这个任务决定在某些语境中哪个词的意义是恰当。

最后,知识的\gls{relation}结合一个推理过程和对自然语言的理解可以让我们建立一个一般的问答系统。
一般的问答系统必须能处理输入信息并记住重要的事实,并以之后能检索和推理的方式组织。
这仍然是一个困难的开放性问题,只能在受限的``玩具''环境下解决。
目前,记住和检索特定声明性事实的最佳方法是使用显式记忆机制,如\secref{sec:explicit_memory}所述。
\gls{memory_network}最开始是被用来解决一个玩具问答任务\citep{Weston2014}。
\citet{Kumar-et-al-arxiv2015} 提出了一种扩展,使用GRU~\gls{recurrent_network}将输入读入存储器并且在给定存储器的内容后产生回答。

\gls{DL}已经应用于其他许多应用(除了这里描述的应用以外),并且肯定会在此之后应用于更多的场景。
我们不可能全面描述与此主题相关的所有应用 。
本项调查尽可能地提供了在本文写作之时的代表性样本

第二部分介绍了涉及\gls{DL}的现代实践,包括了所有非常成功的方法。
一般而言,这些方法使用\gls{cost_function}的梯度寻找模型(近似于某些所期望的函数)的参数。
当具有足够的训练数据时,这种方法是非常强大的。
我们现在转到第三部分,开始进入研究领域,旨在使用较少的训练数据或执行更多样的任务。
而且相比目前为止所描述的情况,其中的挑战更困难并且远远没有解决。



